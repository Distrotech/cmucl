%% cmu-user.tex --- CMUCL User's Manual
%%
%% 2001-04-05  Eric Marsden
%% Modifications to work with hevea and pdflatex. 
%%
%% Aug 1997   Raymond Toy
%% This is a modified version of the original CMUCL User's Manual.
%% The key changes are modification of this file to use standard
%% LaTeX2e.  This means latexinfo isn't going to work anymore.
%% However, Latex2html support has been added.
%%
%% Jan 1998 Paul Werkowski
%% A few of the packages below are not part of the standard LaTeX2e
%% distribution, and must be obtained from a repository. At this time
%% I was able to fetch from
%% ftp.cdrom.com:pub/tex/ctan/macros/latex/contrib/supported/
%%                     camel/index.ins
%%                     camel/index.dtx
%%                     calc/calc.ins
%%                     calc/calc.dtx
%%                     changebar/changebar.ins
%%                     changebar/changebar.dtx
%% One runs latex on the .ins file to produce .tex and/or .sty
%% files that must be put in a path searched by latex.
%%
%% Note all of the required packages are included in the teTeX distribution,
%% and a current version of latex2html can be obtained from:
%% http://saftsack.fs.uni-bayreuth.de/~latex2ht/

%% Delete "[a4paper]" if you don't want this formatted for A4 paper.
\documentclass[a4paper]{report}
\usepackage{xspace}
\usepackage{alltt}
\usepackage{index}
\usepackage{ifthen}
\usepackage{calc}
\usepackage{sectsty}
\usepackage{varioref}
\usepackage[hyperindex=false,colorlinks=false,urlcolor=blue]{hyperref}
%% \usepackage{html}
\usepackage{typehtml}

% macro.tex
%
% LaTeX macros for CMUCL User's Manual
%
% by Raymond Toy

% use Palatino
\renewcommand{\rmdefault}{ppl}
%\ifpdf
%\usepackage{palatino}
%\fi

%% Define the indices.  We need one for Types, Variables, Functions,
%% and a general concept index.
\makeindex
\newindex{types}{tdx}{tnd}{Type Index}
\newindex{vars}{vdx}{vnd}{Variable Index}
\newindex{funs}{fdx}{fnd}{Function Index}
\newindex{concept}{cdx}{cnd}{Concept Index}

\newcommand{\tindexed}[1]{\index[types]{#1}\code{#1}}
\newcommand{\findexed}[1]{\index[funs]{#1}\code{#1}}
\newcommand{\vindexed}[1]{\index[vars]{#1}\code{*#1*}}
\newcommand{\cindex}[1]{\index[concept]{#1}}
\newcommand{\cpsubindex}[2]{\index[concept]{#1!#2}}


%% This code taken from the LaTeX companion.  It's meant as a
%% replacement for the description environment.  We want one that
%% prints description items in a fixed size box and puts the
%% description itself on the same line or the next depending on the
%% size of the item.
\newcommand{\entrylabel}[1]{\mbox{#1}\hfil}
\newenvironment{entry}{%
  \begin{list}{}%
    {\renewcommand{\makelabel}{\entrylabel}%
      \setlength{\labelwidth}{45pt}%
      \setlength{\leftmargin}{\labelwidth+\labelsep}}}%
  {\end{list}}

\newlength{\Mylen}
\newcommand{\Lentrylabel}[1]{%
  \settowidth{\Mylen}{#1}%
  \ifthenelse{\lengthtest{\Mylen > \labelwidth}}%
  {\parbox[b]{\labelwidth}%  term > labelwidth
    {\makebox[0pt][l]{#1}\\}}%
  {#1}%
  \hfil\relax}
\newenvironment{Lentry}{%
  \renewcommand{\entrylabel}{\Lentrylabel}
  \begin{entry}}%
  {\end{entry}}

\newcommand{\fcntype}[1]{\textit{#1}}
\newcommand{\argtype}[1]{\textit{#1}}
\newcommand{\fcnname}[1]{\textsf{#1}}

\newlength{\formnamelen}        % length of a name of a form
\newlength{\pboxargslen}        % length of parbox for arguments
\newlength{\typelen}            % length of the type label for the form

\newcommand{\args}[1]{#1}
\newcommand{\keys}[1]{\code{\&key} \= #1}
\newcommand{\morekeys}[1]{\\ \> #1}
\newcommand{\yetmorekeys}[1]{\\ \> #1}

\newcommand{\defunvspace}{\ifhmode\unskip \par\fi\addvspace{18pt plus 12pt minus 6pt}}


%% \layout[pkg]{name}{param list}{type}
%%
%% This lays out a entry like so:
%%
%% pkg:name arg1 arg2                             [Function]
%%
%% where [Function] is flush right.
%%
\newcommand{\layout}[4][\mbox{}]{%
  \par\noindent
  \fcnname{#1#2\hspace{1em}}%
  \settowidth{\formnamelen}{\fcnname{#1#2\hspace{1em}}}%
  \settowidth{\typelen}{[\argtype{#4}]}%
  \setlength{\pboxargslen}{\linewidth}%
  \addtolength{\pboxargslen}{-1\formnamelen}%
  \addtolength{\pboxargslen}{-1\typelen}%
  \begin{minipage}[t]{\pboxargslen}
    \begin{tabbing}
      #3
    \end{tabbing}
  \end{minipage}
  \hfill[\fcntype{#4}]%
  \par\addvspace{2pt plus 2pt minus 2pt}}

\newcommand{\vrindexbold}[1]{\index[vars]{#1|textbf}}
\newcommand{\fnindexbold}[1]{\index[funs]{#1|textbf}}

%% Define a new type
%%
%% \begin{deftp}{typeclass}{typename}{args}
%%    some description
%% \end{deftp}
\newenvironment{deftp}[3]{%
  \par\bigskip\index[types]{#2|textbf}%
  \layout{#2}{\var{#3}}{#1}
  }{}

%% Define a function with name NAME and given parameters PARAM.  The
%% function is in the package PKG.  If the optional arg SUFFIX is
%% given, this is used as a suffix for the label.  (Useful when you
%% have functions of the same name, such as methods, but want
%% different labels for each version.)
%%
%% The defunx is for additional functions that are related to this one
%% in some way, and we want to group them all together.
%%  
%% \begin{defun}[suffix]{pkg}{name}{params}
%%   \defunx[pkg]{name}{params}
%%   description of function
%% \end{defun}
\newenvironment{defun}[4][]{%
  \par\defunvspace\fnindexbold{#3}\label{FN:#3#1}%
  \layout[#2]{#3}{#4}{Function}
  }{}
\newcommand{\defunx}[3][\mbox{}]{%
  \par\fnindexbold{#2}\label{FN:#2}%
  \layout[#1]{#2}{#3}{Function}}

%% Define a generic function.  Like defun, but for defgeneric.
%%
%% \begin{defgeneric}[suffix]{pkg}{name}{params}
%%   \defgenericx[pkg]{name}{params}
%%   description of function
%% \end{defgeneric}
\newenvironment{defgeneric}[4][]{%
  \par\defunvspace\fnindexbold{#3}\label{FN:#3-generic#1}%
  \layout[#2]{#3}{#4}{Generic Function}
  }{}
\newcommand{\defgenericx}[3][\mbox{}]{%
  \par\fnindexbold{#2}\label{FN:#2}%
  \layout[#1]{#2}{#3}{Generic Function}}

%% Define a method.  Like defgeneric, but for methods.
%%
%% \begin{defmethod}[suffix]{pkg}{name}{params}
%%   \defmethod[pkg]{name}{params}
%%   description of function
%% \end{defmethod}
\newenvironment{defmethod}[4][]{%
  \par\defunvspace\fnindexbold{#3}\label{FN:#3-method#1}%
  \layout[#2]{#3}{#4}{Method}
  }{}
\newcommand{\defmethodx}[3][\mbox{}]{%
  \par\fnindexbold{#2}\label{FN:#2}%
  \layout[#1]{#2}{#3}{Method}}

%% Define a macro
%%
%% \begin{defmac}[suffix]{pkg}{name}{params}
%%   \defmacx[pkg]{name}{params}
%%   description of macro
%% \end{defmac}
\newenvironment{defmac}[4][]{%
  \par\defunvspace\fnindexbold{#3}\label{FN:#3#1}%
  \layout[#2]{#3}{#4}{Macro}}{}
\newcommand{\defmacx}[3][\mbox{}]{%
  \par\fnindexbold{#2}\label{FN:#2}%
  \layout[#1]{#2}{#3}{Macro}}

%% Define a variable
%%
%% \begin{defvar}{pkg}{name}
%%   \defvarx[pkg]{name}
%%   description of defvar
%% \end{defvar}
\newenvironment{defvar}[2]{%
  \par\defunvspace\vrindexbold{#2}\label{VR:#2}
  \layout[#1]{*#2*}{}{Variable}}{}
\newcommand{\defvarx}[2][\mbox{}]{%
  \par\vrindexbold{#2}\label{VR:#2}
  \layout[#1]{*#2*}{}{Variable}}

%% Define a constant
%%
%% \begin{defconst}{pkg}{name}
%%   \ddefconstx[pkg]{name}
%%   description of defconst
%% \end{defconst}
\newcommand{\defconstx}[2][\mbox{}]{%
  \layout[#1]{#2}{}{Constant}}
\newenvironment{defconst}[2]{%
  \defunvspace\defconstx[#1]{#2}}{}

\newcommand{\credits}[1]{%
  \begin{center}
  \textbf{#1}
  \end{center}}

\newenvironment{example}{\begin{quote}\begin{alltt}}{\end{alltt}\end{quote}}
\newenvironment{lisp}{\begin{example}}{\end{example}}

\newcommand{\hide}[1]{}
\newcommand{\trnumber}[1]{#1}
\newcommand{\citationinfo}[1]{#1}
\newcommand{\var}[1]{{\textsf{\textsl{#1}}\xspace}}
\newcommand{\code}[1]{\textnormal{{\sffamily #1}}}
\newcommand{\file}[1]{`\texttt{#1}'}
\newcommand{\kwd}[1]{\code{:#1}}
\newcommand{\F}[1]{\code{#1}}
\newcommand{\w}[1]{\hbox{#1}}
\newcommand{\ctrl}[1]{$\uparrow$\textsf{#1}}
\newcommand{\result}{$\Rightarrow$}
\newcommand{\myequiv}{$\equiv$}
\newcommand{\back}[1]{\(\backslash\)#1}
\newcommand{\pxlref}[1]{see section~\ref{#1}, page~\pageref{#1}}
\newcommand{\xlref}[1]{See section~\ref{#1}, page~\pageref{#1}}
\newcommand{\funref}[1]{\findexed{#1} (page~\pageref{FN:#1})}
\newcommand{\specref}[1]{\findexed{#1} (page~\pageref{FN:#1})}
\newcommand{\macref}[1]{\findexed{#1} (page~\pageref{FN:#1})}
\newcommand{\varref}[1]{\vindexed{#1} (page~\pageref{VR:#1})}
\newcommand{\conref}[1]{\conindexed{#1} (page~\pageref{VR:#1})}

\newcommand{\false}{\code{nil}}
\newcommand{\true}{\code{t}}
\newcommand{\nil}{\false{}}
%% Printed lisp character #\foo
\newcommand{\lispchar}[1]{\code{\#\back{#1}}}

\newcommand{\ampoptional}{\code{\&optional}}
\newcommand{\amprest}{\code{\&rest}}
\newcommand{\ampbody}{\code{\&body}}

\newcommand{\mopt}[1]{{$\,\{$}\textnormal{\textsf{\textsl{#1\/}}}{$\}\,$}}
\newcommand{\mstar}[1]{{$\,\{$}\textnormal{\textsf{\textsl{#1\/}}}{$\}^*\,$}}
\newcommand{\mplus}[1]{{$\,\{$}\textnormal{\textsf{\textsl{#1\/}}}{$\}^+\,$}}
\newcommand{\mgroup}[1]{{$\,\{$}\textnormal{\textsf{\textsl{#1\/}}}{$\}\,$}}
\newcommand{\mor}{$|$}


%% Some common abbreviations
\newcommand{\dash}{---}
\newcommand{\alien}{Alien}
\newcommand{\aliens}{Aliens}
\newcommand{\hemlock}{Hemlock}
\newcommand{\python}{Python}
\newcommand{\cmucl}{\textsc{cmucl}}
\newcommand{\clisp}{Common Lisp}
\newcommand{\llisp}{Common Lisp}
\newcommand{\cltl}{\textit{Common Lisp: The Language}}
\newcommand{\cltltwo}{\textit{Common Lisp: The Language II}}


%% Set up margins
\setlength{\oddsidemargin}{-10pt}
\setlength{\evensidemargin}{-10pt}
\setlength{\topmargin}{-40pt}
\setlength{\headheight}{12pt}
\setlength{\headsep}{25pt}
\setlength{\footskip}{30pt}
\setlength{\textheight}{9.25in}
\setlength{\textwidth}{6.75in}
\setlength{\columnsep}{0.375in}
\setlength{\columnseprule}{0pt}


\setcounter{tocdepth}{2}
\setcounter{secnumdepth}{3}
\def\textfraction{.1}
\def\bottomfraction{.9}         % was .3
\def\topfraction{.9}

%% Allow TeX some stretching space to avoid overfull and underfull
%% boxes.
\setlength{\emergencystretch}{5pt}

%% requires the sectsty package
\allsectionsfont{\bfseries\sffamily}
\chapterfont{\fontfamily{pag}\selectfont}

%% section numbers in the left margin
\makeatletter
\def\@seccntformat#1{\protect\makebox[0pt][r]{\csname
    the#1\endcsname\quad}}
\makeatother



\title{CMUCL User's Manual}
\author{Robert A. MacLachlan, \textit{Editor}}
\newcommand{\keywords}{lisp, Common Lisp, manual, compiler, programming
language implementation, programming environment}

\date{November 2011 \\ 20c}


\begin{document}

\begin{titlepage}
  \makeatletter
  \vspace{60pt}
  \begin{center}
    \rule{\linewidth}{0.7mm}
    \vspace{3em}
    {\Huge \@title \par}
    \vspace{4em}
     {\large
       \begin{tabular}[t]{c}
         \@author
       \end{tabular}\par}
       \vspace{2em}
     {\large \@date \par}
     \vspace{2em}
    \rule{\linewidth}{0.7mm}
  \end{center}
  \vfill

  \begin{quotation}
    \cmucl{} is a free, high-performance implementation of the Common Lisp
    programming language, which runs on most major Unix platforms. It
    mainly conforms to the ANSI Common Lisp Standard. \cmucl{} features a
    sophisticated native-code compiler, a foreign function interface, a
    graphical source-level debugger, an interface to the X11 Window
    System, and an Emacs-like editor.

    \medskip \textbf{Keywords}: \keywords
  \end{quotation}

  \vspace{5cm}
  
  This manual is based on CMU Technical Report CMU-CS-92-161, edited by
  Robert A. MacLachlan, dated July 1992.

  \thispagestyle{empty}
  \makeatother
\end{titlepage}

\ifpdf
\pdfinfo{
/Author (Robert A. MacLachlan, ed)
/Title (CMUCL User's Manual)
/Keywords (\keywords)
}
% Add section numbers to the bookmarks, and open 2 levels by default.
\hypersetup{bookmarksnumbered=true,
            bookmarksopen=true,
            bookmarksopenlevel=2}
\fi

% \maketitle

\pagestyle{headings}
\pagenumbering{roman}
\tableofcontents

\clearpage
\pagenumbering{arabic}

\chapter{Introduction}

\cmucl{} is a free, high-performance implementation of the Common Lisp
programming language which runs on most major Unix platforms. It
mainly conforms to the ANSI Common Lisp standard. Here is a summary of
its main features:

\begin{itemize}
\item a {\em sophisticated native-code compiler} which is capable of
powerful type inferences, and generates code competitive in speed with
C compilers.

\item generational garbage collection and multiprocessing
capability on the x86 ports.

\item a foreign function interface which allows interfacing with C code and
system libraries, including shared libraries on most platforms, and
direct access to Unix system calls.

\item support for interprocess communication and remote procedure
calls.
     
\item an implementation of CLOS, the Common Lisp Object System, which
includes multimethods and a metaobject protocol.

\item a graphical source-level debugger using a Motif interface, and a
code profiler.

\item an interface to the X11 Window System (CLX), and a sophisticated
graphical widget library (Garnet).

\item programmer-extensible input and output streams.
                        
\item an Emacs-like editor implemented in Common Lisp.

\item public domain: free, with full source code and no
strings attached (and no warranty).  Like GNU/Linux and the *BSD
operating systems, \cmucl{} is maintained and improved by a team of
volunteers collaborating over the Internet.
\end{itemize}


This user's manual contains only implementation-specific information
about \cmucl. Users will also need a separate manual describing the
\clisp{} standard, for example, the
\ifpdf
\href{http://www.xanalys.com/software_tools/reference/HyperSpec/FrontMatter/index-text.html}
{Hyperspec}.
\else
\emph{Hyperspec} at \href{http://www.xanalys.com/software_tools}{www.xanalys.com}
\fi


In addition to the language itself, this document describes a number
of useful library modules that run in \cmucl. \hemlock, an Emacs-like
text editor, is included as an integral part of the \cmucl{}
environment. Two documents describe \hemlock{}: the {\it Hemlock
User's Manual}, and the {\it Hemlock Command Implementor's Manual}.


\section{Distribution and Support}

\cmucl{} is developed and maintained by a group of volunteers who
collaborate over the internet. Sources and binary releases for the
various supported platforms can be obtained from
\href{http://www.cons.org/cmucl/}{www.cons.org/cmucl}. These pages
describe how to download by FTP or CVS.

A number of mailing lists are available for users and developers;
please see the web site for more information. 


\section{Command Line Options}
\cindex{command line options}
\label{command-line-options}

The command line syntax and environment is described in the
\verb|lisp(1)| man page in the man/man1 directory of the distribution.
See also \verb|cmucl(1)|. Currently \cmucl{} accepts the following
switches:

\begin{Lentry}
\item[\code{-batch}] specifies batch mode, where all input is
  directed from standard-input.  An error code of 0 is returned upon
  encountering an EOF and 1 otherwise.

\item[\code{-quiet}] enters quiet mode. This implies setting the
  variables \code{*load-verbose*}, \code{*compile-verbose*},
  \code{*compile-print*}, \code{*compile-progress*},
  \code{*require-verbose*} and \code{*gc-verbose*} to NIL, and
  disables the printing of the startup banner.

\item[\code{-core}] requires an argument that should be the name of a
  core file.  Rather than using the default core file, which is searched
  in a number of places, according to the initial value of the
  \code{library:} search-list, the specified core file is loaded.  This
  switch overrides the value of the \code{CMUCLCORE} environment variable,
  if present.
  
\item[\code{-lib}] requires an argument that should be the path to the
  CMUCL library directory, which is going to be used to initialize the
  \code{library:} search-list, among other things.  This switch overrides
  the value of the \code{CMUCLLIB} environment variable, if present.

\item[\code{-dynamic-space-size}] requires an argument that should be
  the number of megabytes (1048576 bytes) that should be allocated to
  the heap.  If not specified, a platform-specific default is used.
  The actual maximum allowed heap size is platform-specific.

  Currently, this option is only available for the x86 and sparc
  platforms. 

\item[\code{-edit}] specifies to enter Hemlock.  A file to edit may be
  specified by placing the name of the file between the program name
  (usually \file{lisp}) and the first switch.
  
\item[\code{-eval}] accepts one argument which should be a Lisp form
  to evaluate during the start up sequence.  The value of the form
  will not be printed unless it is wrapped in a form that does output.
  
\item[\code{-hinit}] accepts an argument that should be the name of
  the hemlock init file to load the first time the function
  \findexed{ed} is invoked.  The default is to load
  \file{hemlock-init.\var{object-type}}, or if that does not exist,
  \file{hemlock-init.lisp} from the user's home directory.  If the
  file is not in the user's home directory, the full path must be
  specified.
  
\item[\code{-init}] accepts an argument that should be the name of an
  init file to load during the normal start up sequence.  The default
  is to load \file{init.\var{object-type}} or, if that does not exist,
  \file{init.lisp} from the user's home directory.  If neither exists,
  \cmucl tries \file{.cmucl-init.\var{object-type}} and then
  \file{.cmucl-init.lisp}.  If the file is not
  in the user's home directory, the full path must be specified.  If
  the file does not exist, \cmucl silently ignores it.
  
\item[\code{-noinit}] accepts no arguments and specifies that an init
  file should not be loaded during the normal start up sequence.
  Also, this switch suppresses the loading of a hemlock init file when
  Hemlock is started up with the \code{-edit} switch.

\item[\code{-nositeinit}] accepts no arguments and specifies that the
  site init file should not be loaded during the normal start up
  sequence. 

\item[\code{-load}] accepts an argument which should be the name of a
  file to load into Lisp before entering Lisp's read-eval-print loop.
  
\item[\code{-slave}] specifies that Lisp should start up as a
  \i{slave} Lisp and try to connect to an editor Lisp.  The name of
  the editor to connect to must be specified\dash{}to find the
  editor's name, use the \hemlock{} ``\code{Accept Slave
    Connections}'' command.  The name for the editor Lisp is of the
  form:
  \begin{example}
    \var{machine-name}\code{:}\var{socket}
  \end{example}
  where \var{machine-name} is the internet host name for the machine
  and \var{socket} is the decimal number of the socket to connect to.

\item[\code{-fpu}] specifies what fpu should be used for x87 machines.
  The possible values are ``\code{x87}'', ``\code{sse2}'', or
  ``\code{auto}'', which is the default.  By default, \cmucl will
  detect if the chip supports the SSE2 instruction set or not.  If so
  or if \code{-fpu sse2} is specified, the SSE2 core will be loaded
  that uses SSE2 for floating-point arithmetic.  If SSE2 is not
  available or if \code{-fpu x87} is given, the legacy x87 core is
  loaded.

\item[\code{--}] indicates that everything after ``\code{--}'' is not
  subject to \cmucl's command line parsing.  Everything after
  ``\code{--}'' is placed in the variable
  \code{ext:*command-line-application-arguments*}.
\end{Lentry}

For more details on the use of the \code{-edit} and \code{-slave}
switches, see the {\it Hemlock User's Manual}.

Arguments to the above switches can be specified in one of two ways:
\w{\var{switch}\code{=}\var{value}} or
\w{\var{switch}<\var{space}>\var{value}}.  For example, to start up
the saved core file mylisp.core use either of the following two
commands:

\begin{example}
   lisp -core=mylisp.core
   lisp -core mylisp.core
\end{example}


\section{Credits}

\cmucl{} was developed at the Computer Science Department of Carnegie
Mellon University. The work was a small autonomous part within the
Mach microkernel-based operating system project, and started more as a
tool development effort than a research project. The project started
out as Spice Lisp, which provided a modern Lisp implementation for use
in the CMU community. \cmucl{} has been under continual development since
the early 1980's (concurrent with the \clisp{} standardization
effort). Most of the CMU Common Lisp implementors went on to work on
the Gwydion environment for Dylan. The CMU team was lead by Scott E.
Fahlman, the \python{} compiler was written by Robert MacLachlan.

\cmucl{}'s CLOS implementation is derived from the PCL reference
implementation written at Xerox PARC:
\begin{quotation}
\noindent Copyright (c) 1985, 1986, 1987, 1988, 1989, 1990 Xerox
Corporation.\\
All rights reserved.

\vspace{1ex}
\noindent Use and copying of this software and preparation of
derivative works based upon this software are permitted.  Any
distribution of this software or derivative works must comply with all
applicable United States export control laws.

\vspace{1ex}
\noindent This software is made available AS IS, and Xerox Corporation
makes no warranty about the software, its performance or its
conformity to any specification.
\end{quotation}
Its implementation of the LOOP macro was derived from code from
Symbolics, which was derived from code written at MIT:
\begin{quotation}
\noindent Portions of LOOP are Copyright (c) 1986 by the Massachusetts
Institute of Technology.\\
All Rights Reserved.

\vspace{1ex}
\noindent Permission to use, copy, modify and distribute this software
and its documentation for any purpose and without fee is hereby granted,
provided that the M.I.T. copyright notice appear in all copies and that
both that copyright notice and this permission notice appear in
supporting documentation.  The names "M.I.T." and "Massachusetts
Institute of Technology" may not be used in advertising or publicity
pertaining to distribution of the software without specific, written
prior permission.  Notice must be given in supporting documentation that
copying distribution is by permission of M.I.T.  M.I.T. makes no
representations about the suitability of this software for any purpose.
It is provided "as is" without express or implied warranty.

\vspace{3ex}
\noindent Portions of LOOP are Copyright (c) 1989, 1990, 1991, 1992 by
Symbolics, Inc.\\
All Rights Reserved.

\vspace{1ex}
\noindent Permission to use, copy, modify and distribute this software
and its documentation for any purpose and without fee is hereby
granted, provided that the Symbolics copyright notice appear in all
copies and that both that copyright notice and this permission notice
appear in supporting documentation.  The name "Symbolics" may not be
used in advertising or publicity pertaining to distribution of the
software without specific, written prior permission.  Notice must be
given in supporting documentation that copying distribution is by
permission of Symbolics.  Symbolics makes no representations about the
suitability of this software for any purpose.  It is provided "as is"
without express or implied warranty.

\vspace{1ex}
\noindent Symbolics, CLOE Runtime, and Minima are trademarks, and
CLOE, Genera, and Zetalisp are registered trademarks of Symbolics,
Inc.
\end{quotation}
The CLX code is copyrighted by Texas Instruments Incorporated:
\begin{quotation}
\noindent Copyright (C) 1987 Texas Instruments Incorporated.

\vspace{1ex}
\noindent Permission is granted to any individual or institution to
use, copy, modify, and distribute this software, provided that this
complete copyright and permission notice is maintained, intact, in all
copies and supporting documentation.

\vspace{1ex}
\noindent Texas Instruments Incorporated provides this software "as
is" without express or implied warranty.
\end{quotation}

\cmucl{} was funded by DARPA under CMU's "Research on Parallel Computing"
contract. Rather than doing pure research on programming languages and
environments, the emphasis was on developing practical programming
tools. Sometimes this required new technology, but much of the work
was in creating a \clisp{} environment that incorporates
state-of-the-art features from existing systems (both Lisp and
non-Lisp). Archives of the project are available online.

The project funding stopped in 1994, so support at Carnegie Mellon
University has been discontinued. All code and documentation developed
at CMU was released into the public domain. The project continues as a
group of users and developers collaborating over the Internet. The
current and previous maintainers include:

\begin{itemize}
\item Marco Antoniotti
\item Martin Cracauer
\item Fred Gilham
\item Alex Goncharov
\item Rob MacLachlan
\item Pierre Mai
\item Eric Marsden
\item Gerd Moellman
\item Tim Moore
\item Carl Shapiro  
\item Raymond Toy
\item Peter Van Eynde
\item Paul Werkowski
\end{itemize}

In particular, Paul Werkowski and Douglas Crosher completed the port
for the x86 architecture for FreeBSD. Peter VanEnyde took the FreeBSD
port and created a Linux version. Other people who have contributed to
the development of \cmucl{} since 1981 are

\begin{itemize}
\item David Axmark
\item Miles Bader
\item Rick Busdiecker
\item Bill Chiles
\item Douglas Thomas Crosher
\item Casper Dik
\item Ted Dunning
\item Scott Fahlman
\item Mike Garland
\item Paul Gleichauf
\item Sean Hallgren
\item Richard Harris
\item Joerg-Cyril Hoehl
\item Chris Hoover
\item John Kolojejchick
\item Todd Kaufmann
\item Simon Leinen
\item Sandra Loosemore
\item William Lott
\item Dave McDonald
\item Tim Moore
\item Skef Wholey
\item Paul Foley
\item Helmut Eller
\end{itemize}

Countless others have contributed to the project by sending in bug
reports, bug fixes, and new features.

This manual is based on CMU Technical Report CMU-CS-92-161, edited by
Robert A. MacLachlan, dated July 1992. Other contributors include
Raymond Toy, Paul Werkowski and Eric Marsden. The Hierarchical
Packages chapter is based on documentation written by Franz. Inc, and
is used with permission. The remainder of the document is in the
public domain.

\chapter{Design Choices and Extensions}

Several design choices in \clisp{} are left to the individual
implementation, and some essential parts of the programming environment
are left undefined.  This chapter discusses the most important design
choices and extensions.

\section{Data Types}

\subsection{Integers}

The \tindexed{fixnum} type is equivalent to \code{(signed-byte 30)}.
Integers outside this range are represented as a \tindexed{bignum} or
a word integer (\pxlref{word-integers}.)  Almost all integers that
appear in programs can be represented as a \code{fixnum}, so integer
number consing is rare.


\subsection{Floats}
\label{ieee-float}

\cmucl{} supports three floating point formats:
\tindexed{single-float}, \tindexed{double-float} and
\tindexed{double-double-float}.  The first two are implemented with
IEEE single and double float arithmetic, respectively.  The last is an
extension; \pxlref{extended-float} for more information.
\code{short-float} is a synonym for \code{single-float}, and
\code{long-float} is a synonym for \code{double-float}.  The initial
value of \vindexed{read-default-float-format} is \code{single-float}.

Both \code{single-float} and \code{double-float} are represented with
a pointer descriptor, so float operations can cause number consing.
Number consing is greatly reduced if programs are written to allow the
use of non-descriptor representations (\pxlref{numeric-types}.)


\subsubsection{IEEE Special Values}

\cmucl{} supports the IEEE infinity and NaN special values.  These
non-numeric values will only be generated when trapping is disabled
for some floating point exception (\pxlref{float-traps}), so users of
the default configuration need not concern themselves with special
values.

\begin{defconst}{extensions:}{short-float-positive-infinity}
  \defconstx[extensions:]{short-float-negative-infinity}
  \defconstx[extensions:]{single-float-positive-infinity}
  \defconstx[extensions:]{single-float-negative-infinity}
  \defconstx[extensions:]{double-float-positive-infinity}
  \defconstx[extensions:]{double-float-negative-infinity}
  \defconstx[extensions:]{long-float-positive-infinity}
  \defconstx[extensions:]{long-float-negative-infinity}
  
  The values of these constants are the IEEE positive and negative
  infinity objects for each float format.
\end{defconst}

\begin{defun}{extensions:}{float-infinity-p}{\args{\var{x}}}
  
  This function returns true if \var{x} is an IEEE float infinity (of
  either sign.)  \var{x} must be a float.
\end{defun}

\begin{defun}{extensions:}{float-nan-p}{\args{\var{x}}}
  \defunx[extensions:]{float-trapping-nan-p}{\args{\var{x}}}
  
  \code{float-nan-p} returns true if \var{x} is an IEEE NaN (Not A
  Number) object.  \code{float-trapping-nan-p} returns true only if
  \var{x} is a trapping NaN.  With either function, \var{x} must be a
  float.
\end{defun}

\subsubsection{Negative Zero}

The IEEE float format provides for distinct positive and negative
zeros.  To test the sign on zero (or any other float), use the
\clisp{} \findexed{float-sign} function.  Negative zero prints as
\code{-0.0f0} or \code{-0.0d0}.

\subsubsection{Denormalized Floats}

\cmucl{} supports IEEE denormalized floats.  Denormalized floats
provide a mechanism for gradual underflow.  The \clisp{}
\findexed{float-precision} function returns the actual precision of a
denormalized float, which will be less than \findexed{float-digits}.
Note that in order to generate (or even print) denormalized floats,
trapping must be disabled for the underflow exception
(\pxlref{float-traps}.)  The \clisp{}
\w{\code{least-positive-}\var{format}-\code{float}} constants are
denormalized.

\begin{defun}{extensions:}{float-denormalized-p}{\args{\var{x}}}
  
  This function returns true if \var{x} is a denormalized float.
  \var{x} must be a float.
\end{defun}


\subsubsection{Floating Point Exceptions}
\label{float-traps}

The IEEE floating point standard defines several exceptions that occur
when the result of a floating point operation is unclear or
undesirable.  Exceptions can be ignored, in which case some default
action is taken, such as returning a special value.  When trapping is
enabled for an exception, a error is signalled whenever that exception
occurs.  These are the possible floating point exceptions:
\begin{Lentry}
  
\item[\kwd{underflow}] This exception occurs when the result of an
  operation is too small to be represented as a normalized float in
  its format.  If trapping is enabled, the
  \tindexed{floating-point-underflow} condition is signalled.
  Otherwise, the operation results in a denormalized float or zero.
  
\item[\kwd{overflow}] This exception occurs when the result of an
  operation is too large to be represented as a float in its format.
  If trapping is enabled, the \tindexed{floating-point-overflow}
  exception is signalled.  Otherwise, the operation results in the
  appropriate infinity.
  
\item[\kwd{inexact}] This exception occurs when the result of a
  floating point operation is not exact, i.e. the result was rounded.
  If trapping is enabled, the \code{extensions:floating-point-inexact}
  condition is signalled.  Otherwise, the rounded result is returned.
  
\item[\kwd{invalid}] This exception occurs when the result of an
  operation is ill-defined, such as \code{\w{(/ 0.0 0.0)}}.  If
  trapping is enabled, the \code{extensions:floating-point-invalid}
  condition is signalled.  Otherwise, a quiet NaN is returned.
  
\item[\kwd{divide-by-zero}] This exception occurs when a float is
  divided by zero.  If trapping is enabled, the
  \tindexed{divide-by-zero} condition is signalled.  Otherwise, the
  appropriate infinity is returned.
\end{Lentry}

\subsubsection{Floating Point Rounding Mode}
\label{float-rounding-modes}

IEEE floating point specifies four possible rounding modes:
\begin{Lentry}
  
\item[\kwd{nearest}] In this mode, the inexact results are rounded to
  the nearer of the two possible result values.  If the neither
  possibility is nearer, then the even alternative is chosen.  This
  form of rounding is also called ``round to even'', and is the form
  of rounding specified for the \clisp{} \findexed{round} function.
  
\item[\kwd{positive-infinity}] This mode rounds inexact results to the
  possible value closer to positive infinity.  This is analogous to
  the \clisp{} \findexed{ceiling} function.
  
\item[\kwd{negative-infinity}] This mode rounds inexact results to the
  possible value closer to negative infinity.  This is analogous to
  the \clisp{} \findexed{floor} function.
  
\item[\kwd{zero}] This mode rounds inexact results to the possible
  value closer to zero.  This is analogous to the \clisp{}
  \findexed{truncate} function.
\end{Lentry}

\paragraph{Warning:}

Although the rounding mode can be changed with
\code{set-floating-point-modes}, use of any value other than the
default (\kwd{nearest}) can cause unusual behavior, since it will
affect rounding done by \llisp{} system code as well as rounding in
user code.  In particular, the unary \code{round} function will stop
doing round-to-nearest on floats, and instead do the selected form of
rounding.

%% \subsubsection{Precision Control}
%% \label{precision-control}
%% 
%% The floating-point unit for the Intel IA-32 architecture supports a
%% precision control mechanism.  The floating-point unit consists of an
%% IEEE extended double-float unit and all operations are always done
%% using his format, and this includes rounding.  However, by setting the
%% precision control mode, the user can control how rounding is done for
%% each basic arithmetic operation like addition, subtraction,
%% multiplication, and division.  The extra instructions for
%% trigonometric, exponential, and logarithmic operations are not
%% affected.  We refer the reader to Intel documentation for more
%% information. 
%% 
%% The possible modes are:
%% \begin{Lentry}
%%   
%% \item[\kwd{24-bit}] In this mode, all basic arithmetic operations like
%%   addition, subtraction, multiplication, and division, are rounded
%%   after each operation as if both the operands were IEEE single
%%   precision numbers.  
%%   
%% \item[\kwd{53-bit}] In this mode, rounding is performed as if the
%%   operands and results were IEEE double precision numbers.
%%   
%% \item[\kwd{64-bit}] In this mode, the default, rounding is performed
%%   on the full IEEE extended double precision format.
%%   
%% \end{Lentry}
%% 
%% \subsubsection{Warning:}
%% 
%% Although the precision mode can be changed with
%% \code{set-floating-point-modes}, use of anything other than
%% \kwd{64-bit} or \kwd{53-bit} can cause unexpected results, especially
%% if external functions or libraries are called.  A setting of
%% \kwd{64-bit} also causes \code{(= 1d0 (+ 1d0 double-float-epsilon))}
%% to return \true{} instead of \false.
%% 
%% 
\subsubsection{Accessing the Floating Point Modes}

These functions can be used to modify or read the floating point modes:

\begin{defun}{extensions:}{set-floating-point-modes}{%
    \keys{\kwd{traps} \kwd{rounding-mode}}
    \morekeys{\kwd{fast-mode} \kwd{accrued-exceptions}}
    \yetmorekeys{\kwd{current-exceptions}}}
  \defunx[extensions:]{get-floating-point-modes}{}
  
  The keyword arguments to \code{set-floating-point-modes} set various
  modes controlling how floating point arithmetic is done:
  \begin{Lentry}
  
  \item[\kwd{traps}] A list of the exception conditions that should
    cause traps.  Possible exceptions are \kwd{underflow},
    \kwd{overflow}, \kwd{inexact}, \kwd{invalid} and
    \kwd{divide-by-zero}.  Initially all traps except \kwd{inexact}
    are enabled.  \xlref{float-traps}.
    
  \item[\kwd{rounding-mode}] The rounding mode to use when the result
    is not exact. Possible values are \kwd{nearest},
    \kwd{positive-infinity}, \kwd{negative-infinity} and \kwd{zero}.
    Initially, the rounding mode is \kwd{nearest}. See the warning in
    section \ref{float-rounding-modes} about use of other rounding
    modes.
  
  \item[\kwd{current-exceptions}, \kwd{accrued-exceptions}] Lists of
    exception keywords used to set the exception flags.  The
    \var{current-exceptions} are the exceptions for the previous
    operation, so setting it is not very useful.  The
    \var{accrued-exceptions} are a cumulative record of the exceptions
    that occurred since the last time these flags were cleared.
    Specifying \code{()} will clear any accrued exceptions.
  
  \item[\kwd{fast-mode}] Set the hardware's ``fast mode'' flag, if
    any.  When set, IEEE conformance or debuggability may be impaired.
    Some machines may not have this feature, in which case the value
    is always \false.  Sparc platforms support a fast mode where
    denormal numbers are silently truncated to zero.
  \end{Lentry}
  If a keyword argument is not supplied, then the associated state is
  not changed.
  
  \code{get-floating-point-modes} returns a list representing the
  state of the floating point modes.  The list is in the same format
  as the keyword arguments to \code{set-floating-point-modes}, so
  \code{apply} could be used with \code{set-floating-point-modes} to
  restore the modes in effect at the time of the call to
  \code{get-floating-point-modes}.
\end{defun}

To make handling control of floating-point exceptions, the following
macro is useful.

\begin{defmac}{ext:}{with-float-traps-masked}{traps \ampbody\ body}
  \code{body} is executed with the selected floating-point exceptions
  given by \code{traps} masked out (disabled).  \code{traps} should be
  a list of possible floating-point exceptions that should be ignored.
  Possible values are \kwd{underflow}, \kwd{overflow}, \kwd{inexact},
  \kwd{invalid} and \kwd{divide-by-zero}.
  
  This is equivalent to saving the current traps from
  \code{get-floating-point-modes}, setting the floating-point modes to
  the desired exceptions, running the \code{body}, and restoring the
  saved floating-point modes.  The advantage of this macro is that it
  causes less consing to occur.

  Some points about the with-float-traps-masked:

  \begin{itemize}
  \item Two approaches are available for detecting FP exceptions:
    \begin{enumerate}
    \item enabling the traps and handling the exceptions
    \item disabling the traps and either handling the return values or
      checking the accrued exceptions.
    \end{enumerate}
    Of these the latter is the most portable because on the alpha port
    it is not possible to enable some traps at run-time.
    
  \item To assist the checking of the exceptions within the body any
    accrued exceptions matching the given traps are cleared at the
    start of the body when the traps are masked.
    
  \item To allow the macros to be nested these accrued exceptions are
    restored at the end of the body to their values at the start of
    the body. Thus any exceptions that occurred within the body will
    not affect the accrued exceptions outside the macro.
    
  \item Note that only the given exceptions are restored at the end of
    the body so other exception will be visible in the accrued
    exceptions outside the body.
    
  \item On the x86, setting the accrued exceptions of an unmasked
    exception would cause a FP trap. The macro behaviour of restoring
    the accrued exceptions ensures than if an accrued exception is
    initially not flagged and occurs within the body it will be
    restored/cleared at the exit of the body and thus not cause a
    trap.
    
  \item On the x86, and, perhaps, the hppa, the FP exceptions may be
    delivered at the next FP instruction which requires a FP
    \code{wait} instruction (\code{x86::float-wait}) if using the lisp
    conditions to catch trap within a \code{handler-bind}.  The
    \code{handler-bind} macro does the right thing and inserts a
    float-wait (at the end of its body on the x86).  The masking and
    noting of exceptions is also safe here.
    
  \item The setting of the FP flags uses the
    \code{(floating-point-modes)} and the \code{(set
      (floating-point-modes)\ldots)} VOPs. These VOPs blindly update
    the flags which may include other state.  We assume this state
    hasn't changed in between getting and setting the state. For
    example, if you used the FP unit between the above calls, the
    state may be incorrectly restored! The
    \code{with-float-traps-masked} macro keeps the intervening code to
    a minimum and uses only integer operations.
    %% Safe byte-compiled?
    %% Perhaps the VOPs (x86) should be smarter and only update some of
    %% the flags, the trap masks and exceptions?
  \end{itemize}

\end{defmac}

\subsection{Extended Floats}
\label{extended-float}

\cmucl{} also has an extension to support \code{double-double-float}
type.  This float format provides extended precision of about 31
decimal digits, with the same exponent range as \code{double-float}.
It is completely integrated into \cmucl{}, and can be used just like
any other floating-point object, including arrays, complex
\code{double-double-float}'s, and special functions.  With appropriate
declarations, no boxing is needed, just like \code{single-float} and
\code{double-float}. 

The exponent marker for a double-double float number is ``W'', so
``1.234w0'' is a double-double float number.


Note that there are a few shortcomings with
\code{double-double-float}'s:
\begin{itemize}
 \item There are no equivalents to \code{most-positive-double-float},
   \code{double-float-positive-infinity}, \textit{etc}.  This is because
   these are not really well defined for \code{double-double-float}'s.
 \item Underflow and overflow may be prematurely signaled.  This is
   due to how \code{double-double-float}'s are implemented.
 \item Basic arithmetic operations are inlined, so the code size is
   fairly large.
 \item \code{double-double-float} arithmetic is quite a bit slower
   than \code{double-float} since there is no hardware support for
   this type.
 \item The constant \code{pi} is still a \code{double-float} instead
   of a \code{double-double-float}.  Use \code{ext:dd-pi} if you
   want a \code{double-double-float} value for $\pi$.
\end{itemize}

\begin{deftp}{float}{extensions:double-double-float}{}
  The \code{double-double-float} type.  It is in the \code{EXTENSIONS}
  package.
\end{deftp}

\begin{defconst}{extensions:}{dd-pi}
  A \code{double-double-float} approximation to $\pi$.
\end{defconst}

\subsection{Characters}

\cmucl{} implements characters according to \cltltwo{}. The
main difference from the first version is that character bits and font
have been eliminated, and the names of the types have been changed.
\tindexed{base-character} is the new equivalent of the old
\tindexed{string-char}. In this implementation, all characters are
base characters (there are no extended characters.) Character codes
range between \code{0} and \code{255}, using the ASCII encoding.
Table~\ref{tbl:chars}~\vpageref{tbl:chars} shows characters recognized
by \cmucl.

\begin{table}[tbhp]
  \begin{center}
    \begin{tabular}{|c|c|l|l|l|l|}
      \hline
      \multicolumn{2}{|c|}{ASCII} & \multicolumn{1}{|c}{Lisp} &
      \multicolumn{3}{|c|}{} \\
      \cline{1-2}
      Name & Code & \multicolumn{1}{|c|}{Name} & \multicolumn{3}{|c|}{\raisebox{1.5ex}{Alternatives}}\\
      \hline
      \hline
      \code{nul} & 0 & \code{\#\back{NULL}} & \code{\#\back{NUL}} & &\\
      \code{bel} & 7 & \code{\#\back{BELL}} & & &\\
      \code{bs} &  8 & \code{\#\back{BACKSPACE}} & \code{\#\back{BS}} & &\\
      \code{tab} & 9 & \code{\#\back{TAB}} & & &\\
      \code{lf} & 10 & \code{\#\back{NEWLINE}} & \code{\#\back{NL}} & \code{\#\back{LINEFEED}} & \code{\#\back{LF}}\\
      \code{ff} & 11 & \code{\#\back{VT}} & \code{\#\back{PAGE}} & \code{\#\back{FORM}} &\\
      \code{cr} & 13 & \code{\#\back{RETURN}} & \code{\#\back{CR}} & &\\
      \code{esc} & 27 & \code{\#\back{ESCAPE}} & \code{\#\back{ESC}} & \code{\#\back{ALTMODE}} & \code{\#\back{ALT}}\\
      \code{sp} & 32 & \code{\#\back{SPACE}} & \code{\#\back{SP}} & &\\
      \code{del} & 127 & \code{\#\back{DELETE}} & \code{\#\back{RUBOUT}} & &\\
      \hline
    \end{tabular}
    \caption{Characters recognized by \cmucl}
    \label{tbl:chars}
  \end{center}
\end{table}


\subsection{Array Initialization}

If no \kwd{initial-value} is specified, arrays are initialized to zero.


\subsection{Hash tables}

The \tindexed{hash-tables} defined by \clisp{} have limited utility because they
are limited to testing their keys using the equality predicates
provided by (pre-CLOS) \clisp{}.  \cmucl{} overcomes this limitation
by allowing its users to specify new hash table tests and hashing
methods.  The hashing method must also be specified, since the
compiler is unable to determine a good hashing function for an
arbitrary equality (equivalence) predicate.

\begin{defun}{extensions:}{define-hash-table-test}%
  {\args{\var{hash-table-test-name} \var{test-function} \var{hash-function}}}
      
      The \var{hash-table-test-name} must be a symbol.
      % I just assumed the above. [2002/10/10:rpg]
      The \var{test-function} takes two objects and returns true
      iff they are the same.  The \var{hash-function} takes one object and
      returns two values: the (positive fixnum) hash value and true if
      the hashing depends on pointer values and will have to be redone
      if the object moves.
      
      To create a hash-table using this new ``test'' (really, a
      test/hash-function pair), use
      \code{(\index[funs]{make-hash-table}make-hash-table :test
        \var{hash-table-test-name} \ldots)}.

      Note that it is the \var{hash-table-test-name} that will be
      returned by the function \findexed{hash-table-test}, when applied to
      a hash-table created using this function.

      This function updates \vindexed{hash-table-tests}, which is now
      internal.  
\end{defun}

\cmucl{} also supports a number of weak hash tables.  These weak
tables are created using the \kwd{weak-p} argument to
\code{make-hash-table}.  Normally, a reference to an object as either
the key or value of the hash-table will prevent that object from being
garbage-collected.  However, in a weak table, if the only reference is
the hash-table, the object can be collected.

The possible values for \kwd{weak-p} are listed below.  An entry in
the table remains if the condition holds
\begin{Lentry}
\item[\kwd{key}] The key is referenced elsewhere
\item[\kwd{value}] The value is referenced elsewhere
\item[\kwd{key-and-value}] Both the key and value are referenced elsewhere
\item[\kwd{key-or-value}] Either the key or value are referenced elsewhere
\item[T] For backward compatibility, this means the same as \kwd{key}.
\end{Lentry}
If the condition does not hold, the object can be removed from the
hash table.  

Weak hash tables can only be created if the test is \code{eq} or
\code{eql}.  An error is signaled if this is not the case.

\begin{defun}{}{make-hash-table}%
  {\args{\keys{\kwd{test} \kwd{size} \kwd{rehash-size} \kwd{rehash-threshold} \kwd{weak-p}}}}
  Creates a hash-table with the specified properties.
\end{defun}
\section{Default Interrupts for Lisp}

\cmucl{} has several interrupt handlers defined when it starts up,
as follows:
\begin{Lentry}
  
\item[\code{SIGINT} (\ctrl{c})] causes Lisp to enter a break loop.
  This puts you into the debugger which allows you to look at the
  current state of the computation.  If you proceed from the break
  loop, the computation will proceed from where it was interrupted.
  
\item[\code{SIGQUIT} (\ctrl{L})] causes Lisp to do a throw to the
  top-level.  This causes the current computation to be aborted, and
  control returned to the top-level read-eval-print loop.
  
\item[\code{SIGTSTP} (\ctrl{z})] causes Lisp to suspend execution and
  return to the Unix shell.  If control is returned to Lisp, the
  computation will proceed from where it was interrupted.
  
\item[\code{SIGILL}, \code{SIGBUS}, \code{SIGSEGV}, and \code{SIGFPE}]
  cause Lisp to signal an error.
\end{Lentry}
For keyboard interrupt signals, the standard interrupt character is in
parentheses.  Your \file{.login} may set up different interrupt
characters.  When a signal is generated, there may be some delay before
it is processed since Lisp cannot be interrupted safely in an arbitrary
place.  The computation will continue until a safe point is reached and
then the interrupt will be processed.  \xlref{signal-handlers} to define
your own signal handlers.


\section{Implementation-Specific Packages}

When \cmucl{} is first started up, the default package is the
\code{common-lisp-user} package. The \code{common-lisp-user} package
uses the \code{common-lisp} and \code{extensions} packages. The
symbols exported from these three packages can be referenced without
package qualifiers. This section describes packages which have
exported interfaces that may concern users. The numerous internal
packages which implement parts of the system are not described here.
Package nicknames are in parenthesis after the full name.

\begin{Lentry}
\item[\code{alien}, \code{c-call}] Export the features of the Alien
  foreign data structure facility (\pxlref{aliens}.)
  
\item[\code{pcl}] This package contains PCL (Portable CommonLoops),
  which is a portable implementation of CLOS (the Common Lisp Object
  System.)  This implements most (but not all) of the features in the
  CLOS chapter of \cltltwo{}.

\item[\code{clos-mop (mop)}] This package contains an implementation
  of the CLOS Metaobject Protocol, as per the book \textit{The Art of
  the Metaobject Protocol}.
  
\item[\code{debug}] The \code{debug} package contains the command-line
  oriented debugger.  It exports utility various functions and
  switches.
  
\item[\code{debug-internals}] The \code{debug-internals} package
  exports the primitives used to write debuggers.
  \xlref{debug-internals}.
  
\item[\code{extensions (ext)}] The \code{extensions} packages exports
  local extensions to \clisp{} that are documented in this manual.
  Examples include the \code{save-lisp} function and time parsing.
  
\item[\code{hemlock (ed)}] The \code{hemlock} package contains all the
  code to implement Hemlock commands.  The \code{hemlock} package
  currently exports no symbols.
  
\item[\code{hemlock-internals (hi)}] The \code{hemlock-internals}
  package contains code that implements low level primitives and
  exports those symbols used to write Hemlock commands.
  
\item[\code{keyword}] The \code{keyword} package contains keywords
  (e.g., \kwd{start}).  All symbols in the \code{keyword} package are
  exported and evaluate to themselves (i.e., the value of the symbol
  is the symbol itself).
  
\item[\code{profile}] The \code{profile} package exports a simple
  run-time profiling facility (\pxlref{profiling}).
  
\item[\code{common-lisp (cl)}] The \code{common-lisp} package
  exports all the symbols defined by \cltl{} and only those symbols.
  Strictly portable Lisp code will depend only on the symbols exported
  from the \code{common-lisp} package.
  
\item[\code{unix}] This package exports system call
  interfaces to Unix (\pxlref{unix-interface}).
  
\item[\code{system (sys)}] The \code{system} package contains
  functions and information necessary for system interfacing.  This
  package is used by the \code{lisp} package and exports several
  symbols that are necessary to interface to system code.
  
\item[\code{xlib}] The \code{xlib} package contains the Common Lisp X
  interface (CLX) to the X11 protocol.  This is mostly Lisp code with
  a couple of functions that are defined in C to connect to the
  server.
  
\item[\code{wire}] The \code{wire} package exports a remote procedure
  call facility (\pxlref{remote}).

\item[\code{stream}] The \code{stream} package exports the public
  interface to the simple-streams implementation (\pxlref{simple-streams}).

\item[\code{xref}] The \code{xref} package exports the public
  interface to the cross-referencing utility (\pxlref{xref}).

\end{Lentry}


\section{Hierarchical Packages}
\cindex{hierarchical packages}


% this section is heavily based on the Franz Inc. documentation for
% the hierarchical packages feature, as per
% <URL:http://www.franz.com/support/tech_corner/hierpackuser.lhtml>
% accessed on 2002-03-18. It is used by permission from Kevin Layer,
% obtained in email to Eric Marsden, in response to spr25795. 
%
% Allegro-specific references in the document have been removed.


\subsection{Introduction}

The \clisp{} package system, designed and standardized several years
ago, is not hierarchical. Since \clisp{} was standardized, other
languages, including Java and Perl, have evolved namespaces which are
hierarchical. This document describes a hierarchical package naming
scheme for \clisp{}. The scheme was proposed by Franz Inc and
implemented in their \textit{Allegro Common Lisp} product; a
compatible implementation of the naming scheme is implemented in
\cmucl{}. This documentation is based on the Franz Inc. documentation,
and is included with permission.

The goals of hierarchical packages in \clisp{} are:

\begin{itemize}
\item
Reduce collisions with user-defined packages: it is a well-known
problem that package names used by the Lisp implementation and those
defined by users can easily conflict. The intent of hierarchical
packages is to reduce such conflicts to a minimum.

\item
Improve modularity: the current organization of packages in various
implementations has grown over the years and appears somewhat random.
Organizing future packages into a hierarchy will help make the
intention of the implementation more clear.

\item 
Foster growth in \clisp{} programs, or modules, available to the CL
community: the Perl and Java communities are able to contribute code
to repositories, with minimal fear of collision, because of the
hierarchical nature of the name spaces used by the contributed code.
We want the Lisp community to benefit from shared modules in the same
way.
\end{itemize}

In a nutshell, a dot (\verb|.|) is used to separate levels in package
names, and a leading dot signifies a relative package name. The choice
of dot follows Java. Perl, another language with hierarchical
packages, uses a colon (\verb|:|) as a delimiter, but the colon is
already reserved in \clisp{}. Absolute package names require no
modifications to the underlying \clisp{} implementation. Relative
package names require only small and simple modifications.


\subsection{Relative package names}

Relative package names are needed for the same reason as relative
pathnames, for brevity and to reduce the brittleness of absolute
names. A relative package name is one that begins with one or more
dots. A single dot means the current package, two dots mean the parent
of the current package, and so on.

Table~\ref{tbl:hierarchical-packages} presents a number of examples,
assuming that packages named \verb|foo|, \verb|foo.bar|,
\verb|mypack|, \verb|mypack.foo|, \verb|mypack.foo.bar|,
\verb|mypack.foo.baz|, \verb|mypack.bar|, and \verb|mypack.bar.baz|,
have all been created.

\begin{table}[h]
\begin{center}
\begin{tabular}{|l|l|l|}
\hline
relative name   &  current package & absolute name of referenced package \\
\hline
foo &               any  &                      foo \\
foo.bar &           any &                       foo.bar \\
.foo &              mypack &                    mypack.foo \\
 .foo.bar &         mypack &                    mypack.foo.bar \\
 ..foo &            mypack.bar &                mypack.foo \\
 ..foo.baz &        mypack.bar &                mypack.foo.baz \\
 ...foo &           mypack.bar.baz &            mypack.foo \\
 . &                mypack.bar.baz &            mypack.bar.baz \\
 .. &               mypack.bar.baz &            mypack.bar \\
 ... &              mypack.bar.baz &            mypack \\
\hline
\end{tabular}
\end{center}
\caption{Examples of hierarchical packages}
\label{tbl:hierarchical-packages}
\end{table}

Additional notes:

\begin{enumerate}
\item
All packages in the hierarchy must exist.

\item
\textbf{Warning about nicknames}: Unless you provide nicknames for
your hierarchical packages (and we recommend against doing so because
the number gets quite large), you can only use the names supplied. You
cannot mix in nicknames or alternate names. \code{cl-user}
is nickname of the \code{common-lisp-user} package.
Consider the following:

\begin{verbatim}
   (defpackage :cl-user.foo)
\end{verbatim}
  
When the current package (the value of the variable \code{*package*})
is \code{common-lisp-user}, you might expect \verb|.foo| to refer to
\verb|cl-user.foo|, but it does not. It refers to the non-existent
package \verb|common-lisp-user.foo|. Note that the purpose of
nicknames is to provide shorter names in place of the longer names
that are designed to be fully descriptive. The hope is that
hierarchical packages makes longer names unnecessary and thus makes
nicknames unnecessary.

\item
Multiple dots can only appear at the beginning of a package name. For
example, \verb|foo.bar..baz| does not mean \verb|foo.baz| -- it is
invalid. (Of course, it is perfectly legal to name a package
\verb|foo.bar..baz|, but \code{cl:find-package} will not process such
a name to find \verb|foo.baz| in the package hierarchy.)
\end{enumerate}


\subsection{Compatibility with ANSI \clisp{}}

The implementation of hierarchical packages modifies the
\code{cl:find-package} function, and provides certain auxiliary
functions, \code{package-parent}, \code{package-children}, and
\code{relative-package-name-to-package}, as described in this section.
The function \code{defpackage} itself requires no modification.

While the changes to \code{cl:find-package} are small and described
below, it is an important consideration for authors who would like
their programs to run on a variety of implementations that using
hierarchical packages will work in an implementation without the
modifications discussed in this document. We show why after
describing the changes to \code{cl:find-package}.

Absolute hierarchical package names require no changes in the
underlying \clisp{} implementation.


\subsubsection{Changes to \code{cl:find-package}}

Using relative hierarchical package names requires a simple
modification of \code{cl:find-package}.

In ANSI \clisp{}, \code{cl:find-package}, if passed a package object,
returns it; if passed a string, \code{cl:find-package} looks for a
package with that string as its name or nickname, and returns the
package if it finds one, or returns nil if it does not; if passed a
symbol, the symbol name (a string) is extracted and
\code{cl:find-package} proceeds as it does with a string.

For implementing hierarchical packages, the behavior when the argument
is a package object (return it) does not change. But when the argument
is a string starting with one or more dots not directly naming a
package, \code{cl:find-package} will, instead of returning nil, check
whether the string can be resolved as naming a relative package, and
if so, return the associated absolute package object. (If the argument
is a symbol, the symbol name is extracted and \code{cl:find-package}
proceeds as it does with a string argument.)

Note that you should not use leading dots in package names when using
hierarchical packages.

\subsubsection{Using hierarchical packages without modifying cl:find-package}

Even without the modifications to \code{cl:find-package}, authors need
not avoid using relative package names, but the ability to reuse
relative package names is restricted. Consider for example a module
\textit{foo} which is composed of the \verb|my.foo.bar| and
\verb|my.foo.baz| packages. In the code for each of the these packages
there are relative package references, \verb|..bar| and \verb|..baz|.

Implementations that have the new \code{cl:find-package} would carry
the keyword \verb|:relative-package-names| on their \code{*features*}
list (this is the case of \cmucl{} releases starting from 18d). Then,
in the \textit{foo} module, there would be definitions of the
\verb|my.foo.bar| and \verb|my.foo.baz| packages like so:

\begin{verbatim}
   (defpackage :my.foo.bar
     #-relative-package-names (:nicknames #:..bar)
     ...)

   (defpackage :my.foo.baz
     #-relative-package-names (:nicknames #:..baz)
     ...)
\end{verbatim}

Then, in a \verb|#-relative-package-names| implementation, the symbol
\verb|my.foo.bar:blam| would be visible from \verb|my.foo.baz| as
\verb|..bar:blam|, just as it would from a
\verb|#+relative-package-names| implementation.

So, even without the implementation of the augmented
\code{cl:find-package}, one can still write \clisp{} code that will
work in both types of implementations, but \verb|..bar| and
\verb|..baz| are now used, so you cannot also have
\verb|otherpack.foo.bar| and \verb|otherpack.foo.baz| and use
\verb|..bar| and \verb|..baz| as relative names. (The point of
hierarchical packages, of course, is to allow reusing relative package
names.)



\section{Package Locks}
\cindex{package locks}

\cmucl{} provides two types of package locks, as an extension to the
ANSI Common Lisp standard. The package-lock protects a package from
changes in its structure (the set of exported symbols, its use list,
etc). The package-definition-lock protects the symbols in the package
from being redefined due to the execution of a \code{defun},
\code{defmacro}, \code{defstruct}, \code{deftype} or \code{defclass}
form.


\subsection{Rationale}

Package locks are an aid to program development, by helping to detect
inadvertent name collisions and function redefinitions. They are
consistent with the principle that a package ``belongs to'' its
implementor, and that noone other than the package's developer should
be making or modifying definitions on symbols in that package. Package
locks are compatible with the ANSI Common Lisp standard, which states
that the consequences of redefining functions in the
\code{COMMON-LISP} package are undefined.

Violation of a package lock leads to a continuable error of type
\code{lisp::package-locked-error} being signaled. The user may choose
to ignore the lock and proceed, or to abort the computation. Two other
restarts are available, one which disables all locks on all packages,
and one to disable only the package-lock or package-definition-lock
that was tripped.

The following transcript illustrates the behaviour seen when
attempting to redefine a standard macro in the \code{COMMON-LISP}
package, or to redefine a function in one of \cmucl{}'s
implementation-defined packages:

{\small
\begin{verbatim}
CL-USER> (defmacro 1+ (x) (* x 2))
Attempt to modify the locked package COMMON-LISP, by defining macro 1+
   [Condition of type LISP::PACKAGE-LOCKED-ERROR]

Restarts:
  0: [continue      ] Ignore the lock and continue
  1: [unlock-package] Disable the package's definition-lock then continue
  2: [unlock-all    ] Unlock all packages, then continue
  3: [abort         ] Return to Top-Level.

CL-USER> (defun ext:gc () t)
Attempt to modify the locked package EXTENSIONS, by redefining function GC
   [Condition of type LISP::PACKAGE-LOCKED-ERROR]

Restarts:
  0: [continue      ] Ignore the lock and continue
  1: [unlock-package] Disable package's definition-lock, then continue
  2: [unlock-all    ] Disable all package locks, then continue
  3: [abort         ] Return to Top-Level.
\end{verbatim}


The following transcript illustrates the behaviour seen when an
attempt to modify the structure of a package is made:

\begin{verbatim}
CL-USER> (unexport 'load-foreign :ext)
Attempt to modify the locked package EXTENSIONS, by unexporting symbols LOAD-FOREIGN
   [Condition of type lisp::package-locked-error]

Restarts:
  0: [continue      ] Ignore the lock and continue
  1: [unlock-package] Disable package's lock then continue
  2: [unlock-all    ] Unlock all packages, then continue
  3: [abort         ] Return to Top-Level.
\end{verbatim}
}

The \code{COMMON-LISP} package and the \cmucl{}-specific
implementation packages are locked on startup. Users can lock their
own packages by using the \code{ext:package-lock} and
\code{ext:package-definition-lock} accessors.



\subsection{Disabling package locks}

A package's locks can be enabled or disabled by using the
\code{ext:package-lock} and \code{ext:package-definition-lock}
accessors, as follows:

\begin{lisp}
   (setf (ext:package-lock (find-package "UNIX")) nil)
   (setf (ext:package-definition-lock (find-package "UNIX")) nil)
\end{lisp}


\begin{defun}{ext:}{package-lock}{\var{package}}
  This function is an accessor for a package's structural lock, which
  protects it against modifications to its list of exported symbols.
\end{defun}


\begin{defun}{ext:}{package-definition-lock}{\var{package}}
  This function is an accessor for a package's definition-lock, which
  protects symbols in that package from redefinition. As well as
  protecting the symbol's fdefinition from change, attempts to change
  the symbol's definition using \code{defstruct}, \code{defclass} or
  \code{deftype} will be trapped.
\end{defun}


\begin{defmac}{ext:}{without-package-locks}{\args{\amprest{} \var{body}}}
  This macro can be used to execute forms with all package locks (both
  structure and definition locks) disabled. 
\end{defmac}


\begin{defun}{ext:}{unlock-all-packages}{}
  This function disables both structure and definition locks on all
  currently defined packages. Note that package locks are reset when
  \cmucl{} is restarted, so the effect of this function is limited to
  the current session. 
\end{defun}


% EOF




\section{The Editor}

The \code{ed} function invokes the Hemlock editor which is described
in {\it Hemlock User's Manual} and {\it Hemlock Command Implementor's
Manual}. Most users at CMU prefer to use Hemlock's slave \llisp{}
mechanism which provides an interactive buffer for the
\code{read-eval-print} loop and editor commands for evaluating and
compiling text from a buffer into the slave \llisp.  Since the editor
runs in the \llisp, using slaves keeps users from trashing their
editor by developing in the same \llisp{} with \hemlock{}.


\section{Garbage Collection}

\cmucl{} uses either a stop-and-copy garbage collector or a
generational, mostly copying garbage collector.  Which collector is
available depends on the platform and the features of the platform.
The stop-and-copy GC is available on all RISC platforms.  The x86
platform supports a conservative stop-and-copy collector, which is now
rarely used, and a generational conservative collector.  On the Sparc
platform, both the stop-and-copy GC and the generational GC are
available, but the stop-and-copy GC is deprecated in favor of the
generational GC.  

The generational GC is available if \var{*features*} contains
\code{:gencgc}.

%% The stop-and-copy GC compacts the items in dynamic space every time it
%% runs. Most users cause the system to garbage collect (GC) frequently,
%% long before space is exhausted. With 16 or 24 megabytes of memory,
%% causing GC's more frequently on less garbage allows the system to GC
%% without much (if any) paging.

The following functions invoke the garbage collector or control whether
automatic garbage collection is in effect:

\begin{defun}[-cheney]{extensions:}{gc}{\args{\ampoptional{} \var{verbose-p}}}
  
  This function runs the garbage collector.  If
  \code{ext:*gc-verbose*} is non-\nil, then it invokes
  \code{ext:*gc-notify-before*} before GC'ing and
  \code{ext:*gc-notify-after*} afterwards.
  
  \code{verbose-p} indicates whether GC statistics are printed or
  not. 

\end{defun}

\begin{defun}{extensions:}{gc-off}{}
  
  This function inhibits automatic garbage collection.  After calling
  it, the system will not GC unless you call \code{ext:gc} or
  \code{ext:gc-on}.
\end{defun}

\begin{defun}{extensions:}{gc-on}{}
  
  This function reinstates automatic garbage collection.  If the
  system would have GC'ed while automatic GC was inhibited, then this
  will call \code{ext:gc}.
\end{defun}

\subsection{GC Parameters}

The following variables control the behavior of the garbage collector:

\begin{defvar}{extensions:}{bytes-consed-between-gcs}
  
  \cmucl{} automatically GC's whenever the amount of memory
  allocated to dynamic objects exceeds the value of an internal
  variable.  After each GC, the system sets this internal variable to
  the amount of dynamic space in use at that point plus the value of
  the variable \code{ext:*bytes-consed-between-gcs*}.  The default
  value is 2000000.
\end{defvar}

\begin{defvar}{extensions:}{gc-verbose}
  
  This variable controls whether \code{ext:gc} invokes the functions
  in \code{ext:*gc-notify-before*} and
  \code{ext:*gc-notify-after*}.  If \code{*gc-verbose*} is \nil,
  \code{ext:gc} foregoes printing any messages.  The default value is
  \code{T}.
\end{defvar}

\begin{defvar}{extensions:}{gc-notify-before}
  
  This variable's value is a function that should notify the user that
  the system is about to GC.  It takes one argument, the amount of
  dynamic space in use before the GC measured in bytes.  The default
  value of this variable is a function that prints a message similar
  to the following:
\begin{verbatim}
   [GC threshold exceeded with 2,107,124 bytes in use.  Commencing GC.]
\end{verbatim}
\end{defvar}

\begin{defvar}{extensions:}{gc-notify-after}
  
  This variable's value is a function that should notify the user when
  a GC finishes.  The function must take three arguments, the amount
  of dynamic spaced retained by the GC, the amount of dynamic space
  freed, and the new threshold which is the minimum amount of space in
  use before the next GC will occur.  All values are byte quantities.
  The default value of this variable is a function that prints a
  message similar to the following:
  \begin{verbatim}
    [GC completed with 25,680 bytes retained and 2,096,808 bytes freed.]
    [GC will next occur when at least 2,025,680 bytes are in use.]
  \end{verbatim}
\end{defvar}

Note that a garbage collection will not happen at exactly the new
threshold printed by the default \code{ext:*gc-notify-after*}
function.  The system periodically checks whether this threshold has
been exceeded, and only then does a garbage collection.

\begin{defvar}{extensions:}{gc-inhibit-hook}
  
  This variable's value is either a function of one argument or \nil.
  When the system has triggered an automatic GC, if this variable is a
  function, then the system calls the function with the amount of
  dynamic space currently in use (measured in bytes).  If the function
  returns \nil, then the GC occurs; otherwise, the system inhibits
  automatic GC as if you had called \code{ext:gc-off}.  The writer of
  this hook is responsible for knowing when automatic GC has been
  turned off and for calling or providing a way to call
  \code{ext:gc-on}.  The default value of this variable is \nil.
\end{defvar}

\begin{defvar}{extensions:}{before-gc-hooks}
  \defvarx[extensions:]{after-gc-hooks}
  
  These variables' values are lists of functions to call before or
  after any GC occurs.  The system provides these purely for
  side-effect, and the functions take no arguments.
\end{defvar}

\subsection{Generational GC}
Generational GC also supports some additional functions and variables
to control it.

\begin{defun}[-gencgc]{extensions:}{gc}{\args{\keys{\kwd{verbose} \kwd{gen} \kwd{full}}}}
  
  This function runs the garbage collector.  If
  \code{ext:*gc-verbose*} is non-\nil, then it invokes
  \code{ext:*gc-notify-before*} before GC'ing and
  \code{ext:*gc-notify-after*} afterwards.

  \begin{Lentry}
  \item[\code{verbose}] Print GC statistics if non-\code{NIL}.
  \item[\code{gen}] The number of generations to be collected.
  \item[\code{full}] If non-\code{NIL}, a full collection of all
    generations is performed.
  \end{Lentry}
\end{defun}

\begin{defun}{lisp::}{gencgc-stats}{\args{\var{generation}}}
  Returns statistics about the generation, as multiple values:
  \begin{enumerate}
  \item Bytes allocated in this generation
  \item The GC trigger for this generation.  When this many bytes have
    been allocated, a GC is started automatically.
  \item The number of bytes consed between GCs.
  \item The number of GCs that have been done on this generation.
    This is reset to zero when the generation is raised.
  \item The trigger age, which is the maximum number of GCs to perform
    before this generation is raised.
  \item The total number of bytes allocated to this generation.
  \item Average age of the objects in this generations.  The average
    age is the cumulative bytes allocated divided by current number of
    bytes allocated.
  \end{enumerate}
\end{defun}

\begin{defun}{lisp::}{set-gc-trigger}{\args{\var{gen} \var{trigger}}}
  Sets the GC trigger value for the specified generation.
\end{defun}

\begin{defun}{lisp::}{set-trigger-age}{\args{\var{gen} \var{trigger-age}}}
  Sets the GC trigger age for the specified generation.
\end{defun}

\begin{defun}{lisp::}{set-min-mem-age}{\args{\var{gen} \var{min-mem-age}}}
  Sets the minimum average memory age for the specified generation.
  If the computed memory age is below this, GC is not performed, which
  helps prevent a GC when a large number of new live objects have been
  added in which case a GC would usually be a waste of time.
\end{defun}

\subsection{Weak Pointers}

A weak pointer provides a way to maintain a reference to an object
without preventing an object from being garbage collected.  If the
garbage collector discovers that the only pointers to an object are
weak pointers, then it breaks the weak pointers and deallocates the
object.

\begin{defun}{extensions:}{make-weak-pointer}{\args{\var{object}}}
  \defunx[extensions:]{weak-pointer-value}{\args{\var{weak-pointer}}}
  
  \code{make-weak-pointer} returns a weak pointer to an object.
  \code{weak-pointer-value} follows a weak pointer, returning the two
  values: the object pointed to (or \false{} if broken) and a boolean
  value which is \false{} if the pointer has been broken, and true
  otherwise.
\end{defun}


\subsection{Finalization}

Finalization provides a ``hook'' that is triggered when the garbage
collector reclaims an object.  It is usually used to recover non-Lisp
resources that were allocated to implement the finalized Lisp object.
For example, when a unix file-descriptor stream is collected,
finalization is used to close the underlying file descriptor.

\begin{defun}{extensions:}{finalize}{\args{\var{object} \var{function}}}
  
  This function registers \var{object} for finalization.
  \var{function} is called with no arguments when \var{object} is
  reclaimed.  Normally \var{function} will be a closure over the
  underlying state that needs to be freed, e.g. the unix file
  descriptor in the fd-stream case.  Note that \var{function} must not
  close over \var{object} itself, as this prevents the object from
  ever becoming garbage.
\end{defun}

\begin{defun}{extensions:}{cancel-finalization}{\args{\var{object}}}
  
  This function cancel any finalization request for \var{object}.
\end{defun}


\section{Describe}

\begin{defun}{}{describe}{ \args{\var{object} \&optional{} \var{stream}}}
  
  The \code{describe} function prints useful information about
  \var{object} on \var{stream}, which defaults to
  \code{*standard-output*}.  For any object, \code{describe} will
  print out the type.  Then it prints other information based on the
  type of \var{object}.  The types which are presently handled are:

  \begin{Lentry}
  
  \item[\tindexed{hash-table}] \code{describe} prints the number of
    entries currently in the hash table and the number of buckets
    currently allocated.
  
  \item[\tindexed{function}] \code{describe} prints a list of the
    function's name (if any) and its formal parameters.  If the name
    has function documentation, then it will be printed.  If the
    function is compiled, then the file where it is defined will be
    printed as well.
  
  \item[\tindexed{fixnum}] \code{describe} prints whether the integer
    is prime or not.
  
  \item[\tindexed{symbol}] The symbol's value, properties, and
    documentation are printed.  If the symbol has a function
    definition, then the function is described.
  \end{Lentry}
  If there is anything interesting to be said about some component of
  the object, describe will invoke itself recursively to describe that
  object.  The level of recursion is indicated by indenting output.
\end{defun}

A number of switches can be used to control \code{describe}'s behavior.

\begin{defvar}{extensions:}{describe-level}

  The maximum level of recursive description allowed.  Initially two.
\end{defvar}

\begin{defvar}{extensions:}{describe-indentation}

The number of spaces to indent for each level of recursive
description, initially three.
\end{defvar}

\begin{defvar}{extensions:}{describe-print-level}
  \defvarx[extensions:]{describe-print-length}
  
  The values of \code{*print-level*} and \code{*print-length*} during
  description.  Initially two and five.
\end{defvar}


\section{The Inspector}

\cmucl{} has both a graphical inspector that uses the X Window System,
and a simple terminal-based inspector.

\begin{defun}{}{inspect}{ \args{\ampoptional{} \var{object}}}
  
  \code{inspect} calls the inspector on the optional argument
  \var{object}.  If \var{object} is unsupplied, \code{inspect}
  immediately returns \false.  Otherwise, the behavior of inspect
  depends on whether Lisp is running under X.  When \code{inspect} is
  eventually exited, it returns some selected Lisp object.
\end{defun}


\subsection{The Graphical Interface}
\label{motif-interface}

\cmucl{} has an interface to Motif which is functionally similar to
CLM, but works better in \cmucl{}.  This interface is documented in
separate manuals \textit{CMUCL Motif Toolkit} and \textit{Design Notes
on the Motif Toolkit}, which are distributed with \cmucl{}.

This motif interface has been used to write the inspector and graphical
debugger.  There is also a Lisp control panel with a simple file management
facility, apropos and inspector dialogs, and controls for setting global
options.  See the \code{interface} and \code{toolkit} packages.

\begin{defun}{interface:}{lisp-control-panel}{}
  
  This function creates a control panel for the Lisp process.
\end{defun}

\begin{defvar}{interface:}{interface-style}
  
  When the graphical interface is loaded, this variable controls
  whether it is used by \code{inspect} and the error system.  If the
  value is \kwd{graphics} (the default) and the \code{DISPLAY}
  environment variable is defined, the graphical inspector and
  debugger will be invoked by \findexed{inspect} or when an error is
  signalled.  Possible values are \kwd{graphics} and {tty}.  If the
  value is \kwd{graphics}, but there is no X display, then we quietly
  use the TTY interface.
\end{defvar}


\subsection{The TTY Inspector}

If X is unavailable, a terminal inspector is invoked.  The TTY inspector
is a crude interface to \code{describe} which allows objects to be
traversed and maintains a history.  This inspector prints information
about and object and a numbered list of the components of the object.
The command-line based interface is a normal
\code{read}--\code{eval}--\code{print} loop, but an integer \var{n}
descends into the \var{n}'th component of the current object, and
symbols with these special names are interpreted as commands:

\begin{Lentry}
\item[U] Move back to the enclosing object.  As you descend into the
components of an object, a stack of all the objects previously seen is
kept.  This command pops you up one level of this stack.

\item[Q, E] Return the current object from \code{inspect}.

\item[R] Recompute object display, and print again.  Useful if the
object may have changed.

\item[D] Display again without recomputing.

\item[H, ?] Show help message.
\end{Lentry}


\section{Load}

\begin{defun}{}{load}{%
    \args{\var{filename}
      \keys{\kwd{verbose} \kwd{print} \kwd{if-does-not-exist}}
      \morekeys{\kwd{if-source-newer} \kwd{contents}}}}
  
  As in standard \clisp{}, this function loads a file containing
  source or object code into the running Lisp.  Several CMU extensions
  have been made to \code{load} to conveniently support a variety of
  program file organizations.  \var{filename} may be a wildcard
  pathname such as \file{*.lisp}, in which case all matching files are
  loaded.
  
  If \var{filename} has a \code{pathname-type} (or extension), then
  that exact file is loaded.  If the file has no extension, then this
  tells \code{load} to use a heuristic to load the ``right'' file.
  The \code{*load-source-types*} and \code{*load-object-types*}
  variables below are used to determine the default source and object
  file types.  If only the source or the object file exists (but not
  both), then that file is quietly loaded.  Similarly, if both the
  source and object file exist, and the object file is newer than the
  source file, then the object file is loaded.  The value of the
  \var{if-source-newer} argument is used to determine what action to
  take when both the source and object files exist, but the object
  file is out of date:
  \begin{Lentry}
  \item[\kwd{load-object}] The object file is loaded even though the
    source file is newer.
    
  \item[\kwd{load-source}] The source file is loaded instead of the
    older object file.
    
  \item[\kwd{compile}] The source file is compiled and then the new
    object file is loaded.
    
  \item[\kwd{query}] The user is asked a yes or no question to
    determine whether the source or object file is loaded.
  \end{Lentry}
  This argument defaults to the value of
  \code{ext:*load-if-source-newer*} (initially \kwd{load-object}.)
  
  The \var{contents} argument can be used to override the heuristic
  (based on the file extension) that normally determines whether to
  load the file as a source file or an object file.  If non-null, this
  argument must be either \kwd{source} or \kwd{binary}, which forces
  loading in source and binary mode, respectively. You really
  shouldn't ever need to use this argument.
\end{defun}

\begin{defvar}{extensions:}{load-source-types}
  \defvarx[extensions:]{load-object-types}
  
  These variables are lists of possible \code{pathname-type} values
  for source and object files to be passed to \code{load}.  These
  variables are only used when the file passed to \code{load} has no
  type; in this case, the possible source and object types are used to
  default the type in order to determine the names of the source and
  object files.
\end{defvar}

\begin{defvar}{extensions:}{load-if-source-newer}
  
  This variable determines the default value of the
  \var{if-source-newer} argument to \code{load}.  Its initial value is
  \kwd{load-object}.
\end{defvar}


\section{The Reader}

\subsection{Reader Extensions}
\cmucl{} supports an ANSI-compatible extension to enable reading of
specialized arrays.  Thus
\begin{example}
  * (setf *print-readably* nil)
  NIL
  * (make-array '(2 2) :element-type '(signed-byte 8))
  #2A((0 0) (0 0))
  * (setf *print-readably* t)
  T
  * (make-array '(2 2) :element-type '(signed-byte 8))
  #A((SIGNED-BYTE 8) (2 2) ((0 0) (0 0)))
  * (type-of (read-from-string "#A((SIGNED-BYTE 8) (2 2) ((0 0) (0 0)))"))
  (SIMPLE-ARRAY (SIGNED-BYTE 8) (2 2))
  * (setf *print-readably* nil)
  NIL
  * (type-of (read-from-string "#A((SIGNED-BYTE 8) (2 2) ((0 0) (0 0)))"))
  (SIMPLE-ARRAY (SIGNED-BYTE 8) (2 2))
\end{example}

\subsection{Reader Parameters}
\begin{defvar}{extensions:}{ignore-extra-close-parentheses}
  
  If this variable is \true{} (the default), then the reader merely
  prints a warning when an extra close parenthesis is detected
  (instead of signalling an error.)
\end{defvar}

\section{Stream Extensions}
\begin{defun}{sys:}{read-n-bytes}{%
    \args{\var{stream buffer start numbytes} 
      \ampoptional{} \var{eof-error-p}}}
  
  On streams that support it, this function reads multiple bytes of
  data into a buffer.  The buffer must be a \code{simple-string} or
  \code{(simple-array (unsigned-byte 8) (*))}.  The argument
  \var{nbytes} specifies the desired number of bytes, and the return
  value is the number of bytes actually read.
  \begin{itemize}
  \item If \var{eof-error-p} is true, an \tindexed{end-of-file}
    condition is signalled if end-of-file is encountered before
    \var{count} bytes have been read.
    
  \item If \var{eof-error-p} is false, \code{read-n-bytes reads} as
    much data is currently available (up to count bytes.)  On pipes or
    similar devices, this function returns as soon as any data is
    available, even if the amount read is less than \var{count} and
    eof has not been hit.  See also \funref{make-fd-stream}.
  \end{itemize}
\end{defun}

\section{Simple Streams}
\cindex{simple-streams}
\label{simple-streams}

[TBD]


\section{Running Programs from Lisp}

It is possible to run programs from Lisp by using the following function.

\begin{defun}{extensions:}{run-program}{%
    \args{\var{program} \var{args}
      \keys{\kwd{env} \kwd{wait} \kwd{pty} \kwd{input}}
      \morekeys{\kwd{if-input-does-not-exist}}
      \yetmorekeys{\kwd{output} \kwd{if-output-exists}}
      \yetmorekeys{\kwd{error} \kwd{if-error-exists}}
      \yetmorekeys{\kwd{status-hook} \kwd{before-execve}}}}
     
  \code{run-program} runs \var{program} in a child process.
  \var{Program} should be a pathname or string naming the program.
  \var{Args} should be a list of strings which this passes to
  \var{program} as normal Unix parameters.  For no arguments, specify
  \var{args} as \nil.  The value returned is either a process
  structure or \nil.  The process interface follows the description of
  \code{run-program}.  If \code{run-program} fails to fork the child
  process, it returns \nil.
  
  Except for sharing file descriptors as explained in keyword argument
  descriptions, \code{run-program} closes all file descriptors in the
  child process before running the program.  When you are done using a
  process, call \code{process-close} to reclaim system resources.  You
  only need to do this when you supply \kwd{stream} for one of
  \kwd{input}, \kwd{output}, or \kwd{error}, or you supply \kwd{pty}
  non-\nil.  You can call \code{process-close} regardless of whether
  you must to reclaim resources without penalty if you feel safer.

  \code{run-program} accepts the following keyword arguments:

  \begin{Lentry}   
  \item[\kwd{env}] This is an a-list mapping keywords and
    simple-strings.  The default is \code{ext:*environment-list*}.  If
    \kwd{env} is specified, \code{run-program} uses the value given
    and does not combine the environment passed to Lisp with the one
    specified.
    
  \item[\kwd{wait}] If non-\nil{} (the default), wait until the child
    process terminates.  If \nil, continue running Lisp while the
    child process runs.
    
  \item[\kwd{pty}] This should be one of \true, \nil, or a stream.  If
    specified non-\nil, the subprocess executes under a Unix PTY.
    If specified as a stream, the system collects all output to this
    pty and writes it to this stream.  If specified as \true, the
    \code{process-pty} slot contains a stream from which you can read
    the program's output and to which you can write input for the
    program.  The default is \nil.
    
  \item[\kwd{input}] This specifies how the program gets its input.
    If specified as a string, it is the name of a file that contains
    input for the child process.  \code{run-program} opens the file as
    standard input.  If specified as \nil{} (the default), then
    standard input is the file \file{/dev/null}.  If specified as
    \true, the program uses the current standard input.  This may
    cause some confusion if \kwd{wait} is \nil{} since two processes
    may use the terminal at the same time.  If specified as
    \kwd{stream}, then the \code{process-input} slot contains an
    output stream.  Anything written to this stream goes to the
    program as input.  \kwd{input} may also be an input stream that
    already contains all the input for the process.  In this case
    \code{run-program} reads all the input from this stream before
    returning, so this cannot be used to interact with the process.
    
  \item[\kwd{if-input-does-not-exist}] This specifies what to do if
    the input file does not exist.  The following values are valid:
    \nil{} (the default) causes \code{run-program} to return \nil{}
    without doing anything; \kwd{create} creates the named file; and
    \kwd{error} signals an error.
    
  \item[\kwd{output}] This specifies what happens with the program's
    output.  If specified as a pathname, it is the name of a file that
    contains output the program writes to its standard output.  If
    specified as \nil{} (the default), all output goes to
    \file{/dev/null}.  If specified as \true, the program writes to
    the Lisp process's standard output.  This may cause confusion if
    \kwd{wait} is \nil{} since two processes may write to the terminal
    at the same time.  If specified as \kwd{stream}, then the
    \code{process-output} slot contains an input stream from which you
    can read the program's output.
    
  \item[\kwd{if-output-exists}] This specifies what to do if the
    output file already exists.  The following values are valid:
    \nil{} causes \code{run-program} to return \nil{} without doing
    anything; \kwd{error} (the default) signals an error;
    \kwd{supersede} overwrites the current file; and \kwd{append}
    appends all output to the file.
    
  \item[\kwd{error}] This is similar to \kwd{output}, except the file
    becomes the program's standard error.  Additionally, \kwd{error}
    can be \kwd{output} in which case the program's error output is
    routed to the same place specified for \kwd{output}.  If specified
    as \kwd{stream}, the \code{process-error} contains a stream
    similar to the \code{process-output} slot when specifying the
    \kwd{output} argument.
    
  \item[\kwd{if-error-exists}] This specifies what to do if the error
    output file already exists.  It accepts the same values as
    \kwd{if-output-exists}.
    
  \item[\kwd{status-hook}] This specifies a function to call whenever
    the process changes status.  This is especially useful when
    specifying \kwd{wait} as \nil.  The function takes the process as
    a required argument.
    
  \item[\kwd{before-execve}] This specifies a function to run in the
    child process before it becomes the program to run.  This is
    useful for actions such as authenticating the child process
    without modifying the parent Lisp process.
  \end{Lentry}
\end{defun}


\subsection{Process Accessors}

The following functions interface the process returned by \code{run-program}:

\begin{defun}{extensions:}{process-p}{\args{\var{thing}}}
  
  This function returns \true{} if \var{thing} is a process.
  Otherwise it returns \nil{}
\end{defun}

\begin{defun}{extensions:}{process-pid}{\args{\var{process}}}
  
  This function returns the process ID, an integer, for the
  \var{process}.
\end{defun}

\begin{defun}{extensions:}{process-status}{\args{\var{process}}}
  
  This function returns the current status of \var{process}, which is
  one of \kwd{running}, \kwd{stopped}, \kwd{exited}, or
  \kwd{signaled}.
\end{defun}

\begin{defun}{extensions:}{process-exit-code}{\args{\var{process}}}
  
  This function returns either the exit code for \var{process}, if it
  is \kwd{exited}, or the termination signal \var{process} if it is
  \kwd{signaled}.  The result is undefined for processes that are
  still alive.
\end{defun}

\begin{defun}{extensions:}{process-core-dumped}{\args{\var{process}}}
  
  This function returns \true{} if someone used a Unix signal to
  terminate the \var{process} and caused it to dump a Unix core image.
\end{defun}

\begin{defun}{extensions:}{process-pty}{\args{\var{process}}}
  
  This function returns either the two-way stream connected to
  \var{process}'s Unix PTY connection or \nil{} if there is none.
\end{defun}

\begin{defun}{extensions:}{process-input}{\args{\var{process}}}
  \defunx[extensions:]{process-output}{\args{\var{process}}}
  \defunx[extensions:]{process-error}{\args{\var{process}}}
  
  If the corresponding stream was created, these functions return the
  input, output or error fd-stream.  \nil{} is returned if there
  is no stream.
\end{defun}

\begin{defun}{extensions:}{process-status-hook}{\args{\var{process}}}
  
  This function returns the current function to call whenever
  \var{process}'s status changes.  This function takes the
  \var{process} as a required argument.  \code{process-status-hook} is
  \code{setf}'able.
\end{defun}

\begin{defun}{extensions:}{process-plist}{\args{\var{process}}}
  
  This function returns annotations supplied by users, and it is
  \code{setf}'able.  This is available solely for users to associate
  information with \var{process} without having to build a-lists or
  hash tables of process structures.
\end{defun}

\begin{defun}{extensions:}{process-wait}{
    \args{\var{process} \ampoptional{} \var{check-for-stopped}}}
  
  This function waits for \var{process} to finish.  If
  \var{check-for-stopped} is non-\nil, this also returns when
  \var{process} stops.
\end{defun}

\begin{defun}{extensions:}{process-kill}{%
    \args{\var{process} \var{signal} \ampoptional{} \var{whom}}}
  
  This function sends the Unix \var{signal} to \var{process}.
  \var{Signal} should be the number of the signal or a keyword with
  the Unix name (for example, \kwd{sigsegv}).  \var{Whom} should be
  one of the following:
  \begin{Lentry}
    
  \item[\kwd{pid}] This is the default, and it indicates sending the
    signal to \var{process} only.
    
  \item[\kwd{process-group}] This indicates sending the signal to
    \var{process}'s group.
    
  \item[\kwd{pty-process-group}] This indicates sending the signal to
    the process group currently in the foreground on the Unix PTY
    connected to \var{process}.  This last option is useful if the
    running program is a shell, and you wish to signal the program
    running under the shell, not the shell itself.  If
    \code{process-pty} of \var{process} is \nil, using this option is
    an error.
  \end{Lentry}
\end{defun}

\begin{defun}{extensions:}{process-alive-p}{\args{\var{process}}}
  
  This function returns \true{} if \var{process}'s status is either
  \kwd{running} or \kwd{stopped}.
\end{defun}

\begin{defun}{extensions:}{process-close}{\args{\var{process}}}
  
  This function closes all the streams associated with \var{process}.
  When you are done using a process, call this to reclaim system
  resources.
\end{defun}


\section{Saving a Core Image}

A mechanism has been provided to save a running Lisp core image and to
later restore it.  This is convenient if you don't want to load several files
into a Lisp when you first start it up.  The main problem is the large
size of each saved Lisp image, typically at least 20 megabytes.

\begin{defun}{extensions:}{save-lisp}{%
    \args{\var{file}
      \keys{\kwd{purify} \kwd{root-structures} \kwd{init-function}}
      \morekeys{\kwd{load-init-file} \kwd{print-herald} \kwd{site-init}}
      \yetmorekeys{\kwd{process-command-line} \kwd{batch-mode}}}}
  
  The \code{save-lisp} function saves the state of the currently
  running Lisp core image in \var{file}.  The keyword arguments have
  the following meaning:
  \begin{Lentry}
    
  \item[\kwd{purify}] If non-\nil{} (the default), the core image is
    purified before it is saved (see \funref{purify}.)  This reduces
    the amount of work the garbage collector must do when the
    resulting core image is being run.  Also, if more than one Lisp is
    running on the same machine, this maximizes the amount of memory
    that can be shared between the two processes.
    
  \item[\kwd{root-structures}]
      This should be a list of the main entry points in any newly
      loaded systems.  This need not be supplied, but locality and/or
      GC performance will be better if they are.  Meaningless if
      \kwd{purify} is \nil.  See \funref{purify}.

  \item[\kwd{init-function}] This is the function that starts running
    when the created core file is resumed.  The default function
    simply invokes the top level read-eval-print loop.  If the
    function returns the lisp will exit.
    
  \item[\kwd{load-init-file}] If non-NIL, then load an init file;
    either the one specified on the command line or
    ``\w{\file{init.}\var{fasl-type}}'', or, if
    ``\w{\file{init.}\var{fasl-type}}'' does not exist,
    \code{init.lisp} from the user's home directory.  If the init file
    is found, it is loaded into the resumed core file before the
    read-eval-print loop is entered.
    
  \item[\kwd{site-init}] If non-NIL, the name of the site init file to
    quietly load.  The default is \file{library:site-init}.  No error
    is signalled if the file does not exist.
    
  \item[\kwd{print-herald}] If non-NIL (the default), then print out
    the standard Lisp herald when starting.
    
  \item[\kwd{process-command-line}] If non-NIL (the default),
    processes the command line switches and performs the appropriate
    actions.

  \item[\kwd{batch-mode}] If NIL (the default), then the presence of
    the -batch command-line switch will invoke batch-mode processing
    upon resuming the saved core.  If non-NIL, the produced core will
    always be in batch-mode, regardless of any command-line switches.

  \item[\kwd{executable}] If non-NIL, an executable image is created.
    Normally, \cmucl{} consists of the C runtime along with a core
    file image.  When \kwd{executable} is non-NIL, the core file is
    incorporated into the C runtime, so one (large) executable is
    created instead of a new separate core file.

    This feature is only available on some platforms.  Currently only
    x86 on Linux and FreeBSD support this.
  \end{Lentry}
\end{defun}

To resume a saved file, type:
\begin{example}
lisp -core file
\end{example}

\begin{defun}{extensions:}{purify}{
    \args{\var{file}
      \keys{\kwd{root-structures} \kwd{environment-name}}}}
  
  This function optimizes garbage collection by moving all currently
  live objects into non-collected storage.  Once statically allocated,
  the objects can never be reclaimed, even if all pointers to them are
  dropped.  This function should generally be called after a large
  system has been loaded and initialized.

  \begin{Lentry}
  \item[\kwd{root-structures}] is an optional list of objects which
    should be copied first to maximize locality.  This should be a
    list of the main entry points for the resulting core image.  The
    purification process tries to localize symbols, functions, etc.,
    in the core image so that paging performance is improved.  The
    default value is NIL which means that Lisp objects will still be
    localized but probably not as optimally as they could be.
  
    \var{defstruct} structures defined with the \code{(:pure t)}
    option are moved into read-only storage, further reducing GC cost.
    List and vector slots of pure structures are also moved into
    read-only storage.
  
  \item[\kwd{environment-name}] is gratuitous documentation for the
    compacted version of the current global environment (as seen in
    \code{c::*info-environment*}.)  If \false{} is supplied, then
    environment compaction is inhibited.
  \end{Lentry}
\end{defun}


\section{Pathnames}

In \clisp{} quite a few aspects of \tindexed{pathname} semantics are left to
the implementation.  


\subsection{Unix Pathnames}
\cpsubindex{unix}{pathnames}

Unix pathnames are always parsed with a \code{unix-host} object as the host and
\code{nil} as the device.  The last two dots (\code{.}) in the namestring mark
the type and version, however if the first character is a dot, it is considered
part of the name.  If the last character is a dot, then the pathname has the
empty-string as its type.  The type defaults to \code{nil} and the version
defaults to \kwd{newest}.

\begin{example}
(defun parse (x)
  (values (pathname-name x) (pathname-type x) (pathname-version x)))

(parse "foo") \result "foo", NIL, :NEWEST
(parse "foo.bar") \result "foo", "bar", :NEWEST
(parse ".foo") \result ".foo", NIL, :NEWEST
(parse ".foo.bar") \result ".foo", "bar", :NEWEST
(parse "..") \result ".", "", :NEWEST
(parse "foo.") \result "foo", "", :NEWEST
(parse "foo.bar.1") \result "foo", "bar", 1
(parse "foo.bar.baz") \result "foo.bar", "baz", :NEWEST
\end{example}

The directory of pathnames beginning with a slash (or a search-list,
\pxlref{search-lists}) is starts \kwd{absolute}, others start with
\kwd{relative}.  The \code{..} directory is parsed as \kwd{up}; there is no
namestring for \kwd{back}:

\begin{example}
(pathname-directory "/usr/foo/bar.baz") \result (:ABSOLUTE "usr" "foo")
(pathname-directory "../foo/bar.baz") \result (:RELATIVE :UP "foo")
\end{example}


\subsection{Wildcard Pathnames}

Wildcards are supported in Unix pathnames.  If `\code{*}' is specified for a
part of a pathname, that is parsed as \kwd{wild}.  `\code{**}' can be used as a
directory name to indicate \kwd{wild-inferiors}.  Filesystem operations
treat \kwd{wild-inferiors} the same as\ \kwd{wild}, but pathname pattern
matching (e.g. for logical pathname translation, \pxlref{logical-pathnames})
matches any number of directory parts with `\code{**}' (see
\pxlref{wildcard-matching}.)

`\code{*}' embedded in a pathname part matches any number of characters.
Similarly, `\code{?}' matches exactly one character, and `\code{[a,b]}'
matches the characters `\code{a}' or `\code{b}'.  These pathname parts are
parsed as \code{pattern} objects.

Backslash can be used as an escape character in namestring
parsing to prevent the next character from being treated as a wildcard.  Note
that if typed in a string constant, the backslash must be doubled, since the
string reader also uses backslash as a quote:

\begin{example}
(pathname-name "foo\(\backslash\backslash\)*bar") => "foo*bar"
\end{example}


\subsection{Logical Pathnames}
\cindex{logical pathnames}
\label{logical-pathnames}

If a namestring begins with the name of a defined logical pathname
host followed by a colon, then it will be parsed as a logical
pathname.  Both `\code{*}' and `\code{**}' wildcards are implemented.
\findexed{load-logical-pathname-translations} on \var{name} looks for a
logical host definition file in
\w{\file{library:\var{name}.translations}}. Note that \file{library:}
designates the search list (\pxlref{search-lists}) initialized to the
\cmucl{} \file{lib/} directory, not a logical pathname.  The format of
the file is a single list of two-lists of the from and to patterns:

\begin{example}
(("foo;*.text" "/usr/ram/foo/*.txt")
 ("foo;*.lisp" "/usr/ram/foo/*.l"))
\end{example}


\subsection{Search Lists}
\cindex{search lists}
\label{search-lists}

Search lists are an extension to \clisp{} pathnames.  They serve a function
somewhat similar to \clisp{} logical pathnames, but work more like Unix PATH
variables.  Search lists are used for two purposes:
\begin{itemize}
\item They provide a convenient shorthand for commonly used directory names,
and

\item They allow the abstract (directory structure independent) specification
of file locations in program pathname constants (similar to logical pathnames.)
\end{itemize}
Each search list has an associated list of directories (represented as
pathnames with no name or type component.)  The namestring for any relative
pathname may be prefixed with ``\var{slist}\code{:}'', indicating that the
pathname is relative to the search list \var{slist} (instead of to the current
working directory.)  Once qualified with a search list, the pathname is no
longer considered to be relative.

When a search list qualified pathname is passed to a file-system operation such
as \code{open}, \code{load} or \code{truename}, each directory in the search
list is successively used as the root of the pathname until the file is
located.  When a file is written to a search list directory, the file is always
written to the first directory in the list.


\subsection{Predefined Search-Lists}

These search-lists are initialized from the Unix environment or when Lisp was
built:
\begin{Lentry}
\item[\code{default:}] The current directory at startup.

\item[\code{home:}] The user's home directory.

\item[\code{library:}] The \cmucl{} \file{lib/} directory (\code{CMUCLLIB} environment
variable.)

\item[\code{path:}] The Unix command path (\code{PATH} environment variable.)

\item[\code{target:}] The root of the tree where \cmucl{} was compiled.
\end{Lentry}
It can be useful to redefine these search-lists, for example, \file{library:}
can be augmented to allow logical pathname translations to be located, and
\file{target:} can be redefined to point to where \cmucl{} system sources are
locally installed. 


\subsection{Search-List Operations}

These operations define and access search-list definitions.  A search-list name
may be parsed into a pathname before the search-list is actually defined, but
the search-list must be defined before it can actually be used in a filesystem
operation.

\begin{defun}{extensions:}{search-list}{\var{name}}
  
  This function returns the list of directories associated with the
  search list \var{name}.  If \var{name} is not a defined search list,
  then an error is signaled.  When set with \code{setf}, the list of
  directories is changed to the new value.  If the new value is just a
  namestring or pathname, then it is interpreted as a one-element
  list.  Note that (unlike Unix pathnames), search list names are
  case-insensitive.
\end{defun}

\begin{defun}{extensions:}{search-list-defined-p}{\var{name}}
  \defunx[extensions:]{clear-search-list}{\var{name}}
  
  \code{search-list-defined-p} returns \true{} if \var{name} is a
  defined search list name, \false{} otherwise.
  \code{clear-search-list} make the search list \var{name} undefined.
\end{defun}

\begin{defmac}{extensions:}{enumerate-search-list}{%
    \args{(\var{var} \var{pathname} \mopt{result}) \mstar{form}}}
  
  This macro provides an interface to search list resolution.  The
  body \var{forms} are executed with \var{var} bound to each
  successive possible expansion for \var{name}.  If \var{name} does
  not contain a search-list, then the body is executed exactly once.
  Everything is wrapped in a block named \nil, so \code{return} can be
  used to terminate early.  The \var{result} form (default \nil) is
  evaluated to determine the result of the iteration.
\end{defmac}


\subsection{Search List Example}

The search list \code{code:} can be defined as follows:
\begin{example}
(setf (ext:search-list "code:") '("/usr/lisp/code/"))
\end{example}
It is now possible to use \code{code:} as an abbreviation for the directory
\file{/usr/lisp/code/} in all file operations.  For example, you can now specify
\code{code:eval.lisp} to refer to the file \file{/usr/lisp/code/eval.lisp}.

To obtain the value of a search-list name, use the function search-list
as follows:
\begin{example}
(ext:search-list \var{name})
\end{example}
Where \var{name} is the name of a search list as described above.  For example,
calling \code{ext:search-list} on \code{code:} as follows:
\begin{example}
(ext:search-list "code:")
\end{example}
returns the list \code{("/usr/lisp/code/")}.


\section{Filesystem Operations}

\cmucl{} provides a number of extensions and optional features beyond those
required by the \clisp{} specification.


\subsection{Wildcard Matching}
\label{wildcard-matching}

Unix filesystem operations such as \code{open} will accept wildcard pathnames
that match a single file (of course, \code{directory} allows any number of
matches.)  Filesystem operations treat \kwd{wild-inferiors} the same as\
\kwd{wild}.

\begin{defun}{}{directory}{\var{wildname} \keys{\kwd{all} \kwd{check-for-subdirs}}
    \kwd{truenamep} \morekeys{\kwd{follow-links}}}
  
  The keyword arguments to this \clisp{} function are a \cmucl{} extension.
  The arguments (all default to \code{t}) have the following
  functions:
  \begin{Lentry}
  \item[\kwd{all}] Include files beginning with dot such as
    \file{.login}, similar to ``\code{ls -a}''.
    
  \item[\kwd{check-for-subdirs}] Test whether files are directories,
    similar to ``\code{ls -F}''.
    
  \item[\kwd{truenamep}] Call \code{truename} on each file, which
    expands out all symbolic links.  Note that this option can easily
    result in pathnames being returned which have a different
    directory from the one in the \var{wildname} argument.

  \item[\kwd{follow-links}] Follow symbolic links when searching for
    matching directories.
  \end{Lentry}
\end{defun}

\begin{defun}{extensions:}{print-directory}{%
    \args{\var{wildname}
      \ampoptional{} \var{stream}
      \keys{\kwd{all} \kwd{verbose}}
      \morekeys{\kwd{return-list}}}}
  
  Print a directory of \var{wildname} listing to \var{stream} (default
  \code{*standard-output*}.)  \kwd{all} and \kwd{verbose} both default
  to \false{} and correspond to the ``\code{-a}'' and ``\code{-l}''
  options of \file{ls}.  Normally this function returns \false{}, but
  if \kwd{return-list} is true, a list of the matched pathnames are
  returned.
\end{defun}


\subsection{File Name Completion}

\begin{defun}{extensions:}{complete-file}{%
    \args{\var{pathname}
      \keys{\kwd{defaults} \kwd{ignore-types}}}}
  
  Attempt to complete a file name to the longest unambiguous prefix.
  If supplied, directory from \kwd{defaults} is used as the ``working
  directory'' when doing completion.  \kwd{ignore-types} is a list of
  strings of the pathname types (a.k.a. extensions) that should be
  disregarded as possible matches (binary file names, etc.)
\end{defun}

\begin{defun}{extensions:}{ambiguous-files}{%
    \args{\var{pathname}
      \ampoptional{} \var{defaults}}}
  
  Return a list of pathnames for all the possible completions of
  \var{pathname} with respect to \var{defaults}.
\end{defun}


\subsection{Miscellaneous Filesystem Operations}

\begin{defun}{extensions:}{default-directory}{}
  
  Return the current working directory as a pathname.  If set with
  \code{setf}, set the working directory.
\end{defun}

\begin{defun}{extensions:}{file-writable}{\var{name}}
  
  This function accepts a pathname and returns \true{} if the current
  process can write it, and \false{} otherwise.
\end{defun}

\begin{defun}{extensions:}{unix-namestring}{%
    \args{\var{pathname}
      \ampoptional{} \var{for-input}}}
  
  This function converts \var{pathname} into a string that can be used
  with UNIX system calls.  Search-lists and wildcards are expanded.
  \var{for-input} controls the treatment of search-lists: when true
  (the default) and the file exists anywhere on the search-list, then
  that absolute pathname is returned; otherwise the first element of
  the search-list is used as the directory.
\end{defun}


\section{Time Parsing and Formatting}

\cindex{time parsing} \cindex{time formatting}
Functions are provided to allow parsing strings containing time information
and printing time in various formats are available.

\begin{defun}{extensions:}{parse-time}{%
    \args{\var{time-string}
      \keys{\kwd{error-on-mismatch} \kwd{default-seconds}}
      \morekeys{\kwd{default-minutes} \kwd{default-hours}}
      \yetmorekeys{\kwd{default-day} \kwd{default-month}}
      \yetmorekeys{\kwd{default-year} \kwd{default-zone}}
      \yetmorekeys{\kwd{default-weekday}}}}
  
  \code{parse-time} accepts a string containing a time (e.g.,
  \w{"\code{Jan 12, 1952}"}) and returns the universal time if it is
  successful.  If it is unsuccessful and the keyword argument
  \kwd{error-on-mismatch} is non-\nil{}, it signals an error.
  Otherwise it returns \nil{}.  The other keyword arguments have the
  following meaning:

  \begin{Lentry}
  \item[\kwd{default-seconds}] specifies the default value for the
    seconds value if one is not provided by \var{time-string}.  The
    default value is 0.
    
  \item[\kwd{default-minutes}] specifies the default value for the
    minutes value if one is not provided by \var{time-string}.  The
    default value is 0.
    
  \item[\kwd{default-hours}] specifies the default value for the hours
    value if one is not provided by \var{time-string}.  The default
    value is 0.
    
  \item[\kwd{default-day}] specifies the default value for the day
    value if one is not provided by \var{time-string}.  The default
    value is the current day.
    
  \item[\kwd{default-month}] specifies the default value for the month
    value if one is not provided by \var{time-string}.  The default
    value is the current month.
    
  \item[\kwd{default-year}] specifies the default value for the year
    value if one is not provided by \var{time-string}.  The default
    value is the current year.
    
  \item[\kwd{default-zone}] specifies the default value for the time
    zone value if one is not provided by \var{time-string}.  The
    default value is the current time zone.
    
  \item[\kwd{default-weekday}] specifies the default value for the day
    of the week if one is not provided by \var{time-string}.  The
    default value is the current day of the week.
  \end{Lentry}
  Any of the above keywords can be given the value \kwd{current} which
  means to use the current value as determined by a call to the
  operating system.
\end{defun}

\begin{defun}{extensions:}{format-universal-time}{
    \args{\var{dest} \var{universal-time}
       \\
       \keys{\kwd{timezone}}
       \morekeys{\kwd{style} \kwd{date-first}}
       \yetmorekeys{\kwd{print-seconds} \kwd{print-meridian}}
       \yetmorekeys{\kwd{print-timezone} \kwd{print-weekday}}}}
   \defunx[extensions:]{format-decoded-time}{
     \args{\var{dest} \var{seconds} \var{minutes} \var{hours} \var{day} \var{month} \var{year}
       \\
       \keys{\kwd{timezone}}
       \morekeys{\kwd{style} \kwd{date-first}}
       \yetmorekeys{\kwd{print-seconds} \kwd{print-meridian}}
       \yetmorekeys{\kwd{print-timezone} \kwd{print-weekday}}}}
   
   \code{format-universal-time} formats the time specified by
   \var{universal-time}.  \code{format-decoded-time} formats the time
   specified by \var{seconds}, \var{minutes}, \var{hours}, \var{day},
   \var{month}, and \var{year}.  \var{Dest} is any destination
   accepted by the \code{format} function.  The keyword arguments have
   the following meaning:
   \begin{Lentry}
     
   \item[\kwd{timezone}] is an integer specifying the hours west of
     Greenwich.  \kwd{timezone} defaults to the current time zone.
     
   \item[\kwd{style}] specifies the style to use in formatting the
     time.  The legal values are:
     \begin{Lentry}
  
     \item[\kwd{short}] specifies to use a numeric date.
  
     \item[\kwd{long}] specifies to format months and weekdays as
       words instead of numbers.
  
     \item[\kwd{abbreviated}] is similar to long except the words are
       abbreviated.
  
     \item[\kwd{government}] is similar to abbreviated, except the
       date is of the form ``day month year'' instead of ``month day,
       year''.
     \end{Lentry}
     
   \item[\kwd{date-first}] if non-\false{} (default) will place the
     date first.  Otherwise, the time is placed first.
  
   \item[\kwd{print-seconds}] if non-\false{} (default) will format
     the seconds as part of the time.  Otherwise, the seconds will be
     omitted.
  
   \item[\kwd{print-meridian}] if non-\false{} (default) will format
     ``AM'' or ``PM'' as part of the time.  Otherwise, the ``AM'' or
     ``PM'' will be omitted.
  
   \item[\kwd{print-timezone}] if non-\false{} (default) will format
     the time zone as part of the time.  Otherwise, the time zone will
     be omitted.

     %%\item[\kwd{print-seconds}]
     %%if non-\false{} (default) will format the seconds as part of
     %%the time.  Otherwise, the seconds will be omitted.
  
   \item[\kwd{print-weekday}] if non-\false{} (default) will format
     the weekday as part of date.  Otherwise, the weekday will be
     omitted.
   \end{Lentry}
\end{defun}


\section{Random Number Generation}
\cindex{random number generation}

\clisp{} includes a random number generator as a standard part of the
language; however, the implementation of the generator is not
specified.

\subsection{MT-19937 Generator}
\cpsubindex{random number generation}{MT-19937 generator}
On all platforms, the random number is \code{MT-19937} generator as indicated by
\kwd{rand-mt19937} being in \code{*features*}.  This is a Lisp
implementation of the MT-19937 generator of Makoto Matsumoto and
T. Nishimura.  We refer the reader to their paper\footnote{``Mersenne
  Twister: A 623-Dimensionally Equidistributed Uniform Pseudorandom
  Number Generator,'' ACM Trans. on Modeling and Computer Simulation,
  Vol. 8, No. 1, January 1998, pp.3--30} or to
their
\ifpdf
\href{http://www.math.sci.hiroshima-u.ac.jp/~m-mat/MT/emt.html}{website}.
\else
website at
\href{http://www.math.keio.ac.jp/~matumoto/emt.html}{\texttt{http://www.math.keio.ac.jp/~matsumoto/emt.html}}.
\fi

When \cmucl{} starts up, \code{*random-state*} is initialized by
reading 627 words from \code{/dev/urandom}, if \code{/dev/urandom} is
available.  If \code{/dev/urandom} is not available, the universal
time is used to initialize \code{*random-state*}.  The initialization
is done as given in Matsumoto's paper.

\section{Lisp Threads}
\cindex{lisp threads}

\cmucl{} supports Lisp threads for the x86 platform.

\section{Lisp Library}
\label{lisp-lib}

The \cmucl{} project maintains a collection of useful or interesting
programs written by users of our system.  The library is in
\file{lib/contrib/}.  Two files there that users should read are:
\begin{Lentry}

\item[CATALOG.TXT]
This file contains a page for each entry in the library.  It
contains information such as the author, portability or dependency issues, how
to load the entry, etc.

\item[READ-ME.TXT]
This file describes the library's organization and all the
possible pieces of information an entry's catalog description could contain.
\end{Lentry}

Hemlock has a command \F{Library Entry} that displays a list of the current
library entries in an editor buffer.  There are mode specific commands that
display catalog descriptions and load entries.  This is a simple and convenient
way to browse the library.


\section{Generalized Function Names}

\begin{defmac}{ext:}{define-function-name-syntax}{name (var) \ampbody\ body}
  Define lists starting with the symbol \code{name} as a new extended
  function name syntax.
  
  \code{body} is executed with \code{var} bound to an actual function
  name of that form, and should return two values:

  \begin{itemize}
  \item A generalized boolean that is true if \code{var} is a valid
    function name.
  \item A symbol that can be used as a \code{block} name in functions
    whose name is \code{var}.  (For some sorts of function names it
    might make sense to return \code{nil} for the block name, or just
    return one value.)
  \end{itemize}
  
  Users should not define function names starting with a symbol that
  \cmucl{} might be using internally.  It is therefore advisable to
  only define new function names starting with a symbol from a
  user-defined package.
\end{defmac}

\begin{defun}{ext:}{valid-function-name-p}{name}
  Returns two values:

  \begin{itemize}
  \item True if \code{name} is a valid function name.
  \item A symbol that can be used as a \code{block} name in
    functions whose name is \code{name}.  This can be \code{nil}
    for some function names.
  \end{itemize}
\end{defun}



\section{CLOS}

\subsection{Primary Method Errors}
\cindex{primary method}

The standard requires that an error is signaled when a generic
function is called and

\begin{itemize}
\item no primary method is applicable to the generic function's actual
  arguments, and
\item the generic function's method combination is either the standard
  method combination or a method combination defined with the short
  form of \code{define-method-combination}.  The latter includes the
  standardized method combinations like \code{progn}, \code{and}, etc.
\end{itemize}

\begin{defgeneric}[-generic]{pcl:}{no-primary-method}{gf \&rest args}
  In \cmucl, this generic function is called in the above erroneous
  cases.  The parameter \code{gf} is the generic function being
  called, and \code{args} is a list of actual arguments in the generic
  function call.
\end{defgeneric}

\begin{defmethod}[-standard]{pcl:}{no-primary-method}{%
    (gf standard-generic-function) \&rest args}
  This method signals a continuable error of type
  \code{pcl:no-primary-method-error}.
\end{defmethod}


\subsection{Slot Type Checking}
\cindex{slot type checking}

Declared slot types are used when 

\begin{itemize}
\item reading slot values with \code{slot-value} in methods, or

\item setting slots with \code{(setf slot-value)} in methods, or 
  
\item creating instances with \code{make-instance}, when slots are
  initialized from initforms.  This currently depends on PCL being
  able to use its internal \code{make-instance} optimization, which it
  usually can.
\end{itemize}

Example:

\begin{example}
(defclass foo ()
  ((a :type fixnum)))

(defmethod bar ((object foo) value)
  (with-slots (a) object
    (setf a value)))

(defmethod baz ((object foo))
  (< (slot-value object 'a) 10))
\end{example}

In method \code{bar}, and with a suitable safety setting, a type error
will occur if \code{value} is not a \code{fixnum}.  In method
\code{baz}, a \code{fixnum} comparison can be used by the compiler.
  
\begin{defvar}{pcl::}{use-slot-types-p}
  Slot type checking can be turned off by setting this variable to
  \false, which can be useful for compiling code containing incorrect
  slot type declarations.
\end{defvar}


\subsection{Slot Access Optimization}
\cindex{slot access optimization}
\cindex{slot declarations}

The declaration \code{ext:slots} is used for optimizing slot access in
methods.

\begin{example}
declare (ext:slots specifier*)

specifier   ::= (quality class-entry*)
quality     ::= SLOT-BOUNDP | INLINE
class-entry ::= class | (class slot-name*)
class       ::= the name of a class
slot-name   ::= the name of a slot
\end{example}

The \code{slot-boundp} quality specifies that all or some slots of a
class are always bound.

The \code{inline} quality specifies that access to all or some slots
of a class should be inlined, using compile-time knowledge of class
layouts.



\subsubsection{\code{slot-boundp} Declaration}
\cpsubindex{slot declaration}{slot-boundp}

Example:

\begin{example}
(defclass foo ()
  (a b))

(defmethod bar ((x foo))
  (declare (ext:slots (slot-boundp foo)))
  (list (slot-value x 'a) (slot-value x 'b)))
\end{example}

The \code{slot-boundp} declaration in method \code{bar} specifies that
the slots \code{a} and \code{b} accessed through parameter \code{x} in
the scope of the declaration are always bound, because parameter
\code{x} is specialized on class \code{foo} to which the
\code{slot-boundp} declaration applies.  The PCL-generated code for
the \code{slot-value} forms will thus not contain tests for the slots
being bound or not.  The consequences are undefined should one of the
accessed slots not be bound.



\subsubsection{\code{inline} Declaration}
\cpsubindex{slot declaration}{inline}

Example:

\begin{example}
(defclass foo ()
  (a b))

(defmethod bar ((x foo))
  (declare (ext:slots (inline (foo a))))
  (list (slot-value x 'a) (slot-value x 'b)))
\end{example}

The \code{inline} declaration in method \code{bar} tells PCL to use
compile-time knowledge of slot locations for accessing slot \code{a}
of class \code{foo}, in the scope of the declaration.

Class \code{foo} must be known at compile time for this optimization
to be possible.  PCL prints a warning and uses normal slot access If
the class is not defined at compile time.

If a class is \code{proclaim}ed to use inline slot access before it is
defined, the class is defined at compile time.  Example:

\begin{example}
(declaim (ext:slots (inline (foo slot-a))))
(defclass foo () ...)
(defclass bar (foo) ...)
\end{example}
  
Class \code{foo} will be defined at compile time because it is
declared to use inline slot access; methods accessing slot
\code{slot-a} of \code{foo} will use inline slot access if otherwise
possible.  Class \code{bar} will be defined at compile time because
its superclass \code{foo} is declared to use inline slot access.  PCL
uses compile-time information from subclasses to warn about situations
where using inline slot access is not possible.

Normal slot access will be used if PCL finds, at method compilation
time, that

\begin{itemize}
\item class \code{foo} has a subclass in which slot \code{a} is at a
  different location, or

\item there exists a \code{slot-value-using-class} method for
  \code{foo} or a subclass of \code{foo}.
\end{itemize}
  
When the declaration is used to optimize calls to slot accessor
generic functions in methods, as opposed to \code{slot-value} or
\code{(setf slot-value)}, the optimization is additionally not used if

\begin{itemize}
\item there exist, at compile time, applicable methods on the
  reader/writer generic function that are not standard accessor
  methods (for instance, there exist around-methods), or
  
\item applicable reader/writer methods access different slots in a
  class accessed inline, and one of its subclasses.
\end{itemize}

The consequences are undefined if the compile-time environment is not
the same as the run-time environment in these respects, or if the
definition of class \code{foo} or any subclass of \code{foo} is
changed in an incompatible way, that is, if slot locations change.

The effect of the \code{inline} optimization combined with the
\code{slot-boundp} optimization is that CLOS slot access becomes as
fast as structure slot access, which is an order of magnitude faster
than normal CLOS slot access.

\begin{defvar}{pcl::}{optimize-inline-slot-access-p}
  This variable controls if inline slot access optimizations are
  performed.  It is true by default.
\end{defvar}



\subsubsection{Automatic Method Recompilation}
\cindex{methods}
\cpsubindex{methods}{auto-compilation}
\cpsubindex{slot declaration}{method recompilation}
  
Methods using inline slot access can be automatically recompiled after
class changes.  Two declarations control which methods are
automatically recompiled.

\begin{example}
declaim (ext:auto-compile specifier*)
declaim (ext:not-auto-compile specifier*)

specifier   ::= gf-name | (gf-name qualifier* (specializer*))
gf-name     ::= the name of a generic function
qualifier   ::= a method qualifier
specializer ::= a method specializer
\end{example}

If no specifier is given, auto-compilation is by default done/not done
for all methods of all generic functions using inline slot access;
current default is that it is not done.  This global policy can be
overridden on a generic function and method basis.  If
\code{specifier} is a generic function name, it applies to all methods
of that generic function.

Examples:

\begin{example}
(declaim (ext:auto-compile foo))
(defmethod foo :around ((x bar)) ...)
\end{example}

The around-method \code{foo} will be automatically recompiled because
the declamation applies to all methods with name \code{foo}.

\begin{example}
(declaim (ext:auto-compile (foo (bar))))
(defmethod foo :around ((x bar)) ...)
(defmethod foo ((x bar)) ...)
\end{example}

The around-method will not be automatically recompiled, but the
primary method will.

\begin{example}
(declaim (ext:auto-compile foo))
(declaim (ext:not-auto-compile (foo :around (bar)))  
(defmethod foo :around ((x bar)) ...)
(defmethod foo ((x bar)) ...)
\end{example}

The around-method will not be automatically recompiled, because it
is explicitly declaimed not to be.  The primary method will be
automatically recompiled because the first declamation applies to
it.

Auto-recompilation works by recording method bodies using inline slot
access.  When PCL determines that a recompilation is necessary, a
\code{defmethod} form is constructed and evaluated.

Auto-compilation can only be done for methods defined in a null
lexical environment.  PCL prints a warning and doesn't record the
method body if a method using inline slot access is defined in a
non-null lexical environment.  Instead of doing a recompilation on
itself, PCL will then print a warning that the method must be
recompiled manually when classes are changed.



\subsection{Inlining Methods in Effective Methods}
\cindex{effective method}
\cpsubindex{methods}{inlining in effective methods}
\cpsubindex{effective method}{inlining of methods}
\cindex{inline}

When a generic function is called, an effective method is constructed
from applicable methods.  The effective method is called with the
original arguments, and itself calls applicable methods according to
the generic function's method combination.  Some of the function call
overhead in effective methods can be removed by inlining methods in
effective methods, at the expense of increased code size.

Inlining of methods is controlled by the usual \code{inline}
declaration.  In the following example, both \code{foo} methods shown
will be inlined in effective methods:

\begin{example}
(declaim (inline (method foo (foo))
                 (method foo :before (foo))))
(defmethod foo ((x foo)) ...)
(defmethod foo :before ((x foo)) ...)
\end{example}

Please note that this form of inlining has no noticeable effect for
effective methods that consist of a primary method only, which doesn't
have keyword arguments.  In such cases, PCL uses the primary method
directly for the effective method.

When the definition of an inlined method is changed, effective methods
are \textbf{not} automatically updated to reflect the change.  This is
just as it is when inlining normal functions.  Different from the
normal case is that users do not have direct access to effective
methods, as it would be the case when a function is inlined somewhere
else.  Because of this, the function \code{pcl:flush-emf-cache} is
provided for forcing such an update of effective methods.

\begin{defun}{pcl:}{flush-emf-cache}{\&optional gf}
  Flush cached effective method functions.  If \code{gf} is supplied,
  it should be a generic function metaobject or the name of a generic
  function, and this function flushes all cached effective methods for
  the given generic function.  If \code{gf} is not supplied, all
  cached effective methods are flushed.
\end{defun}

\begin{defvar}{pcl::}{inline-methods-in-emfs}
  If true, the default, perform method inlining as described above.
  If false, don't.
\end{defvar}



\subsection{Effective Method Precomputation}
\cpsubindex{effective method}{precomputation}
\cpsubindex{methods}{load time}
\cpsubindex{methods}{emf precomputation}

When a generic function is called, the generic function's
discriminating function computes the set of methods applicable to
actual arguments and constructs an effective method function from
applicable methods, using the generic function's method combination.

Effective methods can be precomputed at method load time instead of
when the generic function is called depending on the value of
\code{pcl:*max-emf-precomputation-methods*}.

\begin{defvar}{pcl:}{*max-emf-precomputation-methods*}
  If nonzero, the default value is 100, precompute effective methods
  when methods are loaded, and the method's generic function has less
  than the specified number of methods.
  
  If zero, compute effective methods only when the generic function is
  called.
\end{defvar}



\subsection{Sealing}
\cindex{sealing}
\cpsubindex{sealing}{subclasses}
\cpsubindex{sealing}{methods}
\cpsubindex{methods}{sealing}

Support for sealing classes and generic functions have been
implemented.  Please note that this interface is subject to change.

\begin{defmac}{pcl:}{seal}{name (var) \amprest\ specifiers}
  Seal \code{name} with respect to the given specifiers; \code{name}
  can be the name of a class or generic-function.

  Supported specifiers are \kwd{subclasses} for classes,
  which prevents changing subclasses of a class, and \kwd{methods}
  which prevents changing the methods of a generic function.
  
  Sealing violations signal an error of type \code{pcl:sealed-error}.
\end{defmac}

\begin{defun}{pcl:}{unseal}{name-or-object}
  Remove seals from \code{name-or-object}.
\end{defun}



\subsection{Method Tracing and Profiling}
\label{sec:method-tracing}
\cindex{tracing}
\cpsubindex{tracing}{methods}
\cindex{profiling}
\cpsubindex{profiling}{methods}
\cpsubindex{methods}{tracing}
\cpsubindex{methods}{profiling}

Methods can be traced with \code{trace}, using function names of the
form \code{(method <name> <qualifiers> <specializers>)}.  Example:

\begin{example}
(defmethod foo ((x integer)) x)
(defmethod foo :before ((x integer)) x)

(trace (method foo (integer)))
(trace (method foo :before (integer)))
(untrace (method foo :before (integer)))
\end{example}
  
\code{trace} and \code{untrace} also allow a name specifier
\code{:methods gf-form} for tracing all methods of a generic function:

\begin{example}
(trace :methods 'foo)
(untrace :methods 'foo)
\end{example}

Methods can also be specified for the \kwd{wherein} option to
\code{trace}.  Because this option is a name or a list of names,
methods must be specified as a list.  Thus, to trace all calls of
\code{foo} from the method \code{bar} specialized on integer argument,
use
\begin{example}
  (trace foo :wherein ((method bar (integer))))
\end{example}
Before and after methods are supported as well:
\begin{example}
  (trace foo :wherein ((method bar :before (integer))))
\end{example}

Method profiling is done analogously to \code{trace}:

\begin{example}
(defmethod foo ((x integer)) x)
(defmethod foo :before ((x integer)) x)

(profile:profile (method foo (integer)))
(profile:profile (method foo :before (integer)))
(profile:unprofile (method foo :before (integer)))

(profile:profile :methods 'foo)
(profile:unprofile :methods 'foo)

(profile:profile-all :methods t)
\end{example}



\subsection{Misc}
\cpsubindex{methods}{interpreted}

\begin{defvar}{pcl::}{compile-interpreted-methods-p}
  This variable controls compilation of interpreted method functions,
  e.g. for methods defined interactively at the REPL.  Default is
  true, that is, method functions are compiled.
\end{defvar}





\section{Differences from ANSI Common Lisp}
This section describes some of the known differences between \cmucl{}
and ANSI \clisp{}.  Some may be non-compliance issues; same may be
extensions.

\subsection{Extensions}

\begin{defun}{}{constantly}{value \&optional val1 val2 \&rest
    more-values}
  As an extension, \cmucl{} allows \code{constantly} to accept more
  than one value which are returned as multiple values.
\end{defun}




\section{Function Wrappers}
\cindex{function wrappers}
\cindex{fwrappers}

Function wrappers, fwrappers for short, are a facility for efficiently
encapsulating functions\footnote{This feature was independently
developed, but the interface is modelled after a similar feature in
Allegro.  Some names, however, have been changed.}.

Functions in \cmucl{} are represented by \code{kernel:fdefn}
objects.  Each \code{fdefn} object contains a reference to its
function's actual code, which we call the function's primary function.

A function wrapper replaces the primary function in the \code{fdefn}
object with a function of its own, and records the original function
in an fwrapper object, a funcallable instance.  Thus, when the
function is called, the fwrapper gets called, which in turn might call
the primary function, or a previously installed fwrapper that was
found in the \code{fdefn} object when the second fwrapper was
installed.

Example:

\begin{lisp}
(use-package :fwrappers)

(define-fwrapper foo (x y)
  (format t "x = ~s, y = ~s, user-data = ~s~%"
          x y (fwrapper-user-data fwrapper))
  (let ((value (call-next-function)))
    (format t "value = ~s~%" value)
    value))

(defun bar (x y)
  (+ x y))

(fwrap 'bar #'foo :type 'foo :user-data 42)

(bar 1 2)
 =>
 x = 1, y = 2, user-data = 42
 value = 3
 3   
\end{lisp}

Fwrappers are used in the implementation of \code{trace} and
\code{profile}.

Please note that \code{fdefinition} always returns the primary
definition of a function; if a function is fwrapped,
\code{fdefinition} returns the primary function stored in the
innermost fwrapper object.  Likewise, if a function is fwrapped,
\code{(setf fdefinition)} will set the primary function in the
innermost fwrapper.

\begin{defmac}{fwrappers:}{define-fwrapper}{name lambda-list \ampbody body}
  This macro is like \code{defun}, but defines a function named
  \var{name} that can be used as an fwrapper definition.
  
  In \var{body}, the symbol \code{fwrapper} is bound to the current
  fwrapper object.
  
  The macro \code{call-next-function} can be used to invoke the next
  fwrapper, or the primary function that is being fwrapped.  When
  called with no arguments, \code{call-next-function} invokes the next
  function with the original arguments passed to the fwrapper, unless
  you modify one of the parameters.  When called with arguments,
  \code{call-next-function} invokes the next function with the given
  arguments.
\end{defmac}

\begin{defun}{fwrappers:}{fwrap}{function-name fwrapper \&key type
    user-data}
  This function wraps function \code{function-name} in an fwrapper
  \var{fwrapper} which was defined with \code{define-fwrapper}.

  The value of \var{type}, if supplied, is used as an identifying
  tag that can be used in various other operations.
  
  The value of \var{user-data} is stored as user-supplied data in the
  fwrapper object that is created for the function encapsulation.
  User-data is accessible in the body of fwrappers defined with
  \code{define-fwrapper} as \code{(fwrapper-user-data fwrapper)}.

  Value is the fwrapper object created.
\end{defun}

\begin{defun}{fwrappers:}{funwrap}{function-name \&key type test}
  Remove fwrappers from the function named \var{function-name}.  If
  \var{type} is supplied, remove fwrappers whose type is \code{equal}
  to \var{type}.  If \var{test} is supplied, remove fwrappers
  satisfying \var{test}.
\end{defun}

\begin{defun}{fwrappers:}{find-fwrapper}{function-name \&key type test}
  Find an fwrapper of \var{function-name}.  If \var{type} is supplied,
  find an fwrapper whose type is \code{equal} to \var{type}.  If
  \var{test} is supplied, find an fwrapper satisfying \var{test}.
\end{defun}

\begin{defun}{fwrappers:}{update-fwrapper}{fwrapper}
  Update the funcallable instance function of the fwrapper object
  \var{fwrapper} from the definition of its function that was 
  defined with \code{define-fwrapper}.  This can be used to update
  fwrappers after changing a \code{define-fwrapper}.
\end{defun}

\begin{defun}{fwrappers:}{update-fwrappers}{function-name \&key type test}
  Update fwrappers of \var{function-name}; see \code{update-fwrapper}.
  If \var{type} is supplied, update fwrappers whose type is
  \code{equal} to \var{type}.  If \var{test} is supplied, update fwrappers
  satisfying \var{test}.
\end{defun}

\begin{defun}{fwrappers:}{set-fwrappers}{function-name fwrappers}
  Set \var{function-names}'s fwrappers to elements of the list
  \var{fwrappers}, which is assumed to be ordered from outermost to
  innermost.  \var{fwrappers} null means remove all fwrappers.
\end{defun}

\begin{defun}{fwrappers:}{list-fwrappers}{function-name}
  Return a list of all fwrappers of \var{function-name}, ordered
  from outermost to innermost.
\end{defun}

\begin{defun}{fwrappers:}{push-fwrapper}{fwrapper function-name}
  Prepend fwrapper \var{fwrapper} to the definition of
  \var{function-name}.  Signal an error if \var{function-name} is an
  undefined function.
\end{defun}

\begin{defun}{fwrappers:}{delete-fwrapper}{fwrapper function-name}
  Remove fwrapper \var{fwrapper} from the definition of
  \var{function-name}.  Signal an error if \var{function-name} is an
  undefined function.
\end{defun}

\begin{defmac}{fwrappers:}{do-fwrappers}{(var fdefn \ampoptional
  result) \ampbody body}
  Evaluate \var{body} with \var{var} bound to consecutive fwrappers of
  \var{fdefn}.  Return \var{result} at the end.  Note that \var{fdefn}
  must be an \code{fdefn} object.  You can use
  \code{kernel:fdefn-or-lose}, for instance, to get the \code{fdefn}
  object from a function name.
\end{defmac}

\section{Dynamic-Extent Declarations}
\cindex{dynamic-extent}

\emph{Note:  As of the 19a release, \code{dynamic-extent} is
  unfortunately disabled by default.  It is known to cause some issues
  with CLX and Hemlock.  The cause is not known, but causes random
  errors and brokeness.  Enable at your own risk.  However, it is safe
  enough to build all of CMUCL without problems.}

On x86 and sparc, \cmucl{} can exploit \code{dynamic-extent}
declarations by allocating objects on the stack instead of the heap.

You can tell \cmucl{} to trust or not trust \code{dynamic-extent}
declarations by setting the variable
\var{*trust-dynamic-extent-declarations*}.

\begin{defvar}{ext:}{trust-dynamic-extent-declarations}
  If the value of \var{*trust-dynamic-extent-declarations*} is 
  \code{NIL}, \code{dynamic-extent} declarations are effectively
  ignored.

  If the value of this variable is a function, the function is called
  with four arguments to determine if a \code{dynamic-extent} 
  declaration should be trusted.  The arguments are the safety,
  space, speed, and debug settings at the point where the 
  \code{dynamic-extent} declaration is used.  If the function
  returns true, the declaration is trusted, otherwise it is not
  trusted.

  In all other cases, \code{dynamic-extent} declarations are
  trusted.
\end{defvar}

Please note that stack-allocation is inherently unsafe.  If you make a
mistake, and a stack-allocated object or part of it escapes, \cmucl{}
is likely to crash, or format your hard disk.

\subsection{\code{\&rest} argument lists}
\cpsubindex{dynamic-extent}{rest lists}

Rest argument lists can be allocated on the stack by declaring the
rest argument variable \code{dynamic-extent}.  Examples:

\begin{lisp}
(defun foo (x &rest rest)
  (declare (dynamic-extent rest))
  ...)

(defun bar ()
  (lambda (&rest rest)
    (declare (dynamic-extent rest))
    ...))
\end{lisp}

\subsection{Closures}
\cpsubindex{dynamic-extent}{closures}

Closures for local functions can be allocated on the stack if the
local function is declared \code{dynamic-extent}, and the closure
appears as an argument in the call of a named function.  In the
example:

\begin{lisp}
(defun foo (x)
  (flet ((bar () x))
    (declare (dynamic-extent #'bar))
    (baz #'bar)))
\end{lisp}

the closure passed to function \code{baz} is allocated on the stack.
Likewise in the example:

\begin{lisp}
(defun foo (x)
  (flet ((bar () x))
    (baz #'bar)
    (locally (declare (dynamic-extent #'bar))
      (baz #'bar))))
\end{lisp}

\cpsubindex{dynamic-extent}{known CL functions}

Stack-allocation of closures can also automatically take place when
calling certain known CL functions taking function arguments, for
example \code{some} or \code{find-if}.

\subsection{\code{list}, \code{list*}, and \code{cons}}
\cpsubindex{dynamic-extent}{list, list*, cons}

New conses allocated by \code{list}, \code{list*}, or \code{cons}
which are used to initialize variables can be allocated from the stack
if the variables are declared \code{dynamic-extent}.  In the case of
\code{cons}, only the outermost cons cell is allocated from the stack;
this is an arbitrary restriction.

\begin{lisp}
(let ((x (list 1 2))
      (y (list* 1 2 x))
      (z (cons 1 (cons 2 nil))))
  (declare (dynamic-extent x y z))
  ...
  (setq x (list 2 3))
  ...)
\end{lisp}

Please note that the \code{setq} of \code{x} in the example program
assigns to \code{x} a list that is allocated from the heap.  This is
another arbitrary restriction that exists because other Lisps behave
that way.

\section{Modular Arithmetic}
\cindex{modular-arith}

This section is mostly taken, with permission,  from the documentation
for SBCL.

Some numeric functions have a property: \code{N} lower bits of
the result depend only on \code{N} lower bits of (all or some)
arguments. If the compiler sees an expression of form \code{(logand
exp mask)}, where \code{exp} is a tree of such ``good'' functions
and \code{mask} is known to be of type \code{(unsigned-byte
w)}, where \code{w} is a "good" width, all intermediate results
will be cut to \code{w} bits (but it is not done for variables
and constants!). This often results in an ability to use simple
machine instructions for the functions.

Consider an example.
\begin{lisp}
(defun i (x y)
  (declare (type (unsigned-byte 32) x y))
  (ldb (byte 32 0) (logxor x (lognot y))))
\end{lisp}
The result of \code{(lognot y)} will be negative and of
type \code{(signed-byte 33)}, so a naive implementation on a 32-bit
platform is unable to use 32-bit arithmetic here. But modular
arithmetic optimizer is able to do it: because the result is cut down
to 32 bits, the compiler will replace \code{logxor}
and \code{lognot} with versions cutting results to 32 bits, and
because terminals (here---expressions \code{x} and \code{y})
are also of type \code{(unsigned-byte 32)}, 32-bit machine
arithmetic can be used.


Currently ``good'' functions
are \code{+}, \code{-}, \code{*}; \code{logand}, \code{logior},
\code{logxor}, \code{lognot} and their combinations;
and \code{ash} with the positive second argument. ``Good'' widths
are 32 on HPPA, MIPS, PPC, Sparc and X86 and 64 on Alpha. While it is
possible to support smaller widths as well, currently it is not
implemented.

A more extensive description of modular arithmetic can be found in the
paper ``Efficient Hardware Arithmetic in Common Lisp'' by Alexey
Dejneka, and Christophe Rhodes, to be published.

\section{Extension to REQUIRE}
\cindex{require}

The behavior of \code{require} when called with only one argument is
implementation-defined.  In \cmucl, functions from the list
\var{*module-provider-functions*} are called in order with the
stringified module name as the argument.  The first function to return
non-\var{NIL} is assumed to have loaded the module.

By default the functions \code{module-provide-cmucl-defmodule} and
\code{module-provide- cmucl-library} are on this list of functions, in
that order.

\begin{defvar}{ext:}{module-provider-functions}
  This is a list of functions taking a single argument.
  \code{require} calls each function in turn with the stringified
  module name.  The first function to return non-\var{NIL} indicates
  that the module has been loaded.  The remaining functions, if any,
  are not called.

  To add new providers, push the new provider function onto the
  beginning of this list.
\end{defvar}

\begin{defmac}{ext:}{defmodule}{name \amprest{} files}
  Defines a module by registering the files that need to be loaded
  when the module is required.  If \var{name} is a symbol, its print
  name is used after downcasing it.
\end{defmac}

\begin{defun}{ext:}{module-provide-cmucl-defmodule}{module-name}
  This function is the module-provider for modules registered by a
  \code{ext:defmodule} form.  
\end{defun}

\begin{defun}{ext:}{module-provide-cmucl-library}{module-name}
  This function is the module-provider for \cmucl's libraries,
  including Gray streams, simple streams, CLX, CLM, Hemlock,
  \emph{etc}.
  
  This function causes a file to be loaded whose name is formed by
  merging the search-list ``modules:'' and the concatenation of
  module-name with the suffix ``-LIBRARY''.  Note that both the
  module-name and the suffix are each, separately, converted from
  :case :common to :case :local.  This merged name will be probed with
  both a .lisp and .fasl extensions, calling \code{LOAD} if it exists.
\end{defun}

\chapter{The Debugger}
\cindex{debugger}
\label{debugger}

\credits{by Robert MacLachlan}


\section{Debugger Introduction}

The \cmucl{} debugger is unique in its level of support for source-level
debugging of compiled code.  Although some other debuggers allow access of
variables by name, this seems to be the first \llisp{} debugger that:
\begin{itemize}

\item
Tells you when a variable doesn't have a value because it hasn't been
initialized yet or has already been deallocated, or

\item
Can display the precise source location corresponding to a code
location in the debugged program.
\end{itemize}
These features allow the debugging of compiled code to be made almost
indistinguishable from interpreted code debugging.

The debugger is an interactive command loop that allows a user to examine
the function call stack.  The debugger is invoked when:
\begin{itemize}

\item
A \tindexed{serious-condition} is signaled, and it is not handled, or

\item
\findexed{error} is called, and the condition it signals is not handled, or

\item
The debugger is explicitly invoked with the \clisp{} \findexed{break}
or \findexed{debug} functions.
\end{itemize}

{\it Note: there are two debugger interfaces in \cmucl{}: the TTY
debugger (described below) and the Motif debugger. Since the
difference is only in the user interface, much of this chapter also
applies to the Motif version. \xlref{motif-interface} for a very brief
discussion of the graphical interface.}

When you enter the TTY debugger, it looks something like this:

\begin{example}
Error in function CAR.
Wrong type argument, 3, should have been of type LIST.

Restarts:
  0: Return to Top-Level.

Debug  (type H for help)

(CAR 3)
0]
\end{example}

The first group of lines describe what the error was that put us in the
debugger.  In this case \code{car} was called on \code{3}.  After \code{Restarts:}
is a list of all the ways that we can restart execution after this error.  In
this case, the only option is to return to top-level.  After printing its
banner, the debugger prints the current frame and the debugger prompt.


\section{The Command Loop}

The debugger is an interactive read-eval-print loop much like the normal
top-level, but some symbols are interpreted as debugger commands instead
of being evaluated.  A debugger command starts with the symbol name of
the command, possibly followed by some arguments on the same line.  Some
commands prompt for additional input.  Debugger commands can be
abbreviated by any unambiguous prefix: \code{help} can be typed as
\code{h}, \code{he}, etc.  For convenience, some commands have
ambiguous one-letter abbreviations: \code{f} for \code{frame}.

The package is not significant in debugger commands; any symbol with the
name of a debugger command will work.  If you want to show the value of
a variable that happens also to be the name of a debugger command, you
can use the \code{list-locals} command or the \code{debug:var}
function, or you can wrap the variable in a \code{progn} to hide it from
the command loop.

The debugger prompt is ``\var{frame}\code{]}'', where \var{frame} is the number
of the current frame.  Frames are numbered starting from zero at the top (most
recent call), increasing down to the bottom.  The current frame is the frame
that commands refer to.  The current frame also provides the lexical
environment for evaluation of non-command forms.

\cpsubindex{evaluation}{debugger} The debugger evaluates forms in the lexical
environment of the functions being debugged.  The debugger can only
access variables.  You can't \code{go} or \code{return-from} into a
function, and you can't call local functions.  Special variable
references are evaluated with their current value (the innermost binding
around the debugger invocation)\dash{}you don't get the value that the
special had in the current frame.  \xlref{debug-vars} for more
information on debugger variable access.


\section{Stack Frames}
\cindex{stack frames} \cpsubindex{frames}{stack}

A stack frame is the run-time representation of a call to a function;
the frame stores the state that a function needs to remember what it is
doing.  Frames have:
\begin{itemize}

\item
Variables (\pxlref{debug-vars}), which are the values being operated
on, and

\item
Arguments to the call (which are really just particularly interesting
variables), and

\item
A current location (\pxlref{source-locations}), which is the place in
the program where the function was running when it stopped to call another
function, or because of an interrupt or error.
\end{itemize}


\subsection{Stack Motion}

These commands move to a new stack frame and print the name of the function
and the values of its arguments in the style of a Lisp function call:
\begin{Lentry}

\item[\code{up}]
Move up to the next higher frame.  More recent function calls are considered
to be higher on the stack.

\item[\code{down}]
Move down to the next lower frame.

\item[\code{top}]
Move to the highest frame.

\item[\code{bottom}]
Move to the lowest frame.

\item[\code{frame} [\textit{n}]]
Move to the frame with the specified number.  Prompts for the number if not
supplied.

% \key{S} [\var{function-name} [\var{n}]]
% 
% \item
% Search down the stack for function.  Prompts for the function name if not
% supplied.  Searches an optional number of times, but doesn't prompt for
% this number; enter it following the function.
% 
% \item[\key{R} [\var{function-name} [\var{n}]]]
% Search up the stack for function.  Prompts for the function name if not
% supplied.  Searches an optional number of times, but doesn't prompt for
% this number; enter it following the function.
\end{Lentry}


\subsection{How Arguments are Printed}

A frame is printed to look like a function call, but with the actual argument
values in the argument positions.  So the frame for this call in the source:

\begin{lisp}
(myfun (+ 3 4) 'a)
\end{lisp}

would look like this:

\begin{example}
(MYFUN 7 A)
\end{example}

All keyword and optional arguments are displayed with their actual
values; if the corresponding argument was not supplied, the value will
be the default.  So this call:

\begin{lisp}
(subseq "foo" 1)
\end{lisp}

would look like this:

\begin{example}
(SUBSEQ "foo" 1 3)
\end{example}

And this call:

\begin{lisp}
(string-upcase "test case")
\end{lisp}

would look like this:

\begin{example}
(STRING-UPCASE "test case" :START 0 :END NIL)
\end{example}

The arguments to a function call are displayed by accessing the argument
variables.  Although those variables are initialized to the actual argument
values, they can be set inside the function; in this case the new value will be
displayed.

\code{\amprest} arguments are handled somewhat differently.  The value of
the rest argument variable is displayed as the spread-out arguments to
the call, so:

\begin{lisp}
(format t "~A is a ~A." "This" 'test)
\end{lisp}

would look like this:

\begin{example}
(FORMAT T "~A is a ~A." "This" 'TEST)
\end{example}

Rest arguments cause an exception to the normal display of keyword
arguments in functions that have both \code{\amprest} and \code{\&key}
arguments.  In this case, the keyword argument variables are not
displayed at all; the rest arg is displayed instead.  So for these
functions, only the keywords actually supplied will be shown, and the
values displayed will be the argument values, not values of the
(possibly modified) variables.

If the variable for an argument is never referenced by the function, it will be
deleted.  The variable value is then unavailable, so the debugger prints
\code{\#\textless unused-arg\textgreater} instead of the value.  Similarly, if for any of a number of
reasons (described in more detail in section \ref{debug-vars}) the value of the
variable is unavailable or not known to be available, then
\code{\#\textless unavailable-arg\textgreater} will be printed instead of the argument value.

Printing of argument values is controlled by \code{*debug-print-level*} and
\varref{debug-print-length}.

\subsection{Function Names}
\cpsubindex{function}{names}
\cpsubindex{names}{function}

If a function is defined by \code{defun}, \code{labels}, or \code{flet}, then the
debugger will print the actual function name after the open parenthesis, like:

\begin{example}
(STRING-UPCASE "test case" :START 0 :END NIL)
((SETF AREF) \#\back{a} "for" 1)
\end{example}

Otherwise, the function name is a string, and will be printed in quotes:

\begin{example}
("DEFUN MYFUN" BAR)
("DEFMACRO DO" (DO ((I 0 (1+ I))) ((= I 13))) NIL)
("SETQ *GC-NOTIFY-BEFORE*")
\end{example}

This string name is derived from the \w{\code{def}\var{mumble}} form
that encloses or expanded into the lambda, or the outermost enclosing
form if there is no \w{\code{def}\var{mumble}}.

\subsection{Funny Frames}
\cindex{external entry points}
\cpsubindex{entry points}{external}
\cpsubindex{block compilation}{debugger implications}
\cpsubindex{external}{stack frame kind}
\cpsubindex{optional}{stack frame kind}
\cpsubindex{cleanup}{stack frame kind}

Sometimes the evaluator introduces new functions that are used to implement a
user function, but are not directly specified in the source.  The main place
this is done is for checking argument type and syntax.  Usually these functions
do their thing and then go away, and thus are not seen on the stack in the
debugger.  But when you get some sort of error during lambda-list processing,
you end up in the debugger on one of these funny frames.

These funny frames are flagged by printing ``\code{[}\var{keyword}\code{]}'' after the
parentheses.  For example, this call:

\begin{lisp}
(car 'a 'b)
\end{lisp}

will look like this:

\begin{example}
(CAR 2 A) [:EXTERNAL]
\end{example}

And this call:

\begin{lisp}
(string-upcase "test case" :end)
\end{lisp}

would look like this:

\begin{example}
("DEFUN STRING-UPCASE" "test case" 335544424 1) [:OPTIONAL]
\end{example}

As you can see, these frames have only a vague resemblance to the original
call.  Fortunately, the error message displayed when you enter the debugger
will usually tell you what problem is (in these cases, too many arguments
and odd keyword arguments.)  Also, if you go down the stack to the frame for
the calling function, you can display the original source (\pxlref{source-locations}.)

With recursive or block compiled functions
(\pxlref{block-compilation}), an \kwd{EXTERNAL} frame may appear
before the frame representing the first call to the recursive function
or entry to the compiled block. This is a consequence of the way the
compiler does block compilation: there is nothing odd with your
program. You will also see \kwd{CLEANUP} frames during the execution
of \code{unwind-protect} cleanup code. Note that inline expansion and
open-coding affect what frames are present in the debugger, see
sections \ref{debugger-policy} and \ref{open-coding}.


\subsection{Debug Tail Recursion}
\label{debug-tail-recursion}
\cindex{tail recursion}
\cpsubindex{recursion}{tail}

Both the compiler and the interpreter are ``properly tail recursive.''  If a
function call is in a tail-recursive position, the stack frame will be
deallocated {\em at the time of the call}, rather than after the call returns.
Consider this backtrace:
\begin{example}
(BAR ...) 
(FOO ...)
\end{example}
Because of tail recursion, it is not necessarily the case that
\code{FOO} directly called \code{BAR}.  It may be that \code{FOO} called
some other function \code{FOO2} which then called \code{BAR}
tail-recursively, as in this example:
\begin{example}
(defun foo ()
  ...
  (foo2 ...)
  ...)

(defun foo2 (...)
  ...
  (bar ...))

(defun bar (...)
  ...)
\end{example}

Usually the elimination of tail-recursive frames makes debugging more
pleasant, since theses frames are mostly uninformative.  If there is any
doubt about how one function called another, it can usually be
eliminated by finding the source location in the calling frame (section
\ref{source-locations}.)

The elimination of tail-recursive frames can be prevented by disabling
tail-recursion optimization, which happens when the \code{debug}
optimization quality is greater than \code{2}
(\pxlref{debugger-policy}.)

For a more thorough discussion of tail recursion, \pxlref{tail-recursion}.


\subsection{Unknown Locations and Interrupts}
\label{unknown-locations}
\cindex{unknown code locations}
\cpsubindex{locations}{unknown}
\cindex{interrupts}
\cpsubindex{errors}{run-time}

The debugger operates using special debugging information attached to
the compiled code.  This debug information tells the debugger what it
needs to know about the locations in the code where the debugger can be
invoked.  If the debugger somehow encounters a location not described in
the debug information, then it is said to be \var{unknown}.  If the code
location for a frame is unknown, then some variables may be
inaccessible, and the source location cannot be precisely displayed.

There are three reasons why a code location could be unknown:
\begin{itemize}

\item
There is inadequate debug information due to the value of the \code{debug}
optimization quality.  \xlref{debugger-policy}.

\item
The debugger was entered because of an interrupt such as \code{$\hat{ }C$}.

\item
A hardware error such as ``\code{bus error}'' occurred in code that was
compiled unsafely due to the value of the \code{safety} optimization
quality.  \xlref{optimize-declaration}.
\end{itemize}

In the last two cases, the values of argument variables are accessible,
but may be incorrect.  \xlref{debug-var-validity} for more details on
when variable values are accessible.

It is possible for an interrupt to happen when a function call or return is in
progress.  The debugger may then flame out with some obscure error or insist
that the bottom of the stack has been reached, when the real problem is that
the current stack frame can't be located.  If this happens, return from the
interrupt and try again.

When running interpreted code, all locations should be known.  However,
an interrupt might catch some subfunction of the interpreter at an
unknown location.  In this case, you should be able to go up the stack a
frame or two and reach an interpreted frame which can be debugged.


\section{Variable Access}
\label{debug-vars}
\cpsubindex{variables}{debugger access}
\cindex{debug variables}

There are three ways to access the current frame's local variables in the
debugger.  The simplest is to type the variable's name into the debugger's
read-eval-print loop.  The debugger will evaluate the variable reference as
though it had appeared inside that frame.

The debugger doesn't really understand lexical scoping; it has just one
namespace for all the variables in a function.  If a symbol is the name of
multiple variables in the same function, then the reference appears ambiguous,
even though lexical scoping specifies which value is visible at any given
source location.  If the scopes of the two variables are not nested, then the
debugger can resolve the ambiguity by observing that only one variable is
accessible.

When there are ambiguous variables, the evaluator assigns each one a
small integer identifier.  The \code{debug:var} function and the
\code{list-locals} command use this identifier to distinguish between
ambiguous variables:
\begin{Lentry}

\item[\code{list-locals} \mopt{\var{prefix}}]%%\hfill\\
This command prints the name and value of all variables in the current
frame whose name has the specified \var{prefix}.  \var{prefix} may be a
string or a symbol.  If no \var{prefix} is given, then all available
variables are printed.  If a variable has a potentially ambiguous name,
then the name is printed with a ``\code{\#}\var{identifier}'' suffix, where
\var{identifier} is the small integer used to make the name unique.
\end{Lentry}

\begin{defun}{debug:}{var}{\args{\var{name} \ampoptional{} \var{identifier}}}
  
  This function returns the value of the variable in the current frame
  with the specified \var{name}.  If supplied, \var{identifier}
  determines which value to return when there are ambiguous variables.
  
  When \var{name} is a symbol, it is interpreted as the symbol name of
  the variable, i.e. the package is significant.  If \var{name} is an
  uninterned symbol (gensym), then return the value of the uninterned
  variable with the same name.  If \var{name} is a string,
  \code{debug:var} interprets it as the prefix of a variable name, and
  must unambiguously complete to the name of a valid variable.
  
  This function is useful mainly for accessing the value of uninterned
  or ambiguous variables, since most variables can be evaluated
  directly.
\end{defun}


\subsection{Variable Value Availability}
\label{debug-var-validity}
\cindex{availability of debug variables}
\cindex{validity of debug variables}
\cindex{debug optimization quality}

The value of a variable may be unavailable to the debugger in portions of the
program where \clisp{} says that the variable is defined.  If a variable value is
not available, the debugger will not let you read or write that variable.  With
one exception, the debugger will never display an incorrect value for a
variable.  Rather than displaying incorrect values, the debugger tells you the
value is unavailable.

The one exception is this: if you interrupt (e.g., with \code{$\hat{ }C$}) or if there is
an unexpected hardware error such as ``\code{bus error}'' (which should only happen
in unsafe code), then the values displayed for arguments to the interrupted
frame might be incorrect.\footnote{Since the location of an interrupt or hardware
error will always be an unknown location (\pxlref{unknown-locations}),
non-argument variable values will never be available in the interrupted frame.}
This exception applies only to the interrupted frame: any frame farther down
the stack will be fine.

The value of a variable may be unavailable for these reasons:
\begin{itemize}

\item
The value of the \code{debug} optimization quality may have omitted debug
information needed to determine whether the variable is available.
Unless a variable is an argument, its value will only be available when
\code{debug} is at least \code{2}.

\item
The compiler did lifetime analysis and determined that the value was no longer
needed, even though its scope had not been exited.  Lifetime analysis is
inhibited when the \code{debug} optimization quality is \code{3}.

\item
The variable's name is an uninterned symbol (gensym).  To save space, the
compiler only dumps debug information about uninterned variables when the
\code{debug} optimization quality is \code{3}.

\item
The frame's location is unknown (\pxlref{unknown-locations}) because
the debugger was entered due to an interrupt or unexpected hardware error.
Under these conditions the values of arguments will be available, but might be
incorrect.  This is the exception above.

\item
The variable was optimized out of existence.  Variables with no reads are
always optimized away, even in the interpreter.  The degree to which the
compiler deletes variables will depend on the value of the \code{compile-speed}
optimization quality, but most source-level optimizations are done under all
compilation policies.
\end{itemize}


Since it is especially useful to be able to get the arguments to a function,
argument variables are treated specially when the \code{speed} optimization
quality is less than \code{3} and the \code{debug} quality is at least \code{1}.
With this compilation policy, the values of argument variables are almost
always available everywhere in the function, even at unknown locations.  For
non-argument variables, \code{debug} must be at least \code{2} for values to be
available, and even then, values are only available at known locations.


\subsection{Note On Lexical Variable Access}
\cpsubindex{evaluation}{debugger}
 
When the debugger command loop establishes variable bindings for available
variables, these variable bindings have lexical scope and dynamic
extent.\footnote{The variable bindings are actually created using the \clisp{}
\code{symbol-macrolet} special form.}  You can close over them, but such closures
can't be used as upward funargs.

You can also set local variables using \code{setq}, but if the variable was closed
over in the original source and never set, then setting the variable in the
debugger may not change the value in all the functions the variable is defined
in.  Another risk of setting variables is that you may assign a value of a type
that the compiler proved the variable could never take on.  This may result in
bad things happening.


\section{Source Location Printing}
\label{source-locations}
\cpsubindex{source location printing}{debugger}

One of \cmucl{}'s unique capabilities is source level debugging of compiled
code.  These commands display the source location for the current frame:
\begin{Lentry}

\item[\code{source} \mopt{\var{context}}]%%\hfill\\
This command displays the file that the current frame's function was defined
from (if it was defined from a file), and then the source form responsible for
generating the code that the current frame was executing.  If \var{context} is
specified, then it is an integer specifying the number of enclosing levels of
list structure to print.

\item[\code{vsource} \mopt{\var{context}}]%%\hfill\\
This command is identical to \code{source}, except that it uses the
global values of \code{*print-level*} and \code{*print-length*} instead
of the debugger printing control variables \code{*debug-print-level*}
and \code{*debug-print-length*}.
\end{Lentry}

The source form for a location in the code is the innermost list present
in the original source that encloses the form responsible for generating
that code.  If the actual source form is not a list, then some enclosing
list will be printed.  For example, if the source form was a reference
to the variable \code{*some-random-special*}, then the innermost
enclosing evaluated form will be printed.  Here are some possible
enclosing forms:
\begin{example}
(let ((a *some-random-special*))
  ...)

(+ *some-random-special* ...)
\end{example}

If the code at a location was generated from the expansion of a macro or a
source-level compiler optimization, then the form in the original source that
expanded into that code will be printed.  Suppose the file
\file{/usr/me/mystuff.lisp} looked like this:
\begin{example}
(defmacro mymac ()
  '(myfun))

(defun foo ()
  (mymac)
  ...)
\end{example}
If \code{foo} has called \code{myfun}, and is waiting for it to return, then the
\code{source} command would print:
\begin{example}
; File: /usr/me/mystuff.lisp

(MYMAC)
\end{example}
Note that the macro use was printed, not the actual function call form,
\code{(myfun)}.

If enclosing source is printed by giving an argument to \code{source} or
\code{vsource}, then the actual source form is marked by wrapping it in a list
whose first element is \code{\#:***HERE***}.  In the previous example, 
\w{\code{source 1}} would print:
\begin{example}
; File: /usr/me/mystuff.lisp

(DEFUN FOO ()
  (#:***HERE***
   (MYMAC))
  ...)
\end{example}


\subsection{How the Source is Found}

If the code was defined from \llisp{} by \code{compile} or
\code{eval}, then the source can always be reliably located.  If the
code was defined from a \code{fasl} file created by
\findexed{compile-file}, then the debugger gets the source forms it
prints by reading them from the original source file.  This is a
potential problem, since the source file might have moved or changed
since the time it was compiled.

The source file is opened using the \code{truename} of the source file
pathname originally given to the compiler.  This is an absolute pathname
with all logical names and symbolic links expanded.  If the file can't
be located using this name, then the debugger gives up and signals an
error.

If the source file can be found, but has been modified since the time it was
compiled, the debugger prints this warning:
\begin{example}
; File has been modified since compilation:
;   \var{filename}
; Using form offset instead of character position.
\end{example}
where \var{filename} is the name of the source file.  It then proceeds using a
robust but not foolproof heuristic for locating the source.  This heuristic
works if:
\begin{itemize}

\item
No top-level forms before the top-level form containing the source have been
added or deleted, and

\item
The top-level form containing the source has not been modified much.  (More
precisely, none of the list forms beginning before the source form have been
added or deleted.)
\end{itemize}

If the heuristic doesn't work, the displayed source will be wrong, but will
probably be near the actual source.  If the ``shape'' of the top-level form in
the source file is too different from the original form, then an error will be
signaled.  When the heuristic is used, the the source location commands are
noticeably slowed.

Source location printing can also be confused if (after the source was
compiled) a read-macro you used in the code was redefined to expand into
something different, or if a read-macro ever returns the same \code{eq}
list twice.  If you don't define read macros and don't use \code{\#\#} in
perverted ways, you don't need to worry about this.


\subsection{Source Location Availability}

\cindex{debug optimization quality}
Source location information is only available when the \code{debug}
optimization quality is at least \code{2}.  If source location information is
unavailable, the source commands will give an error message.

If source location information is available, but the source location is
unknown because of an interrupt or unexpected hardware error
(\pxlref{unknown-locations}), then the command will print:

\begin{example}
Unknown location: using block start.
\end{example}

and then proceed to print the source location for the start of the
{\em basic block} enclosing the code location.
\cpsubindex{block}{basic} \cpsubindex{block}{start location} 
It's a bit complicated to explain exactly what a basic block is, but
here are some properties of the block start location:

\begin{itemize}
  
\item The block start location may be the same as the true location.
  
\item The block start location will never be later in the the
  program's flow of control than the true location.
  
\item No conditional control structures (such as \code{if},
  \code{cond}, \code{or}) will intervene between the block start and
  the true location (but note that some conditionals present in the
  original source could be optimized away.)  Function calls {\em do not}
  end basic blocks.
  
\item The head of a loop will be the start of a block.
  
\item The programming language concept of ``block structure'' and the
  \clisp{} \code{block} special form are totally unrelated to the
  compiler's basic block.
\end{itemize}

In other words, the true location lies between the printed location and the
next conditional (but watch out because the compiler may have changed the
program on you.)


\section{Compiler Policy Control}
\label{debugger-policy}
\cpsubindex{policy}{debugger}
\cindex{debug optimization quality}
\cindex{optimize declaration}

The compilation policy specified by \code{optimize} declarations affects the
behavior seen in the debugger.  The \code{debug} quality directly affects the
debugger by controlling the amount of debugger information dumped.  Other
optimization qualities have indirect but observable effects due to changes in
the way compilation is done.

Unlike the other optimization qualities (which are compared in relative value
to evaluate tradeoffs), the \code{debug} optimization quality is directly
translated to a level of debug information.  This absolute interpretation
allows the user to count on a particular amount of debug information being
available even when the values of the other qualities are changed during
compilation.  These are the levels of debug information that correspond to the
values of the \code{debug} quality:
\begin{Lentry}

\item[\code{0}]
Only the function name and enough information to allow the stack to
be parsed.

\item[\code{\w{$>$ 0}}]
Any level greater than \code{0} gives level \code{0} plus all
argument variables.  Values will only be accessible if the argument
variable is never set and
\code{speed} is not \code{3}.  \cmucl{} allows any real value for optimization
qualities.  It may be useful to specify \code{0.5} to get backtrace argument
display without argument documentation.

\item[\code{1}] Level \code{1} provides argument documentation
(printed arglists) and derived argument/result type information.
This makes \findexed{describe} more informative, and allows the
compiler to do compile-time argument count and type checking for any
calls compiled at run-time.

\item[\code{2}]
Level \code{1} plus all interned local variables, source location
information, and lifetime information that tells the debugger when arguments
are available (even when \code{speed} is \code{3} or the argument is set.)  This is
the default.

\item[\code{\w{$>$ 2}}]
Any level greater than \code{2} gives level \code{2} and in addition
disables tail-call optimization, so that the backtrace will contain
frames for all invoked functions, even those in tail positions.

\item[\code{3}]
Level \code{2} plus all uninterned variables.  In addition, lifetime
analysis is disabled (even when \code{speed} is \code{3}), ensuring
that all variable values are available at any known location within
the scope of the binding.  This has a speed penalty in addition to the
obvious space penalty. 
\end{Lentry}

As you can see, if the \code{speed} quality is \code{3}, debugger performance is
degraded.  This effect comes from the elimination of argument variable
special-casing (\pxlref{debug-var-validity}.)  Some degree of
speed/debuggability tradeoff is unavoidable, but the effect is not too drastic
when \code{debug} is at least \code{2}.

\cindex{inline expansion}
\cindex{semi-inline expansion}
In addition to \code{inline} and \code{notinline} declarations, the relative values
of the \code{speed} and \code{space} qualities also change whether functions are
inline expanded (\pxlref{inline-expansion}.)  If a function is inline
expanded, then there will be no frame to represent the call, and the arguments
will be treated like any other local variable.  Functions may also be
``semi-inline'', in which case there is a frame to represent the call, but the
call is to an optimized local version of the function, not to the original
function.


\section{Exiting Commands}

These commands get you out of the debugger.

\begin{Lentry}

\item[\code{quit}]
Throw to top level.

\item[\code{restart} \mopt{\var{n}}]%%\hfill\\
Invokes the \var{n}th restart case as displayed by the \code{error}
command.  If \var{n} is not specified, the available restart cases are
reported.

\item[\code{go}]
Calls \code{continue} on the condition given to \code{debug}.  If there is no
restart case named \var{continue}, then an error is signaled.

\item[\code{abort}]
Calls \code{abort} on the condition given to \code{debug}.  This is
useful for popping debug command loop levels or aborting to top level,
as the case may be.

% (\code{debug:debug-return} \var{expression} \mopt{\var{frame}})
% 
% \item
% From the current or specified frame, return the result of evaluating
% expression.  If multiple values are expected, then this function should be
% called for multiple values.
\end{Lentry}


\section{Information Commands}

Most of these commands print information about the current frame or
function, but a few show general information.

\begin{Lentry}

\item[\code{help}, \code{?}]
Displays a synopsis of debugger commands.

\item[\code{describe}]
Calls \code{describe} on the current function, displays number of local
variables, and indicates whether the function is compiled or interpreted.

\item[\code{print}]
Displays the current function call as it would be displayed by moving to
this frame.

\item[\code{vprint} (or \code{pp}) \mopt{\var{verbosity}}]%%\hfill\\
Displays the current function call using \code{*print-level*} and
\code{*print-length*} instead of \code{*debug-print-level*} and
\code{*debug-print-length*}.  \var{verbosity} is a small integer
(default 2) that controls other dimensions of verbosity.

\item[\code{error}]
Prints the condition given to \code{invoke-debugger} and the active
proceed cases.

\item[\code{backtrace} \mopt{\var{n}}]\hfill\\
Displays all the frames from the current to the bottom.  Only shows
\var{n} frames if specified.  The printing is controlled by
\code{*debug-print-level*} and \code{*debug-print-length*}.

% (\code{debug:debug-function} \mopt{\var{n}})
% 
% \item
% Returns the function from the current or specified frame.
% 
% \item[(\code{debug:function-name} \mopt{\var{n}])]
% Returns the function name from the current or specified frame.
% 
% \item[(\code{debug:pc} \mopt{\var{frame}})]
% Returns the index of the instruction for the function in the current or
% specified frame.  This is useful in conjunction with \code{disassemble}.
% The pc returned points to the instruction after the one that was fatal.
\end{Lentry}


\section{Breakpoint Commands}\cindex{breakpoints}

\cmucl{} supports setting of breakpoints inside compiled functions and
stepping of compiled code.  Breakpoints can only be set at at known
locations (\pxlref{unknown-locations}), so these commands are largely
useless unless the \code{debug} optimize quality is at least \code{2}
(\pxlref{debugger-policy}).  These commands manipulate breakpoints:
\begin{Lentry}
\item[\code{breakpoint} \var{location} \mstar{\var{option} \var{value}}]
%%\hfill\\
Set a breakpoint in some function.  \var{location} may be an integer
code location number (as displayed by \code{list-locations}) or a
keyword.  The keyword can be used to indicate setting a breakpoint at
the function start (\kwd{start}, \kwd{s}) or function end
(\kwd{end}, \kwd{e}).  The \code{breakpoint} command has
\kwd{condition}, \kwd{break}, \kwd{print} and \kwd{function}
options which work similarly to the \code{trace} options.

\item[\code{list-locations} (or \code{ll}) \mopt{\var{function}}]%%\hfill\\
List all the code locations in the current frame's function, or in
\var{function} if it is supplied.  The display format is the code
location number, a colon and then the source form for that location:
\begin{example}
3: (1- N)
\end{example}
If consecutive locations have the same source, then a numeric range like
\code{3-5:} will be printed.  For example, a default function call has a
known location both immediately before and after the call, which would
result in two code locations with the same source.  The listed function
becomes the new default function for breakpoint setting (via the
\code{breakpoint}) command.

\item[\code{list-breakpoints} (or \code{lb})]%%\hfill\\
List all currently active breakpoints with their breakpoint number.

\item[\code{delete-breakpoint} (or \code{db}) \mopt{\var{number}}]%%\hfill\\
Delete a breakpoint specified by its breakpoint number.  If no number is
specified, delete all breakpoints.

\item[\code{step}]%%\hfill\\
Step to the next possible breakpoint location in the current function.
This always steps over function calls, instead of stepping into them
\end{Lentry}


\subsection{Breakpoint Example}

Consider this definition of the factorial function:

\begin{lisp}
(defun ! (n)
  (if (zerop n)
      1
      (* n (! (1- n)))))
\end{lisp}

This debugger session demonstrates the use of breakpoints:

\begin{example}
common-lisp-user> (break) ; Invoke debugger

Break

Restarts:
  0: [CONTINUE] Return from BREAK.
  1: [ABORT   ] Return to Top-Level.

Debug  (type H for help)

(INTERACTIVE-EVAL (BREAK))
0] ll #'!
0: #'(LAMBDA (N) (BLOCK ! (IF # 1 #)))
1: (ZEROP N)
2: (* N (! (1- N)))
3: (1- N)
4: (! (1- N))
5: (* N (! (1- N)))
6: #'(LAMBDA (N) (BLOCK ! (IF # 1 #)))
0] br 2
(* N (! (1- N)))
1: 2 in !
Added.
0] q

common-lisp-user> (! 10) ; Call the function

*Breakpoint hit*

Restarts:
  0: [CONTINUE] Return from BREAK.
  1: [ABORT   ] Return to Top-Level.

Debug  (type H for help)

(! 10) ; We are now in first call (arg 10) before the multiply
Source: (* N (! (1- N)))
3] st

*Step*

(! 10) ; We have finished evaluation of (1- n)
Source: (1- N)
3] st

*Breakpoint hit*

Restarts:
  0: [CONTINUE] Return from BREAK.
  1: [ABORT   ] Return to Top-Level.

Debug  (type H for help)

(! 9) ; We hit the breakpoint in the recursive call
Source: (* N (! (1- N)))
3] 
\end{example}


\section{Function Tracing}
\cindex{tracing}
\cpsubindex{function}{tracing}

The tracer causes selected functions to print their arguments and
their results whenever they are called.  Options allow conditional
printing of the trace information and conditional breakpoints on
function entry or exit.

\begin{defmac}{}{trace}{%
    \args{\mstar{option global-value} \mstar{name \mstar{option
          value}}}}
  
  \code{trace} is a debugging tool that prints information when
  specified functions are called.  In its simplest form:
  \begin{example}
    (trace \var{name-1} \var{name-2} ...)
  \end{example}
  \code{trace} causes a printout on \vindexed{trace-output} each time
  that one of the named functions is entered or returns (the
  \var{names} are not evaluated.)  Trace output is indented according
  to the number of pending traced calls, and this trace depth is
  printed at the beginning of each line of output.  Printing verbosity
  of arguments and return values is controlled by
  \vindexed{debug-print-level} and \vindexed{debug-print-length}.

  Local functions defined by \code{flet} and \code{labels} can be
  traced using the syntax \code{(flet f f1 f2 ...)} or \code{(labels f
    f1 f2 ...)} where \code{f} is the \code{flet} or \code{labels}
  function we want to trace and \code{f1}, \code{f2}, are the
  functions containing the local function \code{f}.
  Invidiual methods can also be traced using the syntax \code{(method
    <name> <qualifiers> <specializers>)}.
  See~\ref{sec:method-tracing} for more information.

  If no \var{names} or \var{options} are are given, \code{trace}
  returns the list of all currently traced functions,
  \code{*traced-function-list*}.
  
  Trace options can cause the normal printout to be suppressed, or
  cause extra information to be printed.  Each option is a pair of an
  option keyword and a value form.  Options may be interspersed with
  function names.  Options only affect tracing of the function whose
  name they appear immediately after.  Global options are specified
  before the first name, and affect all functions traced by a given
  use of \code{trace}.  If an already traced function is traced again,
  any new options replace the old options.  The following options are
  defined:
  \begin{Lentry}
  \item[\kwd{condition} \var{form}, \kwd{condition-after} \var{form},
    \kwd{condition-all} \var{form}] If \kwd{condition} is specified,
    then \code{trace} does nothing unless \var{form} evaluates to true
    at the time of the call.  \kwd{condition-after} is similar, but
    suppresses the initial printout, and is tested when the function
    returns.  \kwd{condition-all} tries both before and after.
    
  \item[\kwd{wherein} \var{names}] If specified, \var{names} is a
    function name or list of names.  \code{trace} does nothing unless
    a call to one of those functions encloses the call to this
    function (i.e. it would appear in a backtrace.)  Anonymous
    functions have string names like \code{"DEFUN FOO"}.  Individual
    methods can also be traced.  See section~\ref{sec:method-tracing}.

  \item[\kwd{wherein-only} \var{names}] If specified, this is just
    like \kwd{wherein}, but trace produces output only if the
    immediate caller of the traced function is one of the functions
    listed in \var{names}.
    
  \item[\kwd{break} \var{form}, \kwd{break-after} \var{form},
    \kwd{break-all} \var{form}] If specified, and \var{form} evaluates
    to true, then the debugger is invoked at the start of the
    function, at the end of the function, or both, according to the
    respective option.
    
  \item[\kwd{print} \var{form}, \kwd{print-after} \var{form},
    \kwd{print-all} \var{form}] In addition to the usual printout, the
    result of evaluating \var{form} is printed at the start of the
    function, at the end of the function, or both, according to the
    respective option.  Multiple print options cause multiple values
    to be printed.
    
  \item[\kwd{function} \var{function-form}] This is a not really an
    option, but rather another way of specifying what function to
    trace.  The \var{function-form} is evaluated immediately, and the
    resulting function is traced.
    
  \item[\kwd{encapsulate \mgroup{:default | t | nil}}] In \cmucl,
    tracing can be done either by temporarily redefining the function
    name (encapsulation), or using breakpoints.  When breakpoints are
    used, the function object itself is destructively modified to
    cause the tracing action.  The advantage of using breakpoints is
    that tracing works even when the function is anonymously called
    via \code{funcall}.
  
    When \kwd{encapsulate} is true, tracing is done via encapsulation.
    \kwd{default} is the default, and means to use encapsulation for
    interpreted functions and funcallable instances, breakpoints
    otherwise.  When encapsulation is used, forms are {\it not}
    evaluated in the function's lexical environment, but
    \code{debug:arg} can still be used.

    Note that if you trace using \kwd{encapsulate}, you will
    only get a trace or breakpoint at the outermost call to the traced
    function, not on recursive calls.

  \end{Lentry}

  In the case of functions where the known return convention is used
  to optimize, encapsulation may be necessary in order to make
  tracing work at all.  The symptom of this occurring is an error
  stating
  \begin{example}
    Error in function \var{foo}: :FUNCTION-END breakpoints are
    currently unsupported for the known return convention.
  \end{example}
  in such cases we recommend using \code{(trace \var{foo} :encapsulate
    t)}
  
  \cpsubindex{tracing}{errors}
  \cpsubindex{breakpoints}{errors}
  \cpsubindex{errors}{breakpoints}
  \cindex{function-end breakpoints}
  \cpsubindex{breakpoints}{function-end}
    

  
  \kwd{condition}, \kwd{break} and \kwd{print} forms are evaluated in
  the lexical environment of the called function; \code{debug:var} and
  \code{debug:arg} can be used.  The \code{-after} and \code{-all}
  forms are evaluated in the null environment.
\end{defmac}

\begin{defmac}{}{untrace}{ \args{\amprest{} \var{function-names}}}
  
  This macro turns off tracing for the specified functions, and
  removes their names from \code{*traced-function-list*}.  If no
  \var{function-names} are given, then all currently traced functions
  are untraced.
\end{defmac}

\begin{defvar}{extensions:}{traced-function-list}
  
  A list of function names maintained and used by \code{trace},
  \code{untrace}, and \code{untrace-all}.  This list should contain
  the names of all functions currently being traced.
\end{defvar}

\begin{defvar}{extensions:}{max-trace-indentation}
  
  The maximum number of spaces which should be used to indent trace
  printout.  This variable is initially set to 40.
\end{defvar}

\begin{defvar}{debug:}{trace-encapsulate-package-names}
  
  A list of package names.  Functions from these packages are traced
  using encapsulation instead of function-end breakpoints.  This list
  should at least include those packages containing functions used
  directly or indirectly in the implementation of \code{trace}.
\end{defvar}


\subsection{Encapsulation Functions}
\cindex{encapsulation}
\cindex{advising}

The encapsulation functions provide a mechanism for intercepting the
arguments and results of a function.  \code{encapsulate} changes the
function definition of a symbol, and saves it so that it can be
restored later.  The new definition normally calls the original
definition.  The \clisp{} \findexed{fdefinition} function always returns
the original definition, stripping off any encapsulation.

The original definition of the symbol can be restored at any time by
the \code{unencapsulate} function.  \code{encapsulate} and \code{unencapsulate}
allow a symbol to be multiply encapsulated in such a way that different
encapsulations can be completely transparent to each other.

Each encapsulation has a type which may be an arbitrary lisp object.
If a symbol has several encapsulations of different types, then any
one of them can be removed without affecting more recent ones.
A symbol may have more than one encapsulation of the same type, but
only the most recent one can be undone.

\begin{defun}{extensions:}{encapsulate}{%
    \args{\var{symbol} \var{type} \var{body}}}
  
  Saves the current definition of \var{symbol}, and replaces it with a
  function which returns the result of evaluating the form,
  \var{body}.  \var{Type} is an arbitrary lisp object which is the
  type of encapsulation.
  
  When the new function is called, the following variables are bound
  for the evaluation of \var{body}:
  \begin{Lentry}
    
  \item[\code{extensions:argument-list}] A list of the arguments to
    the function.
    
  \item[\code{extensions:basic-definition}] The unencapsulated
    definition of the function.
  \end{Lentry}
  The unencapsulated definition may be called with the original
  arguments by including the form
  \begin{lisp}
    (apply extensions:basic-definition extensions:argument-list)
  \end{lisp}

  \code{encapsulate} always returns \var{symbol}.
\end{defun}

\begin{defun}{extensions:}{unencapsulate}{\args{\var{symbol} \var{type}}}
  
  Undoes \var{symbol}'s most recent encapsulation of type \var{type}.
  \var{Type} is compared with \code{eq}.  Encapsulations of other
  types are left in place.
\end{defun}

\begin{defun}{extensions:}{encapsulated-p}{%
    \args{\var{symbol} \var{type}}}
  
  Returns \true{} if \var{symbol} has an encapsulation of type
  \var{type}.  Returns \nil{} otherwise.  \var{type} is compared with
  \code{eq}.
\end{defun}

% section{The Single Stepper}
% 
% \begin{defmac}{}{step}{ \args{\var{form}}}
%   
%   Evaluates form with single stepping enabled or if \var{form} is
%   \code{T}, enables stepping until explicitly disabled.  Stepping can
%   be disabled by quitting to the lisp top level, or by evaluating the
%   form \w{\code{(step ())}}.
%   
%   While stepping is enabled, every call to eval will prompt the user
%   for a single character command.  The prompt is the form which is
%   about to be \code{eval}ed.  It is printed with \code{*print-level*}
%   and \code{*print-length*} bound to \code{*step-print-level*} and
%   \code{*step-print-length*}.  All interaction is done through the
%   stream \code{*query-io*}.  Because of this, the stepper can not be
%   used in Hemlock eval mode.  When connected to a slave Lisp, the
%   stepper can be used from Hemlock.
%   
%   The commands are:
%   \begin{Lentry}
%   
%   \item[\key{n} (next)] Evaluate the expression with stepping still
%     enabled.
%   
%   \item[\key{s} (skip)] Evaluate the expression with stepping
%     disabled.
%   
%   \item[\key{q} (quit)] Evaluate the expression, but disable all
%     further stepping inside the current call to \code{step}.
%   
%   \item[\key{p} (print)] Print current form.  (does not use
%     \code{*step-print-level*} or \code{*step-print-length*}.)
%   
%   \item[\key{b} (break)] Enter break loop, and then prompt for the
%     command again when the break loop returns.
%   
%   \item[\key{e} (eval)] Prompt for and evaluate an arbitrary
%     expression.  The expression is evaluated with stepping disabled.
%   
%   \item[\key{?} (help)] Prints a brief list of the commands.
%   
%   \item[\key{r} (return)] Prompt for an arbitrary value to return as
%     result of the current call to eval.
%   
%   \item[\key{g}] Throw to top level.
%   \end{Lentry}
% \end{defmac}
% 
% \begin{defvar}{extensions:}{step-print-level}
%   \defvarx[extensions:]{step-print-length}
%   
%   \code{*print-level*} and \code{*print-length*} are bound to these
%   values while printing the current form.  \code{*step-print-level*}
%   and \code{*step-print-length*} are initially bound to 4 and 5,
%   respectively.
% \end{defvar}
% 
% \begin{defvar}{extensions:}{max-step-indentation}
%   
%   Step indents the prompts to highlight the nesting of the evaluation.
%   This variable contains the maximum number of spaces to use for
%   indenting.  Initially set to 40.
% \end{defvar}


\section{Specials}
These are the special variables that control the debugger action.

\begin{defvar}{debug:}{debug-print-level}
  \defvarx[debug:]{debug-print-length}
  
  \code{*print-level*} and \code{*print-length*} are bound to these
  values during the execution of some debug commands.  When evaluating
  arbitrary expressions in the debugger, the normal values of
  \code{*print-level*} and \code{*print-length*} are in effect.  These
  variables are initially set to 3 and 5, respectively.
\end{defvar}

\chapter{The Compiler}

\section{Compiler Introduction}

This chapter contains information about the compiler that every \cmucl{} user
should be familiar with.  Chapter \ref{advanced-compiler} goes into greater
depth, describing ways to use more advanced features.

The \cmucl{} compiler (also known as \python{}, not to be confused
with the programming language of the same name) has many features
that are seldom or never supported by conventional \llisp{}
compilers:

\begin{itemize} 
\item Source level debugging of compiled code (see chapter
  \ref{debugger}.)
  
\item Type error compiler warnings for type errors detectable at
  compile time.
  
\item Compiler error messages that provide a good indication of where
  the error appeared in the source.
  
\item Full run-time checking of all potential type errors, with
  optimization of type checks to minimize the cost.
  
\item Scheme-like features such as proper tail recursion and extensive
  source-level optimization.
  
\item Advanced tuning and optimization features such as comprehensive
  efficiency notes, flow analysis, and untagged number representations
  (see chapter \ref{advanced-compiler}.)
\end{itemize}


\section{Calling the Compiler}
\cindex{compiling}

Functions may be compiled using \code{compile}, \code{compile-file}, or 
\code{compile-from-stream}.  

\begin{defun}{}{compile}{ \args{\var{name} \ampoptional{} \var{definition}}}
  
  This function compiles the function whose name is \var{name}.  If
  \var{name} is \false, the compiled function object is returned.  If
  \var{definition} is supplied, it should be a lambda expression that
  is to be compiled and then placed in the function cell of
  \var{name}.  As per the proposed X3J13 cleanup
  ``compile-argument-problems'', \var{definition} may also be an
  interpreted function.
  
  The return values are as per the proposed X3J13 cleanup
  ``compiler-diagnostics''.  The first value is the function name or
  function object.  The second value is \false{} if no compiler
  diagnostics were issued, and \true{} otherwise.  The third value is
  \false{} if no compiler diagnostics other than style warnings were
  issued.  A non-\false{} value indicates that there were ``serious''
  compiler diagnostics issued, or that other conditions of type
  \tindexed{error} or \tindexed{warning} (but not
  \tindexed{style-warning}) were signaled during compilation.
\end{defun}


\begin{defun}{}{compile-file}{
    \args{\var{input-pathname}
      \keys{\kwd{output-file} \kwd{error-file} \kwd{trace-file}}
      \morekeys{\kwd{error-output} \kwd{verbose} \kwd{print} \kwd{progress}}
      \yetmorekeys{\kwd{load} \kwd{block-compile} \kwd{entry-points}}
      \yetmorekeys{\kwd{byte-compile}}}}
  
  The \cmucl{} \code{compile-file} is extended through the addition of
  several new keywords and an additional interpretation of
  \var{input-pathname}:
  \begin{Lentry}
    
  \item[\var{input-pathname}] If this argument is a list of input
    files, rather than a single input pathname, then all the source
    files are compiled into a single object file.  In this case, the
    name of the first file is used to determine the default output
    file names.  This is especially useful in combination with
    \var{block-compile}.
    
  \item[\kwd{output-file}] This argument specifies the name of the
    output file.  \true{} gives the default name, \false{} suppresses
    the output file.
    
  \item[\kwd{error-file}] A listing of all the error output is
    directed to this file.  If there are no errors, then no error file
    is produced (and any existing error file is deleted.)  \true{}
    gives \w{"\var{name}\code{.err}"} (the default), and \false{}
    suppresses the output file.
    
  \item[\kwd{error-output}] If \true{} (the default), then error
    output is sent to \code{*error-output*}.  If a stream, then output
    is sent to that stream instead.  If \false, then error output is
    suppressed.  Note that this error output is in addition to (but
    the same as) the output placed in the \var{error-file}.
    
  \item[\kwd{verbose}] If \true{} (the default), then the compiler
    prints to error output at the start and end of compilation of each
    file.  See \varref{compile-verbose}.
    
  \item[\kwd{print}] If \true{} (the default), then the compiler
    prints to error output when each function is compiled.  See
    \varref{compile-print}.
    
  \item[\kwd{progress}] If \true{} (default \false{}), then the
    compiler prints to error output progress information about the
    phases of compilation of each function.  This is a \cmucl{} extension
    that is useful mainly in large block compilations.  See
    \varref{compile-progress}.
    
  \item[\kwd{trace-file}] If \true{}, several of the intermediate
    representations (including annotated assembly code) are dumped out
    to this file.  \true{} gives \w{"\var{name}\code{.trace}"}.  Trace
    output is off by default.  \xlref{trace-files}.
    
  \item[\kwd{load}] If \true{}, load the resulting output file.
    
  \item[\kwd{block-compile}] Controls the compile-time resolution of
    function calls.  By default, only self-recursive calls are
    resolved, unless an \code{ext:block-start} declaration appears in
    the source file.  \xlref{compile-file-block}.
    
  \item[\kwd{entry-points}] If non-null, then this is a list of the
    names of all functions in the file that should have global
    definitions installed (because they are referenced in other
    files.)  \xlref{compile-file-block}.
    
  \item[\kwd{byte-compile}] If \true{}, compiling to a compact
    interpreted byte code is enabled.  Possible values are \true{},
    \false{}, and \kwd{maybe} (the default.)  See
    \varref{byte-compile-default} and \xlref{byte-compile}.
  \end{Lentry}
  
  The return values are as per the proposed X3J13 cleanup
  ``compiler-diagnostics''.  The first value from \code{compile-file}
  is the truename of the output file, or \false{} if the file could
  not be created.  The interpretation of the second and third values
  is described above for \code{compile}.
\end{defun}

\begin{defvar}{}{compile-verbose}
  \defvarx{compile-print}
  \defvarx{compile-progress}
  
  These variables determine the default values for the \kwd{verbose},
  \kwd{print} and \kwd{progress} arguments to \code{compile-file}.
\end{defvar}

\begin{defun}{extensions:}{compile-from-stream}{%
    \args{\var{input-stream}
      \keys{\kwd{error-stream}}
      \morekeys{\kwd{trace-stream}}
      \yetmorekeys{\kwd{block-compile} \kwd{entry-points}}
      \yetmorekeys{\kwd{byte-compile}}}}
  
  This function is similar to \code{compile-file}, but it takes all
  its arguments as streams.  It reads \llisp{} code from
  \var{input-stream} until end of file is reached, compiling into the
  current environment.  This function returns the same two values as
  the last two values of \code{compile}.  No output files are
  produced.
\end{defun}


\section{Compilation Units}
\cpsubindex{compilation}{units}

\cmucl{} supports the \code{with-compilation-unit} macro added to the
language by the X3J13 ``with-compilation-unit'' compiler cleanup
issue.  This provides a mechanism for eliminating spurious undefined
warnings when there are forward references across files, and also
provides a standard way to access compiler extensions.

\begin{defmac}{}{with-compilation-unit}{%
    \args{(\mstar{\var{key} \var{value}}) \mstar{\var{form}}}}
  
  This macro evaluates the \var{forms} in an environment that causes
  warnings for undefined variables, functions and types to be delayed
  until all the forms have been evaluated.  Each keyword \var{value}
  is an evaluated form.  These keyword options are recognized:
  \begin{Lentry}
  
  \item[\kwd{override}] If uses of \code{with-compilation-unit} are
    dynamically nested, the outermost use will take precedence,
    suppressing printing of undefined warnings by inner uses.
    However, when the \code{override} option is true this shadowing is
    inhibited; an inner use will print summary warnings for the
    compilations within the inner scope.
  
  \item[\kwd{optimize}] This is a \cmucl{} extension that specifies of the
    ``global'' compilation policy for the dynamic extent of the body.
    The argument should evaluate to an \code{optimize} declare form,
    like:
    \begin{lisp}
      (optimize (speed 3) (safety 0))
    \end{lisp}
    \xlref{optimize-declaration}
  
  \item[\kwd{optimize-interface}] Similar to \kwd{optimize}, but
    specifies the compilation policy for function interfaces (argument
    count and type checking) for the dynamic extent of the body.
    \xlref{optimize-interface-declaration}.
  
  \item[\kwd{context-declarations}] This is a \cmucl{} extension that
    pattern-matches on function names, automatically splicing in any
    appropriate declarations at the head of the function definition.
    \xlref{context-declarations}.
  \end{Lentry}
\end{defmac}


\subsection{Undefined Warnings}

\cindex{undefined warnings}
Warnings about undefined variables, functions and types are delayed until the
end of the current compilation unit.  The compiler entry functions
(\code{compile}, etc.) implicitly use \code{with-compilation-unit}, so undefined
warnings will be printed at the end of the compilation unless there is an
enclosing \code{with-compilation-unit}.  In order the gain the benefit of this
mechanism, you should wrap a single \code{with-compilation-unit} around the calls
to \code{compile-file}, i.e.:
\begin{lisp}
(with-compilation-unit ()
  (compile-file "file1")
  (compile-file "file2")
  ...)
\end{lisp}

Unlike for functions and types, undefined warnings for variables are
not suppressed when a definition (e.g. \code{defvar}) appears after
the reference (but in the same compilation unit.)  This is because
doing special declarations out of order just doesn't
work\dash{}although early references will be compiled as special,
bindings will be done lexically.

Undefined warnings are printed with full source context
(\pxlref{error-messages}), which tremendously simplifies the problem
of finding undefined references that resulted from macroexpansion.
After printing detailed information about the undefined uses of each
name, \code{with-compilation-unit} also prints summary listings of the
names of all the undefined functions, types and variables.

\begin{defvar}{}{undefined-warning-limit}
  
  This variable controls the number of undefined warnings for each
  distinct name that are printed with full source context when the
  compilation unit ends.  If there are more undefined references than
  this, then they are condensed into a single warning:
  \begin{example}
    Warning: \var{count} more uses of undefined function \var{name}.
  \end{example}
  When the value is \code{0}, then the undefined warnings are not
  broken down by name at all: only the summary listing of undefined
  names is printed.
\end{defvar}


\section{Interpreting Error Messages}
\label{error-messages}
\cpsubindex{error messages}{compiler}
\cindex{compiler error messages}

One of \python{}'s unique features is the level of source location
information it provides in error messages.  The error messages contain
a lot of detail in a terse format, to they may be confusing at first.
Error messages will be illustrated using this example program:
\begin{lisp}
(defmacro zoq (x)
  `(roq (ploq (+ ,x 3))))

(defun foo (y)
  (declare (symbol y))
  (zoq y))
\end{lisp}
The main problem with this program is that it is trying to add \code{3} to a
symbol.  Note also that the functions \code{roq} and \code{ploq} aren't defined
anywhere.


\subsection{The Parts of the Error Message}

The compiler will produce this warning:

\begin{example}
File: /usr/me/stuff.lisp
In: DEFUN FOO
  (ZOQ Y)
--> ROQ PLOQ + 
==>
  Y
Warning: Result is a SYMBOL, not a NUMBER.
\end{example}

In this example we see each of the six possible parts of a compiler error
message:

\begin{Lentry} 
\item[\w{\code{File: /usr/me/stuff.lisp}}] This is the \var{file} that
  the compiler read the relevant code from.  The file name is
  displayed because it may not be immediately obvious when there is an
  error during compilation of a large system, especially when
  \code{with-compilation-unit} is used to delay undefined warnings.
  
\item[\w{\code{In: DEFUN FOO}}] This is the \var{definition} or
  top-level form responsible for the error.  It is obtained by taking
  the first two elements of the enclosing form whose first element is
  a symbol beginning with ``\code{DEF}''.  If there is no enclosing
  \w{\var{def}mumble}, then the outermost form is used.  If there are
  multiple \w{\var{def}mumbles}, then they are all printed from the
  out in, separated by \code{$=>$}'s.  In this example, the problem
  was in the \code{defun} for \code{foo}.
  
\item[\w{\code{(ZOQ Y)}}] This is the {\em original source} form
  responsible for the error.  Original source means that the form
  directly appeared in the original input to the compiler, i.e. in the
  lambda passed to \code{compile} or the top-level form read from the
  source file.  In this example, the expansion of the \code{zoq} macro
  was responsible for the error.
  
\item[\w{\code{--$>$ ROQ PLOQ +}} ] This is the {\em processing path}
  that the compiler used to produce the errorful code.  The processing
  path is a representation of the evaluated forms enclosing the actual
  source that the compiler encountered when processing the original
  source.  The path is the first element of each form, or the form
  itself if the form is not a list.  These forms result from the
  expansion of macros or source-to-source transformation done by the
  compiler.  In this example, the enclosing evaluated forms are the
  calls to \code{roq}, \code{ploq} and \code{+}.  These calls resulted
  from the expansion of the \code{zoq} macro.
  
\item[\code{==$>$ Y}] This is the {\em actual source} responsible for
  the error.  If the actual source appears in the explanation, then we
  print the next enclosing evaluated form, instead of printing the
  actual source twice.  (This is the form that would otherwise have
  been the last form of the processing path.)  In this example, the
  problem is with the evaluation of the reference to the variable
  \code{y}.
  
\item[\w{\code{Warning: Result is a SYMBOL, not a NUMBER.}}]  This is
  the \var{explanation} the problem.  In this example, the problem is
  that \code{y} evaluates to a \code{symbol}, but is in a context
  where a number is required (the argument to \code{+}).
\end{Lentry}

Note that each part of the error message is distinctively marked:

\begin{itemize} 
\item \code{File:} and \code{In:} mark the file and definition,
  respectively.
  
\item The original source is an indented form with no prefix.
  
\item Each line of the processing path is prefixed with \code{--$>$}.
  
\item The actual source form is indented like the original source, but
  is marked by a preceding \code{==$>$} line.  This is like the
  ``macroexpands to'' notation used in \cltl.
  
\item The explanation is prefixed with the error severity
  (\pxlref{error-severity}), either \code{Error:}, \code{Warning:}, or
  \code{Note:}.
\end{itemize}


Each part of the error message is more specific than the preceding
one.  If consecutive error messages are for nearby locations, then the
front part of the error messages would be the same.  In this case, the
compiler omits as much of the second message as in common with the
first.  For example:

\begin{example}
File: /usr/me/stuff.lisp
In: DEFUN FOO
  (ZOQ Y)
--> ROQ 
==>
  (PLOQ (+ Y 3))
Warning: Undefined function: PLOQ

==>
  (ROQ (PLOQ (+ Y 3)))
Warning: Undefined function: ROQ
\end{example}

In this example, the file, definition and original source are
identical for the two messages, so the compiler omits them in the
second message.  If consecutive messages are entirely identical, then
the compiler prints only the first message, followed by:

\begin{example}
[Last message occurs \var{repeats} times]
\end{example}

where \var{repeats} is the number of times the message was given.

If the source was not from a file, then no file line is printed.  If
the actual source is the same as the original source, then the
processing path and actual source will be omitted.  If no forms
intervene between the original source and the actual source, then the
processing path will also be omitted.


\subsection{The Original and Actual Source}
\cindex{original source}
\cindex{actual source}

The {\em original source} displayed will almost always be a list.  If the actual
source for an error message is a symbol, the original source will be the
immediately enclosing evaluated list form.  So even if the offending symbol
does appear in the original source, the compiler will print the enclosing list
and then print the symbol as the actual source (as though the symbol were
introduced by a macro.)

When the {\em actual source} is displayed (and is not a symbol), it will always
be code that resulted from the expansion of a macro or a source-to-source
compiler optimization.  This is code that did not appear in the original
source program; it was introduced by the compiler.

Keep in mind that when the compiler displays a source form in an error message,
it always displays the most specific (innermost) responsible form.  For
example, compiling this function:

\begin{lisp}
(defun bar (x)
  (let (a)
    (declare (fixnum a))
    (setq a (foo x))
    a))
\end{lisp}

gives this error message:

\begin{example}
In: DEFUN BAR
  (LET (A) (DECLARE (FIXNUM A)) (SETQ A (FOO X)) A)
Warning: The binding of A is not a FIXNUM:
  NIL
\end{example}

This error message is not saying ``there's a problem somewhere in this
\code{let}''\dash{}it is saying that there is a problem with the
\code{let} itself.  In this example, the problem is that \code{a}'s
\false{} initial value is not a \code{fixnum}.


\subsection{The Processing Path}
\cindex{processing path}
\cindex{macroexpansion}
\cindex{source-to-source transformation}

The processing path is mainly useful for debugging macros, so if you don't
write macros, you can ignore the processing path.  Consider this example:

\begin{lisp}
(defun foo (n)
  (dotimes (i n *undefined*)))
\end{lisp}

Compiling results in this error message:

\begin{example}
In: DEFUN FOO
  (DOTIMES (I N *UNDEFINED*))
--> DO BLOCK LET TAGBODY RETURN-FROM 
==>
  (PROGN *UNDEFINED*)
Warning: Undefined variable: *UNDEFINED*
\end{example}

Note that \code{do} appears in the processing path.  This is because \code{dotimes}
expands into:

\begin{lisp}
(do ((i 0 (1+ i)) (#:g1 n))
    ((>= i #:g1) *undefined*)
  (declare (type unsigned-byte i)))
\end{lisp}

The rest of the processing path results from the expansion of \code{do}:

\begin{lisp}
(block nil
  (let ((i 0) (#:g1 n))
    (declare (type unsigned-byte i))
    (tagbody (go #:g3)
     #:g2    (psetq i (1+ i))
     #:g3    (unless (>= i #:g1) (go #:g2))
             (return-from nil (progn *undefined*)))))
\end{lisp}

In this example, the compiler descended into the \code{block},
\code{let}, \code{tagbody} and \code{return-from} to reach the
\code{progn} printed as the actual source.  This is a place where the
``actual source appears in explanation'' rule was applied.  The
innermost actual source form was the symbol \code{*undefined*} itself,
but that also appeared in the explanation, so the compiler backed out
one level.


\subsection{Error Severity}
\label{error-severity}
\cindex{severity of compiler errors}
\cindex{compiler error severity}

There are three levels of compiler error severity:

\begin{Lentry}  
\item[Error] This severity is used when the compiler encounters a
  problem serious enough to prevent normal processing of a form.
  Instead of compiling the form, the compiler compiles a call to
  \code{error}.  Errors are used mainly for signaling syntax errors.
  If an error happens during macroexpansion, the compiler will handle
  it.  The compiler also handles and attempts to proceed from read
  errors.
  
\item[Warning] Warnings are used when the compiler can prove that
  something bad will happen if a portion of the program is executed,
  but the compiler can proceed by compiling code that signals an error
  at runtime if the problem has not been fixed:
  \begin{itemize}
  
  \item Violation of type declarations, or
  
  \item Function calls that have the wrong number of arguments or
    malformed keyword argument lists, or
  
  \item Referencing a variable declared \code{ignore}, or unrecognized
    declaration specifiers.
  \end{itemize}
  
  In the language of the \clisp{} standard, these are situations where
  the compiler can determine that a situation with undefined
  consequences or that would cause an error to be signaled would
  result at runtime.
  
\item[Note] Notes are used when there is something that seems a bit
  odd, but that might reasonably appear in correct programs.
\end{Lentry}

Note that the compiler does not fully conform to the proposed X3J13
``compiler-diagnostics'' cleanup.  Errors, warnings and notes mostly
correspond to errors, warnings and style-warnings, but many things
that the cleanup considers to be style-warnings are printed as
warnings rather than notes.  Also, warnings, style-warnings and most
errors aren't really signaled using the condition system.


\subsection{Errors During Macroexpansion}
\cpsubindex{macroexpansion}{errors during}

The compiler handles errors that happen during macroexpansion, turning
them into compiler errors.  If you want to debug the error (to debug a
macro), you can set \code{*break-on-signals*} to \code{error}.  For
example, this definition:

\begin{lisp}
(defun foo (e l)
  (do ((current l (cdr current))
       ((atom current) nil))
      (when (eq (car current) e) (return current))))
\end{lisp}

gives this error:

\begin{example}
In: DEFUN FOO
  (DO ((CURRENT L #) (# NIL)) (WHEN (EQ # E) (RETURN CURRENT)) )
Error: (during macroexpansion)

Error in function LISP::DO-DO-BODY.
DO step variable is not a symbol: (ATOM CURRENT)
\end{example}


\subsection{Read Errors}
\cpsubindex{read errors}{compiler}

The compiler also handles errors while reading the source.  For example:

\begin{example}
Error: Read error at 2:
 "(,/\back{foo})"
Error in function LISP::COMMA-MACRO.
Comma not inside a backquote.
\end{example}

The ``\code{at 2}'' refers to the character position in the source file at
which the error was signaled, which is generally immediately after the
erroneous text.  The next line, ``\code{(,/\back{foo})}'', is the line in
the source that contains the error file position.  The ``\code{/\back{} }''
indicates the error position within that line (in this example,
immediately after the offending comma.)

When in \hemlock{} (or any other EMACS-like editor), you can go to a
character position with:

\begin{example}
M-< C-u \var{position} C-f
\end{example}

Note that if the source is from a \hemlock{} buffer, then the position
is relative to the start of the compiled region or \code{defun}, not the
file or buffer start.

After printing a read error message, the compiler attempts to recover from the
error by backing up to the start of the enclosing top-level form and reading
again with \code{*read-suppress*} true.  If the compiler can recover from the
error, then it substitutes a call to \code{cerror} for the unreadable form and
proceeds to compile the rest of the file normally.

If there is a read error when the file position is at the end of the file
(i.e., an unexpected EOF error), then the error message looks like this:

\begin{example}
Error: Read error in form starting at 14:
 "(defun test ()"
Error in function LISP::FLUSH-WHITESPACE.
EOF while reading #<Stream for file "/usr/me/test.lisp">
\end{example}

In this case, ``\code{starting at 14}'' indicates the character
position at which the compiler started reading, i.e. the position
before the start of the form that was missing the closing delimiter.
The line \w{"\code{(defun test ()}"} is first line after the starting
position that the compiler thinks might contain the unmatched open
delimiter.


\subsection{Error Message Parameterization}
\cpsubindex{error messages}{verbosity}
\cpsubindex{verbosity}{of error messages}

There is some control over the verbosity of error messages.  See also
\varref{undefined-warning-limit}, \code{*efficiency-note-limit*} and
\varref{efficiency-note-cost-threshold}.

\begin{defvar}{}{enclosing-source-cutoff} 
  
  This variable specifies the number of enclosing actual source forms
  that are printed in full, rather than in the abbreviated processing
  path format.  Increasing the value from its default of \code{1}
  allows you to see more of the guts of the macroexpanded source,
  which is useful when debugging macros.
\end{defvar}

\begin{defvar}{}{error-print-length}
  \defvarx{error-print-level}
  
  These variables are the print level and print length used in
  printing error messages.  The default values are \code{5} and
  \code{3}.  If null, the global values of \code{*print-level*} and
  \code{*print-length*} are used.
\end{defvar}

\begin{defmac}{extensions:}{def-source-context}{%
    \args{\var{name} \var{lambda-list} \mstar{form}}}
  
  This macro defines how to extract an abbreviated source context from
  the \var{name}d form when it appears in the compiler input.
  \var{lambda-list} is a \code{defmacro} style lambda-list used to
  parse the arguments.  The \var{body} should return a list of
  subforms that can be printed on about one line.  There are
  predefined methods for \code{defstruct}, \code{defmethod}, etc.  If
  no method is defined, then the first two subforms are returned.
  Note that this facility implicitly determines the string name
  associated with anonymous functions.
\end{defmac}


\section{Types in Python}
\cpsubindex{types}{in python}

A big difference between \python{} and all other \llisp{} compilers
is the approach to type checking and amount of knowledge about types:
\begin{itemize}
  
\item \python{} treats type declarations much differently that other
  Lisp compilers do.  \python{} doesn't blindly believe type
  declarations; it considers them assertions about the program that
  should be checked.
  
\item \python{} also has a tremendously greater knowledge of the
  \clisp{} type system than other compilers.  Support is incomplete
  only for the \code{not}, \code{and} and \code{satisfies} types.
\end{itemize}
See also sections \ref{advanced-type-stuff} and \ref{type-inference}.


\subsection{Compile Time Type Errors}
\cindex{compile time type errors}
\cpsubindex{type checking}{at compile time}

If the compiler can prove at compile time that some portion of the
program cannot be executed without a type error, then it will give a
warning at compile time.  It is possible that the offending code would
never actually be executed at run-time due to some higher level
consistency constraint unknown to the compiler, so a type warning
doesn't always indicate an incorrect program.  For example, consider
this code fragment:
\begin{lisp}
(defun raz (foo)
  (let ((x (case foo
             (:this 13)
             (:that 9)
             (:the-other 42))))
    (declare (fixnum x))
    (foo x)))
\end{lisp}

Compilation produces this warning:

\begin{example}
In: DEFUN RAZ
  (CASE FOO (:THIS 13) (:THAT 9) (:THE-OTHER 42))
--> LET COND IF COND IF COND IF 
==>
  (COND)
Warning: This is not a FIXNUM:
  NIL
\end{example}

In this case, the warning is telling you that if \code{foo} isn't any
of \kwd{this}, \kwd{that} or \kwd{the-other}, then \code{x} will be
initialized to \false, which the \code{fixnum} declaration makes
illegal.  The warning will go away if \code{ecase} is used instead of
\code{case}, or if \kwd{the-other} is changed to \true.

This sort of spurious type warning happens moderately often in the
expansion of complex macros and in inline functions.  In such cases,
there may be dead code that is impossible to correctly execute.  The
compiler can't always prove this code is dead (could never be
executed), so it compiles the erroneous code (which will always signal
an error if it is executed) and gives a warning.

\begin{defun}{extensions:}{required-argument}{}
  
  This function can be used as the default value for keyword arguments
  that must always be supplied.  Since it is known by the compiler to
  never return, it will avoid any compile-time type warnings that
  would result from a default value inconsistent with the declared
  type.  When this function is called, it signals an error indicating
  that a required keyword argument was not supplied.  This function is
  also useful for \code{defstruct} slot defaults corresponding to
  required arguments.  \xlref{empty-type}.
  
  Although this function is a \cmucl{} extension, it is relatively harmless
  to use it in otherwise portable code, since you can easily define it
  yourself:
  \begin{lisp}
    (defun required-argument ()
      (error "A required keyword argument was not supplied."))
    \end{lisp}
\end{defun}

Type warnings are inhibited when the
\code{extensions:inhibit-warnings} optimization quality is \code{3}
(\pxlref{compiler-policy}.)  This can be used in a local declaration
to inhibit type warnings in a code fragment that has spurious
warnings.


\subsection{Precise Type Checking}
\label{precise-type-checks}
\cindex{precise type checking}
\cpsubindex{type checking}{precise}

With the default compilation policy, all type
assertions\footnote{There are a few circumstances where a type
  declaration is discarded rather than being used as type assertion.
  This doesn't affect safety much, since such discarded declarations
  are also not believed to be true by the compiler.}  are precisely
checked.  Precise checking means that the check is done as though
\code{typep} had been called with the exact type specifier that
appeared in the declaration.  \python{} uses \var{policy} to determine
whether to trust type assertions (\pxlref{compiler-policy}).  Type
assertions from declarations are indistinguishable from the type
assertions on arguments to built-in functions.  In \python, adding
type declarations makes code safer.

If a variable is declared to be \w{\code{(integer 3 17)}}, then its
value must always always be an integer between \code{3} and \code{17}.
If multiple type declarations apply to a single variable, then all the
declarations must be correct; it is as though all the types were
intersected producing a single \code{and} type specifier.

Argument type declarations are automatically enforced.  If you declare
the type of a function argument, a type check will be done when that
function is called.  In a function call, the called function does the
argument type checking, which means that a more restrictive type
assertion in the calling function (e.g., from \code{the}) may be lost.

The types of structure slots are also checked.  The value of a
structure slot must always be of the type indicated in any \kwd{type}
slot option.\footnote{The initial value need not be of this type as
  long as the corresponding argument to the constructor is always
  supplied, but this will cause a compile-time type warning unless
  \code{required-argument} is used.} Because of precise type checking,
the arguments to slot accessors are checked to be the correct type of
structure.

In traditional \llisp{} compilers, not all type assertions are
checked, and type checks are not precise.  Traditional compilers
blindly trust explicit type declarations, but may check the argument
type assertions for built-in functions.  Type checking is not precise,
since the argument type checks will be for the most general type legal
for that argument.  In many systems, type declarations suppress what
little type checking is being done, so adding type declarations makes
code unsafe.  This is a problem since it discourages writing type
declarations during initial coding.  In addition to being more error
prone, adding type declarations during tuning also loses all the
benefits of debugging with checked type assertions.

To gain maximum benefit from \python{}'s type checking, you should
always declare the types of function arguments and structure slots as
precisely as possible.  This often involves the use of \code{or},
\code{member} and other list-style type specifiers.  Paradoxically,
even though adding type declarations introduces type checks, it
usually reduces the overall amount of type checking.  This is
especially true for structure slot type declarations.

\python{} uses the \code{safety} optimization quality (rather than
presence or absence of declarations) to choose one of three levels of
run-time type error checking: \pxlref{optimize-declaration}.
\xlref{advanced-type-stuff} for more information about types in
\python{}.


\subsection{Weakened Type Checking}
\label{weakened-type-checks}
\cindex{weakened type checking}
\cpsubindex{type checking}{weakened}

When the value for the \code{speed} optimization quality is greater
than \code{safety}, and \code{safety} is not \code{0}, then type
checking is weakened to reduce the speed and space penalty.  In
structure-intensive code this can double the speed, yet still catch
most type errors.  Weakened type checks provide a level of safety
similar to that of ``safe'' code in other \llisp{} compilers.

A type check is weakened by changing the check to be for some
convenient supertype of the asserted type.  For example,
\code{\w{(integer 3 17)}} is changed to \code{fixnum},
\code{\w{(simple-vector 17)}} to \code{simple-vector}, and structure
types are changed to \code{structure}.  A complex check like:
\begin{example}
(or node hunk (member :foo :bar :baz))
\end{example}
will be omitted entirely (i.e., the check is weakened to \code{*}.)  If
a precise check can be done for no extra cost, then no weakening is
done.

Although weakened type checking is similar to type checking done by
other compilers, it is sometimes safer and sometimes less safe.
Weakened checks are done in the same places is precise checks, so all
the preceding discussion about where checking is done still applies.
Weakened checking is sometimes somewhat unsafe because although the
check is weakened, the precise type is still input into type
inference.  In some contexts this will result in type inferences not
justified by the weakened check, and hence deletion of some type
checks that would be done by conventional compilers.

For example, if this code was compiled with weakened checks:

\begin{lisp}
(defstruct foo
  (a nil :type simple-string))

(defstruct bar
  (a nil :type single-float))

(defun myfun (x)
  (declare (type bar x))
  (* (bar-a x) 3.0))
\end{lisp}

and \code{myfun} was passed a \code{foo}, then no type error would be
signaled, and we would try to multiply a \code{simple-vector} as
though it were a float (with unpredictable results.)  This is because
the check for \code{bar} was weakened to \code{structure}, yet when
compiling the call to \code{bar-a}, the compiler thinks it knows it
has a \code{bar}.

Note that normally even weakened type checks report the precise type
in error messages.  For example, if \code{myfun}'s \code{bar} check is
weakened to \code{structure}, and the argument is \false{}, then the
error will be:

\begin{example}
Type-error in MYFUN:
  NIL is not of type BAR
\end{example}

However, there is some speed and space cost for signaling a precise
error, so the weakened type is reported if the \code{speed}
optimization quality is \code{3} or \code{debug} quality is less than
\code{1}:

\begin{example}
Type-error in MYFUN:
  NIL is not of type STRUCTURE
\end{example}

\xlref{optimize-declaration} for further discussion of the
\code{optimize} declaration.


\section{Getting Existing Programs to Run}
\cpsubindex{existing programs}{to run}
\cpsubindex{types}{portability}
\cindex{compatibility with other Lisps}

Since \python{} does much more comprehensive type checking than other
Lisp compilers, \python{} will detect type errors in many programs
that have been debugged using other compilers.  These errors are
mostly incorrect declarations, although compile-time type errors can
find actual bugs if parts of the program have never been tested.

Some incorrect declarations can only be detected by run-time type
checking.  It is very important to initially compile programs with
full type checks and then test this version.  After the checking
version has been tested, then you can consider weakening or
eliminating type checks.  {\bf This applies even to previously debugged
  programs.}  \python{} does much more type inference than other
\llisp{} compilers, so believing an incorrect declaration does much
more damage.

The most common problem is with variables whose initial value doesn't
match the type declaration.  Incorrect initial values will always be
flagged by a compile-time type error, and they are simple to fix once
located.  Consider this code fragment:

\begin{example}
(prog (foo)
  (declare (fixnum foo))
  (setq foo ...)
  ...)
\end{example}

Here the variable \code{foo} is given an initial value of \false, but
is declared to be a \code{fixnum}.  Even if it is never read, the
initial value of a variable must match the declared type.  There are
two ways to fix this problem.  Change the declaration:

\begin{example}
(prog (foo)
  (declare (type (or fixnum null) foo))
  (setq foo ...)
  ...)
\end{example}

or change the initial value:

\begin{example}
(prog ((foo 0))
  (declare (fixnum foo))
  (setq foo ...)
  ...)
\end{example}

It is generally preferable to change to a legal initial value rather
than to weaken the declaration, but sometimes it is simpler to weaken
the declaration than to try to make an initial value of the
appropriate type.

Another declaration problem occasionally encountered is incorrect
declarations on \code{defmacro} arguments.  This probably usually
happens when a function is converted into a macro.  Consider this
macro:

\begin{lisp}
(defmacro my-1+ (x)
  (declare (fixnum x))
  `(the fixnum (1+ ,x)))
\end{lisp}

Although legal and well-defined \clisp, this meaning of this
definition is almost certainly not what the writer intended.  For
example, this call is illegal:

\begin{lisp}
(my-1+ (+ 4 5))
\end{lisp}

The call is illegal because the argument to the macro is \w{\code{(+ 4
    5)}}, which is a \code{list}, not a \code{fixnum}.  Because of
macro semantics, it is hardly ever useful to declare the types of
macro arguments.  If you really want to assert something about the
type of the result of evaluating a macro argument, then put a
\code{the} in the expansion:

\begin{lisp}
(defmacro my-1+ (x)
  `(the fixnum (1+ (the fixnum ,x))))
\end{lisp}

In this case, it would be stylistically preferable to change this
macro back to a function and declare it inline.  Macros have no
efficiency advantage over inline functions when using \python{}.
\xlref{inline-expansion}.


Some more subtle problems are caused by incorrect declarations that
can't be detected at compile time.  Consider this code:

\begin{example}
(do ((pos 0 (position #\back{a} string :start (1+ pos))))
    ((null pos))
  (declare (fixnum pos))
  ...)
\end{example}

Although \code{pos} is almost always a \code{fixnum}, it is \false{}
at the end of the loop.  If this example is compiled with full type
checks (the default), then running it will signal a type error at the
end of the loop.  If compiled without type checks, the program will go
into an infinite loop (or perhaps \code{position} will complain
because \w{\code{(1+ nil)}} isn't a sensible start.)  Why?  Because if
you compile without type checks, the compiler just quietly believes
the type declaration.  Since \code{pos} is always a \code{fixnum}, it
is never \nil, so \w{\code{(null pos)}} is never true, and the loop
exit test is optimized away.  Such errors are sometimes flagged by
unreachable code notes (\pxlref{dead-code-notes}), but it is still
important to initially compile any system with full type checks, even
if the system works fine when compiled using other compilers.

In this case, the fix is to weaken the type declaration to
\w{\code{(or fixnum null)}}.\footnote{Actually, this declaration is
  totally unnecessary in \python{}, since it already knows
  \code{position} returns a non-negative \code{fixnum} or \false.}
Note that there is usually little performance penalty for weakening a
declaration in this way.  Any numeric operations in the body can still
assume the variable is a \code{fixnum}, since \false{} is not a legal
numeric argument.  Another possible fix would be to say:

\begin{example}
(do ((pos 0 (position #\back{a} string :start (1+ pos))))
    ((null pos))
  (let ((pos pos))
    (declare (fixnum pos))
    ...))
\end{example}

This would be preferable in some circumstances, since it would allow a
non-standard representation to be used for the local \code{pos}
variable in the loop body (see section \ref{ND-variables}.)

In summary, remember that {\em all} values that a variable {\em ever}
has must be of the declared type, and that you should test using safe
code initially.


\section{Compiler Policy}
\label{compiler-policy}
\cpsubindex{policy}{compiler}
\cindex{compiler policy}

The policy is what tells the compiler \var{how} to compile a program.
This is logically (and often textually) distinct from the program
itself.  Broad control of policy is provided by the \code{optimize}
declaration; other declarations and variables control more specific
aspects of compilation.


\subsection{The Optimize Declaration}
\label{optimize-declaration}
\cindex{optimize declaration}
\cpsubindex{declarations}{\code{optimize}}

The \code{optimize} declaration recognizes six different
\var{qualities}.  The qualities are conceptually independent aspects
of program performance.  In reality, increasing one quality tends to
have adverse effects on other qualities.  The compiler compares the
relative values of qualities when it needs to make a trade-off; i.e.,
if \code{speed} is greater than \code{safety}, then improve speed at
the cost of safety.

The default for all qualities (except \code{debug}) is \code{1}.
Whenever qualities are equal, ties are broken according to a broad
idea of what a good default environment is supposed to be.  Generally
this downplays \code{speed}, \code{compile-speed} and \code{space} in
favor of \code{safety} and \code{debug}.  Novice and casual users
should stick to the default policy.  Advanced users often want to
improve speed and memory usage at the cost of safety and
debuggability.

If the value for a quality is \code{0} or \code{3}, then it may have a
special interpretation.  A value of \code{0} means ``totally
unimportant'', and a \code{3} means ``ultimately important.''  These
extreme optimization values enable ``heroic'' compilation strategies
that are not always desirable and sometimes self-defeating.
Specifying more than one quality as \code{3} is not desirable, since
it doesn't tell the compiler which quality is most important.


These are the optimization qualities:
\begin{Lentry}
  
\item[\code{speed}] \cindex{speed optimization quality}How fast the
  program should is run.  \code{speed 3} enables some optimizations
  that hurt debuggability.
  
\item[\code{compilation-speed}] \cindex{compilation-speed optimization
    quality}How fast the compiler should run.  Note that increasing
  this above \code{safety} weakens type checking.
  
\item[\code{space}] \cindex{space optimization quality}How much space
  the compiled code should take up.  Inline expansion is mostly
  inhibited when \code{space} is greater than \code{speed}.  A value
  of \code{0} enables promiscuous inline expansion.  Wide use of a
  \code{0} value is not recommended, as it may waste so much space
  that run time is slowed.  \xlref{inline-expansion} for a discussion
  of inline expansion.
  
\item[\code{debug}] \cindex{debug optimization quality}How debuggable
  the program should be.  The quality is treated differently from the
  other qualities: each value indicates a particular level of debugger
  information; it is not compared with the other qualities.
  \xlref{debugger-policy} for more details.
  
\item[\code{safety}] \cindex{safety optimization quality}How much
  error checking should be done.  If \code{speed}, \code{space} or
  \code{compilation-speed} is more important than \code{safety}, then
  type checking is weakened (\pxlref{weakened-type-checks}).  If
  \code{safety} if \code{0}, then no run time error checking is done.
  In addition to suppressing type checks, \code{0} also suppresses
  argument count checking, unbound-symbol checking and array bounds
  checks.
  
\item[\code{extensions:inhibit-warnings}] \cindex{inhibit-warnings
    optimization quality}This is a \cmucl{} extension that determines how
  little (or how much) diagnostic output should be printed during
  compilation.  This quality is compared to other qualities to
  determine whether to print style notes and warnings concerning those
  qualities.  If \code{speed} is greater than \code{inhibit-warnings},
  then notes about how to improve speed will be printed, etc.  The
  default value is \code{1}, so raising the value for any standard
  quality above its default enables notes for that quality.  If
  \code{inhibit-warnings} is \code{3}, then all notes and most
  non-serious warnings are inhibited.  This is useful with
  \code{declare} to suppress warnings about unavoidable problems.
\end{Lentry}


\subsection{The Optimize-Interface Declaration}
\label{optimize-interface-declaration}
\cindex{optimize-interface declaration}
\cpsubindex{declarations}{\code{optimize-interface}}

The \code{extensions:optimize-interface} declaration is identical in
syntax to the \code{optimize} declaration, but it specifies the policy
used during compilation of code the compiler automatically generates
to check the number and type of arguments supplied to a function.  It
is useful to specify this policy separately, since even thoroughly
debugged functions are vulnerable to being passed the wrong arguments.
The \code{optimize-interface} declaration can specify that arguments
should be checked even when the general \code{optimize} policy is
unsafe.

Note that this argument checking is the checking of user-supplied
arguments to any functions defined within the scope of the
declaration, \code{not} the checking of arguments to \llisp{}
primitives that appear in those definitions.

The idea behind this declaration is that it allows the definition of
functions that appear fully safe to other callers, but that do no
internal error checking.  Of course, it is possible that arguments may
be invalid in ways other than having incorrect type.  Functions
compiled unsafely must still protect themselves against things like
user-supplied array indices that are out of bounds and improper lists.
See also the \kwd{context-declarations} option to
\macref{with-compilation-unit}.


\section{Open Coding and Inline Expansion}
\label{open-coding}
\cindex{open-coding}
\cindex{inline expansion}
\cindex{static functions}

Since \clisp{} forbids the redefinition of standard functions\footnote{See the
proposed X3J13 ``lisp-symbol-redefinition'' cleanup.}, the compiler can have
special knowledge of these standard functions embedded in it.  This special
knowledge is used in various ways (open coding, inline expansion, source
transformation), but the implications to the user are basically the same:
\begin{itemize}
  
\item Attempts to redefine standard functions may be frustrated, since
  the function may never be called.  Although it is technically
  illegal to redefine standard functions, users sometimes want to
  implicitly redefine these functions when they are debugging using
  the \code{trace} macro.  Special-casing of standard functions can be
  inhibited using the \code{notinline} declaration.
  
\item The compiler can have multiple alternate implementations of
  standard functions that implement different trade-offs of speed,
  space and safety.  This selection is based on the compiler policy,
  \pxlref{compiler-policy}.
\end{itemize}


When a function call is {\em open coded}, inline code whose effect is
equivalent to the function call is substituted for that function call.
When a function call is {\em closed coded}, it is usually left as is,
although it might be turned into a call to a different function with
different arguments.  As an example, if \code{nthcdr} were to be open
coded, then

\begin{lisp}
(nthcdr 4 foobar)
\end{lisp}

might turn into

\begin{lisp}
(cdr (cdr (cdr (cdr foobar))))
\end{lisp}

or even 

\begin{lisp}
(do ((i 0 (1+ i))
     (list foobar (cdr foobar)))
    ((= i 4) list))
\end{lisp}

If \code{nth} is closed coded, then

\begin{lisp}
(nth x l)
\end{lisp}

might stay the same, or turn into something like:

\begin{lisp}
(car (nthcdr x l))
\end{lisp}

In general, open coding sacrifices space for speed, but some functions (such as
\code{car}) are so simple that they are always open-coded.  Even when not
open-coded, a call to a standard function may be transformed into a
different function call (as in the last example) or compiled as {\em
static call}. Static function call uses a more efficient calling
convention that forbids redefinition.

\chapter{Advanced Compiler Use and Efficiency Hints}
\label{advanced-compiler}

\credits{by Robert MacLachlan}


\section{Advanced Compiler Introduction}

In \cmucl{}, as with any language on any computer, the path to efficient
code starts with good algorithms and sensible programming techniques,
but to avoid inefficiency pitfalls, you need to know some of this
implementation's quirks and features.  This chapter is mostly a fairly
long and detailed overview of what optimizations \python{} does.
Although there are the usual negative suggestions of inefficient
features to avoid, the main emphasis is on describing the things that
programmers can count on being efficient.

The optimizations described here can have the effect of speeding up
existing programs written in conventional styles, but the potential
for new programming styles that are clearer and less error-prone is at
least as significant.  For this reason, several sections end with a
discussion of the implications of these optimizations for programming
style.



\subsection{Types}

\python{}'s support for types is unusual in three major ways:
\begin{itemize}
  
\item Precise type checking encourages the specific use of type
  declarations as a form of run-time consistency checking.  This
  speeds development by localizing type errors and giving more
  meaningful error messages.  \xlref{precise-type-checks}.  \python{}
  produces completely safe code; optimized type checking maintains
  reasonable efficiency on conventional hardware
  (\pxlref{type-check-optimization}.)
  
\item Comprehensive support for the \clisp{} type system makes complex
  type specifiers useful.  Using type specifiers such as \code{or} and
  \code{member} has both efficiency and robustness advantages.
  \xlref{advanced-type-stuff}.
  
\item Type inference eliminates the need for some declarations, and
  also aids compile-time detection of type errors.  Given detailed
  type declarations, type inference can often eliminate type checks
  and enable more efficient object representations and code sequences.
  Checking all types results in fewer type checks.  See sections
  \ref{type-inference} and \ref{non-descriptor}.
\end{itemize}


\subsection{Optimization}

The main barrier to efficient Lisp programs is not that there is no
efficient way to code the program in Lisp, but that it is difficult to
arrive at that efficient coding.  \clisp{} is a highly complex
language, and usually has many semantically equivalent ``reasonable''
ways to code a given problem.  It is desirable to make all of these
equivalent solutions have comparable efficiency so that programmers
don't have to waste time discovering the most efficient solution.

Source level optimization increases the number of efficient ways to
solve a problem.  This effect is much larger than the increase in the
efficiency of the ``best'' solution.  Source level optimization
transforms the original program into a more efficient (but equivalent)
program.  Although the optimizer isn't doing anything the programmer
couldn't have done, this high-level optimization is important because:

\begin{itemize} 
\item The programmer can code simply and directly, rather than
  obfuscating code to please the compiler.
  
\item When presented with a choice of similar coding alternatives, the
  programmer can chose whichever happens to be most convenient,
  instead of worrying about which is most efficient.
\end{itemize}

Source level optimization eliminates the need for macros to optimize
their expansion, and also increases the effectiveness of inline
expansion.  See sections \ref{source-optimization} and
\ref{inline-expansion}.

Efficient support for a safer programming style is the biggest
advantage of source level optimization.  Existing tuned programs
typically won't benefit much from source optimization, since their
source has already been optimized by hand.  However, even tuned
programs tend to run faster under \python{} because:

\begin{itemize} 
\item Low level optimization and register allocation provides modest
  speedups in any program.
  
\item Block compilation and inline expansion can reduce function call
  overhead, but may require some program restructuring.  See sections
  \ref{inline-expansion}, \ref{local-call} and
  \ref{block-compilation}.
  
\item Efficiency notes will point out important type declarations that
  are often missed even in highly tuned programs.
  \xlref{efficiency-notes}.
  
\item Existing programs can be compiled safely without prohibitive
  speed penalty, although they would be faster and safer with added
  declarations.  \xlref{type-check-optimization}.
  
\item The context declaration mechanism allows both space and runtime
  of large systems to be reduced without sacrificing robustness by
  semi-automatically varying compilation policy without addition any
  \code{optimize} declarations to the source.
  \xlref{context-declarations}.
  
\item Byte compilation can be used to dramatically reduce the size of
  code that is not speed-critical. \xlref{byte-compile}
\end{itemize}


\subsection{Function Call}

The sort of symbolic programs generally written in \llisp{} often
favor recursion over iteration, or have inner loops so complex that
they involve multiple function calls.  Such programs spend a larger
fraction of their time doing function calls than is the norm in other
languages; for this reason \llisp{} implementations strive to make the
general (or full) function call as inexpensive as possible.  \python{}
goes beyond this by providing two good alternatives to full call:

\begin{itemize} 
\item Local call resolves function references at compile time,
  allowing better calling sequences and optimization across function
  calls.  \xlref{local-call}.
  
\item Inline expansion totally eliminates call overhead and allows
  many context dependent optimizations.  This provides a safe and
  efficient implementation of operations with function semantics,
  eliminating the need for error-prone macro definitions or manual
  case analysis.  Although most \clisp{} implementations support
  inline expansion, it becomes a more powerful tool with \python{}'s
  source level optimization.  See sections \ref{source-optimization}
  and \ref{inline-expansion}.
\end{itemize}


Generally, \python{} provides simple implementations for simple uses
of function call, rather than having only a single calling convention.
These features allow a more natural programming style:

\begin{itemize} 
\item Proper tail recursion.  \xlref{tail-recursion}
  
\item Relatively efficient closures.
  
\item A \code{funcall} that is as efficient as normal named call.
  
\item Calls to local functions such as from \code{labels} are
  optimized:
\begin{itemize}
  
\item Control transfer is a direct jump.
  
\item The closure environment is passed in registers rather than heap
  allocated.
  
\item Keyword arguments and multiple values are implemented more
  efficiently.
\end{itemize}

\xlref{local-call}.
\end{itemize}


\subsection{Representation of Objects}

Sometimes traditional \llisp{} implementation techniques compare so
poorly to the techniques used in other languages that \llisp{} can
become an impractical language choice.  Terrible inefficiencies appear
in number-crunching programs, since \llisp{} numeric operations often
involve number-consing and generic arithmetic.  \python{} supports
efficient natural representations for numbers (and some other types),
and allows these efficient representations to be used in more
contexts.  \python{} also provides good efficiency notes that warn
when a crucial declaration is missing.

See section \ref{non-descriptor} for more about object representations and
numeric types.  Also \pxlref{efficiency-notes} about efficiency notes.


\subsection{Writing Efficient Code}
\label{efficiency-overview}

Writing efficient code that works is a complex and prolonged process.
It is important not to get so involved in the pursuit of efficiency
that you lose sight of what the original problem demands.  Remember
that:
\begin{itemize}
  
\item The program should be correct\dash{}it doesn't matter how
  quickly you get the wrong answer.
  
\item Both the programmer and the user will make errors, so the
  program must be robust\dash{}it must detect errors in a way that
  allows easy correction.
  
\item A small portion of the program will consume most of the
  resources, with the bulk of the code being virtually irrelevant to
  efficiency considerations.  Even experienced programmers familiar
  with the problem area cannot reliably predict where these ``hot
  spots'' will be.
\end{itemize}



The best way to get efficient code that is still worth using, is to separate
coding from tuning.  During coding, you should:
\begin{itemize}
  
\item Use a coding style that aids correctness and robustness without
  being incompatible with efficiency.
  
\item Choose appropriate data structures that allow efficient
  algorithms and object representations
  (\pxlref{object-representation}).  Try to make interfaces abstract
  enough so that you can change to a different representation if
  profiling reveals a need.
  
\item Whenever you make an assumption about a function argument or
  global data structure, add consistency assertions, either with type
  declarations or explicit uses of \code{assert}, \code{ecase}, etc.
\end{itemize}

During tuning, you should:
\begin{itemize}
  
\item Identify the hot spots in the program through profiling (section
  \ref{profiling}.)
  
\item Identify inefficient constructs in the hot spot with efficiency
  notes, more profiling, or manual inspection of the source.  See
  sections \ref{general-efficiency} and \ref{efficiency-notes}.
  
\item Add declarations and consider the application of optimizations.
  See sections \ref{local-call}, \ref{inline-expansion} and
  \ref{non-descriptor}.
  
\item If all else fails, consider algorithm or data structure changes.
  If you did a good job coding, changes will be easy to introduce.
\end{itemize}


\section{More About Types in Python}
\label{advanced-type-stuff}
\cpsubindex{types}{in python}

This section goes into more detail describing what types and declarations are
recognized by \python.  The area where \python{} differs most radically from
previous \llisp{} compilers is in its support for types:
\begin{itemize}
  
\item Precise type checking helps to find bugs at run time.
  
\item Compile-time type checking helps to find bugs at compile time.
  
\item Type inference minimizes the need for generic operations, and
  also increases the efficiency of run time type checking and the
  effectiveness of compile time type checking.
  
\item Support for detailed types provides a wealth of opportunity for
  operation-specific type inference and optimization.
\end{itemize}



\subsection{More Types Meaningful}

\clisp{} has a very powerful type system, but conventional \llisp{}
implementations typically only recognize the small set of types
special in that implementation.  In these systems, there is an
unfortunate paradox: a declaration for a relatively general type like
\code{fixnum} will be recognized by the compiler, but a highly
specific declaration such as \code{\w{(integer 3 17)}} is totally
ignored.

This is obviously a problem, since the user has to know how to specify
the type of an object in the way the compiler wants it.  A very
minimal (but rarely satisfied) criterion for type system support is
that it be no worse to make a specific declaration than to make a
general one.  \python{} goes beyond this by exploiting a number of
advantages obtained from detailed type information.

Using more restrictive types in declarations allows the compiler to do
better type inference and more compile-time type checking.  Also, when
type declarations are considered to be consistency assertions that
should be verified (conditional on policy), then complex types are
useful for making more detailed assertions.

\python{} ``understands'' the list-style \code{or}, \code{member},
\code{function}, array and number type specifiers.  Understanding
means that:
\begin{itemize}
  
\item If the type contains more information than is used in a
  particular context, then the extra information is simply ignored,
  rather than derailing type inference.
  
\item In many contexts, the extra information from these type
  specifier is used to good effect.  In particular, type checking in
  \python{} is \var{precise}, so these complex types can be used
  in declarations to make interesting assertions about functions and
  data structures (\pxlref{precise-type-checks}.)  More specific
  declarations also aid type inference and reduce the cost for type
  checking.
\end{itemize}

For related information, \pxlref{numeric-types} for numeric types, and
section \ref{array-types} for array types.


\subsection{Canonicalization}
\cpsubindex{types}{equivalence}
\cindex{canonicalization of types}
\cindex{equivalence of types}

When given a type specifier, \python{} will often rewrite it into a
different (but equivalent) type.  This is the mechanism that \python{}
uses for detecting type equivalence.  For example, in \python{}'s
canonical representation, these types are equivalent:
\begin{example}
(or list (member :end)) \myequiv (or cons (member nil :end))
\end{example}
This has two implications for the user:
\begin{itemize}
  
\item The standard symbol type specifiers for \code{atom},
  \code{null}, \code{fixnum}, etc., are in no way magical.  The
  \tindexed{null} type is actually defined to be \code{\w{(member
      nil)}}, \tindexed{list} is \code{\w{(or cons null)}}, and
  \tindexed{fixnum} is \code{\w{(signed-byte 30)}}.
  
\item When the compiler prints out a type, it may not look like the
  type specifier that originally appeared in the program.  This is
  generally not a problem, but it must be taken into consideration
  when reading compiler error messages.
\end{itemize}


\subsection{Member Types}
\cindex{member types}

The \tindexed{member} type specifier can be used to represent
``symbolic'' values, analogous to the enumerated types of Pascal.  For
example, the second value of \code{find-symbol} has this type:
\begin{lisp}
(member :internal :external :inherited nil)
\end{lisp}
Member types are very useful for expressing consistency constraints on data
structures, for example:
\begin{lisp}
(defstruct ice-cream
  (flavor :vanilla :type (member :vanilla :chocolate :strawberry)))
\end{lisp}
Member types are also useful in type inference, as the number of members can
sometimes be pared down to one, in which case the value is a known constant.


\subsection{Union Types}
\cindex{union (\code{or}) types}
\cindex{or (union) types}

The \tindexed{or} (union) type specifier is understood, and is
meaningfully applied in many contexts.  The use of \code{or} allows
assertions to be made about types in dynamically typed programs.  For
example:

\begin{lisp}
(defstruct box
  (next nil :type (or box null))
  (top :removed :type (or box-top (member :removed))))
\end{lisp}

The type assertion on the \code{top} slot ensures that an error will be signaled
when there is an attempt to store an illegal value (such as \kwd{rmoved}.)
Although somewhat weak, these union type assertions provide a useful input into
type inference, allowing the cost of type checking to be reduced.  For example,
this loop is safely compiled with no type checks:

\begin{lisp}
(defun find-box-with-top (box)
  (declare (type (or box null) box))
  (do ((current box (box-next current)))
      ((null current))
    (unless (eq (box-top current) :removed)
      (return current))))
\end{lisp}

Union types are also useful in type inference for representing types that are
partially constrained.  For example, the result of this expression:
\begin{lisp}
(if foo
    (logior x y)
    (list x y))
\end{lisp}
can be expressed as \code{\w{(or integer cons)}}.


\subsection{The Empty Type}
\label{empty-type}
\cindex{NIL type}
\cpsubindex{empty type}{the}
\cpsubindex{errors}{result type of}

The type \false{} is also called the empty type, since no object is of
type \false{}.  The union of no types, \code{(or)}, is also empty.
\python{}'s interpretation of an expression whose type is \false{} is
that the expression never yields any value, but rather fails to
terminate, or is thrown out of.  For example, the type of a call to
\code{error} or a use of \code{return} is \false{}.  When the type of
an expression is empty, compile-time type warnings about its value are
suppressed; presumably somebody else is signaling an error.  If a
function is declared to have return type \false{}, but does in fact
return, then (in safe compilation policies) a ``\code{NIL Function
  returned}'' error will be signaled.  See also the function
\funref{required-argument}.


\subsection{Function Types}
\label{function-types}
\cpsubindex{function}{types}
\cpsubindex{types}{function}

\findexed{function} types are understood in the restrictive sense, specifying:
\begin{itemize}
  
\item The argument syntax that the function must be called with.  This
  is information about what argument counts are acceptable, and which
  keyword arguments are recognized.  In \python, warnings about
  argument syntax are a consequence of function type checking.
  
\item The types of the argument values that the caller must pass.  If
  the compiler can prove that some argument to a call is of a type
  disallowed by the called function's type, then it will give a
  compile-time type warning.  In addition to being used for
  compile-time type checking, these type assertions are also used as
  output type assertions in code generation.  For example, if
  \code{foo} is declared to have a \code{fixnum} argument, then the
  \code{1+} in \w{\code{(foo (1+ x))}} is compiled with knowledge that
  the result must be a fixnum.
  
\item The types the values that will be bound to argument variables in
  the function's definition.  Declaring a function's type with
  \code{ftype} implicitly declares the types of the arguments in the
  definition.  \python{} checks for consistency between the definition
  and the \code{ftype} declaration.  Because of precise type checking,
  an error will be signaled when a function is called with an
  argument of the wrong type.
  
\item The type of return value(s) that the caller can expect.  This
  information is a useful input to type inference.  For example, if a
  function is declared to return a \code{fixnum}, then when a call to
  that function appears in an expression, the expression will be
  compiled with knowledge that the call will return a \code{fixnum}.
  
\item The type of return value(s) that the definition must return.
  The result type in an \code{ftype} declaration is treated like an
  implicit \code{the} wrapped around the body of the definition.  If
  the definition returns a value of the wrong type, an error will be
  signaled.  If the compiler can prove that the function returns the
  wrong type, then it will give a compile-time warning.
\end{itemize}

This is consistent with the new interpretation of function types and
the \code{ftype} declaration in the proposed X3J13
``function-type-argument-type-semantics'' cleanup.  Note also, that if
you don't explicitly declare the type of a function using a global
\code{ftype} declaration, then \python{} will compute a function type
from the definition, providing a degree of inter-routine type
inference, \pxlref{function-type-inference}.


\subsection{The Values Declaration}
\cindex{values declaration}

\cmucl{} supports the \code{values} declaration as an extension to
\clisp.  The syntax is {\code{(values \var{type1}
    \var{type2}$\ldots$\var{typen})}}.  This declaration is
semantically equivalent to a \code{the} form wrapped around the body
of the special form in which the \code{values} declaration appears.
The advantage of \code{values} over \findexed{the} is purely
syntactic\dash{}it doesn't introduce more indentation.  For example:

\begin{example}
(defun foo (x)
  (declare (values single-float))
  (ecase x
    (:this ...)
    (:that ...)
    (:the-other ...)))
\end{example}

is equivalent to:

\begin{example}
(defun foo (x)
  (the single-float
       (ecase x
         (:this ...)
         (:that ...)
         (:the-other ...))))
\end{example}

and

\begin{example}
(defun floor (number &optional (divisor 1))
  (declare (values integer real))
  ...)
\end{example}

is equivalent to:

\begin{example}
(defun floor (number &optional (divisor 1))
  (the (values integer real)
       ...))
\end{example}

In addition to being recognized by \code{lambda} (and hence by
\code{defun}), the \code{values} declaration is recognized by all the
other special forms with bodies and declarations: \code{let},
\code{let*}, \code{labels} and \code{flet}.  Macros with declarations
usually splice the declarations into one of the above forms, so they
will accept this declaration too, but the exact effect of a
\code{values} declaration will depend on the macro.

If you declare the types of all arguments to a function, and also
declare the return value types with \code{values}, you have described
the type of the function.  \python{} will use this argument and result
type information to derive a function type that will then be applied
to calls of the function (\pxlref{function-types}.)  This provides a
way to declare the types of functions that is much less syntactically
awkward than using the \code{ftype} declaration with a \code{function}
type specifier.

Although the \code{values} declaration is non-standard, it is
relatively harmless to use it in otherwise portable code, since any
warning in non-CMU implementations can be suppressed with the standard
\code{declaration} proclamation.


\subsection{Structure Types}
\label{structure-types}
\cindex{structure types}
\cindex{defstruct types}
\cpsubindex{types}{structure}

Because of precise type checking, structure types are much better
supported by \python{} than by conventional compilers:

\begin{itemize}  
\item The structure argument to structure accessors is precisely
  checked\dash{}if you call \code{foo-a} on a \code{bar}, an error
  will be signaled.
  
\item The types of slot values are precisely checked\dash{}if you pass
  the wrong type argument to a constructor or a slot setter, then an
  error will be signaled.
\end{itemize}

This error checking is tremendously useful for detecting bugs in
programs that manipulate complex data structures.

An additional advantage of checking structure types and enforcing slot
types is that the compiler can safely believe slot type declarations.
\python{} effectively moves the type checking from the slot access to
the slot setter or constructor call.  This is more efficient since
caller of the setter or constructor often knows the type of the value,
entirely eliminating the need to check the value's type.  Consider
this example:

\begin{lisp}
(defstruct coordinate
  (x nil :type single-float)
  (y nil :type single-float))

(defun make-it ()
  (make-coordinate :x 1.0 :y 1.0))

(defun use-it (it)
  (declare (type coordinate it))
  (sqrt (expt (coordinate-x it) 2) (expt (coordinate-y it) 2)))
\end{lisp}

\code{make-it} and \code{use-it} are compiled with no checking on the
types of the float slots, yet \code{use-it} can use
\code{single-float} arithmetic with perfect safety.  Note that
\code{make-coordinate} must still check the values of \code{x} and
\code{y} unless the call is block compiled or inline expanded
(\pxlref{local-call}.)  But even without this advantage, it is almost
always more efficient to check slot values on structure
initialization, since slots are usually written once and read many
times.


\subsection{The Freeze-Type Declaration}
\cindex{freeze-type declaration}
\label{freeze-type}

The \code{extensions:freeze-type} declaration is a \cmucl{} extension that
enables more efficient compilation of user-defined types by asserting
that the definition is not going to change.  This declaration may only
be used globally (with \code{declaim} or \code{proclaim}).  Currently
\code{freeze-type} only affects structure type testing done by
\code{typep}, \code{typecase}, etc.  Here is an example:

\begin{lisp}
(declaim (freeze-type foo bar))
\end{lisp}

This asserts that the types \code{foo} and \code{bar} and their
subtypes are not going to change.  This allows more efficient type
testing, since the compiler can open-code a test for all possible
subtypes, rather than having to examine the type hierarchy at
run-time.


\subsection{Type Restrictions}
\cpsubindex{types}{restrictions on}

Avoid use of the \code{and}, \code{not} and \code{satisfies} types in
declarations, since type inference has problems with them.  When these
types do appear in a declaration, they are still checked precisely,
but the type information is of limited use to the compiler.
\code{and} types are effective as long as the intersection can be
canonicalized to a type that doesn't use \code{and}.  For example:

\begin{example}
(and fixnum unsigned-byte)
\end{example}

is fine, since it is the same as:

\begin{example}
(integer 0 \var{most-positive-fixnum})
\end{example}

but this type:

\begin{example}
(and symbol (not (member :end)))
\end{example}

will not be fully understood by type interference since the \code{and}
can't be removed by canonicalization.

Using any of these type specifiers in a type test with \code{typep} or
\code{typecase} is fine, since as tests, these types can be translated
into the \code{and} macro, the \code{not} function or a call to the
satisfies predicate.


\subsection{Type Style Recommendations}
\cindex{style recommendations}

\python{} provides good support for some currently unconventional ways of
using the \clisp{} type system.  With \python{}, it is desirable to make
declarations as precise as possible, but type inference also makes
some declarations unnecessary.  Here are some general guidelines for
maximum robustness and efficiency:
\begin{itemize}
  
\item Declare the types of all function arguments and structure slots
  as precisely as possible (while avoiding \code{not}, \code{and} and
  \code{satisfies}).  Put these declarations in during initial coding
  so that type assertions can find bugs for you during debugging.
  
\item Use the \tindexed{member} type specifier where there are a small
  number of possible symbol values, for example: \w{\code{(member :red
      :blue :green)}}.
  
\item Use the \tindexed{or} type specifier in situations where the
  type is not certain, but there are only a few possibilities, for
  example: \w{\code{(or list vector)}}.
  
\item Declare integer types with the tightest bounds that you can,
  such as \code{\w{(integer 3 7)}}.
  
\item Define \findexed{deftype} or \findexed{defstruct} types before
  they are used.  Definition after use is legal (producing no
  ``undefined type'' warnings), but type tests and structure
  operations will be compiled much less efficiently.
  
\item Use the \code{extensions:freeze-type} declaration to speed up
  type testing for structure types which won't have new subtypes added
  later. \xlref{freeze-type}
  
\item In addition to declaring the array element type and simpleness,
  also declare the dimensions if they are fixed, for example:
  \begin{example}
    (simple-array single-float (1024 1024))
  \end{example}
  This bounds information allows array indexing for multi-dimensional
  arrays to be compiled much more efficiently, and may also allow
  array bounds checking to be done at compile time.
  \xlref{array-types}.

\item Avoid use of the \findexed{the} declaration within expressions.
  Not only does it clutter the code, but it is also almost worthless
  under safe policies.  If the need for an output type assertion is
  revealed by efficiency notes during tuning, then you can consider
  \code{the}, but it is preferable to constrain the argument types
  more, allowing the compiler to prove the desired result type.
  
\item Don't bother declaring the type of \findexed{let} or other
  non-argument variables unless the type is non-obvious.  If you
  declare function return types and structure slot types, then the
  type of a variable is often obvious both to the programmer and to
  the compiler.  An important case where the type isn't obvious, and a
  declaration is appropriate, is when the value for a variable is
  pulled out of untyped structure (e.g., the result of \code{car}), or
  comes from some weakly typed function, such as \code{read}.
  
\item Declarations are sometimes necessary for integer loop variables,
  since the compiler can't always prove that the value is of a good
  integer type.  These declarations are best added during tuning, when
  an efficiency note indicates the need.
\end{itemize}


\section{Type Inference}
\label{type-inference}
\cindex{type inference}
\cindex{inference of types}
\cindex{derivation of types}

Type inference is the process by which the compiler tries to figure
out the types of expressions and variables, given an inevitable lack
of complete type information.  Although \python{} does much more type
inference than most \llisp{} compilers, remember that the more precise
and comprehensive type declarations are, the more type inference will
be able to do.


\subsection{Variable Type Inference}
\label{variable-type-inference}

The type of a variable is the union of the types of all the
definitions.  In the degenerate case of a let, the type of the
variable is the type of the initial value.  This inferred type is
intersected with any declared type, and is then propagated to all the
variable's references.  The types of \findexed{multiple-value-bind}
variables are similarly inferred from the types of the individual
values of the values form.

If multiple type declarations apply to a single variable, then all the
declarations must be correct; it is as though all the types were intersected
producing a single \tindexed{and} type specifier.  In this example:
\begin{example}
(defmacro my-dotimes ((var count) &body body)
  `(do ((,var 0 (1+ ,var)))
       ((>= ,var ,count))
     (declare (type (integer 0 *) ,var))
     ,@body))

(my-dotimes (i ...)
  (declare (fixnum i))
  ...)
\end{example}
the two declarations for \code{i} are intersected, so \code{i} is
known to be a non-negative fixnum.

In practice, this type inference is limited to lets and local
functions, since the compiler can't analyze all the calls to a global
function.  But type inference works well enough on local variables so
that it is often unnecessary to declare the type of local variables.
This is especially likely when function result types and structure
slot types are declared.  The main areas where type inference breaks
down are:
\begin{itemize}
  
\item When the initial value of a variable is a untyped expression,
  such as \code{\w{(car x)}}, and
  
\item When the type of one of the variable's definitions is a function
  of the variable's current value, as in: \code{(setq x (1+ x))}
\end{itemize}


\subsection{Local Function Type Inference}
\cpsubindex{local call}{type inference}

The types of arguments to local functions are inferred in the same was
as any other local variable; the type is the union of the argument
types across all the calls to the function, intersected with the
declared type.  If there are any assignments to the argument
variables, the type of the assigned value is unioned in as well.

The result type of a local function is computed in a special way that
takes tail recursion (\pxlref{tail-recursion}) into consideration.
The result type is the union of all possible return values that aren't
tail-recursive calls.  For example, \python{} will infer that the
result type of this function is \code{integer}:

\begin{lisp}
(defun ! (n res)
  (declare (integer n res))
  (if (zerop n)
      res
      (! (1- n) (* n res))))
\end{lisp}

Although this is a rather obvious result, it becomes somewhat less
trivial in the presence of mutual tail recursion of multiple
functions.  Local function result type inference interacts with the
mechanisms for ensuring proper tail recursion mentioned in section
\ref{local-call-return}.


\subsection{Global Function Type Inference}
\label{function-type-inference}
\cpsubindex{function}{type inference}

As described in section \ref{function-types}, a global function type
(\tindexed{ftype}) declaration places implicit type assertions on the
call arguments, and also guarantees the type of the return value.  So
wherever a call to a declared function appears, there is no doubt as
to the types of the arguments and return value.  Furthermore,
\python{} will infer a function type from the function's definition if
there is no \code{ftype} declaration.  Any type declarations on the
argument variables are used as the argument types in the derived
function type, and the compiler's best guess for the result type of
the function is used as the result type in the derived function type.

This method of deriving function types from the definition implicitly assumes
that functions won't be redefined at run-time.  Consider this example:
\begin{lisp}
(defun foo-p (x)
  (let ((res (and (consp x) (eq (car x) 'foo))))
    (format t "It is ~:[not ~;~]foo." res)))

(defun frob (it)
  (if (foo-p it)
      (setf (cadr it) 'yow!)
      (1+ it)))
\end{lisp}

Presumably, the programmer really meant to return \code{res} from
\code{foo-p}, but he seems to have forgotten.  When he tries to call
do \code{\w{(frob (list 'foo nil))}}, \code{frob} will flame out when
it tries to add to a \code{cons}.  Realizing his error, he fixes
\code{foo-p} and recompiles it.  But when he retries his test case, he
is baffled because the error is still there.  What happened in this
example is that \python{} proved that the result of \code{foo-p} is
\code{null}, and then proceeded to optimize away the \code{setf} in
\code{frob}.

Fortunately, in this example, the error is detected at compile time
due to notes about unreachable code (\pxlref{dead-code-notes}.)
Still, some users may not want to worry about this sort of problem
during incremental development, so there is a variable to control
deriving function types.

\begin{defvar}{extensions:}{derive-function-types}
  
  If true (the default), argument and result type information derived
  from compilation of \code{defun}s is used when compiling calls to
  that function.  If false, only information from \code{ftype}
  proclamations will be used.
\end{defvar}


\subsection{Operation Specific Type Inference}
\label{operation-type-inference}
\cindex{operation specific type inference}
\cindex{arithmetic type inference}
\cpsubindex{numeric}{type inference}

Many of the standard \clisp{} functions have special type inference
procedures that determine the result type as a function of the
argument types.  For example, the result type of \code{aref} is the
array element type.  Here are some other examples of type inferences:
\begin{lisp}
(logand x #xFF) \result{} (unsigned-byte 8)

(+ (the (integer 0 12) x) (the (integer 0 1) y)) \result{} (integer 0 13)

(ash (the (unsigned-byte 16) x) -8) \result{} (unsigned-byte 8)
\end{lisp}


\subsection{Dynamic Type Inference}
\label{constraint-propagation}
\cindex{dynamic type inference}
\cindex{conditional type inference}
\cpsubindex{type inference}{dynamic}

\python{} uses flow analysis to infer types in dynamically typed
programs.  For example:

\begin{example}
(ecase x
  (list (length x))
  ...)
\end{example}

Here, the compiler knows the argument to \code{length} is a list,
because the call to \code{length} is only done when \code{x} is a
list.  The most significant efficiency effect of inference from
assertions is usually in type check optimization.

Dynamic type inference has two inputs: explicit conditionals and
implicit or explicit type assertions.  Flow analysis propagates these
constraints on variable type to any code that can be executed only
after passing though the constraint.  Explicit type constraints come
from \findexed{if}s where the test is either a lexical variable or a
function of lexical variables and constants, where the function is
either a type predicate, a numeric comparison or \code{eq}.

If there is an \code{eq} (or \code{eql}) test, then the compiler will
actually substitute one argument for the other in the true branch.
For example:
\begin{lisp}
(when (eq x :yow!) (return x))
\end{lisp}
becomes:
\begin{lisp}
(when (eq x :yow!) (return :yow!))
\end{lisp}
This substitution is done when one argument is a constant, or one
argument has better type information than the other.  This
transformation reveals opportunities for constant folding or
type-specific optimizations.  If the test is against a constant, then
the compiler can prove that the variable is not that constant value in
the false branch, or \w{\code{(not (member :yow!))}}  in the example
above.  This can eliminate redundant tests, for example:
\begin{example}
(if (eq x nil)
    ...
    (if x a b))
\end{example}
is transformed to this:
\begin{example}
(if (eq x nil)
    ...
    a)
\end{example}
Variables appearing as \code{if} tests are interpreted as
\code{\w{(not (eq \var{var} nil))}} tests.  The compiler also converts
\code{=} into \code{eql} where possible.  It is difficult to do
inference directly on \code{=} since it does implicit coercions.

When there is an explicit \code{$<$} or \code{$>$} test on numeric
variables, the compiler makes inferences about the ranges the
variables can assume in the true and false branches. This is mainly
useful when it proves that the values are small enough in magnitude to
allow open-coding of arithmetic operations. For example, in many uses
of \code{dotimes} with a \code{fixnum} repeat count, the compiler
proves that fixnum arithmetic can be used.

Implicit type assertions are quite common, especially if you declare
function argument types.  Dynamic inference from implicit type
assertions sometimes helps to disambiguate programs to a useful
degree, but is most noticeable when it detects a dynamic type error.
For example:

\begin{lisp}
(defun foo (x)
  (+ (car x) x))
\end{lisp} 

results in this warning:

\begin{example}
In: DEFUN FOO
  (+ (CAR X) X)
==>
  X
Warning: Result is a LIST, not a NUMBER.
\end{example}

Note that \llisp{}'s dynamic type checking semantics make dynamic type
inference useful even in programs that aren't really dynamically
typed, for example:

\begin{lisp}
(+ (car x) (length x))
\end{lisp}

Here, \code{x} presumably always holds a list, but in the absence of a
declaration the compiler cannot assume \code{x} is a list simply
because list-specific operations are sometimes done on it.  The
compiler must consider the program to be dynamically typed until it
proves otherwise.  Dynamic type inference proves that the argument to
\code{length} is always a list because the call to \code{length} is
only done after the list-specific \code{car} operation.


\subsection{Type Check Optimization}
\label{type-check-optimization}
\cpsubindex{type checking}{optimization}
\cpsubindex{optimization}{type check}

\python{} backs up its support for precise type checking by minimizing
the cost of run-time type checking.  This is done both through type
inference and though optimizations of type checking itself.

Type inference often allows the compiler to prove that a value is of
the correct type, and thus no type check is necessary.  For example:
\begin{lisp}
(defstruct foo a b c)
(defstruct link
  (foo (required-argument) :type foo)
  (next nil :type (or link null)))

(foo-a (link-foo x))
\end{lisp}

Here, there is no need to check that the result of \code{link-foo} is
a \code{foo}, since it always is.  Even when some type checks are
necessary, type inference can often reduce the number:
\begin{example}
(defun test (x)
  (let ((a (foo-a x))
        (b (foo-b x))
        (c (foo-c x)))
    ...))
\end{example}
In this example, only one \w{\code{(foo-p x)}} check is needed.  This
applies to a lesser degree in list operations, such as:
\begin{lisp}
(if (eql (car x) 3) (cdr x) y)
\end{lisp}
Here, we only have to check that \code{x} is a list once.

Since \python{} recognizes explicit type tests, code that explicitly
protects itself against type errors has little introduced overhead due
to implicit type checking.  For example, this loop compiles with no
implicit checks checks for \code{car} and \code{cdr}:
\begin{lisp}
(defun memq (e l)
  (do ((current l (cdr current)))
      ((atom current) nil)
    (when (eq (car current) e) (return current))))
\end{lisp}

\cindex{complemented type checks}
\python{} reduces the cost of checks that must be done through an
optimization called \var{complementing}.  A complemented check for
\var{type} is simply a check that the value is not of the type
\w{\code{(not \var{type})}}.  This is only interesting when something
is known about the actual type, in which case we can test for the
complement of \w{\code{(and \var{known-type} (not \var{type}))}}, or
the difference between the known type and the assertion.  An example:
\begin{lisp}
(link-foo (link-next x))
\end{lisp}
Here, we change the type check for \code{link-foo} from a test for
\code{foo} to a test for:
\begin{lisp}
(not (and (or foo null) (not foo)))
\end{lisp}
or more simply \w{\code{(not null)}}.  This is probably the most
important use of complementing, since the situation is fairly common,
and a \code{null} test is much cheaper than a structure type test.

Here is a more complicated example that illustrates the combination of
complementing with dynamic type inference:
\begin{lisp}
(defun find-a (a x)
  (declare (type (or link null) x))
  (do ((current x (link-next current)))
      ((null current) nil)
    (let ((foo (link-foo current)))
      (when (eq (foo-a foo) a) (return foo)))))
\end{lisp}
This loop can be compiled with no type checks.  The \code{link} test
for \code{link-foo} and \code{link-next} is complemented to
\w{\code{(not null)}}, and then deleted because of the explicit
\code{null} test.  As before, no check is necessary for \code{foo-a},
since the \code{link-foo} is always a \code{foo}.  This sort of
situation shows how precise type checking combined with precise
declarations can actually result in reduced type checking.


\section{Source Optimization}
\label{source-optimization}
\cindex{optimization}

This section describes source-level transformations that \python{} does on
programs in an attempt to make them more efficient.  Although source-level
optimizations can make existing programs more efficient, the biggest advantage
of this sort of optimization is that it makes it easier to write efficient
programs.  If a clean, straightforward implementation is can be transformed
into an efficient one, then there is no need for tricky and dangerous hand
optimization. 


\subsection{Let Optimization}
\label{let-optimization}

\cindex{let optimization} \cpsubindex{optimization}{let}

The primary optimization of let variables is to delete them when they
are unnecessary.  Whenever the value of a let variable is a constant,
a constant variable or a constant (local or non-notinline) function,
the variable is deleted, and references to the variable are replaced
with references to the constant expression.  This is useful primarily
in the expansion of macros or inline functions, where argument values
are often constant in any given call, but are in general non-constant
expressions that must be bound to preserve order of evaluation.  Let
variable optimization eliminates the need for macros to carefully
avoid spurious bindings, and also makes inline functions just as
efficient as macros.

A particularly interesting class of constant is a local function.
Substituting for lexical variables that are bound to a function can
substantially improve the efficiency of functional programming styles,
for example:
\begin{lisp}
(let ((a #'(lambda (x) (zow x))))
  (funcall a 3))
\end{lisp}
effectively transforms to:
\begin{lisp}
(zow 3)
\end{lisp}
This transformation is done even when the function is a closure, as in:
\begin{lisp}
(let ((a (let ((y (zug)))
           #'(lambda (x) (zow x y)))))
  (funcall a 3))
\end{lisp}
becoming:
\begin{lisp}
(zow 3 (zug))
\end{lisp}

A constant variable is a lexical variable that is never assigned to,
always keeping its initial value.  Whenever possible, avoid setting
lexical variables\dash{}instead bind a new variable to the new value.
Except for loop variables, it is almost always possible to avoid
setting lexical variables.  This form:
\begin{example}
(let ((x (f x)))
  ...)
\end{example}
is \var{more} efficient than this form:
\begin{example}
(setq x (f x))
...
\end{example}
Setting variables makes the program more difficult to understand, both
to the compiler and to the programmer.  \python{} compiles assignments
at least as efficiently as any other \llisp{} compiler, but most let
optimizations are only done on constant variables.

Constant variables with only a single use are also optimized away,
even when the initial value is not constant.\footnote{The source
  transformation in this example doesn't represent the preservation of
  evaluation order implicit in the compiler's internal representation.
  Where necessary, the back end will reintroduce temporaries to
  preserve the semantics.}  For example, this expansion of
\code{incf}:
\begin{lisp}
(let ((#:g3 (+ x 1)))
  (setq x #:G3))
\end{lisp}
becomes:
\begin{lisp}
(setq x (+ x 1))
\end{lisp}
The type semantics of this transformation are more important than the
elimination of the variable itself.  Consider what happens when
\code{x} is declared to be a \code{fixnum}; after the transformation,
the compiler can compile the addition knowing that the result is a
\code{fixnum}, whereas before the transformation the addition would
have to allow for fixnum overflow.

Another variable optimization deletes any variable that is never read.
This causes the initial value and any assigned values to be unused,
allowing those expressions to be deleted if they have no side-effects.

Note that a let is actually a degenerate case of local call
(\pxlref{let-calls}), and that let optimization can be done on calls
that weren't created by a let.  Also, local call allows an applicative
style of iteration that is totally assignment free.


\subsection{Constant Folding}
\cindex{constant folding}
\cpsubindex{folding}{constant}

Constant folding is an optimization that replaces a call of constant
arguments with the constant result of that call.  Constant folding is
done on all standard functions for which it is legal.  Inline
expansion allows folding of any constant parts of the definition, and
can be done even on functions that have side-effects.

It is convenient to rely on constant folding when programming, as in this
example:
\begin{example}
(defconstant limit 42)

(defun foo ()
  (... (1- limit) ...))
\end{example}
Constant folding is also helpful when writing macros or inline
functions, since it usually eliminates the need to write a macro that
special-cases constant arguments.

\cindex{constant-function declaration} Constant folding of a user
defined function is enabled by the \code{extensions:constant-function}
proclamation.  In this example:
\begin{example}
(declaim (ext:constant-function myfun))
(defun myexp (x y)
  (declare (single-float x y))
  (exp (* (log x) y)))

 ... (myexp 3.0 1.3) ...
\end{example}
The call to \code{myexp} is constant-folded to \code{4.1711674}.


\subsection{Unused Expression Elimination}
\cindex{unused expression elimination}
\cindex{dead code elimination}

If the value of any expression is not used, and the expression has no
side-effects, then it is deleted.  As with constant folding, this
optimization applies most often when cleaning up after inline
expansion and other optimizations.  Any function declared an
\code{extensions:constant-function} is also subject to unused
expression elimination.

Note that \python{} will eliminate parts of unused expressions known
to be side-effect free, even if there are other unknown parts.  For
example:
\begin{lisp}
(let ((a (list (foo) (bar))))
  (if t
      (zow)
      (raz a)))
\end{lisp}
becomes:
\begin{lisp}
(progn (foo) (bar))
(zow)
\end{lisp}


\subsection{Control Optimization}
\cindex{control optimization}
\cpsubindex{optimization}{control}

The most important optimization of control is recognizing when an
\findexed{if} test is known at compile time, then deleting the
\code{if}, the test expression, and the unreachable branch of the
\code{if}.  This can be considered a special case of constant folding,
although the test doesn't have to be truly constant as long as it is
definitely not \false.  Note also, that type inference propagates the
result of an \code{if} test to the true and false branches,
\pxlref{constraint-propagation}.

A related \code{if} optimization is this transformation:\footnote{Note
  that the code for \code{x} and \code{y} isn't actually replicated.}
\begin{lisp}
(if (if a b c) x y)
\end{lisp}
into:
\begin{lisp}
(if a
    (if b x y)
    (if c x y))
\end{lisp}
The opportunity for this sort of optimization usually results from a
conditional macro.  For example:
\begin{lisp}
(if (not a) x y)
\end{lisp}
is actually implemented as this:
\begin{lisp}
(if (if a nil t) x y)
\end{lisp}
which is transformed to this:
\begin{lisp}
(if a
    (if nil x y)
    (if t x y))
\end{lisp}
which is then optimized to this:
\begin{lisp}
(if a y x)
\end{lisp}
Note that due to \python{}'s internal representations, the
\code{if}\dash{}\code{if} situation will be recognized even if other
forms are wrapped around the inner \code{if}, like:
\begin{example}
(if (let ((g ...))
      (loop
        ...
        (return (not g))
        ...))
    x y)
\end{example}

In \python, all the \clisp{} macros really are macros, written in
terms of \code{if}, \code{block} and \code{tagbody}, so user-defined
control macros can be just as efficient as the standard ones.
\python{} emits basic blocks using a heuristic that minimizes the
number of unconditional branches.  The code in a \code{tagbody} will
not be emitted in the order it appeared in the source, so there is no
point in arranging the code to make control drop through to the
target.


\subsection{Unreachable Code Deletion}
\label{dead-code-notes}
\cindex{unreachable code deletion}
\cindex{dead code elimination}

\python{} will delete code whenever it can prove that the code can never be
executed.  Code becomes unreachable when:

\begin{itemize}
\item
An \code{if} is optimized away, or

\item
There is an explicit unconditional control transfer such as \code{go} or
\code{return-from}, or

\item
The last reference to a local function is deleted (or there never was any
reference.)
\end{itemize}

When code that appeared in the original source is deleted, the compiler prints
a note to indicate a possible problem (or at least unnecessary code.)  For
example:
\begin{lisp}
(defun foo ()
  (if t
      (write-line "True.")
      (write-line "False.")))
\end{lisp}
will result in this note:
\begin{example}
In: DEFUN FOO
  (WRITE-LINE "False.")
Note: Deleting unreachable code.
\end{example}

It is important to pay attention to unreachable code notes, since they often
indicate a subtle type error.  For example:
\begin{example}
(defstruct foo a b)

(defun lose (x)
  (let ((a (foo-a x))
        (b (if x (foo-b x) :none)))
    ...))
\end{example}
results in this note:
\begin{example}
In: DEFUN LOSE
  (IF X (FOO-B X) :NONE)
==>
  :NONE
Note: Deleting unreachable code.
\end{example}
The \kwd{none} is unreachable, because type inference knows that the argument
to \code{foo-a} must be a \code{foo}, and thus can't be \false.  Presumably the
programmer forgot that \code{x} could be \false{} when he wrote the binding for
\code{a}.

Here is an example with an incorrect declaration:
\begin{lisp}
(defun count-a (string)
  (do ((pos 0 (position #\back{a} string :start (1+ pos)))
       (count 0 (1+ count)))
      ((null pos) count)
    (declare (fixnum pos))))
\end{lisp}
This time our note is:
\begin{example}
In: DEFUN COUNT-A
  (DO ((POS 0 #) (COUNT 0 #))
      ((NULL POS) COUNT)
    (DECLARE (FIXNUM POS)))
--> BLOCK LET TAGBODY RETURN-FROM PROGN 
==>
  COUNT
Note: Deleting unreachable code.
\end{example}

The problem here is that \code{pos} can never be null since it is declared a
\code{fixnum}.

It takes some experience with unreachable code notes to be able to
tell what they are trying to say.  In non-obvious cases, the best
thing to do is to call the function in a way that should cause the
unreachable code to be executed.  Either you will get a type error, or
you will find that there truly is no way for the code to be executed.

Not all unreachable code results in a note:

\begin{itemize} 
\item A note is only given when the unreachable code textually appears
  in the original source.  This prevents spurious notes due to the
  optimization of macros and inline functions, but sometimes also
  foregoes a note that would have been useful.
  
\item Since accurate source information is not available for non-list
  forms, there is an element of heuristic in determining whether or
  not to give a note about an atom.  Spurious notes may be given when
  a macro or inline function defines a variable that is also present
  in the calling function.  Notes about \false{} and \true{} are never
  given, since it is too easy to confuse these constants in expanded
  code with ones in the original source.
  
\item Notes are only given about code unreachable due to control flow.
  There is no note when an expression is deleted because its value is
  unused, since this is a common consequence of other optimizations.
\end{itemize}


Somewhat spurious unreachable code notes can also result when a macro
inserts multiple copies of its arguments in different contexts, for
example:
\begin{lisp}
(defmacro t-and-f (var form)
  `(if ,var ,form ,form))

(defun foo (x)
  (t-and-f x (if x "True." "False.")))
\end{lisp}
results in these notes:
\begin{example}
In: DEFUN FOO
  (IF X "True." "False.")
==>
  "False."
Note: Deleting unreachable code.

==>
  "True."
Note: Deleting unreachable code.
\end{example}

It seems like it has deleted both branches of the \code{if}, but it has really
deleted one branch in one copy, and the other branch in the other copy.  Note
that these messages are only spurious in not satisfying the intent of the rule
that notes are only given when the deleted code appears in the original source;
there is always \var{some} code being deleted when a unreachable code note is
printed.


\subsection{Multiple Values Optimization}
\cindex{multiple value optimization}
\cpsubindex{optimization}{multiple value}

Within a function, \python{} implements uses of multiple values
particularly efficiently.  Multiple values can be kept in arbitrary
registers, so using multiple values doesn't imply stack manipulation
and representation conversion.  For example, this code:
\begin{example}
(let ((a (if x (foo x) u))
      (b (if x (bar x) v)))
  ...)
\end{example}
is actually more efficient written this way:
\begin{example}
(multiple-value-bind
    (a b)
    (if x
        (values (foo x) (bar x))
        (values u v))
  ...)
\end{example}

Also, \pxlref{local-call-return} for information on how local call
provides efficient support for multiple function return values.


\subsection{Source to Source Transformation}
\cindex{source-to-source transformation}
\cpsubindex{transformation}{source-to-source}

The compiler implements a number of operation-specific optimizations as
source-to-source transformations.  You will often see unfamiliar code in error
messages, for example:

\begin{lisp}
(defun my-zerop () (zerop x))
\end{lisp}

gives this warning:

\begin{example}
In: DEFUN MY-ZEROP
  (ZEROP X)
==>
  (= X 0)
Warning: Undefined variable: X
\end{example}

The original \code{zerop} has been transformed into a call to
\code{=}.  This transformation is indicated with the same \code{==$>$}
used to mark macro and function inline expansion.  Although it can be
confusing, display of the transformed source is important, since
warnings are given with respect to the transformed source.  This a
more obscure example:

\begin{lisp}
(defun foo (x) (logand 1 x))
\end{lisp}

gives this efficiency note:

\begin{example}
In: DEFUN FOO
  (LOGAND 1 X)
==>
  (LOGAND C::Y C::X)
Note: Forced to do static-function Two-arg-and (cost 53).
      Unable to do inline fixnum arithmetic (cost 1) because:
      The first argument is a INTEGER, not a FIXNUM.
      etc.
\end{example}

Here, the compiler commuted the call to \code{logand}, introducing
temporaries.  The note complains that the \var{first} argument is not
a \code{fixnum}, when in the original call, it was the second
argument.  To make things more confusing, the compiler introduced
temporaries called \code{c::x} and \code{c::y} that are bound to
\code{y} and \code{1}, respectively.

You will also notice source-to-source optimizations when efficiency
notes are enabled (\pxlref{efficiency-notes}.)  When the compiler is
unable to do a transformation that might be possible if there was more
information, then an efficiency note is printed.  For example,
\code{my-zerop} above will also give this efficiency note:
\begin{example}
In: DEFUN FOO
  (ZEROP X)
==>
  (= X 0)
Note: Unable to optimize because:
      Operands might not be the same type, so can't open code.
\end{example}


\subsection{Style Recommendations}
\cindex{style recommendations}

Source level optimization makes possible a clearer and more relaxed programming
style:
\begin{itemize}
  
\item Don't use macros purely to avoid function call.  If you want an
  inline function, write it as a function and declare it inline.  It's
  clearer, less error-prone, and works just as well.
  
\item Don't write macros that try to ``optimize'' their expansion in
  trivial ways such as avoiding binding variables for simple
  expressions.  The compiler does these optimizations too, and is less
  likely to make a mistake.
  
\item Make use of local functions (i.e., \code{labels} or \code{flet})
  and tail-recursion in places where it is clearer.  Local function
  call is faster than full call.
  
\item Avoid setting local variables when possible.  Binding a new
  \code{let} variable is at least as efficient as setting an existing
  variable, and is easier to understand, both for the compiler and the
  programmer.
  
\item Instead of writing similar code over and over again so that it
  can be hand customized for each use, define a macro or inline
  function, and let the compiler do the work.
\end{itemize}


\section{Tail Recursion}
\label{tail-recursion}
\cindex{tail recursion}
\cindex{recursion}

A call is tail-recursive if nothing has to be done after the the call
returns, i.e. when the call returns, the returned value is immediately
returned from the calling function.  In this example, the recursive
call to \code{myfun} is tail-recursive:
\begin{lisp}
(defun myfun (x)
  (if (oddp (random x))
      (isqrt x)
      (myfun (1- x))))
\end{lisp}

Tail recursion is interesting because it is form of recursion that can be
implemented much more efficiently than general recursion.  In general, a
recursive call requires the compiler to allocate storage on the stack at
run-time for every call that has not yet returned.  This memory consumption
makes recursion unacceptably inefficient for representing repetitive algorithms
having large or unbounded size.  Tail recursion is the special case of
recursion that is semantically equivalent to the iteration constructs normally
used to represent repetition in programs.  Because tail recursion is equivalent
to iteration, tail-recursive programs can be compiled as efficiently as
iterative programs.

So why would you want to write a program recursively when you can write it
using a loop?  Well, the main answer is that recursion is a more general
mechanism, so it can express some solutions simply that are awkward to write as
a loop.  Some programmers also feel that recursion is a stylistically
preferable way to write loops because it avoids assigning variables.
For example, instead of writing:

\begin{lisp}
(defun fun1 (x)
  something-that-uses-x)

(defun fun2 (y)
  something-that-uses-y)

(do ((x something (fun2 (fun1 x))))
    (nil))
\end{lisp}

You can write:

\begin{lisp}
(defun fun1 (x)
  (fun2 something-that-uses-x))

(defun fun2 (y)
  (fun1 something-that-uses-y))

(fun1 something)
\end{lisp}

The tail-recursive definition is actually more efficient, in addition to being
(arguably) clearer.  As the number of functions and the complexity of their
call graph increases, the simplicity of using recursion becomes compelling.
Consider the advantages of writing a large finite-state machine with separate
tail-recursive functions instead of using a single huge \code{prog}.

It helps to understand how to use tail recursion if you think of a
tail-recursive call as a \code{psetq} that assigns the argument values to the
called function's variables, followed by a \code{go} to the start of the called
function.  This makes clear an inherent efficiency advantage of tail-recursive
call: in addition to not having to allocate a stack frame, there is no need to
prepare for the call to return (e.g., by computing a return PC.)

Is there any disadvantage to tail recursion?  Other than an increase
in efficiency, the only way you can tell that a call has been compiled
tail-recursively is if you use the debugger.  Since a tail-recursive
call has no stack frame, there is no way the debugger can print out
the stack frame representing the call.  The effect is that backtrace
will not show some calls that would have been displayed in a
non-tail-recursive implementation.  In practice, this is not as bad as
it sounds\dash{}in fact it isn't really clearly worse, just different.
\xlref{debug-tail-recursion} for information about the debugger
implications of tail recursion, and how to turn it off for the sake of
more conservative backtrace information.

In order to ensure that tail-recursion is preserved in arbitrarily
complex calling patterns across separately compiled functions, the
compiler must compile any call in a tail-recursive position as a
tail-recursive call.  This is done regardless of whether the program
actually exhibits any sort of recursive calling pattern.  In this
example, the call to \code{fun2} will always be compiled as a
tail-recursive call:

\begin{lisp}
(defun fun1 (x)
  (fun2 x))
\end{lisp}

So tail recursion doesn't necessarily have anything to do with recursion
as it is normally thought of.  \xlref{local-tail-recursion} for more
discussion of using tail recursion to implement loops.


\subsection{Tail Recursion Exceptions}

Although \python{} is claimed to be ``properly'' tail-recursive, some
might dispute this, since there are situations where tail recursion is
inhibited:
\begin{itemize}
  
\item When the call is enclosed by a special binding, or
  
\item When the call is enclosed by a \code{catch} or
  \code{unwind-protect}, or
  
\item When the call is enclosed by a \code{block} or \code{tagbody}
  and the block name or \code{go} tag has been closed over.
\end{itemize}
These dynamic extent binding forms inhibit tail recursion because they
allocate stack space to represent the binding.  Shallow-binding
implementations of dynamic scoping also require cleanup code to be
evaluated when the scope is exited.

In addition, optimization of tail-recursive calls is inhibited when
the \code{debug} optimization quality is greater than \code{2}
(\pxlref{debugger-policy}.)


\section{Local Call}
\label{local-call}
\cindex{local call}
\cpsubindex{call}{local}
\cpsubindex{function call}{local}

\python{} supports two kinds of function call: full call and local call.
Full call is the standard calling convention; its late binding and
generality make \llisp{} what it is, but create unavoidable overheads.
When the compiler can compile the calling function and the called
function simultaneously, it can use local call to avoid some of the
overhead of full call.  Local call is really a collection of
compilation strategies.  If some aspect of call overhead is not needed
in a particular local call, then it can be omitted.  In some cases,
local call can be totally free.  Local call provides two main
advantages to the user:
\begin{itemize}
  
\item Local call makes the use of the lexical function binding forms
  \findexed{flet} and \findexed{labels} much more efficient.  A local
  call is always faster than a full call, and in many cases is much
  faster.
  
\item Local call is a natural approach to \textit{block compilation}, a
  compilation technique that resolves function references at compile
  time.  Block compilation speeds function call, but increases
  compilation times and prevents function redefinition.
\end{itemize}



\subsection{Self-Recursive Calls}
\cpsubindex{recursion}{self}

Local call is used when a function defined by \code{defun} calls itself.  For
example:
\begin{lisp}
(defun fact (n)
  (if (zerop n)
      1
      (* n (fact (1- n)))))
\end{lisp}

This use of local call speeds recursion, but can also complicate
debugging, since \findexed{trace} will only show the first call to
\code{fact}, and not the recursive calls.  This is because the
recursive calls directly jump to the start of the function, and don't
indirect through the \code{symbol-function}.  Self-recursive local
call is inhibited when the \kwd{block-compile} argument to
\code{compile-file} is \false{} (\pxlref{compile-file-block}.)


\subsection{Let Calls}
\label{let-calls}
Because local call avoids unnecessary call overheads, the compiler
internally uses local call to implement some macros and special forms
that are not normally thought of as involving a function call.  For
example, this \code{let}:

\begin{example}
(let ((a (foo))
      (b (bar)))
  ...)
\end{example}

is internally represented as though it was macroexpanded into:

\begin{example}
(funcall #'(lambda (a b)
             ...)
         (foo)
         (bar))
\end{example}

This implementation is acceptable because the simple cases of local
call (equivalent to a \code{let}) result in good code.  This doesn't
make \code{let} any more efficient, but does make local calls that are
semantically the same as \code{let} much more efficient than full
calls.  For example, these definitions are all the same as far as the
compiler is concerned:

\begin{example}
(defun foo ()
  ...some other stuff...
  (let ((a something))
    ...some stuff...))

(defun foo ()
  (flet ((localfun (a)
           ...some stuff...))
    ...some other stuff...
    (localfun something)))

(defun foo ()
  (let ((funvar #'(lambda (a)
                    ...some stuff...)))
    ...some other stuff...
    (funcall funvar something)))
\end{example}

Although local call is most efficient when the function is called only
once, a call doesn't have to be equivalent to a \code{let} to be more
efficient than full call.  All local calls avoid the overhead of
argument count checking and keyword argument parsing, and there are a
number of other advantages that apply in many common situations.
\xlref{let-optimization} for a discussion of the optimizations done on
let calls.


\subsection{Closures}
\cindex{closures}

Local call allows for much more efficient use of closures, since the
closure environment doesn't need to be allocated on the heap, or even
stored in memory at all.  In this example, there is no penalty for
\code{localfun} referencing \code{a} and \code{b}:
\begin{lisp}
(defun foo (a b)
  (flet ((localfun (x)
           (1+ (* a b x))))
    (if (= a b)
        (localfun (- x))
        (localfun x))))
\end{lisp}
In local call, the compiler effectively passes closed-over values as
extra arguments, so there is no need for you to ``optimize'' local
function use by explicitly passing in lexically visible values.
Closures may also be subject to let optimization
(\pxlref{let-optimization}.)

Note: indirect value cells are currently always allocated on the heap
when a variable is both assigned to (with \code{setq} or \code{setf})
and closed over, regardless of whether the closure is a local function
or not.  This is another reason to avoid setting variables when you
don't have to.


\subsection{Local Tail Recursion}
\label{local-tail-recursion}
\cindex{tail recursion}
\cpsubindex{recursion}{tail}

Tail-recursive local calls are particularly efficient, since they are
in effect an assignment plus a control transfer.  Scheme programmers
write loops with tail-recursive local calls, instead of using the
imperative \code{go} and \code{setq}.  This has not caught on in the
\clisp{} community, since conventional \llisp{} compilers don't
implement local call.  In \python, users can choose to write loops
such as:
\begin{lisp}
(defun ! (n)
  (labels ((loop (n total)
             (if (zerop n)
                 total
                 (loop (1- n) (* n total)))))
    (loop n 1)))
\end{lisp}

\begin{defmac}{extensions:}{iterate}{%
    \args{\var{name} (\mstar{(\var{var} \var{initial-value})})
      \mstar{\var{declaration}} \mstar{\var{form}}}}
  
  This macro provides syntactic sugar for using \findexed{labels} to
  do iteration.  It creates a local function \var{name} with the
  specified \var{var}s as its arguments and the \var{declaration}s and
  \var{form}s as its body.  This function is then called with the
  \var{initial-values}, and the result of the call is return from the
  macro.

  Here is our factorial example rewritten using \code{iterate}:

  \begin{lisp}
    (defun ! (n)
      (iterate loop
               ((n n)
               (total 1))
        (if (zerop n)
          total
          (loop (1- n) (* n total)))))
  \end{lisp}
      
  The main advantage of using \code{iterate} over \code{do} is that
  \code{iterate} naturally allows stepping to be done differently
  depending on conditionals in the body of the loop.  \code{iterate}
  can also be used to implement algorithms that aren't really
  iterative by simply doing a non-tail call.  For example, the
  standard recursive definition of factorial can be written like this:
\begin{lisp}
(iterate fact
         ((n n))
  (if (zerop n)
      1
      (* n (fact (1- n)))))
\end{lisp}
\end{defmac}


\subsection{Return Values}
\label{local-call-return}
\cpsubindex{return values}{local call}
\cpsubindex{local call}{return values}

One of the more subtle costs of full call comes from allowing
arbitrary numbers of return values.  This overhead can be avoided in
local calls to functions that always return the same number of values.
For efficiency reasons (as well as stylistic ones), you should write
functions so that they always return the same number of values.  This
may require passing extra \false{} arguments to \code{values} in some
cases, but the result is more efficient, not less so.

When efficiency notes are enabled (\pxlref{efficiency-notes}), and the
compiler wants to use known values return, but can't prove that the
function always returns the same number of values, then it will print
a note like this:
\begin{example}
In: DEFUN GRUE
  (DEFUN GRUE (X) (DECLARE (FIXNUM X)) (COND (# #) (# NIL) (T #)))
Note: Return type not fixed values, so can't use known return convention:
  (VALUES (OR (INTEGER -536870912 -1) NULL) &REST T)
\end{example}

In order to implement proper tail recursion in the presence of known
values return (\pxlref{tail-recursion}), the compiler sometimes must
prove that multiple functions all return the same number of values.
When this can't be proven, the compiler will print a note like this:
\begin{example}
In: DEFUN BLUE
  (DEFUN BLUE (X) (DECLARE (FIXNUM X)) (COND (# #) (# #) (# #) (T #)))
Note: Return value count mismatch prevents known return from
      these functions:
  BLUE
  SNOO
\end{example}
\xlref{number-local-call} for the interaction between local call
and the representation of numeric types.


\section{Block Compilation}
\label{block-compilation}
\cindex{block compilation}
\cpsubindex{compilation}{block}

Block compilation allows calls to global functions defined by
\findexed{defun} to be compiled as local calls.  The function call
can be in a different top-level form than the \code{defun}, or even in a
different file.

In addition, block compilation allows the declaration of the \textit{entry points}
to the block compiled portion.  An entry point is any function that may be
called from outside of the block compilation.  If a function is not an entry
point, then it can be compiled more efficiently, since all calls are known at
compile time.  In particular, if a function is only called in one place, then
it will be let converted.  This effectively inline expands the function, but
without the code duplication that results from defining the function normally
and then declaring it inline.

The main advantage of block compilation is that it it preserves efficiency in
programs even when (for readability and syntactic convenience) they are broken
up into many small functions.  There is absolutely no overhead for calling a
non-entry point function that is defined purely for modularity (i.e. called
only in one place.)

Block compilation also allows the use of non-descriptor arguments and return
values in non-trivial programs (\pxlref{number-local-call}).


\subsection{Block Compilation Semantics}

The effect of block compilation can be envisioned as the compiler turning all
the \code{defun}s in the block compilation into a single \code{labels} form:
\begin{example}
(declaim (start-block fun1 fun3))

(defun fun1 ()
  ...)

(defun fun2 ()
  ...
  (fun1)
  ...)

(defun fun3 (x)
  (if x
      (fun1)
      (fun2)))

(declaim (end-block))
\end{example}
becomes:
\begin{example}
(labels ((fun1 ()
           ...)
         (fun2 ()
           ...
           (fun1)
           ...)
         (fun3 (x)
           (if x
               (fun1)
               (fun2))))
  (setf (fdefinition 'fun1) #'fun1)
  (setf (fdefinition 'fun3) #'fun3))
\end{example}
Calls between the block compiled functions are local calls, so changing the
global definition of \code{fun1} will have no effect on what \code{fun2} does;
\code{fun2} will keep calling the old \code{fun1}.

The entry points \code{fun1} and \code{fun3} are still installed in
the \code{symbol-function} as the global definitions of the functions,
so a full call to an entry point works just as before.  However,
\code{fun2} is not an entry point, so it is not globally defined.  In
addition, \code{fun2} is only called in one place, so it will be let
converted.


\subsection{Block Compilation Declarations}
\cpsubindex{declarations}{block compilation}
\cindex{start-block declaration}
\cindex{end-block declaration}

The \code{extensions:start-block} and \code{extensions:end-block}
declarations allow fine-grained control of block compilation.  These
declarations are only legal as a global declarations (\code{declaim}
or \code{proclaim}).

\noindent
\vspace{1 em}
The \code{start-block} declaration has this syntax:
\begin{example}
(start-block \mstar{\var{entry-point-name}})
\end{example}
When processed by the compiler, this declaration marks the start of
block compilation, and specifies the entry points to that block.  If
no entry points are specified, then \var{all} functions are made into
entry points.  If already block compiling, then the compiler ends the
current block and starts a new one.

\noindent
\vspace{1 em}
The \code{end-block} declaration has no arguments:
\begin{lisp}
(end-block)
\end{lisp}
The \code{end-block} declaration ends a block compilation unit without
starting a new one.  This is useful mainly when only a portion of a file
is worth block compiling.


\subsection{Compiler Arguments}
\label{compile-file-block}
\cpsubindex{compile-file}{block compilation arguments}

The \kwd{block-compile} and \kwd{entry-points} arguments to
\code{extensions:compile-from-stream} and \funref{compile-file} provide overall
control of block compilation, and allow block compilation without requiring
modification of the program source.

There are three possible values of the \kwd{block-compile} argument:
\begin{Lentry}
  
\item[\false{}] Do no compile-time resolution of global function
  names, not even for self-recursive calls.  This inhibits any
  \code{start-block} declarations appearing in the file, allowing all
  functions to be incrementally redefined.
  
\item[\true{}] Start compiling in block compilation mode.  This is
  mainly useful for block compiling small files that contain no
  \code{start-block} declarations.  See also the \kwd{entry-points}
  argument.
  
\item[\kwd{specified}] Start compiling in form-at-a-time mode, but
  exploit \code{start-block} declarations and compile self-recursive
  calls as local calls.  Normally \kwd{specified} is the default for
  this argument (see \varref{block-compile-default}.)
\end{Lentry}

The \kwd{entry-points} argument can be used in conjunction with
\w{\kwd{block-compile} \true{}} to specify the entry-points to a
block-compiled file.  If not specified or \nil, all global functions
will be compiled as entry points.  When \kwd{block-compile} is not
\true, this argument is ignored.

\begin{defvar}{}{block-compile-default}
  
  This variable determines the default value for the
  \kwd{block-compile} argument to \code{compile-file} and
  \code{compile-from-stream}.  The initial value of this variable is
  \kwd{specified}, but \false{} is sometimes useful for totally
  inhibiting block compilation.
\end{defvar}


\subsection{Practical Difficulties}

The main problem with block compilation is that the compiler uses
large amounts of memory when it is block compiling.  This places an
upper limit on the amount of code that can be block compiled as a
unit.  To make best use of block compilation, it is necessary to
locate the parts of the program containing many internal calls, and
then add the appropriate \code{start-block} declarations.  When writing
new code, it is a good idea to put in block compilation declarations
from the very beginning, since writing block declarations correctly
requires accurate knowledge of the program's function call structure.
If you want to initially develop code with full incremental
redefinition, you can compile with \varref{block-compile-default} set to
\false.

Note if a \code{defun} appears in a non-null lexical environment, then
calls to it cannot be block compiled.

Unless files are very small, it is probably impractical to block compile
multiple files as a unit by specifying a list of files to \code{compile-file}.
Semi-inline expansion (\pxlref{semi-inline}) provides another way to
extend block compilation across file boundaries.


\subsection{Context Declarations}
\label{context-declarations}
\cindex{context sensitive declarations}
\cpsubindex{declarations}{context-sensitive}

\cmucl{} has a context-sensitive declaration mechanism which is useful
because it allows flexible control of the compilation policy in large
systems without requiring changes to the source files.  The primary
use of this feature is to allow the exported interfaces of a system to
be compiled more safely than the system internals.  The context used
is the name being defined and the kind of definition (function, macro,
etc.)

The \kwd{context-declarations} option to \macref{with-compilation-unit} has
dynamic scope, affecting all compilation done during the evaluation of the
body.  The argument to this option should evaluate to a list of lists of the
form:
\begin{example}
(\var{context-spec} \mplus{\var{declare-form}})
\end{example}
In the indicated context, the specified declare forms are inserted at
the head of each definition.  The declare forms for all contexts that
match are appended together, with earlier declarations getting
precedence over later ones.  A simple example:
\begin{example}
    :context-declarations
    '((:external (declare (optimize (safety 2)))))
\end{example}
This will cause all functions that are named by external symbols to be
compiled with \code{safety 2}.

The full syntax of context specs is:
\begin{Lentry}
  
\item[\kwd{internal}, \kwd{external}] True if the symbol is internal
  (external) in its home package.
  
\item[\kwd{uninterned}] True if the symbol has no home package.
  
\item[\code{\w{(:package \mstar{\var{package-name}})}}] True if the
  symbol's home package is in any of the named packages (false if
  uninterned.)
  
\item[\kwd{anonymous}] True if the function doesn't have any
  interesting name (not \code{defmacro}, \code{defun}, \code{labels}
  or \code{flet}).
  
\item[\kwd{macro}, \kwd{function}] \kwd{macro} is a global
  (\code{defmacro}) macro.  \kwd{function} is anything else.
  
\item[\kwd{local}, \kwd{global}] \kwd{local} is a \code{labels} or
  \code{flet}.  \kwd{global} is anything else.
  
\item[\code{\w{(:or \mstar{\var{context-spec}})}}] True when any
  supplied \var{context-spec} is true.
  
\item[\code{\w{(:and \mstar{\var{context-spec}})}}] True only when all
  supplied \var{context-spec}s are true.
  
\item[\code{\w{(:not \mstar{\var{context-spec}})}}] True when
  \var{context-spec} is false.
  
\item[\code{\w{(:member \mstar{\var{name}})}}] True when the defined
  name is one of these names (\code{equal} test.)
  
\item[\code{\w{(:match \mstar{\var{pattern}})}}] True when any of the
  patterns is a substring of the name.  The name is wrapped with
  \code{\$}'s, so ``\code{\$FOO}'' matches names beginning with
  ``\code{FOO}'', etc.
\end{Lentry}


\subsection{Context Declaration Example}

Here is a more complex example of \code{with-compilation-unit} options:
\begin{example}
:optimize '(optimize (speed 2) (space 2) (inhibit-warnings 2)
                     (debug 1) (safety 0))
:optimize-interface '(optimize-interface (safety 1) (debug 1))
:context-declarations
'(((:or :external (:and (:match "\%") (:match "SET")))
   (declare (optimize-interface (safety 2))))
  ((:or (:and :external :macro)
        (:match "\$PARSE-"))
   (declare (optimize (safety 2)))))
\end{example}
The \code{optimize} and \code{extensions:optimize-interface}
declarations (\pxlref{optimize-declaration}) set up the global
compilation policy.  The bodies of functions are to be compiled
completely unsafe (\code{safety 0}), but argument count and weakened
argument type checking is to be done when a function is called
(\code{speed 2 safety 1}).

The first declaration specifies that all functions that are external
or whose names contain both ``\code{\%}'' and ``\code{SET}'' are to be
compiled compiled with completely safe interfaces (\code{safety 2}).
The reason for this particular \kwd{match} rule is that \code{setf}
inverse functions in this system tend to have both strings in their
name somewhere.  We want \code{setf} inverses to be safe because they
are implicitly called by users even though their name is not exported.

The second declaration makes external macros or functions whose names
start with ``\code{PARSE-}'' have safe bodies (as well as interfaces).
This is desirable because a syntax error in a macro may cause a type
error inside the body.  The \kwd{match} rule is used because macros
often have auxiliary functions whose names begin with this string.

This particular example is used to build part of the standard \cmucl{}
system.  Note however, that context declarations must be set up
according to the needs and coding conventions of a particular system;
different parts of \cmucl{} are compiled with different context
declarations, and your system will probably need its own declarations.
In particular, any use of the \kwd{match} option depends on naming
conventions used in coding.


\section{Inline Expansion}
\label{inline-expansion}
\cindex{inline expansion}
\cpsubindex{expansion}{inline}
\cpsubindex{call}{inline}
\cpsubindex{function call}{inline}
\cpsubindex{optimization}{function call}

\python{} can expand almost any function inline, including functions
with keyword arguments.  The only restrictions are that keyword
argument keywords in the call must be constant, and that global
function definitions (\code{defun}) must be done in a null lexical
environment (not nested in a \code{let} or other binding form.)  Local
functions (\code{flet}) can be inline expanded in any environment.
Combined with \python{}'s source-level optimization, inline expansion
can be used for things that formerly required macros for efficient
implementation.  In \python, macros don't have any efficiency
advantage, so they need only be used where a macro's syntactic
flexibility is required.

Inline expansion is a compiler optimization technique that reduces
the overhead of a function call by simply not doing the call:
instead, the compiler effectively rewrites the program to appear as
though the definition of the called function was inserted at each
call site.  In \llisp, this is straightforwardly expressed by
inserting the \code{lambda} corresponding to the original definition:
\begin{lisp}
(proclaim '(inline my-1+))
(defun my-1+ (x) (+ x 1))

(my-1+ someval) \result{} ((lambda (x) (+ x 1)) someval)
\end{lisp}

When the function expanded inline is large, the program after inline
expansion may be substantially larger than the original program.  If
the program becomes too large, inline expansion hurts speed rather
than helping it, since hardware resources such as physical memory and
cache will be exhausted.  Inline expansion is called for:
\begin{itemize}
  
\item When profiling has shown that a relatively simple function is
  called so often that a large amount of time is being wasted in the
  calling of that function (as opposed to running in that function.)
  If a function is complex, it will take a long time to run relative
  the time spent in call, so the speed advantage of inline expansion
  is diminished at the same time the space cost of inline expansion is
  increased.  Of course, if a function is rarely called, then the
  overhead of calling it is also insignificant.
  
\item With functions so simple that they take less space to inline
  expand than would be taken to call the function (such as
  \code{my-1+} above.)  It would require intimate knowledge of the
  compiler to be certain when inline expansion would reduce space, but
  it is generally safe to inline expand functions whose definition is
  a single function call, or a few calls to simple \clisp{} functions.
\end{itemize}


In addition to this speed/space tradeoff from inline expansion's
avoidance of the call, inline expansion can also reveal opportunities
for optimization.  \python{}'s extensive source-level optimization can
make use of context information from the caller to tremendously
simplify the code resulting from the inline expansion of a function.

The main form of caller context is local information about the actual
argument values: what the argument types are and whether the arguments
are constant.  Knowledge about argument types can eliminate run-time
type tests (e.g., for generic arithmetic.)  Constant arguments in a
call provide opportunities for constant folding optimization after
inline expansion.

A hidden way that constant arguments are often supplied to functions
is through the defaulting of unsupplied optional or keyword arguments.
There can be a huge efficiency advantage to inline expanding functions
that have complex keyword-based interfaces, such as this definition of
the \code{member} function:
\begin{lisp}
(proclaim '(inline member))
(defun member (item list &key
                    (key #'identity)
                    (test #'eql testp)
                    (test-not nil notp))
  (do ((list list (cdr list)))
      ((null list) nil)
    (let ((car (car list)))
      (if (cond (testp
                 (funcall test item (funcall key car)))
                (notp
                 (not (funcall test-not item (funcall key car))))
                (t
                 (funcall test item (funcall key car))))
          (return list)))))

\end{lisp}
After inline expansion, this call is simplified to the obvious code:
\begin{lisp}
(member a l :key #'foo-a :test #'char=) \result{}

(do ((list list (cdr list)))
    ((null list) nil)
  (let ((car (car list)))
    (if (char= item (foo-a car))
        (return list))))
\end{lisp}
In this example, there could easily be more than an order of magnitude
improvement in speed.  In addition to eliminating the original call to
\code{member}, inline expansion also allows the calls to \code{char=}
and \code{foo-a} to be open-coded.  We go from a loop with three tests
and two calls to a loop with one test and no calls.

\xlref{source-optimization} for more discussion of source level
optimization.


\subsection{Inline Expansion Recording}
\cindex{recording of inline expansions}

Inline expansion requires that the source for the inline expanded function to
be available when calls to the function are compiled.  The compiler doesn't
remember the inline expansion for every function, since that would take an
excessive about of space.  Instead, the programmer must tell the compiler to
record the inline expansion before the definition of the inline expanded
function is compiled.  This is done by globally declaring the function inline
before the function is defined, by using the \code{inline} and
\code{extensions:maybe-inline} (\pxlref{maybe-inline-declaration})
declarations.

In addition to recording the inline expansion of inline functions at the time
the function is compiled, \code{compile-file} also puts the inline expansion in
the output file.  When the output file is loaded, the inline expansion is made
available for subsequent compilations; there is no need to compile the
definition again to record the inline expansion.

If a function is declared inline, but no expansion is recorded, then the
compiler will give an efficiency note like:

\begin{example}
Note: MYFUN is declared inline, but has no expansion.
\end{example}

When you get this note, check that the \code{inline} declaration and the
definition appear before the calls that are to be inline expanded.  This note
will also be given if the inline expansion for a \code{defun} could not be
recorded because the \code{defun} was in a non-null lexical environment.


\subsection{Semi-Inline Expansion}
\label{semi-inline}

\python{} supports \var{semi-inline} functions.  Semi-inline expansion
shares a single copy of a function across all the calls in a component
by converting the inline expansion into a local function
(\pxlref{local-call}.)  This takes up less space when there are
multiple calls, but also provides less opportunity for context
dependent optimization.  When there is only one call, the result is
identical to normal inline expansion.  Semi-inline expansion is done
when the \code{space} optimization quality is \code{0}, and the
function has been declared \code{extensions:maybe-inline}.

This mechanism of inline expansion combined with local call also
allows recursive functions to be inline expanded.  If a recursive
function is declared \code{inline}, calls will actually be compiled
semi-inline.  Although recursive functions are often so complex that
there is little advantage to semi-inline expansion, it can still be
useful in the same sort of cases where normal inline expansion is
especially advantageous, i.e. functions where the calling context can
help a lot.


\subsection{The Maybe-Inline Declaration}
\label{maybe-inline-declaration}
\cindex{maybe-inline declaration}

The \code{extensions:maybe-inline} declaration is a \cmucl{}
extension.  It is similar to \code{inline}, but indicates that inline
expansion may sometimes be desirable, rather than saying that inline
expansion should almost always be done.  When used in a global
declaration, \code{extensions:maybe-inline} causes the expansion for
the named functions to be recorded, but the functions aren't actually
inline expanded unless \code{space} is \code{0} or the function is
eventually (perhaps locally) declared \code{inline}.

Use of the \code{extensions:maybe-inline} declaration followed by the
\code{defun} is preferable to the standard idiom of:
\begin{lisp}
(proclaim '(inline myfun))
(defun myfun () ...)
(proclaim '(notinline myfun))

;;; \textit{Any calls to \code{myfun} here are not inline expanded.}

(defun somefun ()
  (declare (inline myfun))
  ;;
  ;; \textit{Calls to \code{myfun} here are inline expanded.}
  ...)
\end{lisp}
The problem with using \code{notinline} in this way is that in
\clisp{} it does more than just suppress inline expansion, it also
forbids the compiler to use any knowledge of \code{myfun} until a
later \code{inline} declaration overrides the \code{notinline}.  This
prevents compiler warnings about incorrect calls to the function, and
also prevents block compilation.

The \code{extensions:maybe-inline} declaration is used like this:
\begin{lisp}
(proclaim '(extensions:maybe-inline myfun))
(defun myfun () ...)

;;; \textit{Any calls to \code{myfun} here are not inline expanded.}

(defun somefun ()
  (declare (inline myfun))
  ;;
  ;; \textit{Calls to \code{myfun} here are inline expanded.}
  ...)

(defun someotherfun ()
  (declare (optimize (space 0)))
  ;;
  ;; \textit{Calls to \code{myfun} here are expanded semi-inline.}
  ...)
\end{lisp}
In this example, the use of \code{extensions:maybe-inline} causes the
expansion to be recorded when the \code{defun} for \code{somefun} is
compiled, and doesn't waste space through doing inline expansion by
default.  Unlike \code{notinline}, this declaration still allows the
compiler to assume that the known definition really is the one that
will be called when giving compiler warnings, and also allows the
compiler to do semi-inline expansion when the policy is appropriate.

When the goal is merely to control whether inline expansion is done by
default, it is preferable to use \code{extensions:maybe-inline} rather
than \code{notinline}.  The \code{notinline} declaration should be
reserved for those special occasions when a function may be redefined
at run-time, so the compiler must be told that the obvious definition
of a function is not necessarily the one that will be in effect at the
time of the call.


\section{Byte Coded Compilation}
\label{byte-compile}
\cindex{byte coded compilation}
\cindex{space optimization}

\python{} supports byte compilation to reduce the size of Lisp
programs by allowing functions to be compiled more compactly.  Byte
compilation provides an extreme speed/space tradeoff: byte code is
typically six times more compact than native code, but runs fifty
times (or more) slower.  This is about ten times faster than the
standard interpreter, which is itself considered fast in comparison to
other \clisp{} interpreters.

Large Lisp systems (such as \cmucl{} itself) often have large amounts
of user-interface code, compile-time (macro) code, debugging code, or
rarely executed special-case code.  This code is a good target for
byte compilation: very little time is spent running in it, but it can
take up quite a bit of space.  Straight-line code with many function
calls is much more suitable than inner loops.

When byte-compiling, the compiler compiles about twice as fast, and
can produce a hardware independent object file (\file{.bytef} type.)
This file can be loaded like a normal fasl file on any implementation
of \cmucl{} with the same byte-ordering.

The decision to byte compile or native compile can be done on a
per-file or per-code-object basis.  The \kwd{byte-compile} argument to
\funref{compile-file} has these possible values:

\begin{Lentry}
\item[\false{}] Don't byte compile anything in this file.
  
\item[\true{}] Byte compile everything in this file and produce a
  processor-independent \file{.bytef} file.
  
\item[\kwd{maybe}] Produce a normal fasl file, but byte compile any
  functions for which the \code{speed} optimization quality is
  \code{0} and the \code{debug} quality is not greater than \code{1}.
\end{Lentry}

\begin{defvar}{extensions:}{byte-compile-top-level}
  
  If this variable is true (the default) and the \kwd{byte-compile}
  argument to \code{compile-file} is \kwd{maybe}, then byte compile
  top-level code (code outside of any \code{defun}, \code{defmethod},
  etc.)
\end{defvar}

\begin{defvar}{extensions:}{byte-compile-default}
  
  This variable determines the default value for the
  \kwd{byte-compile} argument to \code{compile-file}, initially
  \kwd{maybe}.
\end{defvar}


\section{Object Representation}
\label{object-representation}
\cindex{object representation}
\cpsubindex{representation}{object}
\cpsubindex{efficiency}{of objects}

A somewhat subtle aspect of writing efficient \clisp{} programs is
choosing the correct data structures so that the underlying objects
can be implemented efficiently.  This is partly because of the need
for multiple representations for a given value
(\pxlref{non-descriptor}), but is also due to the sheer number of
object types that \clisp{} has built in.  The number of possible
representations complicates the choice of a good representation
because semantically similar objects may vary in their efficiency
depending on how the program operates on them.


\subsection{Think Before You Use a List}
\cpsubindex{lists}{efficiency of}

Although Lisp's creator seemed to think that it was for LISt
Processing, the astute observer may have noticed that the chapter on
list manipulation makes up less that three percent of \cltltwo{}. The
language has grown since Lisp 1.5\dash{}new data types supersede lists
for many purposes.


\subsection{Structure Representation}
\cpsubindex{structure types}{efficiency of} One of the best ways of
building complex data structures is to define appropriate structure
types using \findexed{defstruct}.  In \python, access of structure
slots is always at least as fast as list or vector access, and is
usually faster.  In comparison to a list representation of a tuple,
structures also have a space advantage.

Even if structures weren't more efficient than other representations, structure
use would still be attractive because programs that use structures in
appropriate ways are much more maintainable and robust than programs written
using only lists.  For example:
\begin{lisp}
(rplaca (caddr (cadddr x)) (caddr y))
\end{lisp}
could have been written using structures in this way:
\begin{lisp}
(setf (beverage-flavor (astronaut-beverage x)) (beverage-flavor y))
\end{lisp}
The second version is more maintainable because it is easier to
understand what it is doing.  It is more robust because structures
accesses are type checked.  An \code{astronaut} will never be confused
with a \code{beverage}, and the result of \code{beverage-flavor} is
always a flavor.  See sections \ref{structure-types} and
\ref{freeze-type} for more information about structure types.
\xlref{type-inference} for a number of examples that make clear the
advantages of structure typing.

Note that the structure definition should be compiled before any uses
of its accessors or type predicate so that these function calls can be
efficiently open-coded.


\subsection{Arrays}
\label{array-types}
\cpsubindex{arrays}{efficiency of}

Arrays are often the most efficient representation for collections of objects
because:
\begin{itemize}
  
\item Array representations are often the most compact.  An array is
  always more compact than a list containing the same number of
  elements.
  
\item Arrays allow fast constant-time access.
  
\item Arrays are easily destructively modified, which can reduce
  consing.
  
\item Array element types can be specialized, which reduces both
  overall size and consing (\pxlref{specialized-array-types}.)
\end{itemize}


Access of arrays that are not of type \code{simple-array} is less
efficient, so declarations are appropriate when an array is of a
simple type like \code{simple-string} or \code{simple-bit-vector}.
Arrays are almost always simple, but the compiler may not be able to
prove simpleness at every use.  The only way to get a non-simple array
is to use the \kwd{displaced-to}, \kwd{fill-pointer} or
\code{adjustable} arguments to \code{make-array}.  If you don't use
these hairy options, then arrays can always be declared to be simple.

Because of the many specialized array types and the possibility of
non-simple arrays, array access is much like generic arithmetic
(\pxlref{generic-arithmetic}).  In order for array accesses to be
efficiently compiled, the element type and simpleness of the array
must be known at compile time.  If there is inadequate information,
the compiler is forced to call a generic array access routine.  You
can detect inefficient array accesses by enabling efficiency notes,
\pxlref{efficiency-notes}.


\subsection{Vectors}
\cpsubindex{vectors}{efficiency of}

Vectors (one dimensional arrays) are particularly useful, since in
addition to their obvious array-like applications, they are also well
suited to representing sequences.  In comparison to a list
representation, vectors are faster to access and take up between two
and sixty-four times less space (depending on the element type.)  As
with arbitrary arrays, the compiler needs to know that vectors are not
complex, so you should use \code{simple-string} in preference to
\code{string}, etc.

The only advantage that lists have over vectors for representing
sequences is that it is easy to change the length of a list, add to it
and remove items from it.  Likely signs of archaic, slow lisp code are
\code{nth} and \code{nthcdr}.  If you are using these functions you
should probably be using a vector.


\subsection{Bit-Vectors}
\cpsubindex{bit-vectors}{efficiency of}

Another thing that lists have been used for is set manipulation.  In
applications where there is a known, reasonably small universe of
items bit-vectors can be used to improve performance.  This is much
less convenient than using lists, because instead of symbols, each
element in the universe must be assigned a numeric index into the bit
vector.  Using a bit-vector will nearly always be faster, and can be
tremendously faster if the number of elements in the set is not small.
The logical operations on \code{simple-bit-vector}s are efficient,
since they operate on a word at a time.


\subsection{Hashtables}
\cpsubindex{hash-tables}{efficiency of}

Hashtables are an efficient and general mechanism for maintaining associations
such as the association between an object and its name.  Although hashtables
are usually the best way to maintain associations, efficiency and style
considerations sometimes favor the use of an association list (a-list).

\code{assoc} is fairly fast when the \var{test} argument is \code{eq}
or \code{eql} and there are only a few elements, but the time goes up
in proportion with the number of elements.  In contrast, the
hash-table lookup has a somewhat higher overhead, but the speed is
largely unaffected by the number of entries in the table.  For an
\code{equal} hash-table or alist, hash-tables have an even greater
advantage, since the test is more expensive.  Whatever you do, be sure
to use the most restrictive test function possible.

The style argument observes that although hash-tables and alists
overlap in function, they do not do all things equally well.
\begin{itemize}
  
\item Alists are good for maintaining scoped environments.  They were
  originally invented to implement scoping in the Lisp interpreter,
  and are still used for this in \python.  With an alist one can
  non-destructively change an association simply by consing a new
  element on the front.  This is something that cannot be done with
  hash-tables.
  
\item Hashtables are good for maintaining a global association.  The
  value associated with an entry can easily be changed with
  \code{setf}.  With an alist, one has to go through contortions,
  either \code{rplacd}'ing the cons if the entry exists, or pushing a
  new one if it doesn't.  The side-effecting nature of hash-table
  operations is an advantage here.
\end{itemize}


Historically, symbol property lists were often used for global name
associations.  Property lists provide an awkward and error-prone
combination of name association and record structure.  If you must use
the property list, please store all the related values in a single
structure under a single property, rather than using many properties.
This makes access more efficient, and also adds a modicum of typing
and abstraction.  \xlref{advanced-type-stuff} for information on types
in \cmucl.


\section{Numbers}
\label{numeric-types}
\cpsubindex{numeric}{types}
\cpsubindex{types}{numeric}

Numbers are interesting because numbers are one of the few \llisp{} data types
that have direct support in conventional hardware.  If a number can be
represented in the way that the hardware expects it, then there is a big
efficiency advantage.

Using hardware representations is problematical in \llisp{} due to
dynamic typing (where the type of a value may be unknown at compile
time.)  It is possible to compile code for statically typed portions
of a \llisp{} program with efficiency comparable to that obtained in
statically typed languages such as C, but not all \llisp{}
implementations succeed.  There are two main barriers to efficient
numerical code in \llisp{}:
\begin{itemize}
  
\item The compiler must prove that the numerical expression is in fact
  statically typed, and
  
\item The compiler must be able to somehow reconcile the conflicting
  demands of the hardware mandated number representation with the
  \llisp{} requirements of dynamic typing and garbage-collecting
  dynamic storage allocation.
\end{itemize}

Because of its type inference (\pxlref{type-inference}) and efficiency
notes (\pxlref{efficiency-notes}), \python{} is better than
conventional \llisp{} compilers at ensuring that numerical expressions
are statically typed.  \python{} also goes somewhat farther than existing
compilers in the area of allowing native machine number
representations in the presence of garbage collection.


\subsection{Descriptors}
\cpsubindex{descriptors}{object}
\cindex{object representation}
\cpsubindex{representation}{object}
\cpsubindex{consing}{overhead of}

\llisp{}'s dynamic typing requires that it be possible to represent
any value with a fixed length object, known as a \var{descriptor}.
This fixed-length requirement is implicit in features such as:
\begin{itemize}
  
\item Data types (like \code{simple-vector}) that can contain any type
  of object, and that can be destructively modified to contain
  different objects (of possibly different types.)
  
\item Functions that can be called with any type of argument, and that
  can be redefined at run time.
\end{itemize}

In order to save space, a descriptor is invariably represented as a
single word.  Objects that can be directly represented in the
descriptor itself are said to be \var{immediate}.  Descriptors for
objects larger than one word are in reality pointers to the memory
actually containing the object.

Representing objects using pointers has two major disadvantages:
\begin{itemize}
  
\item The memory pointed to must be allocated on the heap, so it must
  eventually be freed by the garbage collector.  Excessive heap
  allocation of objects (or ``consing'') is inefficient in several
  ways.  \xlref{consing}.
  
\item Representing an object in memory requires the compiler to emit
  additional instructions to read the actual value in from memory, and
  then to write the value back after operating on it.
\end{itemize}

The introduction of garbage collection makes things even worse, since
the garbage collector must be able to determine whether a descriptor
is an immediate object or a pointer.  This requires that a few bits in
each descriptor be dedicated to the garbage collector.  The loss of a
few bits doesn't seem like much, but it has a major efficiency
implication\dash{}objects whose natural machine representation is a
full word (integers and single-floats) cannot have an immediate
representation.  So the compiler is forced to use an unnatural
immediate representation (such as \code{fixnum}) or a natural pointer
representation (with the attendant consing overhead.)


\subsection{Non-Descriptor Representations}
\label{non-descriptor}
\cindex{non-descriptor representations}
\cindex{stack numbers}

From the discussion above, we can see that the standard descriptor
representation has many problems, the worst being number consing.
\llisp{} compilers try to avoid these descriptor efficiency problems by using
\var{non-descriptor} representations.  A compiler that uses non-descriptor
representations can compile this function so that it does no number consing:
\begin{lisp}
(defun multby (vec n)
  (declare (type (simple-array single-float (*)) vec)
           (single-float n))
  (dotimes (i (length vec))
    (setf (aref vec i)
          (* n (aref vec i)))))
\end{lisp}
If a descriptor representation were used, each iteration of the loop might
cons two floats and do three times as many memory references.

As its negative definition suggests, the range of possible non-descriptor
representations is large.  The performance improvement from non-descriptor
representation depends upon both the number of types that have non-descriptor
representations and the number of contexts in which the compiler is forced to
use a descriptor representation.

Many \llisp{} compilers support non-descriptor representations for
float types such as \code{single-float} and \code{double-float}
(section \ref{float-efficiency}.)  \python{} adds support for full
word integers (\pxlref{word-integers}), characters
(\pxlref{characters}) and system-area pointers (unconstrained
pointers, \pxlref{system-area-pointers}.)  Many \llisp{} compilers
support non-descriptor representations for variables (section
\ref{ND-variables}) and array elements (section
\ref{specialized-array-types}.)  \python{} adds support for
non-descriptor arguments and return values in local call
(\pxlref{number-local-call}) and structure slots (\pxlref{raw-slots}).


\subsection{Variables}
\label{ND-variables}
\cpsubindex{variables}{non-descriptor}
\cpsubindex{type declarations}{variable}
\cpsubindex{efficiency}{of numeric variables}

In order to use a non-descriptor representation for a variable or
expression intermediate value, the compiler must be able to prove that
the value is always of a particular type having a non-descriptor
representation.  Type inference (\pxlref{type-inference}) often needs
some help from user-supplied declarations.  The best kind of type
declaration is a variable type declaration placed at the binding
point:
\begin{lisp}
(let ((x (car l)))
  (declare (single-float x))
  ...)
\end{lisp}
Use of \code{the}, or of variable declarations not at the binding form
is insufficient to allow non-descriptor representation of the
variable\dash{}with these declarations it is not certain that all
values of the variable are of the right type.  It is sometimes useful
to introduce a gratuitous binding that allows the compiler to change
to a non-descriptor representation, like:
\begin{lisp}
(etypecase x
  ((signed-byte 32)
   (let ((x x))
     (declare (type (signed-byte 32) x)) 
     ...))
  ...)
\end{lisp}
The declaration on the inner \code{x} is necessary here due to a phase
ordering problem.  Although the compiler will eventually prove that
the outer \code{x} is a \w{\code{(signed-byte 32)}} within that
\code{etypecase} branch, the inner \code{x} would have been optimized
away by that time.  Declaring the type makes let optimization more
cautious.

Note that storing a value into a global (or \code{special}) variable
always forces a descriptor representation.  Wherever possible, you
should operate only on local variables, binding any referenced globals
to local variables at the beginning of the function, and doing any
global assignments at the end.

Efficiency notes signal use of inefficient representations, so
programmer's needn't continuously worry about the details of
representation selection (\pxlref{representation-eff-note}.)


\subsection{Generic Arithmetic}
\label{generic-arithmetic}
\cindex{generic arithmetic}
\cpsubindex{arithmetic}{generic}
\cpsubindex{numeric}{operation efficiency}

In \clisp, arithmetic operations are \var{generic}.\footnote{As Steele
  notes in CLTL II, this is a generic conception of generic, and is
  not to be confused with the CLOS concept of a generic function.}
The \code{+} function can be passed \code{fixnum}s, \code{bignum}s,
\code{ratio}s, and various kinds of \code{float}s and
\code{complex}es, in any combination.  In addition to the inherent
complexity of \code{bignum} and \code{ratio} operations, there is also
a lot of overhead in just figuring out which operation to do and what
contagion and canonicalization rules apply.  The complexity of generic
arithmetic is so great that it is inconceivable to open code it.
Instead, the compiler does a function call to a generic arithmetic
routine, consuming many instructions before the actual computation
even starts.

This is ridiculous, since even \llisp{} programs do a lot of
arithmetic, and the hardware is capable of doing operations on small
integers and floats with a single instruction.  To get acceptable
efficiency, the compiler special-cases uses of generic arithmetic that
are directly implemented in the hardware.  In order to open code
arithmetic, several constraints must be met:
\begin{itemize}
  
\item All the arguments must be known to be a good type of number.
  
\item The result must be known to be a good type of number.
  
\item Any intermediate values such as the result of \w{\code{(+ a b)}}
  in the call \w{\code{(+ a b c)}} must be known to be a good type of
  number.
  
\item All the above numbers with good types must be of the \var{same}
  good type.  Don't try to mix integers and floats or different float
  formats.
\end{itemize}

The ``good types'' are \w{\code{(signed-byte 32)}},
\w{\code{(unsigned-byte 32)}}, \code{single-float} and
\code{double-float}.  See sections \ref{fixnums}, \ref{word-integers}
and \ref{float-efficiency} for more discussion of good numeric types.

\code{float} is not a good type, since it might mean either
\code{single-float} or \code{double-float}.  \code{integer} is not a
good type, since it might mean \code{bignum}.  \code{rational} is not
a good type, since it might mean \code{ratio}.  Note however that
these types are still useful in declarations, since type inference may
be able to strengthen a weak declaration into a good one, when it
would be at a loss if there was no declaration at all
(\pxlref{type-inference}).  The \code{integer} and
\code{unsigned-byte} (or non-negative integer) types are especially
useful in this regard, since they can often be strengthened to a good
integer type.

Arithmetic with \code{complex} numbers is inefficient in comparison to
float and integer arithmetic.  Complex numbers are always represented
with a pointer descriptor (causing consing overhead), and complex
arithmetic is always closed coded using the general generic arithmetic
functions.  But arithmetic with complex types such as:
\begin{lisp}
(complex float)
(complex fixnum)
\end{lisp}
is still faster than \code{bignum} or \code{ratio} arithmetic, since the
implementation is much simpler.

Note: don't use \code{/} to divide integers unless you want the
overhead of rational arithmetic.  Use \code{truncate} even when you
know that the arguments divide evenly.

You don't need to remember all the rules for how to get open-coded
arithmetic, since efficiency notes will tell you when and where there
is a problem\dash{}\pxlref{efficiency-notes}.


\subsection{Fixnums}
\label{fixnums}
\cindex{fixnums}
\cindex{bignums}

A fixnum is a ``FIXed precision NUMber''.  In modern \llisp{}
implementations, fixnums can be represented with an immediate
descriptor, so operating on fixnums requires no consing or memory
references.  Clever choice of representations also allows some
arithmetic operations to be done on fixnums using hardware supported
word-integer instructions, somewhat reducing the speed penalty for
using an unnatural integer representation.

It is useful to distinguish the \code{fixnum} type from the fixnum
representation of integers.  In \python, there is absolutely nothing
magical about the \code{fixnum} type in comparison to other finite
integer types.  \code{fixnum} is equivalent to (is defined with
\code{deftype} to be) \w{\code{(signed-byte 30)}}.  \code{fixnum} is
simply the largest subset of integers that {\em can be represented}
using an immediate fixnum descriptor.

Unlike in other \clisp{} compilers, it is in no way desirable to use
the \code{fixnum} type in declarations in preference to more
restrictive integer types such as \code{bit}, \w{\code{(integer -43
    7)}} and \w{\code{(unsigned-byte 8)}}.  Since \python{} does
understand these integer types, it is preferable to use the more
restrictive type, as it allows better type inference
(\pxlref{operation-type-inference}.)

The small, efficient fixnum is contrasted with bignum, or ``BIG
NUMber''.  This is another descriptor representation for integers, but
this time a pointer representation that allows for arbitrarily large
integers.  Bignum operations are less efficient than fixnum
operations, both because of the consing and memory reference overheads
of a pointer descriptor, and also because of the inherent complexity
of extended precision arithmetic.  While fixnum operations can often
be done with a single instruction, bignum operations are so complex
that they are always done using generic arithmetic.

A crucial point is that the compiler will use generic arithmetic if it
can't \var{prove} that all the arguments, intermediate values, and
results are fixnums.  With bounded integer types such as
\code{fixnum}, the result type proves to be especially problematical,
since these types are not closed under common arithmetic operations
such as \code{+}, \code{-}, \code{*} and \code{/}.  For example,
\w{\code{(1+ (the fixnum x))}} does not necessarily evaluate to a
\code{fixnum}.  Bignums were added to \llisp{} to get around this
problem, but they really just transform the correctness problem ``if
this add overflows, you will get the wrong answer'' to the efficiency
problem ``if this add \var{might} overflow then your program will run
slowly (because of generic arithmetic.)''

There is just no getting around the fact that the hardware only
directly supports short integers.  To get the most efficient open
coding, the compiler must be able to prove that the result is a good
integer type.  This is an argument in favor of using more restrictive
integer types: \w{\code{(1+ (the fixnum x))}} may not always be a
\code{fixnum}, but \w{\code{(1+ (the (unsigned-byte 8) x))}} always
is.  Of course, you can also assert the result type by putting in lots
of \code{the} declarations and then compiling with \code{safety}
\code{0}.


\subsection{Word Integers}
\label{word-integers}
\cindex{word integers}

\python{} is unique in its efficient implementation of arithmetic
on full-word integers through non-descriptor representations and open coding.
Arithmetic on any subtype of these types:

\begin{lisp}
(signed-byte 32)
(unsigned-byte 32)
\end{lisp}

is reasonably efficient, although subtypes of \code{fixnum} remain
somewhat more efficient.

If a word integer must be represented as a descriptor, then the
\code{bignum} representation is used, with its associated consing
overhead.  The support for word integers in no way changes the
language semantics, it just makes arithmetic on small bignums vastly
more efficient.  It is fine to do arithmetic operations with mixed
\code{fixnum} and word integer operands; just declare the most
specific integer type you can, and let the compiler decide what
representation to use.

In fact, to most users, the greatest advantage of word integer
arithmetic is that it effectively provides a few guard bits on the
fixnum representation.  If there are missing assertions on
intermediate values in a fixnum expression, the intermediate results
can usually be proved to fit in a word.  After the whole expression is
evaluated, there will often be a fixnum assertion on the final result,
allowing creation of a fixnum result without even checking for
overflow.

The remarks in section \ref{fixnums} about fixnum result type also
apply to word integers; you must be careful to give the compiler
enough information to prove that the result is still a word integer.
This time, though, when we blow out of word integers we land in into
generic bignum arithmetic, which is much worse than sleazing from
\code{fixnum}s to word integers.  Note that mixing
\w{\code{(unsigned-byte 32)}} arguments with arguments of any signed
type (such as \code{fixnum}) is a no-no, since the result might not be
unsigned.


\subsection{Floating Point Efficiency}
\label{float-efficiency}
\cindex{floating point efficiency}

Arithmetic on objects of type \code{single-float} and \code{double-float} is
efficiently implemented using non-descriptor representations and open coding.
As for integer arithmetic, the arguments must be known to be of the same float
type.  Unlike for integer arithmetic, the results and intermediate values
usually take care of themselves due to the rules of float contagion, i.e.
\w{\code{(1+ (the single-float x))}} is always a \code{single-float}.

Although they are not specially implemented, \code{short-float} and
\code{long-float} are also acceptable in declarations, since they are
synonyms for the \code{single-float} and \code{double-float} types,
respectively.

Some versions of \cmucl{} include extra support for floating
point arithmetic.  In particular, if \code{*features*} includes
\kwd{propagate-float-type}, list-style float type specifiers such as
\w{\code{(single-float 0.0 1.0)}} will be used to good effect.

For example, in this function,

\begin{example}
  (defun square (x)
    (declare (type (single-float 0f0 10f0)))
    (* x x))
\end{example}

\python{} can deduce that the
return type of the function \code{square} is \w{\code{(single-float
    0f0 100f0)}}.

Many union types are also supported so that

\begin{example}
  (+ (the (or (integer 1 1) (integer 5 5)) x)
     (the (or (integer 10 10) (integer 20 20)) y))
\end{example}

has the inferred type \code{(or (integer 11 11) (integer 15 15)
  (integer 21 21) (integer 25 25))}.  This also works for
floating-point numbers.  Member types, however, are not because in
general the member elements do not have to be numbers.  Thus,
instead of \code{(member 1 4)}, you should write \code{(or (integer
  1 1) (integer 4 4))}.
  
In addition, if \kwd{propagate-fun-type} is in \code{*features*},
\python{} knows how to infer types for many mathematical functions
including square root, exponential and logarithmic functions,
trignometric functions and their inverses, and hyperbolic functions
and their inverses.  For numeric code, this can greatly enhance
efficiency by allowing the compiler to use specialized versions of
the functions instead of the generic versions.  The greatest benefit 
of this type inference is determining that the result of the
function is real-valued number instead of possibly being
a complex-valued number.

For example, consider the function
\begin{example}
  (defun fun (x)
    (declare (type (single-float 0f0 100f0) x))
    (values (sqrt x) (log x 10f0)))
\end{example}
With this declaration, the compiler can determine that the argument
to \code{sqrt} and \code{log} are always non-negative so that the result 
is always a \code{single-float}.  In fact, the return type for this
function is derived to be \code{(values (single-float 0f0 10f0)
    (single-float * 2f0))}.

If the declaration were reduced to just \w{\code{(declare
    single-float x)}}, the argument to \code{sqrt} and \code{log}
could be negative.  This forces the use of the generic versions of
these functions because the result could be a complex number.

Union types are not yet supported for functions.  

We note, however, that proper interval arithmetic is not fully
implemented in the compiler so the inferred types may be slightly in
error due to round-off errors.  This round-off error could
accumulate to cause the compiler to erroneously deduce the result
type and cause code to be removed as being
unreachable.\footnote{This, however, has not actually happened, but
  it is a possibility.}%
Thus, the declarations should only be precise enough for the
compiler to deduce that a real-valued argument to a function would
produce a real-valued result.  The efficiency notes
(\pxlref{representation-eff-note}) from the compiler will guide you
on what declarations might be useful.

When a float must be represented as a descriptor, a pointer representation is
used, creating consing overhead.  For this reason, you should try to avoid
situations (such as full call and non-specialized data structures) that force a
descriptor representation.  See sections \ref{specialized-array-types},
\ref{raw-slots} and \ref{number-local-call}.

\xlref{ieee-float} for information on the extensions to support IEEE
floating point.


\subsection{Specialized Arrays}
\label{specialized-array-types}
\cindex{specialized array types}
\cpsubindex{array types}{specialized}
\cpsubindex{types}{specialized array}

\clisp{} supports specialized array element types through the
\kwd{element-type} argument to \code{make-array}.  When an array has a
specialized element type, only elements of that type can be stored in
the array.  From this restriction comes two major efficiency
advantages:
\begin{itemize}
  
\item A specialized array can save space by packing multiple elements
  into a single word.  For example, a \code{base-char} array can have
  4 elements per word, and a \code{bit} array can have 32.  This
  space-efficient representation is possible because it is not
  necessary to separately indicate the type of each element.
  
\item The elements in a specialized array can be given the same
  non-descriptor representation as the one used in registers and on
  the stack, eliminating the need for representation conversions when
  reading and writing array elements.  For objects with pointer
  descriptor representations (such as floats and word integers) there
  is also a substantial consing reduction because it is not necessary
  to allocate a new object every time an array element is modified.
\end{itemize}


These are the specialized element types currently supported:
\begin{lisp}
bit
(unsigned-byte 2)
(unsigned-byte 4)
(unsigned-byte 8)
(unsigned-byte 16)
(unsigned-byte 32)
(signed-byte 8)
(signed-byte 16)
(signed-byte 30)
(signed-byte 32)
base-character
single-float
double-float
(complex single-float)
(complex double-float)
\end{lisp}

Although a \code{simple-vector} can hold any type of object, \true{}
should still be considered a specialized array type, since arrays with
element type \true{} are specialized to hold descriptors.



When using non-descriptor representations, it is particularly
important to make sure that array accesses are open-coded, since in
addition to the generic operation overhead, efficiency is lost when
the array element is converted to a descriptor so that it can be
passed to (or from) the generic access routine.  You can detect
inefficient array accesses by enabling efficiency notes,
\pxlref{efficiency-notes}.  \xlref{array-types}.


\subsection{Specialized Structure Slots}
\label{raw-slots}
\cpsubindex{structure types}{numeric slots}
\cindex{specialized structure slots}

Structure slots declared by the \kwd{type} \code{defstruct} slot option
to have certain known numeric types are also given non-descriptor
representations.  These types (and subtypes of these types) are supported:
\begin{lisp}
(unsigned-byte 32)
single-float
double-float
\end{lisp}

The primary advantage of specialized slot representations is a large
reduction spurious memory allocation and access overhead of programs
that intensively use these types.


\subsection{Interactions With Local Call}
\label{number-local-call}
\cpsubindex{local call}{numeric operands}
\cpsubindex{call}{numeric operands}
\cindex{numbers in local call}

Local call has many advantages (\pxlref{local-call}); one relevant to
our discussion here is that local call extends the usefulness of
non-descriptor representations.  If the compiler knows from the
argument type that an argument has a non-descriptor representation,
then the argument will be passed in that representation.  The easiest
way to ensure that the argument type is known at compile time is to
always declare the argument type in the called function, like:
\begin{lisp}
(defun 2+f (x)
  (declare (single-float x))
  (+ x 2.0))
\end{lisp}
The advantages of passing arguments and return values in a non-descriptor
representation are the same as for non-descriptor representations in general:
reduced consing and memory access (\pxlref{non-descriptor}.)  This
extends the applicative programming styles discussed in section
\ref{local-call} to numeric code.  Also, if source files are kept reasonably
small, block compilation can be used to reduce number consing to a minimum.

Note that non-descriptor return values can only be used with the known return
convention (section \ref{local-call-return}.)  If the compiler can't prove that
a function always returns the same number of values, then it must use the
unknown values return convention, which requires a descriptor representation.
Pay attention to the known return efficiency notes to avoid number consing.
 

\subsection{Representation of Characters}
\label{characters}
\cindex{characters}
\cindex{strings}

\python{} also uses a non-descriptor representation for characters when
convenient.  This improves the efficiency of string manipulation, but is
otherwise pretty invisible; characters have an immediate descriptor
representation, so there is not a great penalty for converting a character to a
descriptor.  Nonetheless, it may sometimes be helpful to declare
character-valued variables as \code{base-character}.


\section{General Efficiency Hints}
\label{general-efficiency}
\cpsubindex{efficiency}{general hints}

This section is a summary of various implementation costs and ways to get
around them.  These hints are relatively unrelated to the use of the \python{}
compiler, and probably also apply to most other \llisp{} implementations.  In
each section, there are references to related in-depth discussion.


\subsection{Compile Your Code}
\cpsubindex{compilation}{why to}

At this point, the advantages of compiling code relative to running it
interpreted probably need not be emphasized too much, but remember that
in \cmucl, compiled code typically runs hundreds of times faster than
interpreted code.  Also, compiled (\code{fasl}) files load significantly faster
than source files, so it is worthwhile compiling files which are loaded many
times, even if the speed of the functions in the file is unimportant.

Even disregarding the efficiency advantages, compiled code is as good or better
than interpreted code.  Compiled code can be debugged at the source level (see
chapter \ref{debugger}), and compiled code does more error checking.  For these
reasons, the interpreter should be regarded mainly as an interactive command
interpreter, rather than as a programming language implementation.

\b{Do not} be concerned about the performance of your program until you
see its speed compiled.  Some techniques that make compiled code run
faster make interpreted code run slower.


\subsection{Avoid Unnecessary Consing}
\label{consing}
\cindex{consing}
\cindex{garbage collection}
\cindex{memory allocation}
\cpsubindex{efficiency}{of memory use}


Consing is another name for allocation of storage, as done by the
\code{cons} function (hence its name.)  \code{cons} is by no means the
only function which conses\dash{}so does \code{make-array} and many
other functions.  Arithmetic and function call can also have hidden
consing overheads.  Consing hurts performance in the following ways:
\begin{itemize}
  
\item Consing reduces memory access locality, increasing paging
  activity.
  
\item Consing takes time just like anything else.
  
\item Any space allocated eventually needs to be reclaimed, either by
  garbage collection or by starting a new \code{lisp} process.
\end{itemize}


Consing is not undiluted evil, since programs do things other than
consing, and appropriate consing can speed up the real work.  It would
certainly save time to allocate a vector of intermediate results that
are reused hundreds of times.  Also, if it is necessary to copy a
large data structure many times, it may be more efficient to update
the data structure non-destructively; this somewhat increases update
overhead, but makes copying trivial.

Note that the remarks in section \ref{efficiency-overview} about the
importance of separating tuning from coding also apply to consing
overhead.  The majority of consing will be done by a small portion of
the program.  The consing hot spots are even less predictable than the
CPU hot spots, so don't waste time and create bugs by doing
unnecessary consing optimization.  During initial coding, avoid
unnecessary side-effects and cons where it is convenient.  If
profiling reveals a consing problem, \var{then} go back and fix the
hot spots.

\xlref{non-descriptor} for a discussion of how to avoid number consing
in \python.


\subsection{Complex Argument Syntax}
\cpsubindex{argument syntax}{efficiency}
\cpsubindex{efficiency}{of argument syntax}
\cindex{keyword argument efficiency}
\cindex{rest argument efficiency}

\clisp{} has very powerful argument passing mechanisms.  Unfortunately, two
of the most powerful mechanisms, rest arguments and keyword arguments, have a
significant performance penalty:

\begin{itemize}
\item
With keyword arguments, the called function has to parse the supplied keywords
by iterating over them and checking them against the desired keywords.

\item
With rest arguments, the function must cons a list to hold the arguments.  If a
function is called many times or with many arguments, large amounts of memory
will be allocated.
\end{itemize}

Although rest argument consing is worse than keyword parsing, neither problem
is serious unless thousands of calls are made to such a function.  The use of
keyword arguments is strongly encouraged in functions with many arguments or
with interfaces that are likely to be extended, and rest arguments are often
natural in user interface functions.

Optional arguments have some efficiency advantage over keyword
arguments, but their syntactic clumsiness and lack of extensibility
has caused many \clisp{} programmers to abandon use of optionals
except in functions that have obviously simple and immutable
interfaces (such as \code{subseq}), or in functions that are only
called in a few places.  When defining an interface function to be
used by other programmers or users, use of only required and keyword
arguments is recommended.

Parsing of \code{defmacro} keyword and rest arguments is done at
compile time, so a macro can be used to provide a convenient syntax
with an efficient implementation.  If the macro-expanded form contains
no keyword or rest arguments, then it is perfectly acceptable in inner
loops.

Keyword argument parsing overhead can also be avoided by use of inline
expansion (\pxlref{inline-expansion}) and block compilation (section
\ref{block-compilation}.)

Note: the compiler open-codes most heavily used system functions which have
keyword or rest arguments, so that no run-time overhead is involved.


\subsection{Mapping and Iteration}
\cpsubindex{mapping}{efficiency of}

One of the traditional \llisp{} programming styles is a highly applicative one,
involving the use of mapping functions and many lists to store intermediate
results.  To compute the sum of the square-roots of a list of numbers, one
might say:

\begin{lisp}
(apply #'+ (mapcar #'sqrt list-of-numbers))
\end{lisp}

This programming style is clear and elegant, but unfortunately results
in slow code.  There are two reasons why:

\begin{itemize} 
\item The creation of lists of intermediate results causes much
  consing (see \ref{consing}).
  
\item Each level of application requires another scan down the list.
  Thus, disregarding other effects, the above code would probably take
  twice as long as a straightforward iterative version.
\end{itemize}


An example of an iterative version of the same code:
\begin{lisp}
(do ((num list-of-numbers (cdr num))
     (sum 0 (+ (sqrt (car num)) sum)))
    ((null num) sum))
\end{lisp}

See sections \ref{variable-type-inference} and \ref{let-optimization}
for a discussion of the interactions of iteration constructs with type
inference and variable optimization.  Also, section
\ref{local-tail-recursion} discusses an applicative style of
iteration.


\subsection{Trace Files and Disassembly}
\label{trace-files}
\cindex{trace files}
\cindex{assembly listing}
\cpsubindex{listing files}{trace}
\cindex{Virtual Machine (VM, or IR2) representation}
\cindex{implicit continuation representation (IR1)}
\cpsubindex{continuations}{implicit representation}

In order to write efficient code, you need to know the relative costs
of different operations.  The main reason why writing efficient
\llisp{} code is difficult is that there are so many operations, and
the costs of these operations vary in obscure context-dependent ways.
Although efficiency notes point out some problem areas, the only way
to ensure generation of the best code is to look at the assembly code
output.

The \code{disassemble} function is a convenient way to get the assembly code for a
function, but it can be very difficult to interpret, since the correspondence
with the original source code is weak.  A better (but more awkward) option is
to use the \kwd{trace-file} argument to \code{compile-file} to generate a trace
file.

A trace file is a dump of the compiler's internal representations,
including annotated assembly code.  Each component in the program gets
four pages in the trace file (separated by ``\code{$\hat{ }L$}''):
\begin{itemize}
  
\item The implicit-continuation (or IR1) representation of the
  optimized source.  This is a dump of the flow graph representation
  used for ``source level'' optimizations.  As you will quickly
  notice, it is not really very close to the source.  This
  representation is not very useful to even sophisticated users.
  
\item The Virtual Machine (VM, or IR2) representation of the program.
  This dump represents the generated code as sequences of ``Virtual
  OPerations'' (VOPs.)  This representation is intermediate between
  the source and the assembly code\dash{}each VOP corresponds fairly
  directly to some primitive function or construct, but a given VOP
  also has a fairly predictable instruction sequence.  An operation
  (such as \code{+}) may have multiple implementations with different
  cost and applicability.  The choice of a particular VOP such as
  \code{+/fixnum} or \code{+/single-float} represents this choice of
  implementation.  Once you are familiar with it, the VM
  representation is probably the most useful for determining what
  implementation has been used.
  
\item An assembly listing, annotated with the VOP responsible for
  generating the instructions.  This listing is useful for figuring
  out what a VOP does and how it is implemented in a particular
  context, but its large size makes it more difficult to read.
  
\item A disassembly of the generated code, which has all
  pseudo-operations expanded out, but is not annotated with VOPs.
\end{itemize}


Note that trace file generation takes much space and time, since the trace file
is tens of times larger than the source file.  To avoid huge confusing trace
files and much wasted time, it is best to separate the critical program portion
into its own file and then generate the trace file from this small file.


\section{Efficiency Notes}
\label{efficiency-notes}
\cindex{efficiency notes}
\cpsubindex{notes}{efficiency}
\cindex{tuning}

Efficiency notes are messages that warn the user that the compiler has
chosen a relatively inefficient implementation for some operation.
Usually an efficiency note reflects the compiler's desire for more
type information.  If the type of the values concerned is known to the
programmer, then additional declarations can be used to get a more
efficient implementation.

Efficiency notes are controlled by the
\code{extensions:inhibit-warnings} (\pxlref{optimize-declaration})
optimization quality. When \code{speed} is greater than
\code{extensions:inhibit-warnings}, efficiency notes are enabled.
Note that this implicitly enables efficiency notes whenever
\code{speed} is increased from its default of \code{1}.

Consider this program with an obscure missing declaration:

\begin{lisp}
(defun eff-note (x y z)
  (declare (fixnum x y z))
  (the fixnum (+ x y z)))
\end{lisp}

If compiled with \code{\w{(speed 3) (safety 0)}}, this note is given:

\begin{example}
In: DEFUN EFF-NOTE
  (+ X Y Z)
==>
  (+ (+ X Y) Z)
Note: Forced to do inline (signed-byte 32) arithmetic (cost 3).
      Unable to do inline fixnum arithmetic (cost 2) because:
      The first argument is a (INTEGER -1073741824 1073741822),
      not a FIXNUM.
\end{example}

This efficiency note tells us that the result of the intermediate
computation \code{\w{(+ x y)}} is not known to be a \code{fixnum}, so
the addition of the intermediate sum to \code{z} must be done less
efficiently.  This can be fixed by changing the definition of
\code{eff-note}:

\begin{lisp}
(defun eff-note (x y z)
  (declare (fixnum x y z))
  (the fixnum (+ (the fixnum (+ x y)) z)))
\end{lisp}


\subsection{Type Uncertainty}
\cpsubindex{types}{uncertainty}
\cindex{uncertainty of types}

The main cause of inefficiency is the compiler's lack of adequate
information about the types of function argument and result values.
Many important operations (such as arithmetic) have an inefficient
general (generic) case, but have efficient implementations that can
usually be used if there is sufficient argument type information.

Type efficiency notes are given when a value's type is uncertain.
There is an important distinction between values that are {\em not
known} to be of a good type (uncertain) and values that are {\em known
not} to be of a good type. Efficiency notes are given mainly for the
first case (uncertain types.) If it is clear to the compiler that that
there is not an efficient implementation for a particular function
call, then an efficiency note will only be given if the
\code{extensions:inhibit-warnings} optimization quality is \code{0}
(\pxlref{optimize-declaration}.)

In other words, the default efficiency notes only suggest that you add
declarations, not that you change the semantics of your program so
that an efficient implementation will apply.  For example, compilation
of this form will not give an efficiency note:
\begin{lisp}
(elt (the list l) i)
\end{lisp}
even though a vector access is more efficient than indexing a list.


\subsection{Efficiency Notes and Type Checking}
\cpsubindex{type checking}{efficiency of}
\cpsubindex{efficiency}{of type checking}
\cpsubindex{optimization}{type check}

It is important that the \code{eff-note} example above used
\w{\code{(safety 0)}}.  When type checking is enabled, you may get apparently
spurious efficiency notes.  With \w{\code{(safety 1)}}, the note has this extra
line on the end:

\begin{example}
The result is a (INTEGER -1610612736 1610612733), not a FIXNUM.
\end{example}

This seems strange, since there is a \code{the} declaration on the result of that
second addition.

In fact, the inefficiency is real, and is a consequence of \python{}'s
treating declarations as assertions to be verified.  The compiler
can't assume that the result type declaration is true\dash{}it must
generate the result and then test whether it is of the appropriate
type.

In practice, this means that when you are tuning a program to run
without type checks, you should work from the efficiency notes
generated by unsafe compilation.  If you want code to run efficiently
with type checking, then you should pay attention to all the
efficiency notes that you get during safe compilation.  Since user
supplied output type assertions (e.g., from \code{the}) are
disregarded when selecting operation implementations for safe code,
you must somehow give the compiler information that allows it to prove
that the result truly must be of a good type.  In our example, it
could be done by constraining the argument types more:

\begin{lisp}
(defun eff-note (x y z)
  (declare (type (unsigned-byte 18) x y z))
  (+ x y z))
\end{lisp}

Of course, this declaration is acceptable only if the arguments to \code{eff-note}
always \var{are} \w{\code{(unsigned-byte 18)}} integers.


\subsection{Representation Efficiency Notes}
\label{representation-eff-note}
\cindex{representation efficiency notes}
\cpsubindex{efficiency notes}{for representation}
\cindex{object representation efficiency notes}
\cindex{stack numbers}
\cindex{non-descriptor representations}
\cpsubindex{descriptor representations}{forcing of}

When operating on values that have non-descriptor representations
(\pxlref{non-descriptor}), there can be a substantial time and consing
penalty for converting to and from descriptor representations.  For
this reason, the compiler gives an efficiency note whenever it is
forced to do a representation coercion more expensive than
\varref{efficiency-note-cost-threshold}.

Inefficient representation coercions may be due to type uncertainty,
as in this example:

\begin{lisp}
(defun set-flo (x)
  (declare (single-float x))
  (prog ((var 0.0))
    (setq var (gorp))
    (setq var x)
    (return var)))
\end{lisp}

which produces this efficiency note:

\begin{example}
In: DEFUN SET-FLO
  (SETQ VAR X)
Note: Doing float to pointer coercion (cost 13) from X to VAR.
\end{example}

The variable \code{var} is not known to always hold values of type
\code{single-float}, so a descriptor representation must be used for its value.
In sort of situation, and adding a declaration will eliminate the inefficiency.

Often inefficient representation conversions are not due to type
uncertainty\dash{}instead, they result from evaluating a
non-descriptor expression in a context that requires a descriptor
result:

\begin{itemize} 
\item Assignment to or initialization of any data structure other than
  a specialized array (\pxlref{specialized-array-types}), or
  
\item Assignment to a \code{special} variable, or
  
\item Passing as an argument or returning as a value in any function
  call that is not a local call (\pxlref{number-local-call}.)
\end{itemize}

If such inefficient coercions appear in a ``hot spot'' in the program, data
structures redesign or program reorganization may be necessary to improve
efficiency.  See sections \ref{block-compilation}, \ref{numeric-types} and
\ref{profiling}.

Because representation selection is done rather late in compilation,
the source context in these efficiency notes is somewhat vague, making
interpretation more difficult.  This is a fairly straightforward
example:

\begin{lisp}
(defun cf+ (x y)
  (declare (single-float x y))
  (cons (+ x y) t))
\end{lisp}

which gives this efficiency note:

\begin{example}
In: DEFUN CF+
  (CONS (+ X Y) T)
Note: Doing float to pointer coercion (cost 13), for:
      The first argument of CONS.
\end{example}

The source context form is almost always the form that receives the value being
coerced (as it is in the preceding example), but can also be the source form
which generates the coerced value.  Compiling this example:

\begin{lisp}
(defun if-cf+ (x y)
  (declare (single-float x y))
  (cons (if (grue) (+ x y) (snoc)) t))
\end{lisp}

produces this note:

\begin{example}
In: DEFUN IF-CF+
  (+ X Y)
Note: Doing float to pointer coercion (cost 13).
\end{example}

In either case, the note's text explanation attempts to include
additional information about what locations are the source and
destination of the coercion.  Here are some example notes:
\begin{example}
  (IF (GRUE) X (SNOC))
Note: Doing float to pointer coercion (cost 13) from X.

  (SETQ VAR X)
Note: Doing float to pointer coercion (cost 13) from X to VAR.
\end{example}
Note that the return value of a function is also a place to which coercions may
have to be done:
\begin{example}
  (DEFUN F+ (X Y) (DECLARE (SINGLE-FLOAT X Y)) (+ X Y))
Note: Doing float to pointer coercion (cost 13) to "<return value>".
\end{example}
Sometimes the compiler is unable to determine a name for the source or
destination, in which case the source context is the only clue.


\subsection{Verbosity Control}
\cpsubindex{verbosity}{of efficiency notes}
\cpsubindex{efficiency notes}{verbosity}

These variables control the verbosity of efficiency notes:

\begin{defvar}{}{efficiency-note-cost-threshold}
  
  Before printing some efficiency notes, the compiler compares the
  value of this variable to the difference in cost between the chosen
  implementation and the best potential implementation.  If the
  difference is not greater than this limit, then no note is printed.
  The units are implementation dependent; the initial value suppresses
  notes about ``trivial'' inefficiencies.  A value of \code{1} will
  note any inefficiency.
\end{defvar}

\begin{defvar}{}{efficiency-note-limit}
  
  When printing some efficiency notes, the compiler reports possible
  efficient implementations.  The initial value of \code{2} prevents
  excessively long efficiency notes in the common case where there is
  no type information, so all implementations are possible.
\end{defvar}


\section{Profiling}
\cindex{profiling}
\cindex{timing}
\cindex{consing}
\cindex{tuning}
\label{profiling}

The first step in improving a program's performance is to profile the
activity of the program to find where it spends its time.  The best
way to do this is to use the profiling utility found in the
\code{profile} package.  This package provides a macro \code{profile}
that encapsulates functions with statistics gathering code.


\subsection{Profile Interface}

\begin{defvar}{profile:}{timed-functions}
  
  This variable holds a list of all functions that are currently being
  profiled.
\end{defvar}

\begin{defmac}{profile:}{profile}{%
    \args{\mstar{\var{name} \mor \kwd{callers} \code{t}}}}
  
  This macro wraps profiling code around the named functions.  As in
  \code{trace}, the \var{name}s are not evaluated.  If a function is
  already profiled, then the function is unprofiled and reprofiled
  (useful to notice function redefinition.)  A warning is printed for
  each name that is not a defined function.
  
  If \kwd{callers \var{t}} is specified, then each function that calls
  this function is recorded along with the number of calls made.
\end{defmac}

\begin{defmac}{profile:}{unprofile}{%
    \args{\mstar{\var{name}}}}
  
  This macro removes profiling code from the named functions.  If no
  \var{name}s are supplied, all currently profiled functions are
  unprofiled.
\end{defmac}

\begin{defmac}{profile:}{profile-all}{%
    \args{\keys{\kwd{package} \kwd{callers-p}}}}
  
  This macro in effect calls \code{profile:profile} for each
  function in the specified package which defaults to
  \code{*package*}.  \kwd{callers-p} has the same meaning as in
  \code{profile:profile}.
\end{defmac}

\begin{defmac}{profile:}{report-time}{\args{\mstar{\var{name}}}}
  
  This macro prints a report for each \var{name}d function of the
  following information:
  \begin{itemize}
  \item The total CPU time used in that function for all calls,
  
  \item the total number of bytes consed in that function for all
    calls,
  
  \item the total number of calls,
  
  \item the average amount of CPU time per call.
  \end{itemize}
  Summary totals of the CPU time, consing and calls columns are
  printed.  An estimate of the profiling overhead is also printed (see
  below).  If no \var{name}s are supplied, then the times for all
  currently profiled functions are printed.
\end{defmac}

\begin{defmac}{}{reset-time}{\args{\mstar{\var{name}}}}
  
  This macro resets the profiling counters associated with the
  \var{name}d functions.  If no \var{name}s are supplied, then all
  currently profiled functions are reset.
\end{defmac}


\subsection{Profiling Techniques}

Start by profiling big pieces of a program, then carefully choose which
functions close to, but not in, the inner loop are to be profiled next.
Avoid profiling functions that are called by other profiled functions, since
this opens the possibility of profiling overhead being included in the reported
times.

If the per-call time reported is less than 1/10 second, then consider the clock
resolution and profiling overhead before you believe the time.  It may be that
you will need to run your program many times in order to average out to a
higher resolution.


\subsection{Nested or Recursive Calls}

The profiler attempts to compensate for nested or recursive calls.  Time and
consing overhead will be charged to the dynamically innermost (most recent)
call to a profiled function.  So profiling a subfunction of a profiled function
will cause the reported time for the outer function to decrease.  However if an
inner function has a large number of calls, some of the profiling overhead may
``leak'' into the reported time for the outer function.  In general, be wary of
profiling short functions that are called many times.


\subsection{Clock resolution}

Unless you are very lucky, the length of your machine's clock ``tick'' is
probably much longer than the time it takes simple function to run.  For
example, on the IBM RT, the clock resolution is 1/50 second.  This means that
if a function is only called a few times, then only the first couple decimal
places are really meaningful.  

Note however, that if a function is called many times, then the statistical
averaging across all calls should result in increased resolution.  For example,
on the IBM RT, if a function is called a thousand times, then a resolution of
tens of microseconds can be expected.

\subsection{Profiling overhead}

The added profiling code takes time to run every time that the profiled
function is called, which can disrupt the attempt to collect timing
information.  In order to avoid serious inflation of the times for functions
that take little time to run, an estimate of the overhead due to profiling is
subtracted from the times reported for each function.

Although this correction works fairly well, it is not totally accurate,
resulting in times that become increasingly meaningless for functions with
short runtimes.  This is only a concern when the estimated profiling overhead
is many times larger than reported total CPU time.

The estimated profiling overhead is not represented in the reported total CPU
time.  The sum of total CPU time and the estimated profiling overhead should be
close to the total CPU time for the entire profiling run (as determined by the
\code{time} macro.)  Time unaccounted for is probably being used by functions that
you forgot to profile.

\subsection{Additional Timing Utilities}

\begin{defmac}{}{time}{ \args{\var{form}}}

  This macro evaluates \var{form}, prints some timing and memory
  allocation information to \code{*trace-output*}, and returns any
  values that \var{form} returns.  The timing information includes
  real time, user run time, and system run time.  This macro executes
  a form and reports the time and consing overhead.  If the
  \code{time} form is not compiled (e.g. it was typed at top-level),
  then \code{compile} will be called on the form to give more accurate
  timing information.  If you really want to time interpreted speed,
  you can say:
\begin{lisp}
(time (eval '\var{form}))
\end{lisp}
Things that execute fairly quickly should be timed more than once,
since there may be more paging overhead in the first timing.  To
increase the accuracy of very short times, you can time multiple
evaluations:
\begin{lisp}
(time (dotimes (i 100) \var{form}))
\end{lisp}
\end{defmac}

\begin{defun}{extensions:}{get-bytes-consed}{}
  
  This function returns the number of bytes allocated since the first
  time you called it.  The first time it is called it returns zero.
  The above profiling routines use this to report consing information.
\end{defun}

\begin{defvar}{extensions:}{gc-run-time}
  
  This variable accumulates the run-time consumed by garbage
  collection, in the units returned by
  \findexed{get-internal-run-time}.
\end{defvar}

\begin{defconst}{}{internal-time-units-per-second}
The value of internal-time-units-per-second is 100.
\end{defconst}

\subsection{A Note on Timing}
\cpsubindex{CPU time}{interpretation of}
\cpsubindex{run time}{interpretation of}
\cindex{interpretation of run time}

There are two general kinds of timing information provided by the
\code{time} macro and other profiling utilities: real time and run
time.  Real time is elapsed, wall clock time.  It will be affected in
a fairly obvious way by any other activity on the machine.  The more
other processes contending for CPU and memory, the more real time will
increase.  This means that real time measurements are difficult to
replicate, though this is less true on a dedicated workstation.  The
advantage of real time is that it is real.  It tells you really how
long the program took to run under the benchmarking conditions.  The
problem is that you don't know exactly what those conditions were.

Run time is the amount of time that the processor supposedly spent
running the program, as opposed to waiting for I/O or running other
processes.  ``User run time'' and ``system run time'' are numbers
reported by the Unix kernel.  They are supposed to be a measure of how
much time the processor spent running your ``user'' program (which
will include GC overhead, etc.), and the amount of time that the
kernel spent running ``on your behalf.''

Ideally, user time should be totally unaffected by benchmarking
conditions; in reality user time does depend on other system activity,
though in rather non-obvious ways.

System time will clearly depend on benchmarking conditions.  In Lisp
benchmarking, paging activity increases system run time (but not by as much
as it increases real time, since the kernel spends some time waiting for
the disk, and this is not run time, kernel or otherwise.)

In my experience, the biggest trap in interpreting kernel/user run time is
to look only at user time.  In reality, it seems that the \var{sum} of kernel
and user time is more reproducible.  The problem is that as system activity
increases, there is a spurious \var{decrease} in user run time.  In effect, as
paging, etc., increases, user time leaks into system time.

So, in practice, the only way to get truly reproducible results is to run
with the same competing activity on the system.  Try to run on a machine
with nobody else logged in, and check with ``ps aux'' to see if there are any
system processes munching large amounts of CPU or memory.  If the ratio
between real time and the sum of user and system time varies much between
runs, then you have a problem.


\subsection{Benchmarking Techniques}
\cindex{benchmarking techniques}

Given these imperfect timing tools, how do should you do benchmarking?  The
answer depends on whether you are trying to measure improvements in the
performance of a single program on the same hardware, or if you are trying to
compare the performance of different programs and/or different hardware.

For the first use (measuring the effect of program modifications with
constant hardware), you should look at \var{both} system+user and real time to
understand what effect the change had on CPU use, and on I/O (including
paging.)  If you are working on a CPU intensive program, the change in
system+user time will give you a moderately reproducible measure of
performance across a fairly wide range of system conditions.  For a CPU
intensive program, you can think of system+user as ``how long it would have
taken to run if I had my own machine.''  So in the case of comparing CPU
intensive programs, system+user time is relatively real, and reasonable to
use.

For programs that spend a substantial amount of their time paging, you
really can't predict elapsed time under a given operating condition without
benchmarking in that condition.  User or system+user time may be fairly
reproducible, but it is also relatively meaningless, since in a paging or
I/O intensive program, the program is spending its time waiting, not
running, and system time and user time are both measures of run time.
A change that reduces run time might increase real time by increasing
paging.

Another common use for benchmarking is comparing the performance of
the same program on different hardware.  You want to know which
machine to run your program on.  For comparing different machines
(operating systems, etc.), the only way to compare that makes sense is
to set up the machines in \var{exactly} the way that they will
\var{normally} be run, and then measure \var{real} time.  If the
program will normally be run along with X, then run X.  If the program
will normally be run on a dedicated workstation, then be sure nobody
else is on the benchmarking machine.  If the program will normally be
run on a machine with three other Lisp jobs, then run three other Lisp
jobs.  If the program will normally be run on a machine with 64MB of
memory, then run with 64MB.  Here, ``normal'' means ``normal for that
machine''.  

If you have a program you believe to be CPU intensive, then you might be
tempted to compare ``run'' times across systems, hoping to get a meaningful
result even if the benchmarking isn't done under the expected running
condition.  Don't to this, for two reasons:

\begin{itemize}  
\item The operating systems might not compute run time in the same
  way.
  
\item Under the real running condition, the program might not be CPU
  intensive after all.
\end{itemize}


In the end, only real time means anything\dash{}it is the amount of time you
have to wait for the result.  The only valid uses for run time are:

\begin{itemize}
\item To develop insight into the program.  For example, if run time
  is much less than elapsed time, then you are probably spending lots
  of time paging.
  
\item To evaluate the relative performance of CPU intensive programs
  in the same environment.
\end{itemize}

\chapter{UNIX Interface}
\label{unix-interface}

\credits{by Robert MacLachlan, Skef Wholey, Bill Chiles and William Lott}


\cmucl{} attempts to make the full power of the underlying
environment available to the Lisp programmer. This is done using
combination of hand-coded interfaces and foreign function calls to C
libraries. Although the techniques differ, the style of interface is
similar. This chapter provides an overview of the facilities available
and general rules for using them, as well as describing specific
features in detail. It is assumed that the reader has a working
familiarity with Unix and X11, as well as access to the standard
system documentation.


\section{Reading the Command Line}

The shell parses the command line with which Lisp is invoked, and
passes a data structure containing the parsed information to Lisp.
This information is then extracted from that data structure and put
into a set of Lisp data structures.

\begin{defvar}{extensions:}{command-line-strings}
  \defvarx[extensions:]{command-line-utility-name}
  \defvarx[extensions:]{command-line-words}
  \defvarx[extensions:]{command-line-switches}
  
  The value of \code{*command-line-words*} is a list of strings that
  make up the command line, one word per string.  The first word on
  the command line, i.e.  the name of the program invoked (usually
  \code{lisp}) is stored in \code{*command-line-utility-name*}.  The
  value of \code{*command-line-switches*} is a list of
  \code{command-line-switch} structures, with a structure for each
  word on the command line starting with a hyphen.  All the command
  line words between the program name and the first switch are stored
  in \code{*command-line-words*}.
\end{defvar}

The following functions may be used to examine \code{command-line-switch}
structures.
\begin{defun}{extensions:}{cmd-switch-name}{\args{\var{switch}}}
  
  Returns the name of the switch, less the preceding hyphen and
  trailing equal sign (if any).
\end{defun}
\begin{defun}{extensions:}{cmd-switch-value}{\args{\var{switch}}}
  
  Returns the value designated using an embedded equal sign, if any.
  If the switch has no equal sign, then this is null.
\end{defun}
\begin{defun}{extensions:}{cmd-switch-words}{\args{\var{switch}}}
  
  Returns a list of the words between this switch and the next switch
  or the end of the command line.
\end{defun}
\begin{defun}{extensions:}{cmd-switch-arg}{\args{\var{switch}}}
  
  Returns the first non-null value from \code{cmd-switch-value}, the
  first element in \code{cmd-switch-words}, or the first word in
  \var{command-line-words}.
\end{defun}

\begin{defun}{extensions:}{get-command-line-switch}{\args{\var{sname}}}
  
  This function takes the name of a switch as a string and returns the
  value of the switch given on the command line.  If no value was
  specified, then any following words are returned.  If there are no
  following words, then \true{} is returned.  If the switch was not
  specified, then \false{} is returned.
\end{defun}

\begin{defmac}{extensions:}{defswitch}{%
    \args{\var{name} \ampoptional{} \var{function}}}
  
  This macro causes \var{function} to be called when the switch
  \var{name} appears in the command line.  Name is a simple-string
  that does not begin with a hyphen (unless the switch name really
  does begin with one.)
  
  If \var{function} is not supplied, then the switch is parsed into
  \var{command-line-switches}, but otherwise ignored.  This suppresses
  the undefined switch warning which would otherwise take place.  The
  warning can also be globally suppressed by
  \var{complain-about-illegal-switches}.
\end{defmac}


\section{Useful Variables}

\begin{defvar}{system:}{stdin}
  \defvarx[system:]{stdout} \defvarx[system:]{stderr}
  
  Streams connected to the standard input, output and error file
  descriptors.
\end{defvar}

\begin{defvar}{system:}{tty}
  
  A stream connected to \file{/dev/tty}.
\end{defvar}

\begin{defvar}{extensions:}{environment-list}
  The environment variables inherited by the current process, as a
  keyword-indexed alist. For example, to access the DISPLAY
  environment variable, you could use

\begin{lisp}
   (cdr (assoc :display ext:*environment-list*))
\end{lisp}

  Note that the case of the variable name is preserved when converting
  to a keyword.  Therefore, you need to specify the keyword properly for
  variable names containing lower-case letters,
\end{defvar}


\section{Lisp Equivalents for C Routines}

The UNIX documentation describes the system interface in terms of C
procedure headers.  The corresponding Lisp function will have a somewhat
different interface, since Lisp argument passing conventions and
datatypes are different.

The main difference in the argument passing conventions is that Lisp does not
support passing values by reference.  In Lisp, all argument and results are
passed by value.  Interface functions take some fixed number of arguments and
return some fixed number of values.  A given ``parameter'' in the C
specification will appear as an argument, return value, or both, depending on
whether it is an In parameter, Out parameter, or In/Out parameter.  The basic
transformation one makes to come up with the Lisp equivalent of a C routine is
to remove the Out parameters from the call, and treat them as extra return
values.  In/Out parameters appear both as arguments and return values.  Since
Out and In/Out parameters are only conventions in C, you must determine the
usage from the documentation.

Thus, the C routine declared as

\begin{example}
kern_return_t lookup(servport, portsname, portsid)
        port        servport;
        char        *portsname;
        int        *portsid;        /* out */
 {
  ...
  *portsid = <expression to compute portsid field>
  return(KERN_SUCCESS);
 }
\end{example}

has as its Lisp equivalent something like

\begin{lisp}
(defun lookup (ServPort PortsName)
  ...
  (values
   success
   <expression to compute portsid field>))
\end{lisp}

If there are multiple out or in-out arguments, then there are multiple
additional returns values.

Fortunately, \cmucl{} programmers rarely have to worry about the
nuances of this translation process, since the names of the arguments and
return values are documented in a way so that the \code{describe} function
(and the \hemlock{} \code{Describe Function Call} command, invoked with
\b{C-M-Shift-A}) will list this information.  Since the names of arguments
and return values are usually descriptive, the information that
\code{describe} prints is usually all one needs to write a
call. Most programmers use this on-line documentation nearly
all of the time, and thereby avoid the need to handle bulky
manuals and perform the translation from barbarous tongues.


\section{Type Translations}
\cindex{aliens}
\cpsubindex{types}{alien}
\cpsubindex{types}{foreign language}

Lisp data types have very different representations from those used by
conventional languages such as C.  Since the system interfaces are
designed for conventional languages, Lisp must translate objects to and
from the Lisp representations.  Many simple objects have a direct
translation: integers, characters, strings and floating point numbers
are translated to the corresponding Lisp object.  A number of types,
however, are implemented differently in Lisp for reasons of clarity and
efficiency.

Instances of enumerated types are expressed as keywords in Lisp.
Records, arrays, and pointer types are implemented with the \alien{}
facility (\pxlref{aliens}).  Access functions are defined
for these types which convert fields of records, elements of arrays,
or data referenced by pointers into Lisp objects (possibly another
object to be referenced with another access function).

One should dispose of \alien{} objects created by constructor
functions or returned from remote procedure calls when they are no
longer of any use, freeing the virtual memory associated with that
object.  Since \alien{}s contain pointers to non-Lisp data, the
garbage collector cannot do this itself.  If the memory
was obtained from \funref{make-alien} or from a foreign function call
to a routine that used \code{malloc}, then \funref{free-alien} should
be used.


\section{System Area Pointers}
\label{system-area-pointers}

\cindex{pointers}\cpsubindex{malloc}{C function}\cpsubindex{free}{C function}
Note that in some cases an address is represented by a Lisp integer, and in
other cases it is represented by a real pointer.  Pointers are usually used
when an object in the current address space is being referred to.  The MACH
virtual memory manipulation calls must use integers, since in principle the
address could be in any process, and Lisp cannot abide random pointers.
Because these types are represented differently in Lisp, one must explicitly
coerce between these representations.

System Area Pointers (SAPs) provide a mechanism that bypasses the
\alien{} type system and accesses virtual memory directly.  A SAP is a
raw byte pointer into the \code{lisp} process address space.  SAPs are
represented with a pointer descriptor, so SAP creation can cause
consing.  However, the compiler uses a non-descriptor representation
for SAPs when possible, so the consing overhead is generally minimal.
\xlref{non-descriptor}.

\begin{defun}{system:}{sap-int}{\args{\var{sap}}}
  \defunx[system:]{int-sap}{\args{\var{int}}}
  
  The function \code{sap-int} is used to generate an integer
  corresponding to the system area pointer, suitable for passing to
  the kernel interfaces (which want all addresses specified as
  integers).  The function \code{int-sap} is used to do the opposite
  conversion.  The integer representation of a SAP is the byte offset
  of the SAP from the start of the address space.
\end{defun}

\begin{defun}{system:}{sap+}{\args{\var{sap} \var{offset}}}
  
  This function adds a byte \var{offset} to \var{sap}, returning a new
  SAP.
\end{defun}

\begin{defun}{system:}{sap-ref-8}{\args{\var{sap} \var{offset}}}
  \defunx[system:]{sap-ref-16}{\args{\var{sap} \var{offset}}}
  \defunx[system:]{sap-ref-32}{\args{\var{sap} \var{offset}}}
  
  These functions return the 8, 16 or 32 bit unsigned integer at
  \var{offset} from \var{sap}.  The \var{offset} is always a byte
  offset, regardless of the number of bits accessed.  \code{setf} may
  be used with the these functions to deposit values into virtual
  memory.
\end{defun}

\begin{defun}{system:}{signed-sap-ref-8}{\args{\var{sap} \var{offset}}}
  \defunx[system:]{signed-sap-ref-16}{\args{\var{sap} \var{offset}}}
  \defunx[system:]{signed-sap-ref-32}{\args{\var{sap} \var{offset}}}
  
  These functions are the same as the above unsigned operations,
  except that they sign-extend, returning a negative number if the
  high bit is set.
\end{defun}


\section{Unix System Calls}

You probably won't have much cause to use them, but all the Unix system
calls are available.  The Unix system call functions are in the
\code{Unix} package.  The name of the interface for a particular system
call is the name of the system call prepended with \code{unix-}.  The
system usually defines the associated constants without any prefix name.
To find out how to use a particular system call, try using
\code{describe} on it.  If that is unhelpful, look at the source in
\file{unix.lisp} or consult your system maintainer.

The Unix system calls indicate an error by returning \false{} as the
first value and the Unix error number as the second value.  If the call
succeeds, then the first value will always be non-\nil, often \code{t}.

For example, to use the \code{chdir} syscall: 

\begin{lisp}
(multiple-value-bind (success errno)
    (unix:unix-chdir "/tmp")
  (unless success
     (error "Can't change working directory: ~a"
            (unix:get-unix-error-msg errno))))
\end{lisp}

\begin{defun}{Unix:}{get-unix-error-msg}{\args{\var{error}}}

  This function returns a string describing the Unix error number
  \var{error} (this is similar to the Unix function \code{perror}). 
\end{defun}


\section{File Descriptor Streams}
\label{sec:fds}

Many of the UNIX system calls return file descriptors.  Instead of using other
UNIX system calls to perform I/O on them, you can create a stream around them.
For this purpose, fd-streams exist.  See also \funref{read-n-bytes}.

\begin{defun}{system:}{make-fd-stream}{%
    \args{\var{descriptor}} \keys{\kwd{input} \kwd{output}
      \kwd{element-type}} \morekeys{\kwd{buffering} \kwd{name}
      \kwd{file} \kwd{original}} \yetmorekeys{\kwd{delete-original}
      \kwd{auto-close}} \yetmorekeys{\kwd{timeout} \kwd{pathname}}}
  
  This function creates a file descriptor stream using
  \var{descriptor}.  If \kwd{input} is non-\nil, input operations are
  allowed.  If \kwd{output} is non-\nil, output operations are
  allowed.  The default is input only.  These keywords are defined:
  \begin{Lentry}
  \item[\kwd{element-type}] is the type of the unit of transaction for
    the stream, which defaults to \code{string-char}.  See the \clisp{}
    description of \code{open} for valid values.
  
  \item[\kwd{buffering}] is the kind of output buffering desired for
    the stream.  Legal values are \kwd{none} for no buffering,
    \kwd{line} for buffering up to each newline, and \kwd{full} for
    full buffering.
  
  \item[\kwd{name}] is a simple-string name to use for descriptive
    purposes when the system prints an fd-stream.  When printing
    fd-streams, the system prepends the streams name with \code{Stream
      for }.  If \var{name} is unspecified, it defaults to a string
    containing \var{file} or \var{descriptor}, in order of preference.
  
  \item[\kwd{file}, \kwd{original}] \var{file} specifies the defaulted
    namestring of the associated file when creating a file stream
    (must be a \code{simple-string}). \var{original} is the
    \code{simple-string} name of a backup file containing the original
    contents of \var{file} while writing \var{file}.
  
    When you abort the stream by passing \true{} to \code{close} as
    the second argument, if you supplied both \var{file} and
    \var{original}, \code{close} will rename the \var{original} name
    to the \var{file} name.  When you \code{close} the stream
    normally, if you supplied \var{original}, and
    \var{delete-original} is non-\nil, \code{close} deletes
    \var{original}.  If \var{auto-close} is true (the default), then
    \var{descriptor} will be closed when the stream is garbage
    collected.
  
  \item[\kwd{pathname}]: The original pathname passed to open and
    returned by \code{pathname}; not defaulted or translated.
  
  \item[\kwd{timeout}] if non-null, then \var{timeout} is an integer
    number of seconds after which an input wait should time out.  If a
    read does time out, then the \code{system:io-timeout} condition is
    signalled.
  \end{Lentry}
\end{defun}

\begin{defun}{system:}{fd-stream-p}{\args{\var{object}}}
  
  This function returns \true{} if \var{object} is an fd-stream, and
  \nil{} if not.  Obsolete: use the portable \code{(typep x
    'file-stream)}.
\end{defun}

\begin{defun}{system:}{fd-stream-fd}{\args{\var{stream}}}
  
  This returns the file descriptor associated with \var{stream}.
\end{defun}


\section{Unix Signals}
\cindex{unix signals} \cindex{signals}

\cmucl{} allows access to all the Unix signals that can be generated
under Unix.  It should be noted that if this capability is abused, it is
possible to completely destroy the running Lisp.  The following macros and
functions allow access to the Unix interrupt system.  The signal names as
specified in section 2 of the {\em Unix Programmer's Manual} are exported
from the Unix package.

\subsection{Changing Signal Handlers}
\label{signal-handlers}

\begin{defmac}{system:}{with-enabled-interrupts}{
    \args{\var{specs} \amprest{} \var{body}}}
  
  This macro should be called with a list of signal specifications,
  \var{specs}.  Each element of \var{specs} should be a list of
  two\hide{ or three} elements: the first should be the Unix signal
  for which a handler should be established, the second should be a
  function to be called when the signal is received\hide{, and the
    third should be an optional character used to generate the signal
    from the keyboard.  This last item is only useful for the SIGINT,
    SIGQUIT, and SIGTSTP signals.}  One or more signal handlers can be
  established in this way.  \code{with-enabled-interrupts} establishes
  the correct signal handlers and then executes the forms in
  \var{body}.  The forms are executed in an unwind-protect so that the
  state of the signal handlers will be restored to what it was before
  the \code{with-enabled-interrupts} was entered.  A signal handler
  function specified as NIL will set the Unix signal handler to the
  default which is normally either to ignore the signal or to cause a
  core dump depending on the particular signal.
\end{defmac}

\begin{defmac}{system:}{without-interrupts}{\args{\amprest{} \var{body}}}
  
  It is sometimes necessary to execute a piece a code that can not be
  interrupted.  This macro the forms in \var{body} with interrupts
  disabled.  Note that the Unix interrupts are not actually disabled,
  rather they are queued until after \var{body} has finished
  executing.
\end{defmac}

\begin{defmac}{system:}{with-interrupts}{\args{\amprest{} \var{body}}}
  
  When executing an interrupt handler, the system disables interrupts,
  as if the handler was wrapped in in a \code{without-interrupts}.
  The macro \code{with-interrupts} can be used to enable interrupts
  while the forms in \var{body} are evaluated.  This is useful if
  \var{body} is going to enter a break loop or do some long
  computation that might need to be interrupted.
\end{defmac}

\begin{defmac}{system:}{without-hemlock}{\args{\amprest{} \var{body}}}
  
  For some interrupts, such as SIGTSTP (suspend the Lisp process and
  return to the Unix shell) it is necessary to leave Hemlock and then
  return to it.  This macro executes the forms in \var{body} after
  exiting Hemlock.  When \var{body} has been executed, control is
  returned to Hemlock.
\end{defmac}

\begin{defun}{system:}{enable-interrupt}{%
    \args{\var{signal} \var{function}\hide{ \ampoptional{}
        \var{character}}}}
  
  This function establishes \var{function} as the handler for
  \var{signal}.
  \hide{The optional \var{character} can be specified
    for the SIGINT, SIGQUIT, and SIGTSTP signals and causes that
    character to generate the appropriate signal from the keyboard.}
  Unless you want to establish a global signal handler, you should use
  the macro \code{with-enabled-interrupts} to temporarily establish a
  signal handler.  \hide{Without \var{character},}
  \code{enable-interrupt} returns the old function associated with the
  signal.  \hide{When \var{character} is specified for SIGINT,
    SIGQUIT, or SIGTSTP, it returns the old character code.}
\end{defun}

\begin{defun}{system:}{ignore-interrupt}{\args{\var{signal}}}
  
  Ignore-interrupt sets the Unix signal mechanism to ignore
  \var{signal} which means that the Lisp process will never see the
  signal.  Ignore-interrupt returns the old function associated with
  the signal or \false{} if none is currently defined.
\end{defun}

\begin{defun}{system:}{default-interrupt}{\args{\var{signal}}}
  
  Default-interrupt can be used to tell the Unix signal mechanism to
  perform the default action for \var{signal}.  For details on what
  the default action for a signal is, see section 2 of the {\em Unix
    Programmer's Manual}.  In general, it is likely to ignore the
  signal or to cause a core dump.
\end{defun}


\subsection{Examples of Signal Handlers}

The following code is the signal handler used by the Lisp system for the
SIGINT signal.

\begin{lisp}
(defun ih-sigint (signal code scp)
  (declare (ignore signal code scp))
  (without-hemlock
   (with-interrupts
    (break "Software Interrupt" t))))
\end{lisp}

The \code{without-hemlock} form is used to make sure that Hemlock is exited before
a break loop is entered.  The \code{with-interrupts} form is used to enable
interrupts because the user may want to generate an interrupt while in the
break loop.  Finally, break is called to enter a break loop, so the user
can look at the current state of the computation.  If the user proceeds
from the break loop, the computation will be restarted from where it was
interrupted.

The following function is the Lisp signal handler for the SIGTSTP signal
which suspends a process and returns to the Unix shell.

\begin{lisp}
(defun ih-sigtstp (signal code scp)
  (declare (ignore signal code scp))
  (without-hemlock
   (Unix:unix-kill (Unix:unix-getpid) Unix:sigstop)))
\end{lisp}

Lisp uses this interrupt handler to catch the SIGTSTP signal because it is
necessary to get out of Hemlock in a clean way before returning to the shell.

To set up these interrupt handlers, the following is recommended:

\begin{lisp}
(with-enabled-interrupts ((Unix:SIGINT #'ih-sigint)
                          (Unix:SIGTSTP #'ih-sigtstp))
  <user code to execute with the above signal handlers enabled.>
)
\end{lisp}

\chapter{Event Dispatching with SERVE-EVENT}
\label{serve-event}

\credits{by Bill Chiles and Robert MacLachlan}


It is common to have multiple activities simultaneously operating in the same
Lisp process.  Furthermore, Lisp programmers tend to expect a flexible
development environment.  It must be possible to load and modify application
programs without requiring modifications to other running programs.  \cmucl{}
achieves this by having a central scheduling mechanism based on an
event-driven, object-oriented paradigm.

An \var{event} is some interesting happening that should cause the Lisp process
to wake up and do something.  These events include X events and activity on
Unix file descriptors.  The object-oriented mechanism is only available with
the first two, and it is optional with X events as described later in this
chapter.  In an X event, the window ID is the object capability and the X event
type is the operation code.  The Unix file descriptor input mechanism simply
consists of an association list of a handler to call when input shows up on a
particular file descriptor.


\section{Object Sets}
\label{object-sets}
\cindex{object sets}

An {\em object set} is a collection of objects that have the same implementation
for each operation.  Externally the object is represented by the object
capability and the operation is represented by the operation code.  Within
Lisp, the object is represented by an arbitrary Lisp object, and the
implementation for the operation is represented by an arbitrary Lisp function.
The object set mechanism maintains this translation from the external to the
internal representation.

\begin{defun}{system:}{make-object-set}{%
    \args{\var{name} \ampoptional{} \var{default-handler}}}
  
  This function makes a new object set.  \var{Name} is a string used
  only for purposes of identifying the object set when it is printed.
  \var{Default-handler} is the function used as a handler when an
  undefined operation occurs on an object in the set.  You can define
  operations with the \code{serve-}\var{operation} functions exported
  the \code{extensions} package for X events
  (\pxlref{x-serve-mumbles}).  Objects are added with
  \code{system:add-xwindow-object}.  Initially the object set has no
  objects and no defined operations.
\end{defun}

\begin{defun}{system:}{object-set-operation}{%
    \args{\var{object-set} \var{operation-code}}}
  
  This function returns the handler function that is the
  implementation of the operation corresponding to
  \var{operation-code} in \var{object-set}.  When set with
  \code{setf}, the setter function establishes the new handler.  The
  \code{serve-}\var{operation} functions exported from the
  \code{extensions} package for X events (\pxlref{x-serve-mumbles})
  call this on behalf of the user when announcing a new operation for
  an object set.
\end{defun}

\begin{defun}{system:}{add-xwindow-object}{%
    \args{\var{window} \var{object} \var{object-set}}}
  
  These functions add \var{port} or \var{window} to \var{object-set}.
  \var{Object} is an arbitrary Lisp object that is associated with the
  \var{port} or \var{window} capability.  \var{Window} is a CLX
  window.  When an event occurs, \code{system:serve-event} passes
  \var{object} as an argument to the handler function.
\end{defun}


\section{The SERVE-EVENT Function}

The \code{system:serve-event} function is the standard way for an application
to wait for something to happen.  For example, the Lisp system calls
\code{system:serve-event} when it wants input from X or a terminal stream.
The idea behind \code{system:serve-event} is that it knows the appropriate
action to take when any interesting event happens.  If an application calls
\code{system:serve-event} when it is idle, then any other applications with
pending events can run.  This allows several applications to run ``at the
same time'' without interference, even though there is only one thread of
control.  Note that if an application is waiting for input of any kind,
then other applications will get events.

\begin{defun}{system:}{serve-event}{\args{\ampoptional{} \var{timeout}}}
  
  This function waits for an event to happen and then dispatches to
  the correct handler function.  If specified, \var{timeout} is the
  number of seconds to wait before timing out.  A time out of zero
  seconds is legal and causes \code{system:serve-event} to poll for
  any events immediately available for processing.
  \code{system:serve-event} returns \true{} if it serviced at least
  one event, and \nil{} otherwise.  Depending on the application, when
  \code{system:serve-event} returns \true, you might want to call it
  repeatedly with a timeout of zero until it returns \nil.
  
  If input is available on any designated file descriptor, then this
  calls the appropriate handler function supplied by
  \code{system:add-fd-handler}.
  
  Since events for many different applications may arrive
  simultaneously, an application waiting for a specific event must
  loop on \code{system:serve-event} until the desired event happens.
  Since programs such as \hemlock{} call \code{system:serve-event} for
  input, applications usually do not need to call
  \code{system:serve-event} at all; \hemlock{} allows other
  application's handlers to run when it goes into an input wait.
\end{defun}

\begin{defun}{system:}{serve-all-events}{\args{\ampoptional{} \var{timeout}}}
  
  This function is similar to \code{system:serve-event}, except it
  serves all the pending events rather than just one.  It returns
  \true{} if it serviced at least one event, and \nil{} otherwise.
\end{defun}


\section{Using SERVE-EVENT with Unix File Descriptors}

Object sets are not available for use with file descriptors, as there are
only two operations possible on file descriptors: input and output.
Instead, a handler for either input or output can be registered with
\code{system:serve-event} for a specific file descriptor.  Whenever any input
shows up, or output is possible on this file descriptor, the function
associated with the handler for that descriptor is funcalled with the
descriptor as it's single argument.

\begin{defun}{system:}{add-fd-handler}{%
    \args{\var{fd} \var{direction} \var{function}}}
  
  This function installs and returns a new handler for the file
  descriptor \var{fd}.  \var{direction} can be either \kwd{input} if
  the system should invoke the handler when input is available or
  \kwd{output} if the system should invoke the handler when output is
  possible.  This returns a unique object representing the handler,
  and this is a suitable argument for \code{system:remove-fd-handler}
  \var{function} must take one argument, the file descriptor.
\end{defun}

\begin{defun}{system:}{remove-fd-handler}{\args{\var{handler}}}

  This function removes \var{handler}, that \code{add-fd-handler} must
  have previously returned.
\end{defun}

\begin{defmac}{system:}{with-fd-handler}{%
    \args{(\var{fd} \var{direction} \var{function})
      \mstar{\var{form}}}}
      
  This macro executes the supplied forms with a handler installed
  using \var{fd}, \var{direction}, and \var{function}.  See
  \code{system:add-fd-handler}.  The forms are wrapped in an
  \code{unwind-protect};  the handler is removed (see
  \code{system:remove-fd-handler}) when done.
\end{defmac}

\begin{defun}{system:}{wait-until-fd-usable}{%
    \args{\var{fd} \var{direction} \ampoptional{} \var{timeout}}}
      
  This function waits for up to \var{timeout} seconds for \var{fd} to
  become usable for \var{direction} (either \kwd{input} or
  \kwd{output}).  If \var{timeout} is \nil{} or unspecified, this
  waits forever.
\end{defun}

\begin{defun}{system:}{invalidate-descriptor}{\args{\var{fd}}}
  
  This function removes all handlers associated with \var{fd}.  This
  should only be used in drastic cases (such as I/O errors, but not
  necessarily EOF).  Normally, you should use \code{remove-fd-handler}
  to remove the specific handler.
\end{defun}




\section{Using SERVE-EVENT with the CLX Interface to X}
\label{x-serve-mumbles}

Remember from section \ref{object-sets}, an object set is a collection of
objects, CLX windows in this case, with some set of operations, event keywords,
with corresponding implementations, the same handler functions.  Since X allows
multiple display connections from a given process, you can avoid using object
sets if every window in an application or display connection behaves the same.
If a particular X application on a single display connection has windows that
want to handle certain events differently, then using object sets is a
convenient way to organize this since you need some way to map the window/event
combination to the appropriate functionality.

The following is a discussion of functions exported from the \code{extensions}
package that facilitate handling CLX events through \code{system:serve-event}.
The first two routines are useful regardless of whether you use
\code{system:serve-event}:
\begin{defun}{ext:}{open-clx-display}{%
    \args{\ampoptional{} \var{string}}}
  
  This function parses \var{string} for an X display specification
  including display and screen numbers.  \var{String} defaults to the
  following:
  \begin{example}
    (cdr (assoc :display ext:*environment-list* :test #'eq))
  \end{example}
  If any field in the display specification is missing, this signals
  an error.  \code{ext:open-clx-display} returns the CLX display and
  screen.
\end{defun}

\begin{defun}{ext:}{flush-display-events}{\args{\var{display}}}
  
  This function flushes all the events in \var{display}'s event queue
  including the current event, in case the user calls this from within
  an event handler.
\end{defun}



\subsection{Without Object Sets}

Since most applications that use CLX, can avoid the complexity of object sets,
these routines are described in a separate section.  The routines described in
the next section that use the object set mechanism are based on these
interfaces.

\begin{defun}{ext:}{enable-clx-event-handling}{%
    \args{\var{display} \var{handler}}} 
  
  This function causes \code{system:serve-event} to notice when there
  is input on \var{display}'s connection to the X11 server.  When this
  happens, \code{system:serve-event} invokes \var{handler} on
  \var{display} in a dynamic context with an error handler bound that
  flushes all events from \var{display} and returns.  By returning,
  the error handler declines to handle the error, but it will have
  cleared all events; thus, entering the debugger will not result in
  infinite errors due to streams that wait via
  \code{system:serve-event} for input.  Calling this repeatedly on the
  same \var{display} establishes \var{handler} as a new handler,
  replacing any previous one for \var{display}.
\end{defun}

\begin{defun}{ext:}{disable-clx-event-handling}{\args{\var{display}}}

  This function undoes the effect of
  \code{ext:enable-clx-event-handling}.
\end{defun}

\begin{defmac}{ext:}{with-clx-event-handling}{%
    \args{(\var{display} \var{handler}) \mstar{form}}}
  
  This macro evaluates each \var{form} in a context where
  \code{system:serve-event} invokes \var{handler} on \var{display}
  whenever there is input on \var{display}'s connection to the X
  server.  This destroys any previously established handler for
  \var{display}.
\end{defmac}


\subsection{With Object Sets}

This section discusses the use of object sets and
\code{system:serve-event} to handle CLX events.  This is necessary
when a single X application has distinct windows that want to handle
the same events in different ways.  Basically, you need some way of
asking for a given window which way you want to handle some event
because this event is handled differently depending on the window.
Object sets provide this feature.

For each CLX event-key symbol-name \i{XXX} (for example,
\var{key-press}), there is a function \code{serve-}\i{XXX} of two
arguments, an object set and a function.  The \code{serve-}\i{XXX}
function establishes the function as the handler for the \kwd{XXX}
event in the object set.  Recall from section \ref{object-sets},
\code{system:add-xwindow-object} associates some Lisp object with a
CLX window in an object set.  When \code{system:serve-event} notices
activity on a window, it calls the function given to
\code{ext:enable-clx-event-handling}.  If this function is
\code{ext:object-set-event-handler}, it calls the function given to
\code{serve-}\i{XXX}, passing the object given to
\code{system:add-xwindow-object} and the event's slots as well as a
couple other arguments described below.

To use object sets in this way:

\begin{itemize} 
\item Create an object set.
  
\item Define some operations on it using the \code{serve-}\i{XXX}
  functions.
  
\item Add an object for every window on which you receive requests.
  This can be the CLX window itself or some structure more meaningful
  to your application.
  
\item Call \code{system:serve-event} to service an X event.
\end{itemize}


\begin{defun}{ext:}{object-set-event-handler}{%
    \args{\var{display}}}
  
  This function is a suitable argument to
  \code{ext:enable-clx-event-handling}.  The actual event handlers
  defined for particular events within a given object set must take an
  argument for every slot in the appropriate event.  In addition to
  the event slots, \code{ext:object-set-event-handler} passes the
  following arguments:
  \begin{itemize}
  \item The object, as established by
    \code{system:add-xwindow-object}, on which the event occurred.
  \item event-key, see \code{xlib:event-case}.
  \item send-event-p, see \code{xlib:event-case}.
  \end{itemize}
  
  Describing any \code{ext:serve-}\var{event-key-name} function, where
  \var{event-key-name} is an event-key symbol-name (for example,
  \code{ext:serve-key-press}), indicates exactly what all the
  arguments are in their correct order.

%%  \begin{comment}
%%    \code{ext:object-set-event-handler} ignores \kwd{no-exposure}
%%    events on pixmaps, issuing a warning if one occurs.  It is only
%%    prepared to dispatch events for windows.
%%  \end{comment}
  
  When creating an object set for use with
  \code{ext:object-set-event-handler}, specify
  \code{ext:default-clx-event-handler} as the default handler for
  events in that object set.  If no default handler is specified, and
  the system invokes the default default handler, it will cause an
  error since this function takes arguments suitable for handling port
  messages.
\end{defun}


\section{A SERVE-EVENT Example}

This section contains two examples using \code{system:serve-event}.  The first
one does not use object sets, and the second, slightly more complicated one
does.

\subsection{Without Object Sets Example}

This example defines an input handler for a CLX display connection.  It only
recognizes \kwd{key-press} events.  The body of the example loops over
\code{system:serve-event} to get input.

\begin{lisp}
(in-package "SERVER-EXAMPLE")

(defun my-input-handler (display)
  (xlib:event-case (display :timeout 0)
    (:key-press (event-window code state)
     (format t "KEY-PRESSED (Window = ~D) = ~S.~%"
                  (xlib:window-id event-window)
             ;; See Hemlock Command Implementor's Manual for convenient
             ;; input mapping function.
             (ext:translate-character display code state))
      ;; Make XLIB:EVENT-CASE discard the event.
      t)))
\end{lisp}

\begin{lisp}
(defun server-example ()
  "An example of using the SYSTEM:SERVE-EVENT function and object sets to
   handle CLX events."
  (let* ((display (ext:open-clx-display))
         (screen (display-default-screen display))
         (black (screen-black-pixel screen))
         (white (screen-white-pixel screen))
         (window (create-window :parent (screen-root screen)
                                :x 0 :y 0 :width 200 :height 200
                                :background white :border black
                                :border-width 2
                                :event-mask
                                (xlib:make-event-mask :key-press))))
    ;; Wrap code in UNWIND-PROTECT, so we clean up after ourselves.
    (unwind-protect
        (progn
          ;; Enable event handling on the display.
          (ext:enable-clx-event-handling display #'my-input-handler)
          ;; Map the windows to the screen.
          (map-window window)
          ;; Make sure we send all our requests.
          (display-force-output display)
          ;; Call serve-event for 100,000 events or immediate timeouts.
          (dotimes (i 100000) (system:serve-event)))
      ;; Disable event handling on this display.
      (ext:disable-clx-event-handling display)
      ;; Get rid of the window.
      (destroy-window window)
      ;; Pick off any events the X server has already queued for our
      ;; windows, so we don't choke since SYSTEM:SERVE-EVENT is no longer
      ;; prepared to handle events for us.
      (loop
       (unless (deleting-window-drop-event *display* window)
        (return)))
      ;; Close the display.
      (xlib:close-display display))))

(defun deleting-window-drop-event (display win)
  "Check for any events on win.  If there is one, remove it from the
   event queue and return t; otherwise, return nil."
  (xlib:display-finish-output display)
  (let ((result nil))
    (xlib:process-event
     display :timeout 0
     :handler #'(lambda (&key event-window &allow-other-keys)
                  (if (eq event-window win)
                      (setf result t)
                      nil)))
    result))
\end{lisp}


\subsection{With Object Sets Example}

This example involves more work, but you get a little more for your effort.  It
defines two objects, \code{input-box} and \code{slider}, and establishes a
\kwd{key-press} handler for each object, \code{key-pressed} and
\code{slider-pressed}.  We have two object sets because we handle events on the
windows manifesting these objects differently, but the events come over the
same display connection.

\begin{lisp}
(in-package "SERVER-EXAMPLE")

(defstruct (input-box (:print-function print-input-box)
                      (:constructor make-input-box (display window)))
  "Our program knows about input-boxes, and it doesn't care how they
   are implemented."
  display        ; The CLX display on which my input-box is displayed.
  window)        ; The CLX window in which the user types.
;;;
(defun print-input-box (object stream n)
  (declare (ignore n))
  (format stream "#<Input-Box ~S>" (input-box-display object)))

(defvar *input-box-windows*
        (system:make-object-set "Input Box Windows"
                                #'ext:default-clx-event-handler))

(defun key-pressed (input-box event-key event-window root child
                    same-screen-p x y root-x root-y modifiers time
                    key-code send-event-p)
  "This is our :key-press event handler."
  (declare (ignore event-key root child same-screen-p x y
                   root-x root-y time send-event-p))
  (format t "KEY-PRESSED (Window = ~D) = ~S.~%"
          (xlib:window-id event-window)
          ;; See Hemlock Command Implementor's Manual for convenient
          ;; input mapping function.
          (ext:translate-character (input-box-display input-box)
                                     key-code modifiers)))
;;;
(ext:serve-key-press *input-box-windows* #'key-pressed)
\end{lisp}

\begin{lisp}
(defstruct (slider (:print-function print-slider)
                   (:include input-box)
                   (:constructor %make-slider
                                    (display window window-width max)))
  "Our program knows about sliders too, and these provide input values
   zero to max."
  bits-per-value  ; bits per discrete value up to max.
  max)            ; End value for slider.
;;;
(defun print-slider (object stream n)
  (declare (ignore n))
  (format stream "#<Slider ~S  0..~D>"
          (input-box-display object)
          (1- (slider-max object))))
;;;
(defun make-slider (display window max)
  (%make-slider display window
                  (truncate (xlib:drawable-width window) max)
                max))

(defvar *slider-windows*
        (system:make-object-set "Slider Windows"
                                #'ext:default-clx-event-handler))

(defun slider-pressed (slider event-key event-window root child
                       same-screen-p x y root-x root-y modifiers time
                       key-code send-event-p)
  "This is our :key-press event handler for sliders.  Probably this is
   a mouse thing, but for simplicity here we take a character typed."
  (declare (ignore event-key root child same-screen-p x y
                   root-x root-y time send-event-p))
  (format t "KEY-PRESSED (Window = ~D) = ~S  -->  ~D.~%"
          (xlib:window-id event-window)
          ;; See Hemlock Command Implementor's Manual for convenient
          ;; input mapping function.
          (ext:translate-character (input-box-display slider)
                                     key-code modifiers)
          (truncate x (slider-bits-per-value slider))))
;;;
(ext:serve-key-press *slider-windows* #'slider-pressed)
\end{lisp}

\begin{lisp}
(defun server-example ()
  "An example of using the SYSTEM:SERVE-EVENT function and object sets to
   handle CLX events."
  (let* ((display (ext:open-clx-display))
         (screen (display-default-screen display))
         (black (screen-black-pixel screen))
         (white (screen-white-pixel screen))
         (iwindow (create-window :parent (screen-root screen)
                                 :x 0 :y 0 :width 200 :height 200
                                 :background white :border black
                                 :border-width 2
                                 :event-mask
                                 (xlib:make-event-mask :key-press)))
         (swindow (create-window :parent (screen-root screen)
                                 :x 0 :y 300 :width 200 :height 50
                                 :background white :border black
                                 :border-width 2
                                 :event-mask
                                 (xlib:make-event-mask :key-press)))
         (input-box (make-input-box display iwindow))
         (slider (make-slider display swindow 15)))
    ;; Wrap code in UNWIND-PROTECT, so we clean up after ourselves.
    (unwind-protect
        (progn
          ;; Enable event handling on the display.
          (ext:enable-clx-event-handling display
                                         #'ext:object-set-event-handler)
          ;; Add the windows to the appropriate object sets.
          (system:add-xwindow-object iwindow input-box
                                       *input-box-windows*)
          (system:add-xwindow-object swindow slider
                                       *slider-windows*)
          ;; Map the windows to the screen.
          (map-window iwindow)
          (map-window swindow)
          ;; Make sure we send all our requests.
          (display-force-output display)
          ;; Call server for 100,000 events or immediate timeouts.
          (dotimes (i 100000) (system:serve-event)))
      ;; Disable event handling on this display.
      (ext:disable-clx-event-handling display)
      (delete-window iwindow display)
      (delete-window swindow display)
      ;; Close the display.
      (xlib:close-display display))))
\end{lisp}

\begin{lisp}
(defun delete-window (window display)
  ;; Remove the windows from the object sets before destroying them.
  (system:remove-xwindow-object window)
  ;; Destroy the window.
  (destroy-window window)
  ;; Pick off any events the X server has already queued for our
  ;; windows, so we don't choke since SYSTEM:SERVE-EVENT is no longer
  ;; prepared to handle events for us.
  (loop
   (unless (deleting-window-drop-event display window)
     (return))))

(defun deleting-window-drop-event (display win)
  "Check for any events on win.  If there is one, remove it from the
   event queue and return t; otherwise, return nil."
  (xlib:display-finish-output display)
  (let ((result nil))
    (xlib:process-event
     display :timeout 0
     :handler #'(lambda (&key event-window &allow-other-keys)
                  (if (eq event-window win)
                      (setf result t)
                      nil)))
    result))
\end{lisp}

\chapter{Alien Objects}
\label{aliens}

\credits{by Robert MacLachlan and William Lott}


\section{Introduction to Aliens}

Because of Lisp's emphasis on dynamic memory allocation and garbage
collection, Lisp implementations use unconventional memory representations
for objects.  This representation mismatch creates problems when a Lisp
program must share objects with programs written in another language.  There
are three different approaches to establishing communication:

\begin{itemize}
\item The burden can be placed on the foreign program (and programmer) by
requiring the use of Lisp object representations.  The main difficulty with
this approach is that either the foreign program must be written with Lisp
interaction in mind, or a substantial amount of foreign ``glue'' code must be
written to perform the translation.

\item The Lisp system can automatically convert objects back and forth
between the Lisp and foreign representations.  This is convenient, but
translation becomes prohibitively slow when large or complex data structures
must be shared.

\item The Lisp program can directly manipulate foreign objects through the
use of extensions to the Lisp language.  Most Lisp systems make use of
this approach, but the language for describing types and expressing
accesses is often not powerful enough for complex objects to be easily
manipulated.
\end{itemize}

\cmucl{} relies primarily on the automatic conversion and direct manipulation
approaches: Aliens of simple scalar types are automatically converted,
while complex types are directly manipulated in their foreign
representation.  Any foreign objects that can't automatically be
converted into Lisp values are represented by objects of type
\code{alien-value}.  Since Lisp is a dynamically typed language, even
foreign objects must have a run-time type; this type information is
provided by encapsulating the raw pointer to the foreign data within an
\code{alien-value} object.

The Alien type language and operations are most similar to those of the
C language, but Aliens can also be used when communicating with most
other languages that can be linked with C.


\section{Alien Types}

Alien types have a description language based on nested list structure.  For
example:

\begin{example}
struct foo \{
    int a;
    struct foo *b[100];
\};
\end{example}

has the corresponding Alien type:

\begin{lisp}
(struct foo
  (a int)
  (b (array (* (struct foo)) 100)))
\end{lisp}


\subsection{Defining Alien Types}

Types may be either named or anonymous.  With structure and union
types, the name is part of the type specifier, allowing recursively
defined types such as:

\begin{lisp}
(struct foo (a (* (struct foo))))
\end{lisp}

An anonymous structure or union type is specified by using the name
\nil{}.  The \funref{with-alien} macro defines a local scope which
``captures'' any named type definitions.  Other types are not
inherently named, but can be given named abbreviations using
\code{def-alien-type}.

\begin{defmac}{alien:}{def-alien-type}{name type}  
  This macro globally defines \var{name} as a shorthand for the Alien
  type \var{type}.  When introducing global structure and union type
  definitions, \var{name} may be \nil, in which case the name to
  define is taken from the type's name.
\end{defmac}


\subsection{Alien Types and Lisp Types}

The Alien types form a subsystem of the \cmucl{} type system.  An
\code{alien} type specifier provides a way to use any Alien type as a
Lisp type specifier.  For example

\begin{lisp}
(typep foo '(alien (* int)))
\end{lisp}

can be used to determine whether \code{foo} is a pointer to an
\code{int}.  \code{alien} type specifiers can be used in the same ways
as ordinary type specifiers (like \code{string}.)  Alien type
declarations are subject to the same precise type checking as any
other declaration (section \xlref{precise-type-checks}.)

Note that the Alien type system overlaps with normal Lisp type
specifiers in some cases.  For example, the type specifier
\code{(alien single-float)} is identical to \code{single-float}, since
Alien floats are automatically converted to Lisp floats.  When
\code{type-of} is called on an Alien value that is not automatically
converted to a Lisp value, then it will return an \code{alien} type
specifier.


\subsection{Alien Type Specifiers}

Some Alien type names are \clisp{} symbols, but the names are
still exported from the \code{alien} package, so it is legal to say
\code{alien:single-float}.  These are the basic Alien type specifiers: 

\begin{deftp}{Alien type}{*}{%
    \args{\var{type}}}
  
  A pointer to an object of the specified \var{type}.  If \var{type}
  is \true, then it means a pointer to anything, similar to
  ``\code{void *}'' in ANSI C.  Currently, the only way to detect a
  null pointer is:
\begin{lisp}
  (zerop (sap-int (alien-sap \var{ptr})))
\end{lisp}
\xlref{system-area-pointers}
\end{deftp}

\begin{deftp}{Alien type}{array}{\var{type} \mstar{\var{dimension}}} 

  An array of the specified \var{dimensions}, holding elements of type
  \var{type}.  Note that \code{(* int)} and \code{(array int)} are
  considered to be different types when type checking is done; pointer
  and array types must be explicitly coerced using \code{cast}.
  
  Arrays are accessed using \code{deref}, passing the indices as
  additional arguments.  Elements are stored in column-major order (as
  in C), so the first dimension determines only the size of the memory
  block, and not the layout of the higher dimensions.  An array whose
  first dimension is variable may be specified by using \nil{} as the
  first dimension.  Fixed-size arrays can be allocated as array
  elements, structure slots or \code{with-alien} variables.  Dynamic
  arrays can only be allocated using \funref{make-alien}.
\end{deftp}

\begin{deftp}{Alien type}{struct}{\var{name} 
    \mstar{(\var{field} \var{type} \mopt{\var{bits}})}}
  
  A structure type with the specified \var{name} and \var{fields}.
  Fields are allocated at the same positions used by the
  implementation's C compiler.  \var{bits} is intended for C-like bit
  field support, but is currently unused.  If \var{name} is \false,
  then the type is anonymous.
  
  If a named Alien \code{struct} specifier is passed to
  \funref{def-alien-type} or \funref{with-alien}, then this defines,
  respectively, a new global or local Alien structure type.  If no
  \var{fields} are specified, then the fields are taken from the
  current (local or global) Alien structure type definition of
  \var{name}.
\end{deftp}

\begin{deftp}{Alien type}{union}{\var{name} 
    \mstar{(\var{field} \var{type} \mopt{\var{bits}})}}
  
  Similar to \code{struct}, but defines a union type.  All fields are
  allocated at the same offset, and the size of the union is the size
  of the largest field.  The programmer must determine which field is
  active from context.
\end{deftp}

\begin{deftp}{Alien type}{enum}{\var{name} \mstar{\var{spec}}}
  
  An enumeration type that maps between integer values and keywords.
  If \var{name} is \false, then the type is anonymous.  Each
  \var{spec} is either a keyword, or a list \code{(\var{keyword}
    \var{value})}.  If \var{integer} is not supplied, then it defaults
  to one greater than the value for the preceding spec (or to zero if
  it is the first spec.)
\end{deftp}

\begin{deftp}{Alien type}{signed}{\mopt{\var{bits}}}  
  A signed integer with the specified number of bits precision.  The
  upper limit on integer precision is determined by the machine's word
  size.  If no size is specified, the maximum size will be used.
\end{deftp}

\begin{deftp}{Alien type}{integer}{\mopt{\var{bits}}}  
  Identical to \code{signed}---the distinction between \code{signed}
  and \code{integer} is purely stylistic.
\end{deftp}

\begin{deftp}{Alien type}{unsigned}{\mopt{\var{bits}}}
  Like \code{signed}, but specifies an unsigned integer.
\end{deftp}

\begin{deftp}{Alien type}{boolean}{\mopt{\var{bits}}}
  Similar to an enumeration type that maps \code{0} to \false{} and
  all other values to \true.  \var{bits} determines the amount of
  storage allocated to hold the truth value.
\end{deftp}

\begin{deftp}{Alien type}{single-float}{}
  A floating-point number in IEEE single format.
\end{deftp}

\begin{deftp}{Alien type}{double-float}{}
  A floating-point number in IEEE double format.
\end{deftp}

\begin{deftp}{Alien type}{function}{\var{result-type} \mstar{\var{arg-type}}}
  \label{alien-function-types}
  A Alien function that takes arguments of the specified
  \var{arg-types} and returns a result of type \var{result-type}.
  Note that the only context where a \code{function} type is directly
  specified is in the argument to \code{alien-funcall} (see section
  \funref{alien-funcall}.)  In all other contexts, functions are
  represented by function pointer types: \code{(* (function ...))}.
\end{deftp}

\begin{deftp}{Alien type}{system-area-pointer}{}
  A pointer which is represented in Lisp as a
  \code{system-area-pointer} object (\pxlref{system-area-pointers}.)
\end{deftp}


\subsection{The C-Call Package}

The \code{c-call} package exports these type-equivalents to the C type
of the same name: \code{char}, \code{short}, \code{int}, \code{long},
\code{unsigned-char}, \code{unsigned-short}, \code{unsigned-int},
\code{unsigned-long}, \code{float}, \code{double}.  \code{c-call} also
exports these types:

\begin{deftp}{Alien type}{void}{}
  This type is used in function types to declare that no useful value
  is returned.  Evaluation of an \code{alien-funcall} form will return
  zero values.
\end{deftp}

\begin{deftp}{Alien type}{c-string}{}
  This type is similar to \code{(* char)}, but is interpreted as a
  null-terminated string, and is automatically converted into a Lisp
  string when accessed.  If the pointer is C \code{NULL} (or 0), then
  accessing gives Lisp \false.
  
  Assigning a Lisp string to a \code{c-string} structure field or
  variable stores the contents of the string to the memory already
  pointed to by that variable.  When an Alien of type \code{(* char)}
  is assigned to a \code{c-string}, then the \code{c-string} pointer
  is assigned to.  This allows \code{c-string} pointers to be
  initialized.  For example:

\begin{lisp}
  (def-alien-type nil (struct foo (str c-string)))
  
  (defun make-foo (str) (let ((my-foo (make-alien (struct foo))))
  (setf (slot my-foo 'str) (make-alien char (length str))) (setf (slot
  my-foo 'str) str) my-foo))
\end{lisp}

Storing Lisp \false{} writes C \code{NULL} to the \code{c-string}
pointer.
\end{deftp}



\section{Alien Operations}

This section describes the basic operations on Alien values.

\subsection{Alien Access Operations}

\begin{defun}{alien:}{deref}{\args{\var{pointer-or-array} \amprest \var{indices}}}
  
  This function returns the value pointed to by an Alien pointer or
  the value of an Alien array element.  If a pointer, an optional
  single index can be specified to give the equivalent of C pointer
  arithmetic; this index is scaled by the size of the type pointed to.
  If an array, the number of indices must be the same as the number of
  dimensions in the array type.  \code{deref} can be set with
  \code{setf} to assign a new value.
\end{defun}
 
\begin{defun}{alien:}{slot}{\args{\var{struct-or-union} \var{slot-name}}}
  
  This function extracts the value of slot \var{slot-name} from the an
  Alien \code{struct} or \code{union}.  If \var{struct-or-union} is a
  pointer to a structure or union, then it is automatically
  dereferenced.  This can be set with \code{setf} to assign a new
  value.  Note that \var{slot-name} is evaluated, and need not be a
  compile-time constant (but only constant slot accesses are
  efficiently compiled.)
\end{defun}


\subsection{Alien Coercion Operations}

\begin{defmac}{alien:}{addr}{\var{alien-expr}}
  
  This macro returns a pointer to the location specified by
  \var{alien-expr}, which must be either an Alien variable, a use of
  \code{deref}, a use of \code{slot}, or a use of
  \funref{extern-alien}.
\end{defmac}

\begin{defmac}{alien:}{cast}{\var{alien} \var{new-type}}
  
  This macro converts \var{alien} to a new Alien with the specified
  \var{new-type}.  Both types must be an Alien pointer, array or
  function type.  Note that the result is not \code{eq} to the
  argument, but does refer to the same data bits.
\end{defmac}

\begin{defmac}{alien:}{sap-alien}{\var{sap} \var{type}}
  \defunx[alien:]{alien-sap}{\var{alien-value}}
  
  \code{sap-alien} converts \var{sap} (a system area pointer
  \pxlref{system-area-pointers}) to an Alien value with the specified
  \var{type}.  \var{type} is not evaluated.

\code{alien-sap} returns the SAP which points to \var{alien-value}'s
data.

The \var{type} to \code{sap-alien} and the type of the \var{alien-value} to
\code{alien-sap} must some Alien pointer, array or record type.
\end{defmac}


\subsection{Alien Dynamic Allocation}

Dynamic Aliens are allocated using the \code{malloc} library, so foreign code
can call \code{free} on the result of \code{make-alien}, and Lisp code can
call \code{free-alien} on objects allocated by foreign code.

\begin{defmac}{alien:}{make-alien}{\var{type} \mopt{\var{size}}}
  
  This macro returns a dynamically allocated Alien of the specified
  \var{type} (which is not evaluated.)  The allocated memory is not
  initialized, and may contain arbitrary junk.  If supplied,
  \var{size} is an expression to evaluate to compute the size of the
  allocated object.  There are two major cases:
  \begin{itemize}
  \item When \var{type} is an array type, an array of that type is
    allocated and a \var{pointer} to it is returned.  Note that you
    must use \code{deref} to change the result to an array before you
    can use \code{deref} to read or write elements:

\begin{lisp}
(defvar *foo* (make-alien (array char 10)))

(type-of *foo*) \result{} (alien (* (array (signed 8) 10)))

(setf (deref (deref foo) 0) 10) \result{} 10
\end{lisp}

    If supplied, \var{size} is used as the first dimension for the
    array.
    
  \item When \var{type} is any other type, then then an object for
    that type is allocated, and a \var{pointer} to it is returned.  So
    \code{(make-alien int)} returns a \code{(* int)}.  If \var{size}
    is specified, then a block of that many objects is allocated, with
    the result pointing to the first one.
  \end{itemize}
\end{defmac}
 
\begin{defun}{alien:}{free-alien}{\var{alien}}

  This function frees the storage for \var{alien} (which must have
  been allocated with \code{make-alien} or \code{malloc}.)
\end{defun}

See also \funref{with-alien}, which stack-allocates Aliens.


\section{Alien Variables}

Both local (stack allocated) and external (C global) Alien variables are
supported.


\subsection{Local Alien Variables}

\begin{defmac}{alien:}{with-alien}{\mstar{(\var{name} \var{type} 
      \mopt{\var{initial-value}})} \mstar{form}}
  
  This macro establishes local alien variables with the specified
  Alien types and names for dynamic extent of the body.  The variable
  \var{names} are established as symbol-macros; the bindings have
  lexical scope, and may be assigned with \code{setq} or \code{setf}.
  This form is analogous to defining a local variable in C: additional
  storage is allocated, and the initial value is copied.
  
  \code{with-alien} also establishes a new scope for named structures
  and unions.  Any \var{type} specified for a variable may contain
  name structure or union types with the slots specified.  Within the
  lexical scope of the binding specifiers and body, a locally defined
  structure type \var{foo} can be referenced by its name using:
\begin{lisp}
  (struct foo)
\end{lisp}
\end{defmac}


\subsection{External Alien Variables} 
\label{external-aliens}

External Alien names are strings, and Lisp names are symbols.  When an
external Alien is represented using a Lisp variable, there must be a
way to convert from one name syntax into the other.  The macros
\code{extern-alien}, \code{def-alien-variable} and
\funref{def-alien-routine} use this conversion heuristic:
\begin{itemize}
\item Alien names are converted to Lisp names by uppercasing and
  replacing underscores with hyphens.
  
\item Conversely, Lisp names are converted to Alien names by
  lowercasing and replacing hyphens with underscores.
  
\item Both the Lisp symbol and Alien string names may be separately
  specified by using a list of the form:
\begin{lisp}
  (\var{alien-string} \var{lisp-symbol})
\end{lisp}
\end{itemize}

\begin{defmac}{alien:}{def-alien-variable}{\var{name} \var{type}}
  
  This macro defines \var{name} as an external Alien variable of the
  specified Alien \var{type}.  \var{name} and \var{type} are not
  evaluated.  The Lisp name of the variable (see above) becomes a
  global Alien variable in the Lisp namespace.  Global Alien variables
  are effectively ``global symbol macros''; a reference to the
  variable fetches the contents of the external variable.  Similarly,
  setting the variable stores new contents---the new contents must be
  of the declared \var{type}.
  
  For example, it is often necessary to read the global C variable
  \code{errno} to determine why a particular function call failed.  It
  is possible to define errno and make it accessible from Lisp by the
  following:
\begin{lisp}
(def-alien-variable "errno" int)

;; Now it is possible to get the value of the C variable errno simply by
;; referencing that Lisp variable:
;;
(print errno)
\end{lisp}
\end{defmac}

\begin{defmac}{alien:}{extern-alien}{\var{name} \var{type}}
  
  This macro returns an Alien with the specified \var{type} which
  points to an externally defined value.  \var{name} is not evaluated,
  and may be specified either as a string or a symbol.  \var{type} is
  an unevaluated Alien type specifier.
\end{defmac}


\section{Alien Data Structure Example}

Now that we have Alien types, operations and variables, we can manipulate
foreign data structures.  This C declaration can be translated into the
following Alien type:

\begin{lisp}
struct foo \{
    int a;
    struct foo *b[100];
\};

 \myequiv

(def-alien-type nil
  (struct foo
    (a int)
    (b (array (* (struct foo)) 100))))
\end{lisp}

With this definition, the following C expression can be translated in this way:

\begin{example}
struct foo f;
f.b[7].a

 \myequiv

(with-alien ((f (struct foo)))
  (slot (deref (slot f 'b) 7) 'a)
  ;;
  ;; Do something with f...
  )
\end{example}


Or consider this example of an external C variable and some accesses:

\begin{example}
struct c_struct \{
        short x, y;
        char a, b;
        int z;
        c_struct *n;
\};

extern struct c_struct *my_struct;

my_struct->x++;
my_struct->a = 5;
my_struct = my_struct->n;
\end{example}

which can be made be manipulated in Lisp like this:

\begin{lisp}
(def-alien-type nil
  (struct c-struct
          (x short)
          (y short)
          (a char)
          (b char)
          (z int)
          (n (* c-struct))))

(def-alien-variable "my_struct" (* c-struct))

(incf (slot my-struct 'x))
(setf (slot my-struct 'a) 5)
(setq my-struct (slot my-struct 'n))
\end{lisp}


\section{Loading Unix Object Files}

Foreign object files are loaded into the running Lisp process by
\code{load-foreign}.  First, it runs the linker on the files and
libraries, creating an absolute Unix object file.  This object file is
then loaded into into the currently running Lisp.  The external
symbols defining routines and variables are made available for future
external references (e.g.  by \code{extern-alien}.)
\code{load-foreign} must be run before any of the defined symbols are
referenced.

Note that if a Lisp core image is saved (using \funref{save-lisp}), all
loaded foreign code is lost when the image is restarted.

\begin{defun}{ext:}{load-foreign}{%
    \args{\var{files} \keys{\kwd{libraries} \kwd{base-file} \kwd{env}}}}
  
  \var{files} is a \code{simple-string} or list of
  \code{simple-string}s specifying the names of the object files.
  \var{libraries} is a list of \code{simple-string}s specifying
  libraries in a format that \code{ld}, the Unix linker, expects.  The
  default value for \var{libraries} is \code{("-lc")} (i.e., the
  standard C library).  \var{base-file} is the file to use for the
  initial symbol table information.  The default is the Lisp start up
  code: \file{path:lisp}.  \var{env} should be a list of simple
  strings in the format of Unix environment variables (i.e.,
  \code{\var{A}=\var{B}}, where \var{A} is an environment variable and
  \var{B} is its value).  The default value for \var{env} is the
  environment information available at the time Lisp was invoked.
  Unless you are certain that you want to change this, you should just
  use the default.
\end{defun}


\section{Alien Function Calls}

The foreign function call interface allows a Lisp program to call functions
written in other languages.  The current implementation of the foreign
function call interface assumes a C calling convention and thus routines
written in any language that adheres to this convention may be called from
Lisp.

Lisp sets up various interrupt handling routines and other environment
information when it first starts up, and expects these to be in place at all
times.  The C functions called by Lisp should either not change the
environment, especially the interrupt entry points, or should make sure
that these entry points are restored when the C function returns to Lisp.
If a C function makes changes without restoring things to the way they were
when the C function was entered, there is no telling what will happen.


\subsection{The alien-funcall Primitive}

\begin{defun}{alien:}{alien-funcall}{%
    \args{\var{alien-function} \amprest{} \var{arguments}}}
  
  This function is the foreign function call primitive:
  \var{alien-function} is called with the supplied \var{arguments} and
  its value is returned.  The \var{alien-function} is an arbitrary
  run-time expression; to call a constant function, use
  \funref{extern-alien} or \code{def-alien-routine}.
  
  The type of \var{alien-function} must be \code{(alien (function
    ...))} or \code{(alien (* (function ...)))},
  \xlref{alien-function-types}.  The function type is used to
  determine how to call the function (as though it was declared with
  a prototype.)  The type need not be known at compile time, but only
  known-type calls are efficiently compiled.  Limitations:
  \begin{itemize}
  \item Structure type return values are not implemented.
  \item Passing of structures by value is not implemented.
  \end{itemize}
\end{defun}

Here is an example which allocates a \code{(struct foo)}, calls a foreign
function to initialize it, then returns a Lisp vector of all the
\code{(* (struct foo))} objects filled in by the foreign call:

\begin{lisp}
;; Allocate a foo on the stack.
(with-alien ((f (struct foo)))
  ;;
  ;; Call some C function to fill in foo fields.
  (alien-funcall (extern-alien "mangle_foo" (function void (* foo)))
                 (addr f))
  ;;
  ;; Find how many foos to use by getting the A field.
  (let* ((num (slot f 'a))
         (result (make-array num)))
    ;;
    ;; Get a pointer to the array so that we don't have to keep
    ;; extracting it:
    (with-alien ((a (* (array (* (struct foo)) 100)) (addr (slot f 'b))))
      ;;
      ;; Loop over the first N elements and stash them in the
      ;; result vector.
      (dotimes (i num)
        (setf (svref result i) (deref (deref a) i)))
      result)))
\end{lisp}


\subsection{The def-alien-routine Macro}

\begin{defmac}{alien:}{def-alien-routine}{\var{name} \var{result-type}
    \mstar{(\var{aname} \var{atype} \mopt{style})}}
  
  This macro is a convenience for automatically generating Lisp
  interfaces to simple foreign functions.  The primary feature is the
  parameter style specification, which translates the C
  pass-by-reference idiom into additional return values.
  
  \var{name} is usually a string external symbol, but may also be a
  symbol Lisp name or a list of the foreign name and the Lisp name.
  If only one name is specified, the other is automatically derived,
  (\pxlref{external-aliens}.)
  
  \var{result-type} is the Alien type of the return value.  Each
  remaining subform specifies an argument to the foreign function.
  \var{aname} is the symbol name of the argument to the constructed
  function (for documentation) and \var{atype} is the Alien type of
  corresponding foreign argument.  The semantics of the actual call
  are the same as for \funref{alien-funcall}.  \var{style} should be
  one of the following:
  \begin{Lentry}
  \item[\kwd{in}] specifies that the argument is passed by value.
    This is the default.  \kwd{in} arguments have no corresponding
    return value from the Lisp function.
  
  \item[\kwd{out}] specifies a pass-by-reference output value.  The
    type of the argument must be a pointer to a fixed sized object
    (such as an integer or pointer).  \kwd{out} and \kwd{in-out}
    cannot be used with pointers to arrays, records or functions.  An
    object of the correct size is allocated, and its address is passed
    to the foreign function.  When the function returns, the contents
    of this location are returned as one of the values of the Lisp
    function.
  
  \item[\kwd{copy}] is similar to \kwd{in}, but the argument is copied
    to a pre-allocated object and a pointer to this object is passed
    to the foreign routine.
  
  \item[\kwd{in-out}] is a combination of \kwd{copy} and \kwd{out}.
    The argument is copied to a pre-allocated object and a pointer to
    this object is passed to the foreign routine.  On return, the
    contents of this location is returned as an additional value.
  \end{Lentry}
  Any efficiency-critical foreign interface function should be inline
  expanded by preceding \code{def-alien-routine} with:

\begin{lisp}
(declaim (inline \var{lisp-name}))
\end{lisp}

  In addition to avoiding the Lisp call overhead, this allows
  pointers, word-integers and floats to be passed using non-descriptor
  representations, avoiding consing (\pxlref{non-descriptor}.)
\end{defmac}


\subsection{def-alien-routine Example}

Consider the C function \code{cfoo} with the following calling convention:

\begin{example}
/* a for update
 * i out
 */
void cfoo (char *str, char *a, int *i);
\end{example}

which can be described by the following call to \code{def-alien-routine}:

\begin{lisp}
(def-alien-routine "cfoo" void
  (str c-string)
  (a char :in-out)
  (i int :out))
\end{lisp}

The Lisp function \code{cfoo} will have two arguments (\var{str} and \var{a})
and two return values (\var{a} and \var{i}).


\subsection{Calling Lisp from C}

Calling Lisp functions from C is sometimes possible, but is rather hackish.
See \code{funcall0} ... \code{funcall3} in the \file{lisp/arch.h}.  The
arguments must be valid \cmucl{} object descriptors (e.g.  fixnums must be
left-shifted by 2.)  See \file{compiler/generic/objdef.lisp} or the derived
file \file{lisp/internals.h} for details of the object representation.
\file{lisp/internals.h} is mechanically generated, and is not part of the
source distribution.  It is distributed in the \file{docs/} directory of the
binary distribution.

Note that the garbage collector moves objects, and won't be able to fix up any
references in C variables, so either turn GC off or don't keep Lisp pointers
in C data unless they are to statically allocated objects.  You can use
\funref{purify} to place live data structures in static space so that they
won't move during GC.

\subsection{Accessing Lisp Arrays}

Due to the way \cmucl{} manages memory, the amount of memory that can
be dynamically allocated by \code{malloc} or \funref{make-alien} is
limited\footnote{\cmucl{} mmaps a large piece of memory for its own
use and this memory is typically about 256~MB above the start of the C
heap. Thus, only about 256~MB of memory can be dynamically allocated.
In earlier versions, this limit was closer to 8~MB.}.

To overcome this limitation, it is possible to access the content of
Lisp arrays which are limited only by the amount of physical memory
and swap space available.  However, this technique is only useful if
the foreign function takes pointers to memory instead of allocating
memory for itself.  In latter case, you will have to modify the
foreign functions.

This technique takes advantage of the fact that \cmucl{} has
specialized array types (\pxlref{specialized-array-types}) that match
a typical C array.  For example, a \code{(simple-array double-float
  (100))} is stored in memory in essentially the same way as the C
array \code{double x[100]} would be.  The following function allows us
to get the physical address of such a Lisp array:

\begin{example}
(defun array-data-address (array)
  "Return the physical address of where the actual data of an array is
stored.

ARRAY must be a specialized array type in \cmucl{}.  This means ARRAY
must be an array of one of the following types:

                  double-float
                  single-float
                  (unsigned-byte 32)
                  (unsigned-byte 16)
                  (unsigned-byte  8)
                  (signed-byte 32)
                  (signed-byte 16)
                  (signed-byte  8)
"
  (declare (type (or (simple-array (signed-byte 8))
                     (simple-array (signed-byte 16))
                     (simple-array (signed-byte 32))
                     (simple-array (unsigned-byte 8))
                     (simple-array (unsigned-byte 16))
                     (simple-array (unsigned-byte 32))
                     (simple-array single-float)
                     (simple-array double-float)
                     (simple-array (complex single-float))
                     (simple-array (complex double-float)))
                 array)
           (optimize (speed 3) (safety 0))
           (ext:optimize-interface (safety 3)))
  ;; with-array-data will get us to the actual data.  However, because
  ;; the array could have been displaced, we need to know where the
  ;; data starts.
  (lisp::with-array-data ((data array)
                          (start)
                          (end))
    (declare (ignore end))
    ;; DATA is a specialized simple-array.  Memory is laid out like this:
    ;;
    ;;   byte offset    Value
    ;;        0         type code (should be 70 for double-float vector)
    ;;        4         4 * number of elements in vector
    ;;        8         1st element of vector
    ;;      ...         ...
    ;;
    (let ((addr (+ 8 (logandc1 7 (kernel:get-lisp-obj-address data))))
          (type-size
           (let ((type (array-element-type data)))
             (cond ((or (equal type '(signed-byte 8))
                        (equal type '(unsigned-byte 8)))
                    1)
                   ((or (equal type '(signed-byte 16))
                        (equal type '(unsigned-byte 16)))
                    2)
                   ((or (equal type '(signed-byte 32))
                        (equal type '(unsigned-byte 32)))
                    4)
                   ((equal type 'single-float)
                    4)
                   ((equal type 'double-float)
                    8)
                   (t
                    (error "Unknown specialized array element type"))))))
      (declare (type (unsigned-byte 32) addr)
               (optimize (speed 3) (safety 0) (ext:inhibit-warnings 3)))
      (system:int-sap (the (unsigned-byte 32)
                        (+ addr (* type-size start)))))))
\end{example}

We note, however, that the system function
\findexed{system:vector-sap} will do the same thing as above does.

Assume we have the C function below that we wish to use:

\begin{example}
  double dotprod(double* x, double* y, int n)
  \{
    int k;
    double sum = 0;

    for (k = 0; k < n; ++k) \{
      sum += x[k] * y[k];
    \}
    return sum;
  \}
\end{example}

The following example generates two large arrays in Lisp, and calls the C
function to do the desired computation.  This would not have been
possible using \code{malloc} or \code{make-alien} since we need about
16 MB of memory to hold the two arrays.

\begin{example}
  (alien:def-alien-routine "dotprod" c-call:double
    (x (* double-float) :in)
    (y (* double-float) :in)
    (n c-call:int :in))
    
  (defun test-dotprod ()
    (let ((x (make-array 10000 :element-type 'double-float :initial-element 2d0))
          (y (make-array 10000 :element-type 'double-float :initial-element 10d0)))
        (sys:without-gcing
          (let ((x-addr (sys:vector-sap x))
                (y-addr (sys:vector-sap y)))
            (dotprod x-addr y-addr 10000)))))
\end{example}

In this example, we have used \code{sys:vector-sap} instead of
\code{array-data-address}, but we could have used \code{(sys:int-sap
  (array-data-address x))} as well.

Also, we have wrapped the inner \code{let} expression in a
\code{sys:without-gcing} that disables garbage collection for the
duration of the body.  This will prevent garbage collection from
moving \code{x} and \code{y} arrays after we have obtained the (now
erroneous) addresses but before the call to \code{dotprod} is made.


\section{Step-by-Step Alien Example}

This section presents a complete example of an interface to a somewhat
complicated C function.  This example should give a fairly good idea
of how to get the effect you want for almost any kind of C function.
Suppose you have the following C function which you want to be able to
call from Lisp in the file \file{test.c}:

\begin{verbatim}                
struct c_struct
{
  int x;
  char *s;
};
 
struct c_struct *c_function (i, s, r, a)
    int i;
    char *s;
    struct c_struct *r;
    int a[10];
{
  int j;
  struct c_struct *r2;
 
  printf("i = %d\n", i);
  printf("s = %s\n", s);
  printf("r->x = %d\n", r->x);
  printf("r->s = %s\n", r->s);
  for (j = 0; j < 10; j++) printf("a[%d] = %d.\n", j, a[j]);
  r2 = (struct c_struct *) malloc (sizeof(struct c_struct));
  r2->x = i + 5;
  r2->s = "A C string";
  return(r2);
};
\end{verbatim}

It is possible to call this function from Lisp using the file \file{test.lisp}
whose contents is:

\begin{lisp}
;;; -*- Package: test-c-call -*-
(in-package "TEST-C-CALL")
(use-package "ALIEN")
(use-package "C-CALL")

;;; Define the record c-struct in Lisp.
(def-alien-type nil
    (struct c-struct
            (x int)
            (s c-string)))

;;; Define the Lisp function interface to the C routine.  It returns a
;;; pointer to a record of type c-struct.  It accepts four parameters:
;;; i, an int; s, a pointer to a string; r, a pointer to a c-struct
;;; record; and a, a pointer to the array of 10 ints.
;;;
;;; The INLINE declaration eliminates some efficiency notes about heap
;;; allocation of Alien values.
(declaim (inline c-function))
(def-alien-routine c-function
    (* (struct c-struct))
  (i int)
  (s c-string)
  (r (* (struct c-struct)))
  (a (array int 10)))

;;; A function which sets up the parameters to the C function and
;;; actually calls it.
(defun call-cfun ()
  (with-alien ((ar (array int 10))
               (c-struct (struct c-struct)))
    (dotimes (i 10)                     ; Fill array.
      (setf (deref ar i) i))
    (setf (slot c-struct 'x) 20)
    (setf (slot c-struct 's) "A Lisp String")

    (with-alien ((res (* (struct c-struct))
                 (c-function 5 "Another Lisp String" (addr c-struct) ar)))
      (format t "Returned from C function.~%")
      (multiple-value-prog1
          (values (slot res 'x)
                  (slot res 's))
        ;;              
        ;; Deallocate result {\em after} we are done using it.
        (free-alien res)))))
\end{lisp}

To execute the above example, it is necessary to compile the C routine as
follows:

\begin{example}
cc -c test.c
\end{example}

In order to enable incremental loading with some linkers, you may need to say:

\begin{example}
cc -G 0 -c test.c
\end{example}

Once the C code has been compiled, you can start up Lisp and load it in:

\begin{example}
% lisp
;;; Lisp should start up with its normal prompt.

;;; Compile the Lisp file.  This step can be done separately.  You don't have
;;; to recompile every time.
* (compile-file "test.lisp")

;;; Load the foreign object file to define the necessary symbols.  This must
;;; be done before loading any code that refers to these symbols.  next block
;;; of comments are actually the output of LOAD-FOREIGN.  Different linkers
;;; will give different warnings, but some warning about redefining the code
;;; size is typical.
* (load-foreign "test.o")

;;; Running library:load-foreign.csh...
;;; Loading object file...
;;; Parsing symbol table...
Warning:  "_gp" moved from #x00C082C0 to #x00C08460.
Warning:  "end" moved from #x00C00340 to #x00C004E0.

;;; o.k. now load the compiled Lisp object file.
* (load "test")

;;; Now we can call the routine that sets up the parameters and calls the C
;;; function.
* (test-c-call::call-cfun)

;;; The C routine prints the following information to standard output.
i = 5
s = Another Lisp string
r->x = 20
r->s = A Lisp string
a[0] = 0.
a[1] = 1.
a[2] = 2.
a[3] = 3.
a[4] = 4.
a[5] = 5.
a[6] = 6.
a[7] = 7.
a[8] = 8.
a[9] = 9.
;;; Lisp prints out the following information.
Returned from C function.
;;; Return values from the call to test-c-call::call-cfun.
10
"A C string"
*
\end{example}

If any of the foreign functions do output, they should not be called
from within \hemlock{}. Depending on the situation, various strange
behavior occurs. Under X, the output goes to the window in which Lisp
was started; on a terminal, the output will overwrite the \hemlock{}
screen image; in a \hemlock{} slave, standard output is
\file{/dev/null} by default, so any output is discarded.

\chapter{Interprocess Communication under LISP}
\label{remote}

\credits{by William Lott and Bill Chiles}


\cmucl{} offers a facility for interprocess communication (IPC)
on top of using Unix system calls and the complications of that level
of IPC.  There is a simple remote-procedure-call (RPC) package build
on top of TCP/IP sockets.


\section{The REMOTE Package}

The \code{remote} package provides simple RPC facility including
interfaces for creating servers, connecting to already existing
servers, and calling functions in other Lisp processes.  The routines
for establishing a connection between two processes,
\code{create-request-server} and \code{connect-to-remote-server},
return \var{wire} structures.  A wire maintains the current state of
a connection, and all the RPC forms require a wire to indicate where
to send requests.


\subsection{Connecting Servers and Clients}

Before a client can connect to a server, it must know the network address on
which the server accepts connections.  Network addresses consist of a host
address or name, and a port number.  Host addresses are either a string of the
form \code{VANCOUVER.SLISP.CS.CMU.EDU} or a 32 bit unsigned integer.  Port
numbers are 16 bit unsigned integers.  Note: \var{port} in this context has
nothing to do with Mach ports and message passing.

When a process wants to receive connection requests (that is, become a
server), it first picks an integer to use as the port.  Only one server
(Lisp or otherwise) can use a given port number on a given machine at
any particular time.  This can be an iterative process to find a free
port: picking an integer and calling \code{create-request-server}.  This
function signals an error if the chosen port is unusable.  You will
probably want to write a loop using \code{handler-case}, catching
conditions of type error, since this function does not signal more
specific conditions.

\begin{defun}{wire:}{create-request-server}{%
    \args{\var{port} \ampoptional{} \var{on-connect}}}

  \code{create-request-server} sets up the current Lisp to accept
  connections on the given port.  If port is unavailable for any
  reason, this signals an error.  When a client connects to this port,
  the acceptance mechanism makes a wire structure and invokes the
  \var{on-connect} function.  Invoking this function has a couple of
  purposes, and \var{on-connect} may be \nil{} in which case the
  system foregoes invoking any function at connect time.
  
  The \var{on-connect} function is both a hook that allows you access
  to the wire created by the acceptance mechanism, and it confirms the
  connection.  This function takes two arguments, the wire and the
  host address of the connecting process.  See the section on host
  addresses below.  When \var{on-connect} is \nil, the request server
  allows all connections.  When it is non-\nil, the function returns
  two values, whether to accept the connection and a function the
  system should call when the connection terminates.  Either value may
  be \nil, but when the first value is \nil, the acceptance mechanism
  destroys the wire.
  
  \code{create-request-server} returns an object that
  \code{destroy-request-server} uses to terminate a connection.
\end{defun}

\begin{defun}{wire:}{destroy-request-server}{\args{\var{server}}}
  
  \code{destroy-request-server} takes the result of
  \code{create-request-server} and terminates that server.  Any
  existing connections remain intact, but all additional connection
  attempts will fail.
\end{defun}

\begin{defun}{wire:}{connect-to-remote-server}{%
    \args{\var{host} \var{port} \ampoptional{} \var{on-death}}}
  
  \code{connect-to-remote-server} attempts to connect to a remote
  server at the given \var{port} on \var{host} and returns a wire
  structure if it is successful.  If \var{on-death} is non-\nil, it is
  a function the system invokes when this connection terminates.
\end{defun}


\subsection{Remote Evaluations}

After the server and client have connected, they each have a wire
allowing function evaluation in the other process.  This RPC mechanism
has three flavors: for side-effect only, for a single value, and for
multiple values.

Only a limited number of data types can be sent across wires as
arguments for remote function calls and as return values: integers
inclusively less than 32 bits in length, symbols, lists, and
\var{remote-objects} (\pxlref{remote-objs}).  The system sends symbols
as two strings, the package name and the symbol name, and if the
package doesn't exist remotely, the remote process signals an error.
The system ignores other slots of symbols.  Lists may be any tree of
the above valid data types.  To send other data types you must
represent them in terms of these supported types.  For example, you
could use \code{prin1-to-string} locally, send the string, and use
\code{read-from-string} remotely.

\begin{defmac}{wire:}{remote}{%
    \args{\var{wire} \mstar{call-specs}}}
  
  The \code{remote} macro arranges for the process at the other end of
  \var{wire} to invoke each of the functions in the \var{call-specs}.
  To make sure the system sends the remote evaluation requests over
  the wire, you must call \code{wire-force-output}.
  
  Each of \var{call-specs} looks like a function call textually, but
  it has some odd constraints and semantics.  The function position of
  the form must be the symbolic name of a function.  \code{remote}
  evaluates each of the argument subforms for each of the
  \var{call-specs} locally in the current context, sending these
  values as the arguments for the functions.
  
  Consider the following example:

\begin{verbatim}
(defun write-remote-string (str)
  (declare (simple-string str))
  (wire:remote wire
    (write-string str)))
\end{verbatim}

  The value of \code{str} in the local process is passed over the wire
  with a request to invoke \code{write-string} on the value.  The
  system does not expect to remotely evaluate \code{str} for a value
  in the remote process.
\end{defmac}

\begin{defun}{wire:}{wire-force-output}{\args{\var{wire}}}
  
  \code{wire-force-output} flushes all internal buffers associated
  with \var{wire}, sending the remote requests.  This is necessary
  after a call to \code{remote}.
\end{defun}

\begin{defmac}{wire:}{remote-value}{\args{\var{wire} \var{call-spec}}}
  
  The \code{remote-value} macro is similar to the \code{remote} macro.
  \code{remote-value} only takes one \var{call-spec}, and it returns
  the value returned by the function call in the remote process.  The
  value must be a valid type the system can send over a wire, and
  there is no need to call \code{wire-force-output} in conjunction
  with this interface.
  
  If client unwinds past the call to \code{remote-value}, the server
  continues running, but the system ignores the value the server sends
  back.
  
  If the server unwinds past the remotely requested call, instead of
  returning normally, \code{remote-value} returns two values, \nil{}
  and \true.  Otherwise this returns the result of the remote
  evaluation and \nil.
\end{defmac}

\begin{defmac}{wire:}{remote-value-bind}{%
    \args{\var{wire} (\mstar{variable}) remote-form
      \mstar{local-forms}}}
  
  \code{remote-value-bind} is similar to \code{multiple-value-bind}
  except the values bound come from \var{remote-form}'s evaluation in
  the remote process.  The \var{local-forms} execute in an implicit
  \code{progn}.
  
  If the client unwinds past the call to \code{remote-value-bind}, the
  server continues running, but the system ignores the values the
  server sends back.
  
  If the server unwinds past the remotely requested call, instead of
  returning normally, the \var{local-forms} never execute, and
  \code{remote-value-bind} returns \nil.
\end{defmac}


\subsection{Remote Objects}
\label{remote-objs}

The wire mechanism only directly supports a limited number of data
types for transmission as arguments for remote function calls and as
return values: integers inclusively less than 32 bits in length,
symbols, lists.  Sometimes it is useful to allow remote processes to
refer to local data structures without allowing the remote process
to operate on the data.  We have \var{remote-objects} to support
this without the need to represent the data structure in terms of
the above data types, to send the representation to the remote
process, to decode the representation, to later encode it again, and
to send it back along the wire.

You can convert any Lisp object into a remote-object.  When you send
a remote-object along a wire, the system simply sends a unique token
for it.  In the remote process, the system looks up the token and
returns a remote-object for the token.  When the remote process
needs to refer to the original Lisp object as an argument to a
remote call back or as a return value, it uses the remote-object it
has which the system converts to the unique token, sending that
along the wire to the originating process.  Upon receipt in the
first process, the system converts the token back to the same
(\code{eq}) remote-object.

\begin{defun}{wire:}{make-remote-object}{\args{\var{object}}}
  
  \code{make-remote-object} returns a remote-object that has
  \var{object} as its value.  The remote-object can be passed across
  wires just like the directly supported wire data types.
\end{defun}

\begin{defun}{wire:}{remote-object-p}{\args{\var{object}}}
  
  The function \code{remote-object-p} returns \true{} if \var{object}
  is a remote object and \nil{} otherwise.
\end{defun}

\begin{defun}{wire:}{remote-object-local-p}{\args{\var{remote}}}
  
  The function \code{remote-object-local-p} returns \true{} if
  \var{remote} refers to an object in the local process.  This is can
  only occur if the local process created \var{remote} with
  \code{make-remote-object}.
\end{defun}

\begin{defun}{wire:}{remote-object-eq}{\args{\var{obj1} \var{obj2}}}
  
  The function \code{remote-object-eq} returns \true{} if \var{obj1} and
  \var{obj2} refer to the same (\code{eq}) lisp object, regardless of
  which process created the remote-objects.
\end{defun}

\begin{defun}{wire:}{remote-object-value}{\args{\var{remote}}}
  
  This function returns the original object used to create the given
  remote object.  It is an error if some other process originally
  created the remote-object.
\end{defun}

\begin{defun}{wire:}{forget-remote-translation}{\args{\var{object}}}
  
  This function removes the information and storage necessary to
  translate remote-objects back into \var{object}, so the next
  \code{gc} can reclaim the memory.  You should use this when you no
  longer expect to receive references to \var{object}.  If some remote
  process does send a reference to \var{object},
  \code{remote-object-value} signals an error.
\end{defun}


% This stuff has been moved to internet.tex.  *** Remove me someday ***
% \subsection{Host Addresses}

% The operating system maintains a database of all the valid host
% addresses.  You can use this database to convert between host names
% and addresses and vice-versa.

% \begin{defun}{ext:}{lookup-host-entry}{\args{\var{host}}}
  
%   \code{lookup-host-entry} searches the database for the given
%   \var{host} and returns a host-entry structure for it.  If it fails
%   to find \var{host} in the database, it returns \nil.  \var{Host} is
%   either the address (as an integer) or the name (as a string) of the
%   desired host.
% \end{defun}

% \begin{defun}{ext:}{host-entry-name}{\args{\var{host-entry}}}
%   \defunx[ext:]{host-entry-aliases}{\args{\var{host-entry}}}
%   \defunx[ext:]{host-entry-addr-list}{\args{\var{host-entry}}}
%   \defunx[ext:]{host-entry-addr}{\args{\var{host-entry}}}

%   \code{host-entry-name}, \code{host-entry-aliases}, and
%   \code{host-entry-addr-list} each return the indicated slot from the
%   host-entry structure.  \code{host-entry-addr} returns the primary
%   (first) address from the list returned by
%   \code{host-entry-addr-list}.
% \end{defun}


\section{The WIRE Package}

The \code{wire} package provides for sending data along wires.  The
\code{remote} package sits on top of this package.  All data sent
with a given output routine must be read in the remote process with
the complementary fetching routine.  For example, if you send so a
string with \code{wire-output-string}, the remote process must know
to use \code{wire-get-string}.  To avoid rigid data transfers and
complicated code, the interface supports sending
\var{tagged} data.  With tagged data, the system sends a tag
announcing the type of the next data, and the remote system takes
care of fetching the appropriate type.

When using interfaces at the wire level instead of the RPC level,
the remote process must read everything sent by these routines.  If
the remote process leaves any input on the wire, it will later
mistake the data for an RPC request causing unknown lossage.


\subsection{Untagged Data}

When using these routines both ends of the wire know exactly what types are
coming and going and in what order. This data is restricted to the following
types:

\begin{itemize}
\item
8 bit unsigned bytes.

\item
32 bit unsigned bytes.

\item
32 bit integers.

\item
simple-strings less than 65535 in length.
\end{itemize}

\begin{defun}{wire:}{wire-output-byte}{\args{\var{wire} \var{byte}}}
  \defunx[wire:]{wire-get-byte}{\args{\var{wire}}}
  \defunx[wire:]{wire-output-number}{\args{\var{wire} \var{number}}}
  \defunx[wire:]{wire-get-number}{\args{\var{wire} \ampoptional{}
      \var{signed}}}
  \defunx[wire:]{wire-output-string}{\args{\var{wire} \var{string}}}
  \defunx[wire:]{wire-get-string}{\args{\var{wire}}}
  
  These functions either output or input an object of the specified
  data type.  When you use any of these output routines to send data
  across the wire, you must use the corresponding input routine
  interpret the data.
\end{defun}


\subsection{Tagged Data}

When using these routines, the system automatically transmits and interprets
the tags for you, so both ends can figure out what kind of data transfers
occur.  Sending tagged data allows a greater variety of data types: integers
inclusively less than 32 bits in length, symbols, lists, and \var{remote-objects}
(\pxlref{remote-objs}).  The system sends symbols as two strings, the
package name and the symbol name, and if the package doesn't exist remotely,
the remote process signals an error.  The system ignores other slots of
symbols.  Lists may be any tree of the above valid data types.  To send other
data types you must represent them in terms of these supported types.  For
example, you could use \code{prin1-to-string} locally, send the string, and use
\code{read-from-string} remotely.

\begin{defun}{wire:}{wire-output-object}{%
    \args{\var{wire} \var{object} \ampoptional{} \var{cache-it}}}
  \defunx[wire:]{wire-get-object}{\args{\var{wire}}}
  
  The function \code{wire-output-object} sends \var{object} over
  \var{wire} preceded by a tag indicating its type.
  
  If \var{cache-it} is non-\nil, this function only sends \var{object}
  the first time it gets \var{object}.  Each end of the wire
  associates a token with \var{object}, similar to remote-objects,
  allowing you to send the object more efficiently on successive
  transmissions.  \var{cache-it} defaults to \true{} for symbols and
  \nil{} for other types.  Since the RPC level requires function
  names, a high-level protocol based on a set of function calls saves
  time in sending the functions' names repeatedly.
  
  The function \code{wire-get-object} reads the results of
  \code{wire-output-object} and returns that object.
\end{defun}


\subsection{Making Your Own Wires}

You can create wires manually in addition to the \code{remote}
package's interface creating them for you. To create a wire, you need
a Unix {\em file descriptor}. If you are unfamiliar with Unix file
descriptors, see section 2 of the Unix manual pages.

\begin{defun}{wire:}{make-wire}{\args{\var{descriptor}}}

  The function \code{make-wire} creates a new wire when supplied with
  the file descriptor to use for the underlying I/O operations.
\end{defun}

\begin{defun}{wire:}{wire-p}{\args{\var{object}}}
  
  This function returns \true{} if \var{object} is indeed a wire,
  \nil{} otherwise.
\end{defun}

\begin{defun}{wire:}{wire-fd}{\args{\var{wire}}}
  
  This function returns the file descriptor used by the \var{wire}.
\end{defun}


\section{Out-Of-Band Data}

The TCP/IP protocol allows users to send data asynchronously, otherwise
known as \var{out-of-band} data.  When using this feature, the operating
system interrupts the receiving process if this process has chosen to be
notified about out-of-band data.  The receiver can grab this input
without affecting any information currently queued on the socket.
Therefore, you can use this without interfering with any current
activity due to other wire and remote interfaces.

Unfortunately, most implementations of TCP/IP are broken, so use of
out-of-band data is limited for safety reasons.  You can only reliably
send one character at a time.

The Wire package is built on top of \cmucl{}s networking support. In
view of this, it is possible to use the routines described in section
\ref{internet-oob} for handling and sending out-of-band data. These
all take a Unix file descriptor instead of a wire, but you can fetch a
wire's file descriptor with \code{wire-fd}.

\chapter{Networking Support}
\label{internet}

\credits{by Mario S. Mommer}

This chapter documents the IPv4 networking and local sockets support
offered by \cmucl{}. It covers most of the basic sockets interface
functionality in a convenient and transparent way.

For reasons of space it would be impossible to include a thorough
introduction to network programming, so we assume some basic knowledge
of the matter.

\section{Byte Order Converters}

These are the functions that convert integers from host byte order to
network byte order (big-endian).

\begin{defun}{extensions:}{htonl}{%
    \args{\var{integer}}}
  
  Converts a $32$ bit integer from host byte order to network byte
  order.

\end{defun}

\begin{defun}{extensions:}{htons}{%
    \args{\var{integer}}}
  
  Converts a $16$ bit integer from host byte order to network byte
  order.

\end{defun}

\begin{defun}{extensions:}{ntohs}{%
    \args{\var{integer}}}
  
  Converts a $32$ bit integer from network byte order to host
  byte order.

\end{defun}

\begin{defun}{extensions:}{ntohl}{%
    \args{\var{integer}}}
  
  Converts a $32$ bit integer from network byte order to host byte
  order.

\end{defun}

\section{Domain Name Services (DNS)}

The networking support of \cmucl{} includes the possibility of doing
DNS lookups. The function 

\begin{defun}{extensions:}{lookup-host-entry}{%
    \args{\var{host}}}
 
  returns a structure of type \var{host-entry} (explained below) for
  the given \var{host}.  If \var{host} is an integer, it will be
  assumed to be the IP address in host (byte-)order. If it is a string,
  it can contain either the host name or the IP address in dotted
  format.
  
  This function works by completing the structure \var{host-entry}.
  That is, if the user provides the IP address, then the structure will
  contain that information and also the domain names. If the user
  provides the domain name, the structure will be complemented with
  the IP addresses along with the any aliases the host might have.

\end{defun}

\newpage

\begin{deftp}{structure}{host-entry}\args{\var{name} \var{aliases}
    \var{addr-type} \var{addr-list}}
  
  This structure holds all information available at request time on a
  given host. The entries are self-explanatory. Aliases is a list of
  strings containing alternative names of the host, and addr-list a
  list of addresses stored in host byte order. The field
  \var{addr-type} contains the number of the address family, as
  specified in {\tt socket.h}, to which the addresses belong. Since
  only addresses of the IPv4 family are currently supported, this slot
  always has the value $2$.

\end{deftp}

\begin{defun}{extensions:}{ip-string}{%
    \args{\var{addr}}}
  
  This function takes an IP address in host order and returns a string
  containing it in dotted format.

\end{defun}

\section{Binding to Interfaces}

In this section, functions for creating sockets bound to an interface
are documented.

\begin{defun}{extensions:}{create-inet-listener}{%
    \args{\var{port} \ampoptional{} \var{kind}  %
      \keys{\kwd{reuse-address} \kwd{backlog} \kwd{host}}}}
                     
  Creates a socket and binds it to a port, prepared to receive
  connections of kind \var{kind} (which defaults to \kwd{stream}),
  queuing up to \var{backlog} of them. If \kwd{reuse-address} \var{T}
  is used, the option SO\_REUSEADDR is used in the call to \var{bind}.
  If no value is given for \kwd{host}, it will try to bind to the
  default IP address of the machine where the Lisp process is running.

\end{defun}

\begin{defun}{extensions:}{create-unix-listener}{%
    \args{\var{path} \ampoptional{} \var{kind} \keys{
      \kwd{backlog}}}}
  
  Creates a socket and binds it to the file name given by \var{path},
  prepared to receive connections of kind \var{kind} (which defaults
  to \kwd{stream}), queuing up to \var{backlog} of them.
%  If
%  \kwd{reuse-address} \var{T} is used, then the file given by
%  \var{path} is unlinked first.

\end{defun}

\section{Accepting Connections}

Once a socket is bound to its interface, we have to explicitly accept
connections. This task is performed by the functions we document here.

\begin{defun}{extensions:}{accept-tcp-connection}{%
    \args{\var{unconnected}}}
  
  Waits until a connection arrives on the (internet family) socket
  \var{unconnected}. Returns the file descriptor of the connection.
  These can be conveniently encapsulated using file descriptor
  streams; see \ref{sec:fds}.

\end{defun}

\begin{defun}{extensions:}{accept-unix-connection}{%
    \args{\var{unconnected}}}

  Waits until a connection arrives on the (unix family) socket
  \var{unconnected}. Returns the file descriptor of the connection.
  These can be conveniently encapsulated using file descriptor
  streams; see \ref{sec:fds}.

\end{defun}

\begin{defun}{extensions:}{accept-network-stream}{%
    \args{\var{socket} \keys{\kwd{buffering} \kwd{timeout} \kwd{wait-max}}}}

  Accept a connect from the specified \var{socket} and returns a stream 
  connected to connection.  
\end{defun}

\section{Connecting}

The task performed by the functions we present next is connecting to
remote hosts.

\begin{defun}{extensions:}{connect-to-inet-socket}{%
    \args{\var{host} \var{port} \ampoptional{} \var{kind}
          \keys{\kwd{local-host} \kwd{local-port}}}}
  
  Tries to open a connection to the remote host \var{host} (which may
  be an IP address in host order, or a string with either a host name
  or an IP address in dotted format) on port \var{port}. Returns the
  file descriptor of the connection.  The optional parameter
  \var{kind} can be either \kwd{stream} (the default) or \kwd{datagram}.

  If \var{local-host} and \var{local-port} are specified, the socket
  that is created is also bound to the specified \var{local-host} and
  \var{port}.

\end{defun}

\begin{defun}{extensions:}{connect-to-unix-socket}{%
    \args{\var{path} \ampoptional{} \var{kind}}}
  
  Opens a connection to the unix ``address'' given by \var{path}.
  Returns the file descriptor of the connection.  The type of
  connection is given by \var{kind}, which can be either \kwd{stream}
  (the default) or \kwd{datagram}.

\end{defun}

\begin{defun}{extensions:}{open-network-stream}{%
    \args{\var{host} \var{port} \keys{\kwd{buffering} \kwd{timeout}}}}
  
   Return a stream connected to the specified \var{port} on the given \var{host}.
\end{defun}


\section{Out-of-Band Data}
\label{internet-oob}

Out-of-band data is data transmitted with a higher priority than
ordinary data. This is usually used by either side of the connection
to signal exceptional conditions. Due to the fact that most TCP/IP
implementations are broken in this respect, only single characters can
reliably be sent this way.

\begin{defun}{extensions:}{add-oob-handler}{%
    \args{\var{fd} \var{char} \var{handler}}}
  
  Sets the function passed in \var{handler} as a handler for the
  character \var{char} on the connection whose descriptor is \var{fd}.
  In case this character arrives, the function in \var{handler} is
  called without any argument.

\end{defun}

\begin{defun}{extensions:}{remove-oob-handler}{%
    \args{\var{fd} \var{char}}}
  
  Removes the handler for the character \var{char} from the connection
  with the file descriptor \var{fd}

\end{defun}

\begin{defun}{extensions:}{remove-all-oob-handlers}{%
    \args{\var{fd}}}

  After calling this function, the connection whose descriptor is
  \var{fd} will ignore any out-of-band character it receives.

\end{defun}

\begin{defun}{extensions:}{send-character-out-of-band}{%
    \args{\var{fd} \var{char}}}

  Sends the character \var{char} through the connection \var{fd} out
  of band.

\end{defun}

\section{Unbound Sockets, Socket Options, and Closing Sockets}

These functions create unbound sockets. This is usually not necessary,
since connectors and listeners create their own.

\begin{defun}{extensions:}{create-unix-socket}{%
    \args{\ampoptional{} \var{type}}}
  
  Creates a unix socket for the unix address family, of type
  \var{:stream} and (on success) returns its file descriptor.

\end{defun}

\begin{defun}{extensions:}{create-inet-socket}{%
    \args{\ampoptional{} \var{kind}}}
  
  Creates a unix socket for the internet address family, of type
  \var{:stream} and (on success) returns its file descriptor.

\end{defun}
\bigskip

Once a socket is created, it is sometimes useful to bind the socket to a 
local address using \code{bind-inet-socket}:

\begin{defun}{extensions:}{bind-inet-socket}{%
  \args{\var{socket} \var{host} \var{port}}}

  Bind the \var{socket} to a local interface address specified
  by \var{host} and \var{port}. 

\end{defun}
\bigskip

Further, it is desirable to be able to change socket options. This is
performed by the following two functions, which are essentially
wrappers for system calls to {\tt getsockopt} and {\tt setsockopt}.

\begin{defun}{extensions:}{get-socket-option}{%
    \args{\var{socket} \var{level} \var{optname}}}
  
  Gets the value of option \var{optname} from the socket \var{socket}.

\end{defun}

\begin{defun}{extensions:}{set-socket-option}{%
    \args{\var{socket} \var{level} \var{optname} \var{optval}}}
  
  Sets the value of option \var{optname} from the socket \var{socket}
  to the value \var{optval}.

\end{defun}
\bigskip

For information on possible options and values we refer to the
manpages of {\tt getsockopt} and {\tt setsockopt}, and to {\tt
 socket.h}

Finally, the function

\begin{defun}{extensions:}{close-socket}{%
    \args{\var{socket}}}

  Closes the socket given by the file descriptor \var{socket}.

\end{defun}

\section{Unix Datagrams}

Datagram network is supported with the following functions.

\begin{defun}{extensions:}{inet-recvfrom}{%
	\args{\var{fd} \var{buffer} \var{size}}
	\keys{\kwd{flags}}}
   A simple interface to the Unix \code{recvfrom} function. Returns
   three values: bytecount, source address as integer, and source
   port. Bytecount can of course be negative, to indicate faults.
\end{defun}

\begin{defun}{extensions:}{inet-sendto}{%
	\args{\var{fd} \var{buffer} \var{size} \var{addr} \var{port}}
	\keys{\kwd{flags}}}
   A simple interface to the Unix \code{sendto} function.
\end{defun}

\begin{defun}{extensions:}{inet-shutdown}{%
	\args{\var{fd} \var{level}}}

   A simple interface to the Unix \code{shutdown} function.  For
   \code{level}, you may use the following symbols to close one or
   both ends of a socket: \code{shut-rd}, \code{shut-wr},
   \code{shut-rdwr}.

\end{defun}

\section{Errors}

Errors that occur during socket operations signal a
\code{socket-error} condition, a subtype of the \code{error}
condition.  Currently this condition includes just the Unix
\code{errno} associated with the error.

\chapter{Debugger Programmer's Interface}
\label{debug-internals}

The debugger programmers interface is exported from from the
\code{DEBUG-INTERNALS} or \code{DI} package.  This is a CMU
extension that allows debugging tools to be written without detailed
knowledge of the compiler or run-time system.

Some of the interface routines take a code-location as an argument.  As
described in the section on code-locations, some code-locations are
unknown.  When a function calls for a \var{basic-code-location}, it
takes either type, but when it specifically names the argument
\var{code-location}, the routine will signal an error if you give it an
unknown code-location.


\section{DI Exceptional Conditions}

Some of these operations fail depending on the availability debugging
information.  In the most severe case, when someone saved a Lisp image
stripping all debugging data structures, no operations are valid.  In
this case, even backtracing and finding frames is impossible.  Some
interfaces can simply return values indicating the lack of information,
or their return values are naturally meaningful in light missing data.
Other routines, as documented below, will signal
\code{serious-condition}s when they discover awkward situations.  This
interface does not provide for programs to detect these situations other
than by calling a routine that detects them and signals a condition.
These are serious-conditions because the program using the interface
must handle them before it can correctly continue execution.  These
debugging conditions are not errors since it is no fault of the
programmers that the conditions occur.

\subsection{Debug-conditions}

The debug internals interface signals conditions when it can't adhere
to its contract.  These are serious-conditions because the program
using the interface must handle them before it can correctly continue
execution.  These debugging conditions are not errors since it is no
fault of the programmers that the conditions occur.  The interface
does not provide for programs to detect these situations other than
calling a routine that detects them and signals a condition.


\begin{deftp}{Condition}{debug-condition}{}

This condition inherits from serious-condition, and all debug-conditions
inherit from this.  These must be handled, but they are not programmer errors.
\end{deftp}


\begin{deftp}{Condition}{no-debug-info}{}

This condition indicates there is absolutely no debugging information
available.
\end{deftp}


\begin{deftp}{Condition}{no-debug-function-returns}{}

This condition indicates the system cannot return values from a frame since
its debug-function lacks debug information details about returning values.
\end{deftp}


\begin{deftp}{Condition}{no-debug-blocks}{}
This condition indicates that a function was not compiled with debug-block
information, but this information is necessary necessary for some requested
operation.
\end{deftp}

\begin{deftp}{Condition}{no-debug-variables}{}
Similar to \code{no-debug-blocks}, except that variable information was
requested.
\end{deftp}

\begin{deftp}{Condition}{lambda-list-unavailable}{}
Similar to \code{no-debug-blocks}, except that lambda list information was
requested.
\end{deftp}

\begin{deftp}{Condition}{invalid-value}{}

This condition indicates a debug-variable has \kwd{invalid} or \kwd{unknown}
value in a particular frame.
\end{deftp}


\begin{deftp}{Condition}{ambiguous-variable-name}{}

This condition indicates a user supplied debug-variable name identifies more
than one valid variable in a particular frame.
\end{deftp}


\subsection{Debug-errors}

These are programmer errors resulting from misuse of the debugging tools'
programmers' interface.  You could have avoided an occurrence of one of these
by using some routine to check the use of the routine generating the error.


\begin{deftp}{Condition}{debug-error}{}
This condition inherits from error, and all user programming errors inherit
from this condition.
\end{deftp}


\begin{deftp}{Condition}{unhandled-condition}{}
This error results from a signalled \code{debug-condition} occurring
without anyone handling it.
\end{deftp}


\begin{deftp}{Condition}{unknown-code-location}{}
This error indicates the invalid use of an unknown-code-location.
\end{deftp}


\begin{deftp}{Condition}{unknown-debug-variable}{}

This error indicates an attempt to use a debug-variable in conjunction with an
inappropriate debug-function; for example, checking the variable's validity
using a code-location in the wrong debug-function will signal this error.
\end{deftp}


\begin{deftp}{Condition}{frame-function-mismatch}{}

This error indicates you called a function returned by
\code{preprocess-for-eval}
on a frame other than the one for which the function had been prepared.
\end{deftp}


\section{Debug-variables}

Debug-variables represent the constant information about where the system
stores argument and local variable values.  The system uniquely identifies with
an integer every instance of a variable with a particular name and package.  To
access a value, you must supply the frame along with the debug-variable since
these are particular to a function, not every instance of a variable on the
stack.

\begin{defun}{}{debug-variable-name}{\args{\var{debug-variable}}}
  
  This function returns the name of the \var{debug-variable}.  The
  name is the name of the symbol used as an identifier when writing
  the code.
\end{defun}


\begin{defun}{}{debug-variable-package}{\args{\var{debug-variable}}}
  
  This function returns the package name of the \var{debug-variable}.
  This is the package name of the symbol used as an identifier when
  writing the code.
\end{defun}


\begin{defun}{}{debug-variable-symbol}{\args{\var{debug-variable}}}
  
  This function returns the symbol from interning
  \code{debug-variable-name} in the package named by
  \code{debug-variable-package}.
\end{defun}


\begin{defun}{}{debug-variable-id}{\args{\var{debug-variable}}}
  
  This function returns the integer that makes \var{debug-variable}'s
  name and package name unique with respect to other
  \var{debug-variable}'s in the same function.
\end{defun}


\begin{defun}{}{debug-variable-validity}{%
    \args{\var{debug-variable} \var{basic-code-location}}}
  
  This function returns three values reflecting the validity of
  \var{debug-variable}'s value at \var{basic-code-location}:
  \begin{Lentry}
  \item[\kwd{valid}] The value is known to be available.
  \item[\kwd{invalid}] The value is known to be unavailable.
  \item[\kwd{unknown}] The value's availability is unknown.
  \end{Lentry}
\end{defun}


\begin{defun}{}{debug-variable-value}{\args{\var{debug-variable}
      \var{frame}}}
  
  This function returns the value stored for \var{debug-variable} in
  \var{frame}.  The value may be invalid.  This is \code{SETF}'able.
\end{defun}


\begin{defun}{}{debug-variable-valid-value}{%
    \args{\var{debug-variable} \var{frame}}}
  
  This function returns the value stored for \var{debug-variable} in
  \var{frame}.  If the value is not \kwd{valid}, then this signals an
  \code{invalid-value} error.
\end{defun}


\section{Frames}

Frames describe a particular call on the stack for a particular thread.  This
is the environment for name resolution, getting arguments and locals, and
returning values.  The stack conceptually grows up, so the top of the stack is
the most recently called function.

\code{top-frame}, \code{frame-down}, \code{frame-up}, and
\code{frame-debug-function} can only fail when there is absolutely no
debug information available.  This can only happen when someone saved a
Lisp image specifying that the system dump all debugging data.


\begin{defun}{}{top-frame}{}
  
  This function never returns the frame for itself, always the frame
  before calling \code{top-frame}.
\end{defun}


\begin{defun}{}{frame-down}{\args{\var{frame}}}
  
  This returns the frame immediately below \var{frame} on the stack.
  When \var{frame} is the bottom of the stack, this returns \nil.
\end{defun}


\begin{defun}{}{frame-up}{\args{\var{frame}}}
  
  This returns the frame immediately above \var{frame} on the stack.
  When \var{frame} is the top of the stack, this returns \nil.
\end{defun}


\begin{defun}{}{frame-debug-function}{\args{\var{frame}}}
  
  This function returns the debug-function for the function whose call
  \var{frame} represents.
\end{defun}


\begin{defun}{}{frame-code-location}{\args{\var{frame}}}
  
  This function returns the code-location where \var{frame}'s
  debug-function will continue running when program execution returns
  to \var{frame}.  If someone interrupted this frame, the result could
  be an unknown code-location.
\end{defun}


\begin{defun}{}{frame-catches}{\args{\var{frame}}}
  
  This function returns an a-list for all active catches in
  \var{frame} mapping catch tags to the code-locations at which the
  catch re-enters.
\end{defun}


\begin{defun}{}{eval-in-frame}{\args{\var{frame} \var{form}}}
  
  This evaluates \var{form} in \var{frame}'s environment.  This can
  signal several different debug-conditions since its success relies
  on a variety of inexact debug information: \code{invalid-value},
  \code{ambiguous-variable-name}, \code{frame-function-mismatch}.  See
  also \funref{preprocess-for-eval}.
\end{defun}

%   \begin{defun}{}{return-from-frame}{\args{\var{frame} \var{values}}}
%     
%     This returns the elements in the list \var{values} as multiple
%     values from \var{frame} as if the function \var{frame} represents
%     returned these values.  This signals a
%     \code{no-debug-function-returns} condition when \var{frame}'s
%     debug-function lacks information on returning values.
%     
%     \i{Not Yet Implemented}
%   \end{defun}


\section {Debug-functions}

Debug-functions represent the static information about a function determined at
compile time---argument and variable storage, their lifetime information,
etc.  The debug-function also contains all the debug-blocks representing
basic-blocks of code, and these contains information about specific
code-locations in a debug-function.

\begin{defmac}{}{do-debug-function-blocks}{%
    \args{(\var{block-var} \var{debug-function} \mopt{result-form})
      \mstar{form}}}
  
  This executes the forms in a context with \var{block-var} bound to
  each debug-block in \var{debug-function} successively.
  \var{Result-form} is an optional form to execute for a return value,
  and \code{do-debug-function-blocks} returns \nil if there is no
  \var{result-form}.  This signals a \code{no-debug-blocks} condition
  when the \var{debug-function} lacks debug-block information.
\end{defmac}


\begin{defun}{}{debug-function-lambda-list}{\args{\var{debug-function}}}
  
  This function returns a list representing the lambda-list for
  \var{debug-function}.  The list has the following structure:
  \begin{example}
    (required-var1 required-var2
    ...
    (:optional var3 suppliedp-var4)
    (:optional var5)
    ...
    (:rest var6) (:rest var7)
    ...
    (:keyword keyword-symbol var8 suppliedp-var9)
    (:keyword keyword-symbol var10)
    ...
    )
  \end{example}
  Each \code{var}\var{n} is a debug-variable; however, the symbol
  \kwd{deleted} appears instead whenever the argument remains
  unreferenced throughout \var{debug-function}.
  
  If there is no lambda-list information, this signals a
  \code{lambda-list-unavailable} condition.
\end{defun}


\begin{defmac}{}{do-debug-function-variables}{%
    \args{(\var{var} \var{debug-function} \mopt{result})
      \mstar{form}}}
  
  This macro executes each \var{form} in a context with \var{var}
  bound to each debug-variable in \var{debug-function}.  This returns
  the value of executing \var{result} (defaults to \nil).  This may
  iterate over only some of \var{debug-function}'s variables or none
  depending on debug policy; for example, possibly the compilation
  only preserved argument information.
\end{defmac}


\begin{defun}{}{debug-variable-info-available}{\args{\var{debug-function}}}
  
  This function returns whether there is any variable information for
  \var{debug-function}.  This is useful for distinguishing whether
  there were no locals in a function or whether there was no variable
  information.  For example, if \code{do-debug-function-variables}
  executes its forms zero times, then you can use this function to
  determine the reason.
\end{defun}


\begin{defun}{}{debug-function-symbol-variables}{%
    \args{\var{debug-function} \var{symbol}}}
  
  This function returns a list of debug-variables in
  \var{debug-function} having the same name and package as
  \var{symbol}.  If \var{symbol} is uninterned, then this returns a
  list of debug-variables without package names and with the same name
  as \var{symbol}.  The result of this function is limited to the
  availability of variable information in \var{debug-function}; for
  example, possibly \var{debug-function} only knows about its
  arguments.
\end{defun}


\begin{defun}{}{ambiguous-debug-variables}{%
    \args{\var{debug-function} \var{name-prefix-string}}}
  
  This function returns a list of debug-variables in
  \var{debug-function} whose names contain \var{name-prefix-string} as
  an initial substring.  The result of this function is limited to the
  availability of variable information in \var{debug-function}; for
  example, possibly \var{debug-function} only knows about its
  arguments.
\end{defun}


\begin{defun}{}{preprocess-for-eval}{%
    \args{\var{form} \var{basic-code-location}}}
  
  This function returns a function of one argument that evaluates
  \var{form} in the lexical context of \var{basic-code-location}.
  This allows efficient repeated evaluation of \var{form} at a certain
  place in a function which could be useful for conditional breaking.
  This signals a \code{no-debug-variables} condition when the
  code-location's debug-function has no debug-variable information
  available.  The returned function takes a frame as an argument.  See
  also \funref{eval-in-frame}.
\end{defun}


\begin{defun}{}{function-debug-function}{\args{\var{function}}}
  
  This function returns a debug-function that represents debug
  information for \var{function}.
\end{defun}


\begin{defun}{}{debug-function-kind}{\args{\var{debug-function}}}
  
  This function returns the kind of function \var{debug-function}
  represents.  The value is one of the following:
  \begin{Lentry}
  \item[\kwd{optional}] This kind of function is an entry point to an
    ordinary function.  It handles optional defaulting, parsing
    keywords, etc.
  \item[\kwd{external}] This kind of function is an entry point to an
    ordinary function.  It checks argument values and count and calls
    the defined function.
  \item[\kwd{top-level}] This kind of function executes one or more
    random top-level forms from a file.
  \item[\kwd{cleanup}] This kind of function represents the cleanup
    forms in an \code{unwind-protect}.
  \item[\nil] This kind of function is not one of the above; that is,
    it is not specially marked in any way.
  \end{Lentry}
\end{defun}


\begin{defun}{}{debug-function-function}{\args{\var{debug-function}}}
  
  This function returns the Common Lisp function associated with the
  \var{debug-function}.  This returns \nil{} if the function is
  unavailable or is non-existent as a user callable function object.
\end{defun}


\begin{defun}{}{debug-function-name}{\args{\var{debug-function}}}
  
  This function returns the name of the function represented by
  \var{debug-function}.  This may be a string or a cons; do not assume
  it is a symbol.
\end{defun}


\section{Debug-blocks}

Debug-blocks contain information pertinent to a specific range of code in a
debug-function.

\begin{defmac}{}{do-debug-block-locations}{%
    \args{(\var{code-var} \var{debug-block} \mopt{result})
      \mstar{form}}}
  
  This macro executes each \var{form} in a context with \var{code-var}
  bound to each code-location in \var{debug-block}.  This returns the
  value of executing \var{result} (defaults to \nil).
\end{defmac}


\begin{defun}{}{debug-block-successors}{\args{\var{debug-block}}}
  
  This function returns the list of possible code-locations where
  execution may continue when the basic-block represented by
  \var{debug-block} completes its execution.
\end{defun}


\begin{defun}{}{debug-block-elsewhere-p}{\args{\var{debug-block}}}
  
  This function returns whether \var{debug-block} represents elsewhere
  code.  This is code the compiler has moved out of a function's code
  sequence for optimization reasons.  Code-locations in these blocks
  are unsuitable for stepping tools, and the first code-location has
  nothing to do with a normal starting location for the block.
\end{defun}


\section{Breakpoints}

A breakpoint represents a function the system calls with the current frame when
execution passes a certain code-location.  A break point is active or inactive
independent of its existence.  They also have an extra slot for users to tag
the breakpoint with information.

\begin{defun}{}{make-breakpoint}{%
    \args{\var{hook-function} \var{what} \keys{\kwd{kind} \kwd{info}
        \kwd{function-end-cookie}}}}
  
  This function creates and returns a breakpoint.  When program
  execution encounters the breakpoint, the system calls
  \var{hook-function}.  \var{hook-function} takes the current frame
  for the function in which the program is running and the breakpoint
  object.
  
  \var{what} and \var{kind} determine where in a function the system
  invokes \var{hook-function}.  \var{what} is either a code-location
  or a debug-function.  \var{kind} is one of \kwd{code-location},
  \kwd{function-start}, or \kwd{function-end}.  Since the starts and
  ends of functions may not have code-locations representing them,
  designate these places by supplying \var{what} as a debug-function
  and \var{kind} indicating the \kwd{function-start} or
  \kwd{function-end}.  When \var{what} is a debug-function and
  \var{kind} is \kwd{function-end}, then hook-function must take two
  additional arguments, a list of values returned by the function and
  a function-end-cookie.
  
  \var{info} is information supplied by and used by the user.
  
  \var{function-end-cookie} is a function.  To implement function-end
  breakpoints, the system uses starter breakpoints to establish the
  function-end breakpoint for each invocation of the function.  Upon
  each entry, the system creates a unique cookie to identify the
  invocation, and when the user supplies a function for this argument,
  the system invokes it on the cookie.  The system later invokes the
  function-end breakpoint hook on the same cookie.  The user may save
  the cookie when passed to the function-end-cookie function for later
  comparison in the hook function.
  
  This signals an error if \var{what} is an unknown code-location.
  
  {\em Note: Breakpoints in interpreted code or byte-compiled code are
    not implemented.  Function-end breakpoints are not implemented for
    compiled functions that use the known local return convention
    (e.g. for block-compiled or self-recursive functions.)}

\end{defun}


\begin{defun}{}{activate-breakpoint}{\args{\var{breakpoint}}}
  
  This function causes the system to invoke the \var{breakpoint}'s
  hook-function until the next call to \code{deactivate-breakpoint} or
  \code{delete-breakpoint}.  The system invokes breakpoint hook
  functions in the opposite order that you activate them.
\end{defun}


\begin{defun}{}{deactivate-breakpoint}{\args{\var{breakpoint}}}
  
  This function stops the system from invoking the \var{breakpoint}'s
  hook-function.
\end{defun}


\begin{defun}{}{breakpoint-active-p}{\args{\var{breakpoint}}}
  
  This returns whether \var{breakpoint} is currently active.
\end{defun}


\begin{defun}{}{breakpoint-hook-function}{\args{\var{breakpoint}}}
  
  This function returns the \var{breakpoint}'s function the system
  calls when execution encounters \var{breakpoint}, and it is active.
  This is \code{SETF}'able.
\end{defun}


\begin{defun}{}{breakpoint-info}{\args{\var{breakpoint}}}
  
  This function returns \var{breakpoint}'s information supplied by the
  user.  This is \code{SETF}'able.
\end{defun}


\begin{defun}{}{breakpoint-kind}{\args{\var{breakpoint}}}

  This function returns the \var{breakpoint}'s kind specification.
\end{defun}


\begin{defun}{}{breakpoint-what}{\args{\var{breakpoint}}}
  
  This function returns the \var{breakpoint}'s what specification.
\end{defun}


\begin{defun}{}{delete-breakpoint}{\args{\var{breakpoint}}}
  
  This function frees system storage and removes computational
  overhead associated with \var{breakpoint}.  After calling this,
  \var{breakpoint} is useless and can never become active again.
\end{defun}


\section{Code-locations}

Code-locations represent places in functions where the system has correct
information about the function's environment and where interesting operations
can occur---asking for a local variable's value, setting breakpoints,
evaluating forms within the function's environment, etc.

Sometimes the interface returns unknown code-locations.  These
represent places in functions, but there is no debug information
associated with them.  Some operations accept these since they may
succeed even with missing debug data.  These operations' argument is
named \var{basic-code-location} indicating they take known and unknown
code-locations.  If an operation names its argument
\var{code-location}, and you supply an unknown one, it will signal an
error.  For example, \code{frame-code-location} may return an unknown
code-location if someone interrupted Lisp in the given frame.  The
system knows where execution will continue, but this place in the code
may not be a place for which the compiler dumped debug information.

\begin{defun}{}{code-location-debug-function}{\args{\var{basic-code-location}}}
  
  This function returns the debug-function representing information
  about the function corresponding to the code-location.
\end{defun}


\begin{defun}{}{code-location-debug-block}{\args{\var{basic-code-location}}}
  
  This function returns the debug-block containing code-location if it
  is available.  Some debug policies inhibit debug-block information,
  and if none is available, then this signals a \code{no-debug-blocks}
  condition.
\end{defun}


\begin{defun}{}{code-location-top-level-form-offset}{%
    \args{\var{code-location}}}
  
  This function returns the number of top-level forms before the one
  containing \var{code-location} as seen by the compiler in some
  compilation unit.  A compilation unit is not necessarily a single
  file, see the section on debug-sources.
\end{defun}


\begin{defun}{}{code-location-form-number}{\args{\var{code-location}}}
  
  This function returns the number of the form corresponding to
  \var{code-location}.  The form number is derived by walking the
  subforms of a top-level form in depth-first order.  While walking
  the top-level form, count one in depth-first order for each subform
  that is a cons.  See \funref{form-number-translations}.
\end{defun}


\begin{defun}{}{code-location-debug-source}{\args{\var{code-location}}}
  
  This function returns \var{code-location}'s debug-source.
\end{defun}


\begin{defun}{}{code-location-unknown-p}{\args{\var{basic-code-location}}}
  
  This function returns whether \var{basic-code-location} is unknown.
  It returns \nil{} when the code-location is known.
\end{defun}


\begin{defun}{}{code-location=}{\args{\var{code-location1}
      \var{code-location2}}}
  
  This function returns whether the two code-locations are the same.
\end{defun}


\section{Debug-sources}

Debug-sources represent how to get back the source for some code.  The
source is either a file (\code{compile-file} or \code{load}), a
lambda-expression (\code{compile}, \code{defun}, \code{defmacro}), or
a stream (something particular to \cmucl{}, \code{compile-from-stream}).

When compiling a source, the compiler counts each top-level form it
processes, but when the compiler handles multiple files as one block
compilation, the top-level form count continues past file boundaries.
Therefore \code{code-location-top-level-form-offset} returns an offset
that does not always start at zero for the code-location's
debug-source.  The offset into a particular source is
\code{code-location-top-level-form-offset} minus
\code{debug-source-root-number}.

Inside a top-level form, a code-location's form number indicates the
subform corresponding to the code-location.

\begin{defun}{}{debug-source-from}{\args{\var{debug-source}}}
  
  This function returns an indication of the type of source.  The
  following are the possible values:
  \begin{Lentry}
  \item[\kwd{file}] from a file (obtained by \code{compile-file} if
    compiled).
  \item[\kwd{lisp}] from Lisp (obtained by \code{compile} if
    compiled).
  \item[\kwd{stream}] from a non-file stream (\cmucl{} supports
    \code{compile-from-stream}).
  \end{Lentry}
\end{defun}


\begin{defun}{}{debug-source-name}{\args{\var{debug-source}}}
  
  This function returns the actual source in some sense represented by
  debug-source, which is related to \code{debug-source-from}:
  \begin{Lentry}
  \item[\kwd{file}] the pathname of the file.
  \item[\kwd{lisp}] a lambda-expression.
  \item[\kwd{stream}] some descriptive string that's otherwise
    useless.
\end{Lentry}
\end{defun}


\begin{defun}{}{debug-source-created}{\args{\var{debug-source}}}
  
  This function returns the universal time someone created the source.
  This may be \nil{} if it is unavailable.
\end{defun}


\begin{defun}{}{debug-source-compiled}{\args{\var{debug-source}}}
  
  This function returns the time someone compiled the source.  This is
  \nil{} if the source is uncompiled.
\end{defun}


\begin{defun}{}{debug-source-root-number}{\args{\var{debug-source}}}
  
  This returns the number of top-level forms processed by the compiler
  before compiling this source.  If this source is uncompiled, this is
  zero.  This may be zero even if the source is compiled since the
  first form in the first file compiled in one compilation, for
  example, must have a root number of zero---the compiler saw no other
  top-level forms before it.
\end{defun}


\section{Source Translation Utilities}

These two functions provide a mechanism for converting the rather
obscure (but highly compact) representation of source locations into an
actual source form:

\begin{defun}{}{debug-source-start-positions}{\args{\var{debug-source}}}
  
  This function returns the file position of each top-level form as a
  vector if \var{debug-source} is from a \kwd{file}.  If
  \code{debug-source-from} is \kwd{lisp} or \kwd{stream}, or the file
  is byte-compiled, then the result is \false{}.
\end{defun}


\begin{defun}{}{form-number-translations}{\args{\var{form}
      \var{tlf-number}}}
  
  This function returns a table mapping form numbers (see
  \code{code-location-form-number}) to source-paths.  A source-path
  indicates a descent into the top-level-form \var{form}, going
  directly to the subform corresponding to a form number.
  \var{tlf-number} is the top-level-form number of \var{form}.
\end{defun}


\begin{defun}{}{source-path-context}{%
    \args{\var{form} \var{path} \var{context}}}
  
  This function returns the subform of \var{form} indicated by the
  source-path.  \var{Form} is a top-level form, and \var{path} is a
  source-path into it.  \var{Context} is the number of enclosing forms
  to return instead of directly returning the source-path form.  When
  \var{context} is non-zero, the form returned contains a marker,
  \code{\#:****HERE****}, immediately before the form indicated by
  \var{path}.
\end{defun}

\chapter{Cross-Referencing Facility}
\label{xref}
\cindex{cross-referencing}
\credits{by Eric Marsden}

The \cmucl{} cross-referencing facility (abbreviated XREF) assists in
the analysis of static dependency relationships in a program. It
provides introspection capabilities such as the ability to know which
functions may call a given function, and the program contexts in which
a particular global variable is used. The compiler populates a
database of cross-reference information, which can be queried by the
user to know:

\begin{itemize}
\item
the list of program contexts (functions, macros, top-level forms)
where a given function may be called at runtime, either directly or
indirectly (via its function-object);

\item
the list of program contexts where a given global variable may be
read;

\item
the list of program contexts that bind a global variable;

\item
the list of program contexts where a given global variable may be
modified during the execution of the program.
\end{itemize}

A global variable is either a dynamic variable or a constant variable,
for instance declared using \code{defvar} or \code{defparameter} or
\code{defconstant}.


\section{Populating the cross-reference database}

\begin{defvar}{c:}{record-xref-info}
   When non-NIL, code that is compiled (either using
   \code{compile-file}, or by calling \code{compile} from the
   listener), will be analyzed for cross-references. Defaults to
   \nil{}.
\end{defvar}

Cross-referencing information is only generated by the compiler; the
interpreter does not populate the cross-reference database. XREF
analysis is independent of whether the compiler is generating native
code or byte code, and of whether it is compiling from a file, from a
stream, or is invoked interactively from the listener. 

\begin{defun}{xref:}{init-xref-database}{}
  Reinitializes the database of cross-references. This can be used to
  reclaim the space occupied by the database contents, or to discard
  stale cross-reference information.
\end{defun}



\section{Querying the cross-reference database}

\cmucl{} provides a number of functions in the XREF package that may
be used to query the cross-reference database:

\begin{defun}{xref:}{who-calls}{\args \var{function}}
   Returns the list of xref-contexts where \var{function} (either a
   symbol that names a function, or a function object) may be called
   at runtime. XREF does not record calls to macro-functions (such as
   \code{defun}) or to special forms (such as \code{eval-when}).
\end{defun}

\begin{defun}{xref:}{who-references}{\args \var{global-variable}}
   Returns the list of program contexts that may reference
   \var{global-variable}. 
\end{defun}

\begin{defun}{xref:}{who-binds}{\args \var{global-variable}}
  Returns a list of program contexts where the specified global
  variable may be bound at runtime (for example using \code{LET}).
\end{defun}

\begin{defun}{xref:}{who-sets}{\args \var{global-variable}}
  Returns a list of program contexts where the given global variable
  may be modified at runtime (for example using \code{SETQ}). 
\end{defun}

An \textit{xref-context} is the originating site of a cross-reference.
It identifies a portion of a program, and is defined by an
\code{xref-context} structure, that comprises a name, a source file and a
source-path. 

\begin{defun}{xref:}{xref-context-name}{\args \var{context}}
  Returns the name slot of an xref-context, which is one of:
\begin{itemize}
\item
a global function, which is named by a symbol or by a list of the form
\code{(setf\ foo)}. 

\item
a macro, named by a list \verb|(:macro foo)|.

\item
an inner function (\code{flet}, \code{labels}, anonymous lambdas) that
is named by a list of the form \verb|(:internal outer innner)|.

\item
a method, named by a list of the form
\verb|(:method foo (specializer1 specializer2)|. 

\item
a string \verb|"Top-Level Form"| that identifies a reference from a
top-level form. Note that multiple references from top-level forms
will only be listed once. 

\item
a compiler-macro, named by a string of the form
\verb|(:compiler-macro foo)|. 

\item
a string such as \verb|"DEFSTRUCT FOO"|, identifying a reference from
within a structure accessor or constructor or copier.

\item
a string such as 
\begin{verbatim}
  "Creation Form for #<KERNEL::CLASS-CELL STRUCT-FOO>"
\end{verbatim}

\item
a string such as \verb|"defun foo"|, or \verb|"defmethod bar (t)"|,
that identifies a reference from within code that has been generated
by the compiler for that form. For example, the compilation of a
\code{defclass} form causes accessor functions to be generated by the
compiler; this code is compiler-generated (it does not appear in the
source file), and so is identified by the XREF facility by a string. 
\end{itemize}
\end{defun}


\begin{defun}{xref:}{xref-context-file}{context}
  Return the truename (in the sense of the variable
   \vindexed{compile-file-truename}) of the source file from which the
   referencing forms were compiled. This slot will be \nil{} if the
   code was compiled from a stream, or interactively from the
   listener.
\end{defun}

\begin{defun}{xref:}{xref-context-source-path}{context}
  Return a list of positive integers identifying the form that
  contains the cross-reference. The first integer in the source-path
  is the number of the top-level form containing the cross-reference
  (for example, 2 identifies the second top-level form in the source
  file). The second integer in the source-path identifies the form
  within this top-level form that contains the cross-reference, and so
  on. This function will always return \nil{} if the file slot of an
  xref-context is \nil{}.

% While walking the top-level form, count one in depth-first order for
% each subform that is a cons.
\end{defun}




\section{Example usage}

In this section, we will illustrate use of the XREF facility on a
number of simple examples.

Consider the following program fragment, that defines a global
variable and a function.

\begin{verbatim}
  (defvar *variable-one* 42)
  
  (defun function-one (x)
     (princ (* x *variable-one*)))
\end{verbatim}

We save this code in a file named \code{example.lisp}, enable
cross-referencing, clear any previous cross-reference information,
compile the file, and can then query the cross-reference database
(output has been modified for readability).

\begin{verbatim}
  USER> (setf c:*record-xref-info* t)
  USER> (xref:init-xref-database)
  USER> (compile-file "example")
  USER> (xref:who-calls 'princ)
  (#<xref-context function-one in #p"example.lisp">)
  USER> (xref:who-references '*variable-one*)
  (#<xref-context function-one in #p"example.lisp">)
\end{verbatim}

From this example, we see that the compiler has noted the call to the
global function \code{princ} in \code{function-one}, and the reference
to the global variable \code{*variable-one*}. 

Suppose that we add the following code to the previous file. 

\begin{verbatim}
(defconstant +constant-one+ 1)
  
(defstruct struct-one
  slot-one
  (slot-two +constant-one+ :type integer)
  (slot-three 42 :read-only t))

(defmacro with-different-one (&body body)
  `(let ((*variable-one* 666))
      ,@body))

(defun get-variable-one () *variable-one*)

(defun (setf get-variable-one) (new-value)
  (setq *variable-one* new-value))
\end{verbatim}

In the following example, we detect references x and y.


% FIXME add function with LABELS, a binding, a set



The following function illustrates the effect that various forms of
optimization carried out by the \cmucl{} compiler can have on the
cross-references that are reported for a particular program. The
compiler is able to detect that the evaluated condition is always
false, and that the first clause of the \code{if} will never be taken
(this optimization is called dead-code elimination). XREF will
therefore not register a call to the function \code{sin} from the
function \code{foo}. Likewise, no calls to the functions \code{sqrt}
and \code{\textless} are registered, because the compiler has eliminated the
code that evaluates the condition. Finally, no call to the function
\code{expt} is generated, because the compiler was able to evaluate
the result of the expression \code{(expt 3 2)} at compile-time (though
a process called constant-folding).

\begin{verbatim}
;; zero call references are registered for this function!
(defun constantly-nine (x)
  (if (< (sqrt x) 0)
      (sin x)
      (expt 3 2)))
\end{verbatim}


\section{Limitations of the cross-referencing facility}

No cross-reference information is available for interpreted functions.
The cross-referencing database is not persistent: unless you save an
image using \code{save-lisp}, the database will be empty each time
\cmucl{} is restarted. There is no mechanism that saves
cross-reference information in FASL files, so loading a system from
compiled code will not populate the cross-reference database.

The cross-referencing facility is only able to analyze the static
dependencies in a program; it does not provide any information about
runtime (dynamic) dependencies. For instance, XREF is able to identify
the list of program contexts where a given function may be called, but
is not able to determine which contexts will be activated when the
program is executed with a specific set of input parameters. However,
the static analysis that is performed by the \cmucl{} compiler does
allow XREF to provide more information than would be available from a
mere syntactic analysis of a program. References that occur from
within unreachable code will not be displayed by XREF, because the
\cmucl{} compiler deletes dead code before cross-references are
analyzed. Certain ``trivial'' function calls (where the result of the
function call can be evaluated at compile-time) may be eliminated by
optimizations carried out by the compiler; see the example below.

If you examine the entire database of cross-reference information (by
accessing undocumented internals of the XREF package), you will note
that XREF notes ``bogus'' cross-references to function calls that are
inserted by the compiler. For example, in safe code, the \cmucl{}
compiler inserts a call to an internal function called
\code{c::\%verify-argument-count}, so that the number of arguments
passed to the function is checked each time it is called. The XREF
facility does not distinguish between user code and these forms that
are introduced during compilation. This limitation should not be
visible if you use the documented functions in the XREF package. 

As of the 18e release of \cmucl{}, the cross-referencing facility is
experimental; expect details of its implementation to change in future
releases. In particular, the names given to CLOS methods and to inner
functions will change in future releases. 


\chapter{Internationalization}
\label{i18n}
\cindex{Internationalization}

\cmucl{} supports internationalization by supporting Unicode
characters internally and by adding support for external formats to
convert from the internal format to an appropriate external character
coding format.

To understand the support for Unicode, we refer the reader to the
\ifpdf
\href{http://www.unicode.org/}{Unicode standard}.
\else
\emph{Unicode standard} at \href{http://www.unicode.org}
\fi
\section{Changes}

To support internationalization, the following changes to Common Lisp
functions have been done.


\subsection{Design Choices}

To support Unicode, there are many approaches.  One choice is to
support both 8-bit \code{base-char} and a 21-bit (or larger)
\code{character} since Unicode codepoints use 21 bits.  This generally
means strings are much larger, and complicates the compiler by having
to support both \code{base-char} and \code{character} types and the
corresponding string types.  This also adds complexity for the user to
understand the difference between the different string and character
types.

Another choice is to have just one character and string type that can
hold the entire Unicode codepoint.  While simplifying the compiler and
reducing the burden on the user, this significantly increases memory
usage for strings.

The solution chosen by \cmucl{} is to tradeoff the size and complexity
by having only 16-bit characters.  Most of the important languages can
be encoded using only 16-bits.  The rest of the codepoints are for
rare languages or ancient scripts.  Thus, the memory usage is
significantly reduced while still supporting the the most important
languages.  Compiler complexity is also reduced since \code{base-char}
and \code{character} are the same as are the string types..  But we
still want to support the full Unicode character set.  This is
achieved by making strings be UTF-16 strings internally.  Hence, Lisp
strings are UTF-16 strings, and Lisp characters are UTF-16 code-units.


\subsection{Characters}
\label{sec:i18n:characters}

Characters are now 16 bits long instead of 8 bits, and \code{base-char}
and \code{character} types are the same.  This difference is
naturally indicated by changing \code{char-code-limit} from 256 to
65536.

\subsection{Strings}
\label{sec:i18n:strings}

In \cmucl{} there is only one type of string---\code{base-string} and
\code{string} are the same.  

Internally, the strings are encoded using UTF-16.  This means that in
some rare cases the number of Lisp characters in a string is not the
same as the number of codepoints in the string.


\section{External Formats}

To be able to communicate to the external world, \cmucl{} supports
external formats to convert to and from the external world to
\cmucl{}'s string format.  The external format is specified in several
ways.  The standard streams \var{*standard-input*},
\var{*standard-output*}, and \var{*standard-error*} take the format
from the value specified by \var{*default-external-format*}.  The
default value of \var{*default-external-format*} is \kwd{iso8859-1}.

For files, \code{OPEN} takes the \kwd{external-format}
parameter to specify the format.  The default external format is
\kwd{default}. 

\begin{defun}{stream:}{set-system-external-format}{\var{terminal} \ampoptional{} \var{filenames}}
  This function changes the external format used for
  \var{*standard-input*}, \var{*standard-output*}, and
  \var{*standard-error*} to the external format specified by
  \var{terminal}.  Additionally, the Unix file name encoding can be
  set to the value specified by \var{filenames} if non-\nil.
\end{defun}

\subsection{Available External Formats}

The available external formats are listed below in
Table~\ref{table:external-formats}.  The first column gives the
external format, and the second column gives a list of aliases that
can be used for this format.  The set of aliases can be changed by
changing the \file{aliases} file.

For all of these formats, if an illegal sequence is encountered, no
error or warning is signaled.  Instead, the offending sequence is
silently replaced with the Unicode REPLACEMENT CHARACTER (U$+$FFFD).

\begin{table}
  \centering
  \begin{tabular}{|l|l|p{3in}|}
    \hline
    \textbf{Format} & \textbf{Aliases} & \textbf{Description} \\
    \hline
    \hline
    \kwd{iso8859-1} & \kwd{latin1} \kwd{latin-1} \kwd{iso-8859-1} & ISO8859-1 \\
    \hline
    \kwd{iso8859-2} & \kwd{latin2} \kwd{latin-2} \kwd{iso-8859-2} & ISO8859-2 \\
    \hline
    \kwd{iso8859-3} & \kwd{latin3} \kwd{latin-3} \kwd{iso-8859-3} & ISO8859-3 \\
    \hline
    \kwd{iso8859-4} & \kwd{latin4} \kwd{latin-4} \kwd{iso-8859-4} & ISO8859-4 \\
    \hline
    \kwd{iso8859-5} & \kwd{cyrillic} \kwd{iso-8859-5} & ISO8859-5 \\
    \hline
    \kwd{iso8859-6} & \kwd{arabic} \kwd{iso-8859-6} & ISO8859-6 \\
    \hline
    \kwd{iso8859-7} & \kwd{greek} \kwd{iso-8859-7} & ISO8859-7 \\
    \hline
    \kwd{iso8859-8} & \kwd{hebrew} \kwd{iso-8859-8} & ISO8859-8 \\
    \hline
    \kwd{iso8859-9} & \kwd{latin5} \kwd{latin-5} \kwd{iso-8859-9} & ISO8859-9 \\
    \hline
    \kwd{iso8859-10} & \kwd{latin6} \kwd{latin-6} \kwd{iso-8859-10} & ISO8859-10 \\
    \hline
    \kwd{iso8859-13} & \kwd{latin7} \kwd{latin-7} \kwd{iso-8859-13} & ISO8859-13 \\
    \hline
    \kwd{iso8859-14} & \kwd{latin8} \kwd{latin-8} \kwd{iso-8859-14} & ISO8859-14 \\
    \hline
    \kwd{iso8859-15} & \kwd{latin9} \kwd{latin-9} \kwd{iso-8859-15} & ISO8859-15 \\
    \hline
    \kwd{utf-8} & \kwd{utf} \kwd{utf8} & UTF-8 \\
    \hline
    \kwd{utf-16} & \kwd{utf16} & UTF-16 with optional BOM \\
    \hline
    \kwd{utf-16-be} & \kwd{utf-16be} \kwd{utf16-be} & UTF-16 big-endian (without BOM) \\
    \hline
    \kwd{utf-16-le} & \kwd{utf-16le} \kwd{utf16-le} & UTF-16 little-endian (without BOM) \\
    \hline
    \kwd{utf-32} & \kwd{utf32} & UTF-32 with optional BOM \\
    \hline
    \kwd{utf-32-be} & \kwd{utf-32be} \kwd{utf32-be} & UTF-32 big-endian (without BOM) \\
    \hline
    \kwd{utf-32-le} & \kwd{utf-32le} \kwd{utf32-le} & UTF-32 little-endian (without BOM) \\
    \hline
    \kwd{cp1250} & & \\
    \hline
    \kwd{cp1251} & & \\
    \hline
    \kwd{cp1252} & \kwd{windows-1252} \kwd{windows-cp1252} \kwd{windows-latin1} & \\
    \hline
    \kwd{cp1253} & & \\
    \hline
    \kwd{cp1254} & & \\
    \hline
    \kwd{cp1255} & & \\
    \hline
    \kwd{cp1256} & & \\
    \hline
    \kwd{cp1257} & & \\
    \hline
    \kwd{cp1258} & & \\
    \hline
    \kwd{koi8-r} & & \\
    \hline
    \kwd{mac-cyrillic} & & \\
    \hline
    \kwd{mac-greek} & & \\
    \hline
    \kwd{mac-icelandic} & & \\
    \hline
    \kwd{mac-latin2} & & \\
    \hline
    \kwd{mac-roman} & & \\
    \hline
    \kwd{mac-turkish} & & \\
    \hline
  \end{tabular}
  \caption{External formats}
  \label{table:external-formats}
\end{table}

\subsection{Composing External Formats}

A composing external format is an external format that converts between
one codepoint and another, rather than between codepoints and octets.
A composing external format must be used in conjunction with another
(octet-producing) external format.  This is specified by
using a list as the external format.  For example, we can use
\code{'(\kwd{latin1} \kwd{crlf})} as the external format. In this
particular example, the external format is latin1, but whenever a
carriage-return/linefeed sequence is read, it is converted to the Lisp
\lispchar{Newline} character.  Conversely, whenever a string is written,
a Lisp \lispchar{Newline} character is converted to a
carriage-return/linefeed sequence.  Without the \kwd{crlf} composing
format, the carriage-return and linefeed will be read in as separate
characters, and on output the Lisp \lispchar{Newline} character is
output as a single linefeed character.

Table~\ref{table:composing-formats} lists the available composing formats.

\begin{table}
  \centering
  \begin{tabular}{|l|l|p{3in}|}
    \hline
    \textbf{Format} & \textbf{Aliases} & \textbf{Description} \\
    \hline
    \hline
    \kwd{crlf} & \kwd{dos} & Composing format for converting to/from DOS (CR/LF)
    end-of-line sequence to \lispchar{Newline}\\
    \hline
    \kwd{beta-gk} & & Composing format that translates (lower-case) Beta
    code (an ASCII encoding of ancient Greek) \\
    \hline
    \kwd{final-sigma} & & Composing format that attempts to detect sigma in
    word-final position and change it from U+3C3 to U+3C2\\
    \hline
  \end{tabular}
  \caption{Composing external formats}
  \label{table:composing-formats}
\end{table}

\section{Dictionary}

\subsection{Variables}

\begin{defvar}{extensions:}{default-external-format}
   This is the default external format to use for all newly opened
   files.  It is also the default format to use for
   \var{*standard-input*}, \var{*standard-output*}, and
   \var{*standard-error*}.  The default value is \kwd{iso8859-1}.

   Setting this will cause the standard streams to start using the new
   format immediately.  If a stream has been created with external
   format \kwd{default}, then setting \var{*default-external-format*}
   will cause all subsequent input and output to use the new value of
   \var{*default-external-format*}.
\end{defvar}
\subsection{Characters}

Remember that \cmucl{}'s characters are only 16-bits long but Unicode
codepoints are up to 21 bits long.  Hence there are codepoints that
cannot be represented via Lisp characters.  Operating on individual
characters is not recommended.  Operations on strings are better.
(This would be true even if \cmucl{}'s characters could hold a
full Unicode codepoint.)

\begin{defun}{}{char-equal}{\amprest{} \var{characters}}
   \defunx{char-not-equal}{\amprest{} \var{characters}}
   \defunx{char-lessp}{\amprest{} \var{characters}}
   \defunx{char-greaterp}{\amprest{} \var{characters}}
   \defunx{char-not-greaterp}{\amprest{} \var{characters}}
   \defunx{char-not-lessp}{\amprest{} \var{characters}}
   For the comparison, the characters are converted to lowercase and
   the corresponding \code{char-code} are compared.
\end{defun}

\begin{defun}{}{alpha-char-p}{\args \var{character}}
  Returns non-nil{} if the Unicode category is a letter category.
\end{defun}

\begin{defun}{}{alphanumericp}{\args \var{character}}
  Returns non-nil{} if the Unicode category is a letter category or an ASCII
  digit.
\end{defun}

\begin{defun}{}{digit-char-p}{\args \var{character} \ampoptional{} \var{radix}}
   Only recognizes ASCII digits (and ASCII letters if the radix is larger
   than 10).
\end{defun}

\begin{defun}{}{graphic-char-p}{\args \var{character}}
  Returns non-nil{} if the Unicode category is a graphic category.
\end{defun}

\begin{defun}{}{upper-case-p}{\args \var{character}}
  \defunx{lower-case-p}{\args \var{character}}
  Returns non-nil{} if the Unicode category is an uppercase
  (lowercase) character.
\end{defun}

\begin{defun}{lisp:}{title-case-p}{\args \var{character}}
  Returns non-nil{} if the Unicode category is a titlecase character.
\end{defun}

\begin{defun}{}{both-case-p}{\args \var{character}}
  Returns non-nil{} if the Unicode category is an uppercase,
  lowercase, or titlecase character.
\end{defun}

\begin{defun}{}{char-upcase}{\args \var{character}}
  \defunx{char-downcase}{\args \var{character}}
  The Unicode uppercase (lowercase) letter is returned.
\end{defun}

\begin{defun}{lisp:}{char-titlecase}{\args \var{character}}
  \defunx{char-downcase}{\args \var{character}}
  The Unicode titlecase letter is returned.
\end{defun}

\begin{defun}{}{char-name}{\args \var{char}}
   If possible the name of the character \var{char} is returned.  If
   there is a Unicode name, the Unicode name is returned, except
   spaces are converted to underscores and the string is capitalized
   via \code{string-capitalize}.  If there is no Unicode name, the
   form \lispchar{U+xxxx} is returned where ``xxxx'' is the
   \code{char-code} of the character, in hexadecimal.
\end{defun}

\begin{defun}{}{name-char}{\args \var{name}}
  The inverse to \code{char-name}.  If no character has the name
  \var{name}, then \nil{} is returned.  Unicode names are not
  case-sensitive, and spaces and underscores are optional.
\end{defun}
\subsection{Strings}

Strings in \cmucl{} are UTF-16 strings.  That is, for Unicode code
points greater than 65535, surrogate pairs are used.  We refer the
reader to the Unicode standard for more information about surrogate
pairs.  We just want to make a note that because of the UTF-16
encoding of strings, there is a distinction between Lisp characters
and Unicode codepoints.  The standard string operations know about
this encoding and handle the surrogate pairs correctly.


\begin{defun}{}{string-upcase}{\args \var{string} \keys{\kwd{start}
      \kwd{end} \kwd{casing}}}
  \defunx{string-downcase}{\args \var{string} \keys{\kwd{start}
      \kwd{end} \kwd{casing}}}
  \defunx{string-capitalize}{\args \var{string} \keys{\kwd{start}
      \kwd{end} \kwd{casing}}}
  The case of the \var{string} is changed appropriately.  Surrogate
  pairs are handled correctly.  The conversion to the appropriate case
  is done based on the Unicode conversion.  The additional argument
  \kwd{casing} controls how case conversion is done.  The default
  value is \kwd{:simple}, which uses simple Unicode case conversion.
  If \kwd{casing} is \kwd{:full}, then full Unicode case conversion is
  done where the string may actually increase in length.
\end{defun}

\begin{defun}{}{nstring-upcase}{\args \var{string} \keys{\kwd{start} \kwd{end}}}
  \defunx{nstring-downcase}{\args \var{string} \keys{\kwd{start} \kwd{end}}}
  \defunx{nstring-capitalize}{\args \var{string} \keys{\kwd{start}
      \kwd{end}}}
  The case of the \var{string} is changed appropriately.  Surrogate
  pairs are handled correctly.  The conversion to the appropriate case
  is done based on the Unicode conversion.  (Full casing is not
  available because the string length cannot be increased when needed.)
\end{defun}

\begin{defun}{}{string=}{\args \var{s1} \var{s2} \keys{\kwd{start1}
      \kwd{end1} \kwd{start2} \kwd{end2}}}
  \defunx{string/=}{\args \var{s1} \var{s2} \keys{\kwd{start1} \kwd{end1} \kwd{start2} \kwd{end2}}}
  \defunx{string$<$}{\args \var{s1} \var{s2} \keys{\kwd{start1} \kwd{end1} \kwd{start2} \kwd{end2}}}
  \defunx{string$>$}{\args \var{s1} \var{s2} \keys{\kwd{start1} \kwd{end1} \kwd{start2} \kwd{end2}}}
  \defunx{string$<$=}{\args \var{s1} \var{s2} \keys{\kwd{start1} \kwd{end1} \kwd{start2} \kwd{end2}}}
  \defunx{string$>$=}{\args \var{s1} \var{s2} \keys{\kwd{start1} \kwd{end1} \kwd{start2} \kwd{end2}}}
  The string comparison is done in codepoint order.  (This is
  different from just comparing the order of the individual characters
  due to surrogate pairs.)  Unicode collation is not done.
\end{defun}

\begin{defun}{}{string-equal}{\args \var{s1} \var{s2} \keys{\kwd{start1}
      \kwd{end1} \kwd{start2} \kwd{end2}}}
  \defunx{string-not-equal}{\args \var{s1} \var{s2} \keys{\kwd{start1} \kwd{end1} \kwd{start2} \kwd{end2}}}
  \defunx{string-lessp}{\args \var{s1} \var{s2} \keys{\kwd{start1} \kwd{end1} \kwd{start2} \kwd{end2}}}
  \defunx{string-greaterp}{\args \var{s1} \var{s2} \keys{\kwd{start1} \kwd{end1} \kwd{start2} \kwd{end2}}}
  \defunx{string-not-greaterp}{\args \var{s1} \var{s2} \keys{\kwd{start1} \kwd{end1} \kwd{start2} \kwd{end2}}}
  \defunx{string-not-lessp}{\args \var{s1} \var{s2} \keys{\kwd{start1} \kwd{end1} \kwd{start2} \kwd{end2}}}
  Each codepoint in each string is converted to lowercase and the
  appropriate comparison of the codepoint values is done.  Unicode
  collation is not done.
\end{defun}

\begin{defun}{}{string-left-trim}{\args \var{bag} \var{string}}
  \defunx{string-right-trim}{\args \var{bag} \var{string}}
  \defunx{string-trim}{\args \var{bag} \var{string}}
  Removes any characters in \code{bag} from the left, right, or both
  ends of the string \code{string}, respectively.  This has potential
  problems if you want to remove a surrogate character from the
  string, since a single character cannot represent a surrogate.  As
  an extension, if \code{bag} is a string, we properly handle
  surrogate characters in the \code{bag}.
\end{defun}

\subsection{Sequences}

Since strings are also sequences, the sequence functions can be used
on strings.  We note here some issues with these functions.  Most
issues are due to the fact that strings are UTF-16 strings and
characters are UTF-16 code units, not Unicode codepoints.

\begin{defun}{}{remove-duplicates}{\args \var{sequence}
    \keys{\kwd{from-end} \kwd{test} \kwd{test-not} \kwd{start}
      \kwd{end} \kwd{key}}}
  \defunx{delete-duplicates}{\args \var{sequence}
    \keys{\kwd{from-end} \kwd{test} \kwd{test-not} \kwd{start}
      \kwd{end} \kwd{key}}}
  Because of surrogate pairs these functions may remove a high or low
  surrogate value, leaving the string in an invalid state.  Use these
  functions carefully with strings.
\end{defun}


\subsection{Reader}

To support Unicode characters, the reader has been extended to
recognize characters written in hexadecimal.  Thus \lispchar{U+41} is
the ASCII capital letter ``A'', since 41 is the hexadecimal code for
that letter.  The Unicode name of the character is also recognized,
except spaces in the name are replaced by underscores.

Recall, however, that characters in \cmucl{} are only 16 bits long so
many Unicode characters cannot be represented.  However, strings can
represent all Unicode characters.

When symbols are read, the symbol name is converted to Unicode NFC
form before interning the symbol into the package.  Hence,
\code{symbol-name (intern ``string'')} may produce a string that is
not \code{string=} to ``string''.  However, after conversion to NFC
form, the strings will be identical.

\subsection{Printer}

When printing characters, if the character is a graphic character, the
character is printed.  Thus \lispchar{U+41} is printed as
\lispchar{A}.  If the character is not a graphic character, the Lisp
name (e.g., \lispchar{Tab}) is used if possible;
if there is no Lisp name, the Unicode name is used.  If there is no
Unicode name, the hexadecimal char-code is
printed.  For example, \lispchar{U+34e}, which is not a graphic
character, is printed as \lispchar{Combining\_Upwards\_Arrow\_Below},
and \lispchar{U+9f} which is not a graphic character and does not have a
Unicode name, is printed as \lispchar{U+009F}.

\subsection{Miscellaneous}


\subsubsection{Files}

\cmucl{} loads external formats using the search-list
\file{ext-formats:}.  The \file{aliases} file is also located using
this search-list.

The Unicode data base is stored in compressed form in the file
\file{ext-formats:unidata.bin}.  If this file is not found, Unicode
support is severely reduced; you can only use ASCII characters.

\subsubsection{Utilities}

Since strings are UTF-16 and hence may contain surrogate pairs, some
utility functions are provided to make access easier.

\begin{defun}{lisp:}{codepoint}{\args \var{string} \var{i}
    \ampoptional{} \var{end}}
  Return the codepoint value from \var{string} at position \var{i}.
  If code unit at that position is a surrogate value, it is combined
  with either the previous or following code unit (when possible) to
  compute the codepoint.  The first return value is the codepoint
  itself.  The second return value is \nil{} if the position is not a
  surrogate pair.  Otherwise, $+1$ or $-1$ is returned if the position
  is the high (leading) or low (trailing) surrogate value, respectively.

  This is useful for iterating through a string in codepoint sequence.
\end{defun}

\begin{defun}{lisp:}{surrogates-to-codepoint}{\args \var{hi} \var{lo}}
  Convert the given \var{hi} and \var{lo} surrogate characters to the
  corresponding codepoint value
\end{defun}

\begin{defun}{lisp:}{surrogates}{\args \var{codepoint}}
  Convert the given \var{codepoint} value to the corresponding high
  and low surrogate characters.  If the codepoint is less than 65536,
  the second value is \nil{} since the codepoint does not need to be
  represented as a surrogate pair.
\end{defun}

\begin{defun}{stream:}{string-encode}{\args \var{string}
    \var{external-format} \ampoptional{} (\var{start} 0) \var{end}}
  \code{string-encode} encodes \var{string} using the format
  \var{external-format}, producing an array of octets.  Each octet is
  converted to a character via \code{code-char} and the resulting
  string is returned.

  The optional argument \var{start}, defaulting to 0, specifies the
  starting index and \var{end}, defaulting to the length of the
  string, is the end of the string.
\end{defun}

\begin{defun}{stream:}{string-decode}{\args \var{string}
    \var{external-format} \ampoptional{} (\var{start} 0) \var{end}}
  \code{string-decode} decodes \var{string} using the format
  \var{external-format} and produces a new string.  Each character of
  \var{string} is converted to octet (by \code{char-code}) and the
  resulting array of octets is used by the external format to produce
  a string.  This is the inverse of \code{string-encode}.

  The optional argument \var{start}, defaulting to 0, specifies the
  starting index and \var{end}, defaulting to the length of the
  string, is the end of the string.

  \var{string} must consist of characters whose \code{char-code} is
  less than 256.
\end{defun}

\begin{defun}{}{string-to-octets}{\args \var{string} \keys{\kwd{start}
      \kwd{end} \kwd{external-format} \kwd{buffer}}}
  \code{string-to-octets} converts \var{string} to a sequence of
  octets according to the external format specified by
  \var{external-format}.  The string to be converted is bounded by
  \var{start}, which defaults to 0, and \var{end}, which defaults to
  the length of the string.  If \var{buffer} is specified, the octets
  are placed in \var{buffer}.  If \var{buffer} is not specified, a new
  array is allocated to hold the octets.  In all cases the buffer is
  returned.
\end{defun}

\begin{defun}{}{octets-to-string}{\args \var{octets} \keys{\kwd{start}
      \kwd{end} \kwd{external-format} \kwd{string}}}
  \code{octets-to-string} converts the sequence of octets in
  \var{octets} to a string.   \var{octets} must be a
  \code{(simple-array (unsigned-byte 8) (*))}.  The octets to be
  converted are bounded by \var{start} and \var{end}, which default to
  0 and the length of the array, respectively.  The conversion is
  performed according to the external format specified by
  \var{external-format}.  If \var{string} is specified, the octets are
  converted and stored in \var{string}.  \var{string} must be
  \code{simple-string}.  Otherwise, a new string is
  created.  

  Three values are returned: the string, the number of characters
  written to the string, and the number of octets consumed to produce
  the characters.
\end{defun}

\section{Writing External Formats}

\subsection{External Formats}
Users may write their own external formats.  It is probably easiest to
look at existing external formats to see how do this.

An external format basically needs two functions:
\code{octets-to-code} to convert octets to Unicode codepoints and
\code{code-to-octets} to convert Unicode codepoints to octets.  The
external format is defined using the macro
\code{stream::define-external-format}.

\begin{defmac}[base]{stream:}{define-external-format}{\args \var{name}
    (\keys{\var{min} \var{max} \var{size}}) (\amprest{} \var{slots})
    \morekeys{\var{octets-to-code} \var{code-to-octets}
              \var{flush-state} \var{copy-state}}}
  \defmacx[stream:]{define-external-format}{\args \var{name}
    (\var{base}) (\amprest{} \var{slots})}

  The first defines a new external format of the name \kwd{name}.
  \var{min}, \var{max}, and \var{size} are the minimum and maximum
  number of octets that make up a character.  (\code{\kwd{size} n} is
  just a short cut for \code{\kwd{min} n \kwd{max} n}.)  The arguments
  \var{octets-to-code} and \var{code-to-octets} are not optional in
  this case.  They specify how to convert octets to codepoints and
  vice versa, respectively.  These should be backquoted forms for the
  body of a function to do the conversion.  See the description below
  for these functions.  Some good examples are the external format for
  \kwd{utf-8} or \kwd{utf-16}.  The \kwd{slots} argument is a list of
  read-only slots, similar to defstruct.  The slot names are available as
  local variables inside the \var{code-to-octets} and \var{octets-to-code}
  bodies.

  The second form above defines an external format with the name
  \kwd{name} that is based on a previously defined format \kwd{base}.
  The \var{slots} are inherited from the \kwd{base} format by default,
  although the definition may alter their values and add new slots.
  See, for example, the \kwd{mac-greek} external format.

\end{defmac}

\begin{defmac}{}{octets-to-code}{\args \var{state} \var{input}
    \var{unput} \amprest{} \var{args}}
  This defines a form to be used by an external format to convert
  octets to a code point.  \var{state} is a form that can be used by
  the body to access the state variable of a stream.  This can be used
  for any reason to hold anything needed by \code{octets-to-code}.
  \var{input} is a form that returns one octet from the input stream.
  \var{unput} will put back \var{N} octets to the stream.  \var{args} is a
  list of variables that need to be defined for any symbols in the
  body of the macro.
\end{defmac}

\begin{defmac}{}{code-to-octets}{\args \var{code} \var{state}
    \var{output} \amprest{} \var{args}}
  Defines a form to be used by the external format to convert a code
  point to octets for output.  \var{code} is the code point to be
  converted.  \var{state} is a form to access the current value of the
  stream's state variable.  \var{output} is a form that writes one
  octet to the output stream.
\end{defmac}

\begin{defmac}{}{flush-state}{\args \var{state}
    \var{output} \amprest{} \var{args}}
  Defines a form to be used by the external format to flush out
  any state when an output stream is closed.  Similar to
  \code{code-to-octets}, but there is no code point to be output.

  If \false, then nothing special is needed to flush the state to the
  output.

  This is called only when an output character stream is being closed.
\end{defmac}

\begin{defmac}{}{copy-state}{\args \var{state} \amprest{} \var{args}}
  Defines a form to copy any state needed by the external format.
  This should probably be a deep copy so that if the original
  state is modified, the copy is not.

  If not given, then nothing special is needed to copy the state
  either because there is no state for the external format or that no
  special copier is needed.
\end{defmac}

\subsection{Composing External Formats}

\begin{defmac}{stream:}{define-composing-external-format}{\args \var{name}
    (\keys{\var{min} \var{max} \var{size}}) \var{input}
    \var{output}}
  This is the same as \code{define-external-format}, except that a
  composing external format is created.
\end{defmac}


\twocolumn
\cindex{Function Index}
\printindex[funs]

\twocolumn
\cindex{Variable Index}
\printindex[vars]

\twocolumn
\cindex{Type Index}
\printindex[types]

\onecolumn
\cindex{Concept Index}
\printindex[concept]

\end{document}
