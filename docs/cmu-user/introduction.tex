\chapter{Introduction}

\cmucl{} is a free, high-performance implementation of the Common Lisp
programming language which runs on most major Unix platforms. It
mainly conforms to the ANSI Common Lisp standard. Here is a summary of
its main features:

\begin{itemize}
\item a {\em sophisticated native-code compiler} which is capable of
powerful type inferences, and generates code competitive in speed with
C compilers.

\item generational garbage collection and multiprocessing
capability on the x86 ports.

\item a foreign function interface which allows interfacing with C code and
system libraries, including shared libraries on most platforms, and
direct access to Unix system calls.

\item support for interprocess communication and remote procedure
calls.
     
\item an implementation of CLOS, the Common Lisp Object System, which
includes multimethods and a metaobject protocol.

\item a graphical source-level debugger using a Motif interface, and a
code profiler.

\item an interface to the X11 Window System (CLX), and a sophisticated
graphical widget library (Garnet).

\item programmer-extensible input and output streams.
                        
\item an Emacs-like editor implemented in Common Lisp.

\item public domain: free, with full source code and no
strings attached (and no warranty).  Like GNU/Linux and the *BSD
operating systems, \cmucl{} is maintained and improved by a team of
volunteers collaborating over the Internet.
\end{itemize}


This user's manual contains only implementation-specific information
about \cmucl. Users will also need a separate manual describing the
\clisp{} standard, for example, the
\ifpdf
\href{http://www.xanalys.com/software_tools/reference/HyperSpec/FrontMatter/index-text.html}
{Hyperspec}.
\else
\emph{Hyperspec} at \href{http://www.xanalys.com/software_tools}{www.xanalys.com}
\fi


In addition to the language itself, this document describes a number
of useful library modules that run in \cmucl. \hemlock, an Emacs-like
text editor, is included as an integral part of the \cmucl{}
environment. Two documents describe \hemlock{}: the {\it Hemlock
User's Manual}, and the {\it Hemlock Command Implementor's Manual}.


\section{Distribution and Support}

\cmucl{} is developed and maintained by a group of volunteers who
collaborate over the internet. Sources and binary releases for the
various supported platforms can be obtained from
\href{http://www.cons.org/cmucl/}{www.cons.org/cmucl}. These pages
describe how to download by FTP or CVS.

A number of mailing lists are available for users and developers;
please see the web site for more information. 


\section{Command Line Options}
\cindex{command line options}
\label{command-line-options}

The command line syntax and environment is described in the
\verb|lisp(1)| man page in the man/man1 directory of the distribution.
See also \verb|cmucl(1)|. Currently \cmucl{} accepts the following
switches:

\begin{Lentry}
\item[\code{-batch}] specifies batch mode, where all input is
  directed from standard-input.  An error code of 0 is returned upon
  encountering an EOF and 1 otherwise.

\item[\code{-quiet}] enters quiet mode. This implies setting the
  variables \code{*load-verbose*}, \code{*compile-verbose*},
  \code{*compile-print*}, \code{*compile-progress*},
  \code{*require-verbose*} and \code{*gc-verbose*} to NIL, and
  disables the printing of the startup banner.

\item[\code{-core}] requires an argument that should be the name of a
  core file.  Rather than using the default core file, which is searched
  in a number of places, according to the initial value of the
  \code{library:} search-list, the specified core file is loaded.  This
  switch overrides the value of the \code{CMUCLCORE} environment variable,
  if present.
  
\item[\code{-lib}] requires an argument that should be the path to the
  CMUCL library directory, which is going to be used to initialize the
  \code{library:} search-list, among other things.  This switch overrides
  the value of the \code{CMUCLLIB} environment variable, if present.

\item[\code{-dynamic-space-size}] requires an argument that should be
  the number of megabytes (1048576 bytes) that should be allocated to
  the heap.  If not specified, a platform-specific default is used.
  The actual maximum allowed heap size is platform-specific.

  Currently, this option is only available for the x86 and sparc
  platforms. 

\item[\code{-edit}] specifies to enter Hemlock.  A file to edit may be
  specified by placing the name of the file between the program name
  (usually \file{lisp}) and the first switch.
  
\item[\code{-eval}] accepts one argument which should be a Lisp form
  to evaluate during the start up sequence.  The value of the form
  will not be printed unless it is wrapped in a form that does output.
  
\item[\code{-hinit}] accepts an argument that should be the name of
  the hemlock init file to load the first time the function
  \findexed{ed} is invoked.  The default is to load
  \file{hemlock-init.\var{object-type}}, or if that does not exist,
  \file{hemlock-init.lisp} from the user's home directory.  If the
  file is not in the user's home directory, the full path must be
  specified.
  
\item[\code{-init}] accepts an argument that should be the name of an
  init file to load during the normal start up sequence.  The default
  is to load \file{init.\var{object-type}} or, if that does not exist,
  \file{init.lisp} from the user's home directory.  If neither exists,
  \cmucl tries \file{.cmucl-init.\var{object-type}} and then
  \file{.cmucl-init.lisp}.  If the file is not
  in the user's home directory, the full path must be specified.  If
  the file does not exist, \cmucl silently ignores it.
  
\item[\code{-noinit}] accepts no arguments and specifies that an init
  file should not be loaded during the normal start up sequence.
  Also, this switch suppresses the loading of a hemlock init file when
  Hemlock is started up with the \code{-edit} switch.

\item[\code{-nositeinit}] accepts no arguments and specifies that the
  site init file should not be loaded during the normal start up
  sequence. 

\item[\code{-load}] accepts an argument which should be the name of a
  file to load into Lisp before entering Lisp's read-eval-print loop.
  
\item[\code{-slave}] specifies that Lisp should start up as a
  \i{slave} Lisp and try to connect to an editor Lisp.  The name of
  the editor to connect to must be specified\dash{}to find the
  editor's name, use the \hemlock{} ``\code{Accept Slave
    Connections}'' command.  The name for the editor Lisp is of the
  form:
  \begin{example}
    \var{machine-name}\code{:}\var{socket}
  \end{example}
  where \var{machine-name} is the internet host name for the machine
  and \var{socket} is the decimal number of the socket to connect to.

\item[\code{-fpu}] specifies what fpu should be used for x87 machines.
  The possible values are ``\code{x87}'', ``\code{sse2}'', or
  ``\code{auto}'', which is the default.  By default, \cmucl will
  detect if the chip supports the SSE2 instruction set or not.  If so
  or if \code{-fpu sse2} is specified, the SSE2 core will be loaded
  that uses SSE2 for floating-point arithmetic.  If SSE2 is not
  available or if \code{-fpu x87} is given, the legacy x87 core is
  loaded.

\item[\code{--}] indicates that everything after ``\code{--}'' is not
  subject to \cmucl's command line parsing.  Everything after
  ``\code{--}'' is placed in the variable
  \code{ext:*command-line-application-arguments*}.
\end{Lentry}

For more details on the use of the \code{-edit} and \code{-slave}
switches, see the {\it Hemlock User's Manual}.

Arguments to the above switches can be specified in one of two ways:
\w{\var{switch}\code{=}\var{value}} or
\w{\var{switch}<\var{space}>\var{value}}.  For example, to start up
the saved core file mylisp.core use either of the following two
commands:

\begin{example}
   lisp -core=mylisp.core
   lisp -core mylisp.core
\end{example}


\section{Credits}

\cmucl{} was developed at the Computer Science Department of Carnegie
Mellon University. The work was a small autonomous part within the
Mach microkernel-based operating system project, and started more as a
tool development effort than a research project. The project started
out as Spice Lisp, which provided a modern Lisp implementation for use
in the CMU community. \cmucl{} has been under continual development since
the early 1980's (concurrent with the \clisp{} standardization
effort). Most of the CMU Common Lisp implementors went on to work on
the Gwydion environment for Dylan. The CMU team was lead by Scott E.
Fahlman, the \python{} compiler was written by Robert MacLachlan.

\cmucl{}'s CLOS implementation is derived from the PCL reference
implementation written at Xerox PARC:
\begin{quotation}
\noindent Copyright (c) 1985, 1986, 1987, 1988, 1989, 1990 Xerox
Corporation.\\
All rights reserved.

\vspace{1ex}
\noindent Use and copying of this software and preparation of
derivative works based upon this software are permitted.  Any
distribution of this software or derivative works must comply with all
applicable United States export control laws.

\vspace{1ex}
\noindent This software is made available AS IS, and Xerox Corporation
makes no warranty about the software, its performance or its
conformity to any specification.
\end{quotation}
Its implementation of the LOOP macro was derived from code from
Symbolics, which was derived from code written at MIT:
\begin{quotation}
\noindent Portions of LOOP are Copyright (c) 1986 by the Massachusetts
Institute of Technology.\\
All Rights Reserved.

\vspace{1ex}
\noindent Permission to use, copy, modify and distribute this software
and its documentation for any purpose and without fee is hereby granted,
provided that the M.I.T. copyright notice appear in all copies and that
both that copyright notice and this permission notice appear in
supporting documentation.  The names "M.I.T." and "Massachusetts
Institute of Technology" may not be used in advertising or publicity
pertaining to distribution of the software without specific, written
prior permission.  Notice must be given in supporting documentation that
copying distribution is by permission of M.I.T.  M.I.T. makes no
representations about the suitability of this software for any purpose.
It is provided "as is" without express or implied warranty.

\vspace{3ex}
\noindent Portions of LOOP are Copyright (c) 1989, 1990, 1991, 1992 by
Symbolics, Inc.\\
All Rights Reserved.

\vspace{1ex}
\noindent Permission to use, copy, modify and distribute this software
and its documentation for any purpose and without fee is hereby
granted, provided that the Symbolics copyright notice appear in all
copies and that both that copyright notice and this permission notice
appear in supporting documentation.  The name "Symbolics" may not be
used in advertising or publicity pertaining to distribution of the
software without specific, written prior permission.  Notice must be
given in supporting documentation that copying distribution is by
permission of Symbolics.  Symbolics makes no representations about the
suitability of this software for any purpose.  It is provided "as is"
without express or implied warranty.

\vspace{1ex}
\noindent Symbolics, CLOE Runtime, and Minima are trademarks, and
CLOE, Genera, and Zetalisp are registered trademarks of Symbolics,
Inc.
\end{quotation}
The CLX code is copyrighted by Texas Instruments Incorporated:
\begin{quotation}
\noindent Copyright (C) 1987 Texas Instruments Incorporated.

\vspace{1ex}
\noindent Permission is granted to any individual or institution to
use, copy, modify, and distribute this software, provided that this
complete copyright and permission notice is maintained, intact, in all
copies and supporting documentation.

\vspace{1ex}
\noindent Texas Instruments Incorporated provides this software "as
is" without express or implied warranty.
\end{quotation}

\cmucl{} was funded by DARPA under CMU's "Research on Parallel Computing"
contract. Rather than doing pure research on programming languages and
environments, the emphasis was on developing practical programming
tools. Sometimes this required new technology, but much of the work
was in creating a \clisp{} environment that incorporates
state-of-the-art features from existing systems (both Lisp and
non-Lisp). Archives of the project are available online.

The project funding stopped in 1994, so support at Carnegie Mellon
University has been discontinued. All code and documentation developed
at CMU was released into the public domain. The project continues as a
group of users and developers collaborating over the Internet. The
current and previous maintainers include:

\begin{itemize}
\item Marco Antoniotti
\item Martin Cracauer
\item Fred Gilham
\item Alex Goncharov
\item Rob MacLachlan
\item Pierre Mai
\item Eric Marsden
\item Gerd Moellman
\item Tim Moore
\item Carl Shapiro  
\item Robert Swindells
\item Raymond Toy
\item Peter Van Eynde
\item Paul Werkowski
\end{itemize}

In particular, Paul Werkowski and Douglas Crosher completed the port
for the x86 architecture for FreeBSD. Peter VanEnyde took the FreeBSD
port and created a Linux version. Other people who have contributed to
the development of \cmucl{} since 1981 are

\begin{itemize}
\item David Axmark
\item Miles Bader
\item Rick Busdiecker
\item Bill Chiles
\item Douglas Thomas Crosher
\item Casper Dik
\item Ted Dunning
\item Scott Fahlman
\item Mike Garland
\item Paul Gleichauf
\item Sean Hallgren
\item Richard Harris
\item Joerg-Cyril Hoehl
\item Chris Hoover
\item John Kolojejchick
\item Todd Kaufmann
\item Simon Leinen
\item Sandra Loosemore
\item William Lott
\item Dave McDonald
\item Tim Moore
\item Skef Wholey
\item Paul Foley
\item Helmut Eller
\item Jan Rychter
\end{itemize}

Countless others have contributed to the project by sending in bug
reports, bug fixes, and new features.

This manual is based on CMU Technical Report CMU-CS-92-161, edited by
Robert A. MacLachlan, dated July 1992. Other contributors include
Raymond Toy, Paul Werkowski and Eric Marsden. The Hierarchical
Packages chapter is based on documentation written by Franz. Inc, and
is used with permission. The remainder of the document is in the
public domain.
