\chapter{Event Dispatching with SERVE-EVENT}
\label{serve-event}

\credits{by Bill Chiles and Robert MacLachlan}


It is common to have multiple activities simultaneously operating in the same
Lisp process.  Furthermore, Lisp programmers tend to expect a flexible
development environment.  It must be possible to load and modify application
programs without requiring modifications to other running programs.  \cmucl{}
achieves this by having a central scheduling mechanism based on an
event-driven, object-oriented paradigm.

An \var{event} is some interesting happening that should cause the Lisp process
to wake up and do something.  These events include X events and activity on
Unix file descriptors.  The object-oriented mechanism is only available with
the first two, and it is optional with X events as described later in this
chapter.  In an X event, the window ID is the object capability and the X event
type is the operation code.  The Unix file descriptor input mechanism simply
consists of an association list of a handler to call when input shows up on a
particular file descriptor.


\section{Object Sets}
\label{object-sets}
\cindex{object sets}

An {\em object set} is a collection of objects that have the same implementation
for each operation.  Externally the object is represented by the object
capability and the operation is represented by the operation code.  Within
Lisp, the object is represented by an arbitrary Lisp object, and the
implementation for the operation is represented by an arbitrary Lisp function.
The object set mechanism maintains this translation from the external to the
internal representation.

\begin{defun}{system:}{make-object-set}{%
    \args{\var{name} \ampoptional{} \var{default-handler}}}
  
  This function makes a new object set.  \var{Name} is a string used
  only for purposes of identifying the object set when it is printed.
  \var{Default-handler} is the function used as a handler when an
  undefined operation occurs on an object in the set.  You can define
  operations with the \code{serve-}\var{operation} functions exported
  the \code{extensions} package for X events
  (\pxlref{x-serve-mumbles}).  Objects are added with
  \code{system:add-xwindow-object}.  Initially the object set has no
  objects and no defined operations.
\end{defun}

\begin{defun}{system:}{object-set-operation}{%
    \args{\var{object-set} \var{operation-code}}}
  
  This function returns the handler function that is the
  implementation of the operation corresponding to
  \var{operation-code} in \var{object-set}.  When set with
  \code{setf}, the setter function establishes the new handler.  The
  \code{serve-}\var{operation} functions exported from the
  \code{extensions} package for X events (\pxlref{x-serve-mumbles})
  call this on behalf of the user when announcing a new operation for
  an object set.
\end{defun}

\begin{defun}{system:}{add-xwindow-object}{%
    \args{\var{window} \var{object} \var{object-set}}}
  
  These functions add \var{port} or \var{window} to \var{object-set}.
  \var{Object} is an arbitrary Lisp object that is associated with the
  \var{port} or \var{window} capability.  \var{Window} is a CLX
  window.  When an event occurs, \code{system:serve-event} passes
  \var{object} as an argument to the handler function.
\end{defun}


\section{The SERVE-EVENT Function}

The \code{system:serve-event} function is the standard way for an application
to wait for something to happen.  For example, the Lisp system calls
\code{system:serve-event} when it wants input from X or a terminal stream.
The idea behind \code{system:serve-event} is that it knows the appropriate
action to take when any interesting event happens.  If an application calls
\code{system:serve-event} when it is idle, then any other applications with
pending events can run.  This allows several applications to run ``at the
same time'' without interference, even though there is only one thread of
control.  Note that if an application is waiting for input of any kind,
then other applications will get events.

\begin{defun}{system:}{serve-event}{\args{\ampoptional{} \var{timeout}}}
  
  This function waits for an event to happen and then dispatches to
  the correct handler function.  If specified, \var{timeout} is the
  number of seconds to wait before timing out.  A time out of zero
  seconds is legal and causes \code{system:serve-event} to poll for
  any events immediately available for processing.
  \code{system:serve-event} returns \true{} if it serviced at least
  one event, and \nil{} otherwise.  Depending on the application, when
  \code{system:serve-event} returns \true, you might want to call it
  repeatedly with a timeout of zero until it returns \nil.
  
  If input is available on any designated file descriptor, then this
  calls the appropriate handler function supplied by
  \code{system:add-fd-handler}.
  
  Since events for many different applications may arrive
  simultaneously, an application waiting for a specific event must
  loop on \code{system:serve-event} until the desired event happens.
  Since programs such as \hemlock{} call \code{system:serve-event} for
  input, applications usually do not need to call
  \code{system:serve-event} at all; \hemlock{} allows other
  application's handlers to run when it goes into an input wait.
\end{defun}

\begin{defun}{system:}{serve-all-events}{\args{\ampoptional{} \var{timeout}}}
  
  This function is similar to \code{system:serve-event}, except it
  serves all the pending events rather than just one.  It returns
  \true{} if it serviced at least one event, and \nil{} otherwise.
\end{defun}


\section{Using SERVE-EVENT with Unix File Descriptors}

Object sets are not available for use with file descriptors, as there are
only two operations possible on file descriptors: input and output.
Instead, a handler for either input or output can be registered with
\code{system:serve-event} for a specific file descriptor.  Whenever any input
shows up, or output is possible on this file descriptor, the function
associated with the handler for that descriptor is funcalled with the
descriptor as it's single argument.

\begin{defun}{system:}{add-fd-handler}{%
    \args{\var{fd} \var{direction} \var{function}}}
  
  This function installs and returns a new handler for the file
  descriptor \var{fd}.  \var{direction} can be either \kwd{input} if
  the system should invoke the handler when input is available or
  \kwd{output} if the system should invoke the handler when output is
  possible.  This returns a unique object representing the handler,
  and this is a suitable argument for \code{system:remove-fd-handler}
  \var{function} must take one argument, the file descriptor.
\end{defun}

\begin{defun}{system:}{remove-fd-handler}{\args{\var{handler}}}

  This function removes \var{handler}, that \code{add-fd-handler} must
  have previously returned.
\end{defun}

\begin{defmac}{system:}{with-fd-handler}{%
    \args{(\var{direction} \var{fd} \var{function})
      \mstar{\var{form}}}}
      
  This macro executes the supplied forms with a handler installed
  using \var{fd}, \var{direction}, and \var{function}.  See
  \code{system:add-fd-handler}.
\end{defmac}

\begin{defun}{system:}{wait-until-fd-usable}{%
    \args{\var{direction} \var{fd} \ampoptional{} \var{timeout}}}
      
  This function waits for up to \var{timeout} seconds for \var{fd} to
  become usable for \var{direction} (either \kwd{input} or
  \kwd{output}).  If \var{timeout} is \nil{} or unspecified, this
  waits forever.
\end{defun}

\begin{defun}{system:}{invalidate-descriptor}{\args{\var{fd}}}
  
  This function removes all handlers associated with \var{fd}.  This
  should only be used in drastic cases (such as I/O errors, but not
  necessarily EOF).  Normally, you should use \code{remove-fd-handler}
  to remove the specific handler.
\end{defun}




\section{Using SERVE-EVENT with the CLX Interface to X}
\label{x-serve-mumbles}

Remember from section \ref{object-sets}, an object set is a collection of
objects, CLX windows in this case, with some set of operations, event keywords,
with corresponding implementations, the same handler functions.  Since X allows
multiple display connections from a given process, you can avoid using object
sets if every window in an application or display connection behaves the same.
If a particular X application on a single display connection has windows that
want to handle certain events differently, then using object sets is a
convenient way to organize this since you need some way to map the window/event
combination to the appropriate functionality.

The following is a discussion of functions exported from the \code{extensions}
package that facilitate handling CLX events through \code{system:serve-event}.
The first two routines are useful regardless of whether you use
\code{system:serve-event}:
\begin{defun}{ext:}{open-clx-display}{%
    \args{\ampoptional{} \var{string}}}
  
  This function parses \var{string} for an X display specification
  including display and screen numbers.  \var{String} defaults to the
  following:
  \begin{example}
    (cdr (assoc :display ext:*environment-list* :test #'eq))
  \end{example}
  If any field in the display specification is missing, this signals
  an error.  \code{ext:open-clx-display} returns the CLX display and
  screen.
\end{defun}

\begin{defun}{ext:}{flush-display-events}{\args{\var{display}}}
  
  This function flushes all the events in \var{display}'s event queue
  including the current event, in case the user calls this from within
  an event handler.
\end{defun}



\subsection{Without Object Sets}

Since most applications that use CLX, can avoid the complexity of object sets,
these routines are described in a separate section.  The routines described in
the next section that use the object set mechanism are based on these
interfaces.

\begin{defun}{ext:}{enable-clx-event-handling}{%
    \args{\var{display} \var{handler}}} 
  
  This function causes \code{system:serve-event} to notice when there
  is input on \var{display}'s connection to the X11 server.  When this
  happens, \code{system:serve-event} invokes \var{handler} on
  \var{display} in a dynamic context with an error handler bound that
  flushes all events from \var{display} and returns.  By returning,
  the error handler declines to handle the error, but it will have
  cleared all events; thus, entering the debugger will not result in
  infinite errors due to streams that wait via
  \code{system:serve-event} for input.  Calling this repeatedly on the
  same \var{display} establishes \var{handler} as a new handler,
  replacing any previous one for \var{display}.
\end{defun}

\begin{defun}{ext:}{disable-clx-event-handling}{\args{\var{display}}}

  This function undoes the effect of
  \code{ext:enable-clx-event-handling}.
\end{defun}

\begin{defmac}{ext:}{with-clx-event-handling}{%
    \args{(\var{display} \var{handler}) \mstar{form}}}
  
  This macro evaluates each \var{form} in a context where
  \code{system:serve-event} invokes \var{handler} on \var{display}
  whenever there is input on \var{display}'s connection to the X
  server.  This destroys any previously established handler for
  \var{display}.
\end{defmac}


\subsection{With Object Sets}

This section discusses the use of object sets and
\code{system:serve-event} to handle CLX events.  This is necessary
when a single X application has distinct windows that want to handle
the same events in different ways.  Basically, you need some way of
asking for a given window which way you want to handle some event
because this event is handled differently depending on the window.
Object sets provide this feature.

For each CLX event-key symbol-name \i{XXX} (for example,
\var{key-press}), there is a function \code{serve-}\i{XXX} of two
arguments, an object set and a function.  The \code{serve-}\i{XXX}
function establishes the function as the handler for the \kwd{XXX}
event in the object set.  Recall from section \ref{object-sets},
\code{system:add-xwindow-object} associates some Lisp object with a
CLX window in an object set.  When \code{system:serve-event} notices
activity on a window, it calls the function given to
\code{ext:enable-clx-event-handling}.  If this function is
\code{ext:object-set-event-handler}, it calls the function given to
\code{serve-}\i{XXX}, passing the object given to
\code{system:add-xwindow-object} and the event's slots as well as a
couple other arguments described below.

To use object sets in this way:

\begin{itemize} 
\item Create an object set.
  
\item Define some operations on it using the \code{serve-}\i{XXX}
  functions.
  
\item Add an object for every window on which you receive requests.
  This can be the CLX window itself or some structure more meaningful
  to your application.
  
\item Call \code{system:serve-event} to service an X event.
\end{itemize}


\begin{defun}{ext:}{object-set-event-handler}{%
    \args{\var{display}}}
  
  This function is a suitable argument to
  \code{ext:enable-clx-event-handling}.  The actual event handlers
  defined for particular events within a given object set must take an
  argument for every slot in the appropriate event.  In addition to
  the event slots, \code{ext:object-set-event-handler} passes the
  following arguments:
  \begin{itemize}
  \item The object, as established by
    \code{system:add-xwindow-object}, on which the event occurred.
  \item event-key, see \code{xlib:event-case}.
  \item send-event-p, see \code{xlib:event-case}.
  \end{itemize}
  
  Describing any \code{ext:serve-}\var{event-key-name} function, where
  \var{event-key-name} is an event-key symbol-name (for example,
  \code{ext:serve-key-press}), indicates exactly what all the
  arguments are in their correct order.

%%  \begin{comment}
%%    \code{ext:object-set-event-handler} ignores \kwd{no-exposure}
%%    events on pixmaps, issuing a warning if one occurs.  It is only
%%    prepared to dispatch events for windows.
%%  \end{comment}
  
  When creating an object set for use with
  \code{ext:object-set-event-handler}, specify
  \code{ext:default-clx-event-handler} as the default handler for
  events in that object set.  If no default handler is specified, and
  the system invokes the default default handler, it will cause an
  error since this function takes arguments suitable for handling port
  messages.
\end{defun}


\section{A SERVE-EVENT Example}

This section contains two examples using \code{system:serve-event}.  The first
one does not use object sets, and the second, slightly more complicated one
does.

\subsection{Without Object Sets Example}

This example defines an input handler for a CLX display connection.  It only
recognizes \kwd{key-press} events.  The body of the example loops over
\code{system:serve-event} to get input.

\begin{lisp}
(in-package "SERVER-EXAMPLE")

(defun my-input-handler (display)
  (xlib:event-case (display :timeout 0)
    (:key-press (event-window code state)
     (format t "KEY-PRESSED (Window = ~D) = ~S.~%"
                  (xlib:window-id event-window)
             ;; See Hemlock Command Implementor's Manual for convenient
             ;; input mapping function.
             (ext:translate-character display code state))
      ;; Make XLIB:EVENT-CASE discard the event.
      t)))
\end{lisp}

\begin{lisp}
(defun server-example ()
  "An example of using the SYSTEM:SERVE-EVENT function and object sets to
   handle CLX events."
  (let* ((display (ext:open-clx-display))
         (screen (display-default-screen display))
         (black (screen-black-pixel screen))
         (white (screen-white-pixel screen))
         (window (create-window :parent (screen-root screen)
                                :x 0 :y 0 :width 200 :height 200
                                :background white :border black
                                :border-width 2
                                :event-mask
                                (xlib:make-event-mask :key-press))))
    ;; Wrap code in UNWIND-PROTECT, so we clean up after ourselves.
    (unwind-protect
        (progn
          ;; Enable event handling on the display.
          (ext:enable-clx-event-handling display #'my-input-handler)
          ;; Map the windows to the screen.
          (map-window window)
          ;; Make sure we send all our requests.
          (display-force-output display)
          ;; Call serve-event for 100,000 events or immediate timeouts.
          (dotimes (i 100000) (system:serve-event)))
      ;; Disable event handling on this display.
      (ext:disable-clx-event-handling display)
      ;; Get rid of the window.
      (destroy-window window)
      ;; Pick off any events the X server has already queued for our
      ;; windows, so we don't choke since SYSTEM:SERVE-EVENT is no longer
      ;; prepared to handle events for us.
      (loop
       (unless (deleting-window-drop-event *display* window)
        (return)))
      ;; Close the display.
      (xlib:close-display display))))

(defun deleting-window-drop-event (display win)
  "Check for any events on win.  If there is one, remove it from the
   event queue and return t; otherwise, return nil."
  (xlib:display-finish-output display)
  (let ((result nil))
    (xlib:process-event
     display :timeout 0
     :handler #'(lambda (&key event-window &allow-other-keys)
                  (if (eq event-window win)
                      (setf result t)
                      nil)))
    result))
\end{lisp}


\subsection{With Object Sets Example}

This example involves more work, but you get a little more for your effort.  It
defines two objects, \code{input-box} and \code{slider}, and establishes a
\kwd{key-press} handler for each object, \code{key-pressed} and
\code{slider-pressed}.  We have two object sets because we handle events on the
windows manifesting these objects differently, but the events come over the
same display connection.

\begin{lisp}
(in-package "SERVER-EXAMPLE")

(defstruct (input-box (:print-function print-input-box)
                      (:constructor make-input-box (display window)))
  "Our program knows about input-boxes, and it doesn't care how they
   are implemented."
  display        ; The CLX display on which my input-box is displayed.
  window)        ; The CLX window in which the user types.
;;;
(defun print-input-box (object stream n)
  (declare (ignore n))
  (format stream "#<Input-Box ~S>" (input-box-display object)))

(defvar *input-box-windows*
        (system:make-object-set "Input Box Windows"
                                #'ext:default-clx-event-handler))

(defun key-pressed (input-box event-key event-window root child
                    same-screen-p x y root-x root-y modifiers time
                    key-code send-event-p)
  "This is our :key-press event handler."
  (declare (ignore event-key root child same-screen-p x y
                   root-x root-y time send-event-p))
  (format t "KEY-PRESSED (Window = ~D) = ~S.~%"
          (xlib:window-id event-window)
          ;; See Hemlock Command Implementor's Manual for convenient
          ;; input mapping function.
          (ext:translate-character (input-box-display input-box)
                                     key-code modifiers)))
;;;
(ext:serve-key-press *input-box-windows* #'key-pressed)
\end{lisp}

\begin{lisp}
(defstruct (slider (:print-function print-slider)
                   (:include input-box)
                   (:constructor %make-slider
                                    (display window window-width max)))
  "Our program knows about sliders too, and these provide input values
   zero to max."
  bits-per-value  ; bits per discrete value up to max.
  max)            ; End value for slider.
;;;
(defun print-slider (object stream n)
  (declare (ignore n))
  (format stream "#<Slider ~S  0..~D>"
          (input-box-display object)
          (1- (slider-max object))))
;;;
(defun make-slider (display window max)
  (%make-slider display window
                  (truncate (xlib:drawable-width window) max)
                max))

(defvar *slider-windows*
        (system:make-object-set "Slider Windows"
                                #'ext:default-clx-event-handler))

(defun slider-pressed (slider event-key event-window root child
                       same-screen-p x y root-x root-y modifiers time
                       key-code send-event-p)
  "This is our :key-press event handler for sliders.  Probably this is
   a mouse thing, but for simplicity here we take a character typed."
  (declare (ignore event-key root child same-screen-p x y
                   root-x root-y time send-event-p))
  (format t "KEY-PRESSED (Window = ~D) = ~S  -->  ~D.~%"
          (xlib:window-id event-window)
          ;; See Hemlock Command Implementor's Manual for convenient
          ;; input mapping function.
          (ext:translate-character (input-box-display slider)
                                     key-code modifiers)
          (truncate x (slider-bits-per-value slider))))
;;;
(ext:serve-key-press *slider-windows* #'slider-pressed)
\end{lisp}

\begin{lisp}
(defun server-example ()
  "An example of using the SYSTEM:SERVE-EVENT function and object sets to
   handle CLX events."
  (let* ((display (ext:open-clx-display))
         (screen (display-default-screen display))
         (black (screen-black-pixel screen))
         (white (screen-white-pixel screen))
         (iwindow (create-window :parent (screen-root screen)
                                 :x 0 :y 0 :width 200 :height 200
                                 :background white :border black
                                 :border-width 2
                                 :event-mask
                                 (xlib:make-event-mask :key-press)))
         (swindow (create-window :parent (screen-root screen)
                                 :x 0 :y 300 :width 200 :height 50
                                 :background white :border black
                                 :border-width 2
                                 :event-mask
                                 (xlib:make-event-mask :key-press)))
         (input-box (make-input-box display iwindow))
         (slider (make-slider display swindow 15)))
    ;; Wrap code in UNWIND-PROTECT, so we clean up after ourselves.
    (unwind-protect
        (progn
          ;; Enable event handling on the display.
          (ext:enable-clx-event-handling display
                                         #'ext:object-set-event-handler)
          ;; Add the windows to the appropriate object sets.
          (system:add-xwindow-object iwindow input-box
                                       *input-box-windows*)
          (system:add-xwindow-object swindow slider
                                       *slider-windows*)
          ;; Map the windows to the screen.
          (map-window iwindow)
          (map-window swindow)
          ;; Make sure we send all our requests.
          (display-force-output display)
          ;; Call server for 100,000 events or immediate timeouts.
          (dotimes (i 100000) (system:serve-event)))
      ;; Disable event handling on this display.
      (ext:disable-clx-event-handling display)
      (delete-window iwindow display)
      (delete-window swindow display)
      ;; Close the display.
      (xlib:close-display display))))
\end{lisp}

\begin{lisp}
(defun delete-window (window display)
  ;; Remove the windows from the object sets before destroying them.
  (system:remove-xwindow-object window)
  ;; Destroy the window.
  (destroy-window window)
  ;; Pick off any events the X server has already queued for our
  ;; windows, so we don't choke since SYSTEM:SERVE-EVENT is no longer
  ;; prepared to handle events for us.
  (loop
   (unless (deleting-window-drop-event display window)
     (return))))

(defun deleting-window-drop-event (display win)
  "Check for any events on win.  If there is one, remove it from the
   event queue and return t; otherwise, return nil."
  (xlib:display-finish-output display)
  (let ((result nil))
    (xlib:process-event
     display :timeout 0
     :handler #'(lambda (&key event-window &allow-other-keys)
                  (if (eq event-window win)
                      (setf result t)
                      nil)))
    result))
\end{lisp}
