%%\documentstyle[cmu-titlepage]{report} % -*- Dictionary: design -*-
%\documentstyle{report} % -*- Dictionary: design -*-

\documentclass{report}
\usepackage{ifthen}
\usepackage{calc}
\usepackage{palatino}
\usepackage[hyperindex=false,colorlinks=false,urlcolor=blue]{hyperref}

% define a new conditional statement which allows us to include
% stuff conditionally when compiling to PDF. 
\newif\ifpdf
\ifx\pdfoutput\undefined
   \pdffalse
\else
   \pdfoutput=1
   \pdftrue
\fi



\title{Design of CMU Common Lisp}
\date{January 15, 2003}
\author{Robert A. MacLachlan (ed)}

\ifpdf
\pdfinfo{
/Author (Robert A. MacLachlan, ed)
/Title (Design of CMU Common Lisp)
}
% Add section numbers to the bookmarks, and open 2 levels by default.
\hypersetup{bookmarksnumbered=true,
            bookmarksopen=true,
            bookmarksopenlevel=2}
\fi
%%\trnumber{CMU-CS-91-???}

%% This code taken from the LaTeX companion.  It's meant as a
%% replacement for the description environment.  We want one that
%% prints description items in a fixed size box and puts the
%% description itself on the same line or the next depending on the
%% size of the item.
\newcommand{\entrylabel}[1]{\mbox{#1}\hfil}
\newenvironment{entry}{%
  \begin{list}{}%
    {\renewcommand{\makelabel}{\entrylabel}%
      \setlength{\labelwidth}{45pt}%
      \setlength{\leftmargin}{\labelwidth+\labelsep}}}%
  {\end{list}}

\newlength{\Mylen}
\newcommand{\Lentrylabel}[1]{%
  \settowidth{\Mylen}{#1}%
  \ifthenelse{\lengthtest{\Mylen > \labelwidth}}%
  {\parbox[b]{\labelwidth}%  term > labelwidth
    {\makebox[0pt][l]{#1}\\}}%
  {#1}%
  \hfil\relax}
\newenvironment{Lentry}{%
  \renewcommand{\entrylabel}{\Lentrylabel}
  \begin{entry}}%
  {\end{entry}}

\setcounter{tocdepth}{2}
\setcounter{secnumdepth}{3}
\def\textfraction{.1}
\def\bottomfraction{.9}         % was .3
\def\topfraction{.9}

\newcommand{\code}[1]{\textnormal{{\sffamily #1}}}
%% Some common abbreviations
\newcommand{\cmucl}{\textsc{cmucl}}

%% Set up margins
\setlength{\oddsidemargin}{-10pt}
\setlength{\evensidemargin}{-10pt}
\setlength{\topmargin}{-40pt}
\setlength{\headheight}{12pt}
\setlength{\headsep}{25pt}
\setlength{\footskip}{30pt}
\setlength{\textheight}{9.25in}
\setlength{\textwidth}{6.75in}
\setlength{\columnsep}{0.375in}
\setlength{\columnseprule}{0pt}


\begin{document}
\maketitle
\abstract{This report documents internal details of the CMU Common Lisp
compiler and run-time system.  CMU Common Lisp is a public domain
implementation of Common Lisp that runs on various Unix workstations.
This document is a work in progress: neither the contents nor the
presentation are completed. Nevertheless, it provides some useful
background information, in particular regarding the \cmucl{} compiler.}

\tableofcontents
\part{System Architecture}% -*- Dictionary: int:design -*-

\chapter{Package and File Structure}

\section{RCS and build areas}

The CMU CL sources are maintained using RCS in a hierarchical directory
structure which supports:
\begin{itemize}
\item    shared RCS config file across a build area, 

\item    frozen sources for multiple releases, and 

\item    separate system build areas for different architectures.
\end{itemize}

Since this organization maintains multiple copies of the source, it is somewhat
space intensive.  But it is easy to delete and later restore a copy of the
source using RCS snapshots.

There are three major subtrees of the root \verb|/afs/cs/project/clisp|:
\begin{description}
\item[rcs] holds the RCS source (suffix \verb|,v|) files.

\item[src] holds ``checked out'' (but not locked) versions of the source files,
and is subdivided by release.  Each release directory in the source tree has a
symbolic link named ``{\tt RCS}'' which points to the RCS subdirectory of the
corresponding directory in the ``{\tt rcs} tree.  At top-level in a source tree
is the ``{\tt RCSconfig}'' file for that area.  All subdirectories also have a
symbolic link to this RCSconfig file, allowing the configuration for an area to
be easily changed.

\item[build] compiled object files are placed in this tree, which is subdivided
by machine type and version.  The CMU CL search-list mechanism is used to allow
the source files to be located in a different tree than the object files.  C
programs are compiled by using the \verb|tools/dupsrcs| command to make
symbolic links to the corresponding source tree.
\end{description}

On order to modify an file in RCS, it must be checked out with a lock to
produce a writable working file.  Each programmer checks out files into a
personal ``play area'' subtree of \verb|clisp/hackers|.  These tree duplicate
the structure of source trees, but are normally empty except for files actively
being worked on.

See \verb|/afs/cs/project/clisp/pmax_mach/alpha/tools/| for
various tools we use for RCS hacking:
\begin{description}
\item[rcs.lisp] Hemlock (editor) commands for RCS file manipulation

\item[rcsupdate.c] Program to check out all files in a tree that have been
modified since last checkout.

\item[updates] Shell script to produce a single listing of all RCS log
 entries in a tree since a date.

\item[snapshot-update.lisp] Lisp program to generate a shell script which
generates a listing of updates since a particular RCS snapshot ({\tt RCSSNAP})
file was created.
\end{description}

You can easily operate on all RCS files in a subtree using:
\begin{verbatim}
find . -follow -name '*,v' -exec <some command> {} \;
\end{verbatim}

\subsection{Configuration Management}

config files are useful, especially in combinarion with ``{\tt snapshot}''.  You
can shapshot any particular version, giving an RCSconfig that designates that
configuration.  You can also use config files to specify the system as of a
particular date.  For example:
\begin{verbatim}
<3-jan-91
\end{verbatim}
in the the config file will cause the version as of that 3-jan-91 to be checked
out, instead of the latest version.

\subsection{RCS Branches}

Branches and named revisions are used together to allow multiple paths of
development to be supported.  Each separate development has a branch, and each
branch has a name.  This project uses branches in two somewhat different cases
of divergent development:
\begin{itemize}
\item For systems that we have imported from the outside, we generally assign a
``{\tt cmu}'' branch for our local modifications.  When a new release comes
along, we check it in on the trunk, and then merge our branch back in.

\item For the early development and debugging of major system changes, where
the development and debugging is expected to take long enough that we wouldn't
want the trunk to be in an inconsistent state for that long.
\end{itemize}

\section{Releases}

We name releases according to the normal alpha, beta, default convention.
Alpha releases are frequent, intended primarily for internal use, and are thus
not subject to as high high documentation and configuration management
standards.  Alpha releases are designated by the date on which the system was
built; the alpha releases for different systems may not be in exact
correspondence, since they are built at different times.

Beta and default releases are always based on a snapshot, ensuring that all
systems are based on the same sources.  A release name is an integer and a
letter, like ``15d''.  The integer is the name of the source tree which the
system was built from, and the letter represents the release from that tree:
``a'' is the first release, etc.  Generally the numeric part increases when
there are major system changes, whereas changes in the letter represent
bug-fixes and minor enhancements.

\section{Source Tree Structure}

A source tree (and the master ``{\tt rcs}'' tree) has subdirectories for each
major subsystem:
\begin{description}
\item[{\tt assembly/}] Holds the CMU CL source-file assembler, and has machine
specific subdirectories holding assembly code for that architecture.

\item[{\tt clx/}] The CLX interface to the X11 window system.

\item[{\tt code/}] The Lisp code for the runtime system and standard CL
utilities.

\item[{\tt compiler/}] The Python compiler.  Has architecture-specific
subdirectories which hold backends for different machines.  The {\tt generic}
subdirectory holds code that is shared across most backends.

\item[{\tt hemlock/}] The Hemlock editor.

\item[{\tt ldb/}] The C runtime system code and low-level ``Lisp DeBugger''.

\item[{\tt pcl/}] CMU version of the PCL implementation of CLOS.

\item[{\tt tools/}] System building command files and source management tools.
\end{description}


\section{Package structure}

Goals: with the single exception of LISP, we want to be able to export from the
package that the code lives in.

\begin{description}
\item[Mach, CLX...] --- These Implementation-dependent system-interface
packages provide direct access to specific features available in the operating
system environment, but hide details of how OS communication is done.

\item[system] contains code that must know about the operating system
environment: I/O, etc.  Hides the operating system environment.  Provides OS
interface extensions such as {\tt print-directory}, etc.

\item[kernel] hides state and types used for system integration: package
system, error system, streams (?), reader, printer.  Also, hides the VM, in
that we don't export anything that reveals the VM interface.  Contains code
that needs to use the VM and SYSTEM interface, but is independent of OS and VM
details.  This code shouldn't need to be changed in any port of CMU CL, but
won't work when plopped into an arbitrary CL.  Uses SYSTEM, VM, EXTENSIONS.  We
export "hidden" symbols related to implementation of CL: setf-inverses,
possibly some global variables.

The boundary between KERNEL and VM is fuzzy, but this fuzziness reflects the
fuzziness in the definition of the VM.  We can make the VM large, and bring
everything inside, or we make make it small.  Obviously, we want the VM to be
as small as possible, subject to efficiency constraints.  Pretty much all of
the code in KERNEL could be put in VM.  The issue is more what VM hides from
KERNEL: VM knows about everything.

\item[lisp]  Originally, this package had all the system code in it.  The
current ideal is that this package should have {\it no} code in it, and only
exist to export the standard interface.  Note that the name has been changed by
x3j13 to common-lisp.

\item[extensions] contains code that any random user could have written: list
operations, syntactic sugar macros.  Uses only LISP, so code in EXTENSIONS is
pure CL.  Exports everything defined within that is useful elsewhere.  This
package doesn't hide much, so it is relatively safe for users to use
EXTENSIONS, since they aren't getting anything they couldn't have written
themselves.  Contrast this to KERNEL, which exports additional operations on
CL's primitive data structures: PACKAGE-INTERNAL-SYMBOL-COUNT, etc.  Although
some of the functionality exported from KERNEL could have been defined in CL,
the kernel implementation is much more efficient because it knows about
implementation internals.  Currently this package contains only extensions to
CL, but in the ideal scheme of things, it should contain the implementations of
all CL functions that are in KERNEL (the library.)

\item[VM] hides information about the hardware and data structure
representations.  Contains all code that knows about this sort of thing: parts
of the compiler, GC, etc.  The bulk of the code is the compiler back-end.
Exports useful things that are meaningful across all implementations, such as
operations for examining compiled functions, system constants.  Uses COMPILER
and whatever else it wants.  Actually, there are different {\it machine}{\tt
-VM} packages for each target implementation.  VM is a nickname for whatever
implementation we are currently targeting for.


\item[compiler] hides the algorithms used to map Lisp semantics onto the
operations supplied by the VM.  Exports the mechanisms used for defining the
VM.  All the VM-independent code in the compiler, partially hiding the compiler
intermediate representations.  Uses KERNEL.

\item[eval] holds code that does direct execution of the compiler's ICR.  Uses
KERNEL, COMPILER.  Exports debugger interface to interpreted code.

\item[debug-internals] presents a reasonable, unified interface to
manipulation of the state of both compiled and interpreted code.  (could be in
KERNEL) Uses VM, INTERPRETER, EVAL, KERNEL.

\item[debug] holds the standard debugger, and exports the debugger 
\end{description}

\chapter{System Building}

It's actually rather easy to build a CMU CL core with exactly what you want in
it.  But to do this you need two things: the source and a working CMU CL.

Basically, you use the working copy of CMU CL to compile the sources,
then run a process call ``genesis'' which builds a ``kernel'' core.
You then load whatever you want into this kernel core, and save it.

In the \verb|tools/| directory in the sources there are several files that
compile everything, and build cores, etc.  The first step is to compile the C
startup code.

{\bf Note:} {\it the various scripts mentioned below have hard-wired paths in
them set up for our directory layout here at CMU.  Anyone anywhere else will
have to edit them before they will work.}

\section{Compiling the C Startup Code}

There is a circular dependancy between ldb/lisp.h and ldb/ldb.map that causes
bootstrapping problems.  To the easiest way to get around this problem is to
make a fake ldb.map file that has nothing in it by a version number:
\begin{verbatim}
	% echo "Map file for ldb version 0" > ldb.map
\end{verbatim}
and then run genesis with NIL for the list of files:
\begin{verbatim}
	* (load ".../compiler/generic/genesis")
	* (lisp::genesis nil ".../ldb/ldb.map" "/dev/null"
		".../ldb/lisp.map" ".../ldb/lisp.h")
\end{verbatim}
It will generate
a whole bunch of warnings about things being undefined, but ignore
that, because it will also generate a correct lisp.h.  You can then
compile ldb producing a correct ldb.map:
\begin{verbatim}
	% make
\end{verbatim}
and the use \verb|tools/do-worldbuild| and \verb|tools/mk-lisp| to build
\verb|kernel.core| and \verb|lisp.core| (see section \ref[building-cores].)

\section{Compiling the Lisp Code}

The \verb|tools| directory contains various lisp and C-shell utilities for
building CMU CL:
\begin{description}
\item[compile-all*] Will compile lisp files and build a kernel core.  It has
numerous command-line options to control what to compile and how.  Try -help to
see a description.  It runs a separate Lisp process to compile each
subsystem.  Error output is generated in files with ``{\tt .log}'' extension in
the root of the build area.

\item[setup.lisp] Some lisp utilities used for compiling changed files in batch
mode and collecting the error output Sort of a crude defsystem.  Loads into the
``user'' package.  See {\tt with-compiler-log-file} and {\tt comf}.

\item[{\it foo}com.lisp] Each system has a ``\verb|.lisp|'' file in
\verb|tools/| which compiles that system.
\end{description}

\section{Building Core Images}
\label{building-cores}
Both the kernel and final core build are normally done using shell script
drivers:
\begin{description}
\item[do-worldbuild*] Builds a kernel core for the current machine.  The
version to build is indicated by an optional argument, which defaults to
``alpha''.  The \verb|kernel.core| file is written either in the \verb|ldb/|
directory in the build area, or in \verb|/usr/tmp/|.  The directory which
already contains \verb|kernel.core| is chosen.  You can create a dummy version
with e.g. ``touch'' to select the initial build location.

\item[mk-lisp*] Builds a full core, with conditional loading of subsystems.
The version is the first argument, which defaults to ``alpha''.  Any additional
arguments are added to the \verb|*features*| list, which controls system
loading (among other things.)  The \verb|lisp.core| file is written in the
current working directory.
\end{description}

These scripts load Lisp command files.  When \verb|tools/worldbuild.lisp| is
loaded, it calls genesis with the correct arguments to build a kernel core.
Similarly, \verb|worldload.lisp|
builds a full core.  Adding certain symbols to \verb|*features*| before
loading worldload.lisp suppresses loading of different parts of the
system.  These symbols are:
\begin{description}
\item[:no-compiler] don't load the compiler.
\item[:no-clx] don't load CLX.
\item[:no-hemlock] don't load hemlock.
\item[:no-pcl] don't load PCL.
\end{description}

Note: if you don't load the compiler, you can't (successfully) load the
pretty-printer or pcl.  And if you compiled hemlock with CLX loaded, you can't
load it without CLX also being loaded.

\chapter{The Compiler}

\section{Compiler Introduction}

This chapter contains information about the compiler that every \cmucl{} user
should be familiar with.  Chapter \ref{advanced-compiler} goes into greater
depth, describing ways to use more advanced features.

The \cmucl{} compiler (also known as \python{}, not to be confused
with the programming language of the same name) has many features
that are seldom or never supported by conventional \llisp{}
compilers:

\begin{itemize} 
\item Source level debugging of compiled code (see chapter
  \ref{debugger}.)
  
\item Type error compiler warnings for type errors detectable at
  compile time.
  
\item Compiler error messages that provide a good indication of where
  the error appeared in the source.
  
\item Full run-time checking of all potential type errors, with
  optimization of type checks to minimize the cost.
  
\item Scheme-like features such as proper tail recursion and extensive
  source-level optimization.
  
\item Advanced tuning and optimization features such as comprehensive
  efficiency notes, flow analysis, and untagged number representations
  (see chapter \ref{advanced-compiler}.)
\end{itemize}


\section{Calling the Compiler}
\cindex{compiling}

Functions may be compiled using \code{compile}, \code{compile-file}, or 
\code{compile-from-stream}.  

\begin{defun}{}{compile}{ \args{\var{name} \ampoptional{} \var{definition}}}
  
  This function compiles the function whose name is \var{name}.  If
  \var{name} is \false, the compiled function object is returned.  If
  \var{definition} is supplied, it should be a lambda expression that
  is to be compiled and then placed in the function cell of
  \var{name}.  As per the proposed X3J13 cleanup
  ``compile-argument-problems'', \var{definition} may also be an
  interpreted function.
  
  The return values are as per the proposed X3J13 cleanup
  ``compiler-diagnostics''.  The first value is the function name or
  function object.  The second value is \false{} if no compiler
  diagnostics were issued, and \true{} otherwise.  The third value is
  \false{} if no compiler diagnostics other than style warnings were
  issued.  A non-\false{} value indicates that there were ``serious''
  compiler diagnostics issued, or that other conditions of type
  \tindexed{error} or \tindexed{warning} (but not
  \tindexed{style-warning}) were signaled during compilation.
\end{defun}


\begin{defun}{}{compile-file}{
    \args{\var{input-pathname}
      \keys{\kwd{output-file} \kwd{error-file} \kwd{trace-file}}
      \morekeys{\kwd{error-output} \kwd{verbose} \kwd{print} \kwd{progress}}
      \yetmorekeys{\kwd{load} \kwd{block-compile} \kwd{entry-points}}
      \yetmorekeys{\kwd{byte-compile}}}}
  
  The \cmucl{} \code{compile-file} is extended through the addition of
  several new keywords and an additional interpretation of
  \var{input-pathname}:
  \begin{Lentry}
    
  \item[\var{input-pathname}] If this argument is a list of input
    files, rather than a single input pathname, then all the source
    files are compiled into a single object file.  In this case, the
    name of the first file is used to determine the default output
    file names.  This is especially useful in combination with
    \var{block-compile}.
    
  \item[\kwd{output-file}] This argument specifies the name of the
    output file.  \true{} gives the default name, \false{} suppresses
    the output file.
    
  \item[\kwd{error-file}] A listing of all the error output is
    directed to this file.  If there are no errors, then no error file
    is produced (and any existing error file is deleted.)  \true{}
    gives \w{"\var{name}\code{.err}"} (the default), and \false{}
    suppresses the output file.
    
  \item[\kwd{error-output}] If \true{} (the default), then error
    output is sent to \code{*error-output*}.  If a stream, then output
    is sent to that stream instead.  If \false, then error output is
    suppressed.  Note that this error output is in addition to (but
    the same as) the output placed in the \var{error-file}.
    
  \item[\kwd{verbose}] If \true{} (the default), then the compiler
    prints to error output at the start and end of compilation of each
    file.  See \varref{compile-verbose}.
    
  \item[\kwd{print}] If \true{} (the default), then the compiler
    prints to error output when each function is compiled.  See
    \varref{compile-print}.
    
  \item[\kwd{progress}] If \true{} (default \false{}), then the
    compiler prints to error output progress information about the
    phases of compilation of each function.  This is a \cmucl{} extension
    that is useful mainly in large block compilations.  See
    \varref{compile-progress}.
    
  \item[\kwd{trace-file}] If \true{}, several of the intermediate
    representations (including annotated assembly code) are dumped out
    to this file.  \true{} gives \w{"\var{name}\code{.trace}"}.  Trace
    output is off by default.  \xlref{trace-files}.
    
  \item[\kwd{load}] If \true{}, load the resulting output file.
    
  \item[\kwd{block-compile}] Controls the compile-time resolution of
    function calls.  By default, only self-recursive calls are
    resolved, unless an \code{ext:block-start} declaration appears in
    the source file.  \xlref{compile-file-block}.
    
  \item[\kwd{entry-points}] If non-null, then this is a list of the
    names of all functions in the file that should have global
    definitions installed (because they are referenced in other
    files.)  \xlref{compile-file-block}.
    
  \item[\kwd{byte-compile}] If \true{}, compiling to a compact
    interpreted byte code is enabled.  Possible values are \true{},
    \false{}, and \kwd{maybe} (the default.)  See
    \varref{byte-compile-default} and \xlref{byte-compile}.
  \end{Lentry}
  
  The return values are as per the proposed X3J13 cleanup
  ``compiler-diagnostics''.  The first value from \code{compile-file}
  is the truename of the output file, or \false{} if the file could
  not be created.  The interpretation of the second and third values
  is described above for \code{compile}.
\end{defun}

\begin{defvar}{}{compile-verbose}
  \defvarx{compile-print}
  \defvarx{compile-progress}
  
  These variables determine the default values for the \kwd{verbose},
  \kwd{print} and \kwd{progress} arguments to \code{compile-file}.
\end{defvar}

\begin{defun}{extensions:}{compile-from-stream}{%
    \args{\var{input-stream}
      \keys{\kwd{error-stream}}
      \morekeys{\kwd{trace-stream}}
      \yetmorekeys{\kwd{block-compile} \kwd{entry-points}}
      \yetmorekeys{\kwd{byte-compile}}}}
  
  This function is similar to \code{compile-file}, but it takes all
  its arguments as streams.  It reads \llisp{} code from
  \var{input-stream} until end of file is reached, compiling into the
  current environment.  This function returns the same two values as
  the last two values of \code{compile}.  No output files are
  produced.
\end{defun}


\section{Compilation Units}
\cpsubindex{compilation}{units}

\cmucl{} supports the \code{with-compilation-unit} macro added to the
language by the X3J13 ``with-compilation-unit'' compiler cleanup
issue.  This provides a mechanism for eliminating spurious undefined
warnings when there are forward references across files, and also
provides a standard way to access compiler extensions.

\begin{defmac}{}{with-compilation-unit}{%
    \args{(\mstar{\var{key} \var{value}}) \mstar{\var{form}}}}
  
  This macro evaluates the \var{forms} in an environment that causes
  warnings for undefined variables, functions and types to be delayed
  until all the forms have been evaluated.  Each keyword \var{value}
  is an evaluated form.  These keyword options are recognized:
  \begin{Lentry}
  
  \item[\kwd{override}] If uses of \code{with-compilation-unit} are
    dynamically nested, the outermost use will take precedence,
    suppressing printing of undefined warnings by inner uses.
    However, when the \code{override} option is true this shadowing is
    inhibited; an inner use will print summary warnings for the
    compilations within the inner scope.
  
  \item[\kwd{optimize}] This is a \cmucl{} extension that specifies of the
    ``global'' compilation policy for the dynamic extent of the body.
    The argument should evaluate to an \code{optimize} declare form,
    like:
    \begin{lisp}
      (optimize (speed 3) (safety 0))
    \end{lisp}
    \xlref{optimize-declaration}
  
  \item[\kwd{optimize-interface}] Similar to \kwd{optimize}, but
    specifies the compilation policy for function interfaces (argument
    count and type checking) for the dynamic extent of the body.
    \xlref{optimize-interface-declaration}.
  
  \item[\kwd{context-declarations}] This is a \cmucl{} extension that
    pattern-matches on function names, automatically splicing in any
    appropriate declarations at the head of the function definition.
    \xlref{context-declarations}.
  \end{Lentry}
\end{defmac}


\subsection{Undefined Warnings}

\cindex{undefined warnings}
Warnings about undefined variables, functions and types are delayed until the
end of the current compilation unit.  The compiler entry functions
(\code{compile}, etc.) implicitly use \code{with-compilation-unit}, so undefined
warnings will be printed at the end of the compilation unless there is an
enclosing \code{with-compilation-unit}.  In order the gain the benefit of this
mechanism, you should wrap a single \code{with-compilation-unit} around the calls
to \code{compile-file}, i.e.:
\begin{lisp}
(with-compilation-unit ()
  (compile-file "file1")
  (compile-file "file2")
  ...)
\end{lisp}

Unlike for functions and types, undefined warnings for variables are
not suppressed when a definition (e.g. \code{defvar}) appears after
the reference (but in the same compilation unit.)  This is because
doing special declarations out of order just doesn't
work\dash{}although early references will be compiled as special,
bindings will be done lexically.

Undefined warnings are printed with full source context
(\pxlref{error-messages}), which tremendously simplifies the problem
of finding undefined references that resulted from macroexpansion.
After printing detailed information about the undefined uses of each
name, \code{with-compilation-unit} also prints summary listings of the
names of all the undefined functions, types and variables.

\begin{defvar}{}{undefined-warning-limit}
  
  This variable controls the number of undefined warnings for each
  distinct name that are printed with full source context when the
  compilation unit ends.  If there are more undefined references than
  this, then they are condensed into a single warning:
  \begin{example}
    Warning: \var{count} more uses of undefined function \var{name}.
  \end{example}
  When the value is \code{0}, then the undefined warnings are not
  broken down by name at all: only the summary listing of undefined
  names is printed.
\end{defvar}


\section{Interpreting Error Messages}
\label{error-messages}
\cpsubindex{error messages}{compiler}
\cindex{compiler error messages}

One of \python{}'s unique features is the level of source location
information it provides in error messages.  The error messages contain
a lot of detail in a terse format, to they may be confusing at first.
Error messages will be illustrated using this example program:
\begin{lisp}
(defmacro zoq (x)
  `(roq (ploq (+ ,x 3))))

(defun foo (y)
  (declare (symbol y))
  (zoq y))
\end{lisp}
The main problem with this program is that it is trying to add \code{3} to a
symbol.  Note also that the functions \code{roq} and \code{ploq} aren't defined
anywhere.


\subsection{The Parts of the Error Message}

The compiler will produce this warning:

\begin{example}
File: /usr/me/stuff.lisp
In: DEFUN FOO
  (ZOQ Y)
--> ROQ PLOQ + 
==>
  Y
Warning: Result is a SYMBOL, not a NUMBER.
\end{example}

In this example we see each of the six possible parts of a compiler error
message:

\begin{Lentry} 
\item[\w{\code{File: /usr/me/stuff.lisp}}] This is the \var{file} that
  the compiler read the relevant code from.  The file name is
  displayed because it may not be immediately obvious when there is an
  error during compilation of a large system, especially when
  \code{with-compilation-unit} is used to delay undefined warnings.
  
\item[\w{\code{In: DEFUN FOO}}] This is the \var{definition} or
  top-level form responsible for the error.  It is obtained by taking
  the first two elements of the enclosing form whose first element is
  a symbol beginning with ``\code{DEF}''.  If there is no enclosing
  \w{\var{def}mumble}, then the outermost form is used.  If there are
  multiple \w{\var{def}mumbles}, then they are all printed from the
  out in, separated by \code{$=>$}'s.  In this example, the problem
  was in the \code{defun} for \code{foo}.
  
\item[\w{\code{(ZOQ Y)}}] This is the {\em original source} form
  responsible for the error.  Original source means that the form
  directly appeared in the original input to the compiler, i.e. in the
  lambda passed to \code{compile} or the top-level form read from the
  source file.  In this example, the expansion of the \code{zoq} macro
  was responsible for the error.
  
\item[\w{\code{--$>$ ROQ PLOQ +}} ] This is the {\em processing path}
  that the compiler used to produce the errorful code.  The processing
  path is a representation of the evaluated forms enclosing the actual
  source that the compiler encountered when processing the original
  source.  The path is the first element of each form, or the form
  itself if the form is not a list.  These forms result from the
  expansion of macros or source-to-source transformation done by the
  compiler.  In this example, the enclosing evaluated forms are the
  calls to \code{roq}, \code{ploq} and \code{+}.  These calls resulted
  from the expansion of the \code{zoq} macro.
  
\item[\code{==$>$ Y}] This is the {\em actual source} responsible for
  the error.  If the actual source appears in the explanation, then we
  print the next enclosing evaluated form, instead of printing the
  actual source twice.  (This is the form that would otherwise have
  been the last form of the processing path.)  In this example, the
  problem is with the evaluation of the reference to the variable
  \code{y}.
  
\item[\w{\code{Warning: Result is a SYMBOL, not a NUMBER.}}]  This is
  the \var{explanation} the problem.  In this example, the problem is
  that \code{y} evaluates to a \code{symbol}, but is in a context
  where a number is required (the argument to \code{+}).
\end{Lentry}

Note that each part of the error message is distinctively marked:

\begin{itemize} 
\item \code{File:} and \code{In:} mark the file and definition,
  respectively.
  
\item The original source is an indented form with no prefix.
  
\item Each line of the processing path is prefixed with \code{--$>$}.
  
\item The actual source form is indented like the original source, but
  is marked by a preceding \code{==$>$} line.  This is like the
  ``macroexpands to'' notation used in \cltl.
  
\item The explanation is prefixed with the error severity
  (\pxlref{error-severity}), either \code{Error:}, \code{Warning:}, or
  \code{Note:}.
\end{itemize}


Each part of the error message is more specific than the preceding
one.  If consecutive error messages are for nearby locations, then the
front part of the error messages would be the same.  In this case, the
compiler omits as much of the second message as in common with the
first.  For example:

\begin{example}
File: /usr/me/stuff.lisp
In: DEFUN FOO
  (ZOQ Y)
--> ROQ 
==>
  (PLOQ (+ Y 3))
Warning: Undefined function: PLOQ

==>
  (ROQ (PLOQ (+ Y 3)))
Warning: Undefined function: ROQ
\end{example}

In this example, the file, definition and original source are
identical for the two messages, so the compiler omits them in the
second message.  If consecutive messages are entirely identical, then
the compiler prints only the first message, followed by:

\begin{example}
[Last message occurs \var{repeats} times]
\end{example}

where \var{repeats} is the number of times the message was given.

If the source was not from a file, then no file line is printed.  If
the actual source is the same as the original source, then the
processing path and actual source will be omitted.  If no forms
intervene between the original source and the actual source, then the
processing path will also be omitted.


\subsection{The Original and Actual Source}
\cindex{original source}
\cindex{actual source}

The {\em original source} displayed will almost always be a list.  If the actual
source for an error message is a symbol, the original source will be the
immediately enclosing evaluated list form.  So even if the offending symbol
does appear in the original source, the compiler will print the enclosing list
and then print the symbol as the actual source (as though the symbol were
introduced by a macro.)

When the {\em actual source} is displayed (and is not a symbol), it will always
be code that resulted from the expansion of a macro or a source-to-source
compiler optimization.  This is code that did not appear in the original
source program; it was introduced by the compiler.

Keep in mind that when the compiler displays a source form in an error message,
it always displays the most specific (innermost) responsible form.  For
example, compiling this function:

\begin{lisp}
(defun bar (x)
  (let (a)
    (declare (fixnum a))
    (setq a (foo x))
    a))
\end{lisp}

gives this error message:

\begin{example}
In: DEFUN BAR
  (LET (A) (DECLARE (FIXNUM A)) (SETQ A (FOO X)) A)
Warning: The binding of A is not a FIXNUM:
  NIL
\end{example}

This error message is not saying ``there's a problem somewhere in this
\code{let}''\dash{}it is saying that there is a problem with the
\code{let} itself.  In this example, the problem is that \code{a}'s
\false{} initial value is not a \code{fixnum}.


\subsection{The Processing Path}
\cindex{processing path}
\cindex{macroexpansion}
\cindex{source-to-source transformation}

The processing path is mainly useful for debugging macros, so if you don't
write macros, you can ignore the processing path.  Consider this example:

\begin{lisp}
(defun foo (n)
  (dotimes (i n *undefined*)))
\end{lisp}

Compiling results in this error message:

\begin{example}
In: DEFUN FOO
  (DOTIMES (I N *UNDEFINED*))
--> DO BLOCK LET TAGBODY RETURN-FROM 
==>
  (PROGN *UNDEFINED*)
Warning: Undefined variable: *UNDEFINED*
\end{example}

Note that \code{do} appears in the processing path.  This is because \code{dotimes}
expands into:

\begin{lisp}
(do ((i 0 (1+ i)) (#:g1 n))
    ((>= i #:g1) *undefined*)
  (declare (type unsigned-byte i)))
\end{lisp}

The rest of the processing path results from the expansion of \code{do}:

\begin{lisp}
(block nil
  (let ((i 0) (#:g1 n))
    (declare (type unsigned-byte i))
    (tagbody (go #:g3)
     #:g2    (psetq i (1+ i))
     #:g3    (unless (>= i #:g1) (go #:g2))
             (return-from nil (progn *undefined*)))))
\end{lisp}

In this example, the compiler descended into the \code{block},
\code{let}, \code{tagbody} and \code{return-from} to reach the
\code{progn} printed as the actual source.  This is a place where the
``actual source appears in explanation'' rule was applied.  The
innermost actual source form was the symbol \code{*undefined*} itself,
but that also appeared in the explanation, so the compiler backed out
one level.


\subsection{Error Severity}
\label{error-severity}
\cindex{severity of compiler errors}
\cindex{compiler error severity}

There are three levels of compiler error severity:

\begin{Lentry}  
\item[Error] This severity is used when the compiler encounters a
  problem serious enough to prevent normal processing of a form.
  Instead of compiling the form, the compiler compiles a call to
  \code{error}.  Errors are used mainly for signaling syntax errors.
  If an error happens during macroexpansion, the compiler will handle
  it.  The compiler also handles and attempts to proceed from read
  errors.
  
\item[Warning] Warnings are used when the compiler can prove that
  something bad will happen if a portion of the program is executed,
  but the compiler can proceed by compiling code that signals an error
  at runtime if the problem has not been fixed:
  \begin{itemize}
  
  \item Violation of type declarations, or
  
  \item Function calls that have the wrong number of arguments or
    malformed keyword argument lists, or
  
  \item Referencing a variable declared \code{ignore}, or unrecognized
    declaration specifiers.
  \end{itemize}
  
  In the language of the \clisp{} standard, these are situations where
  the compiler can determine that a situation with undefined
  consequences or that would cause an error to be signaled would
  result at runtime.
  
\item[Note] Notes are used when there is something that seems a bit
  odd, but that might reasonably appear in correct programs.
\end{Lentry}

Note that the compiler does not fully conform to the proposed X3J13
``compiler-diagnostics'' cleanup.  Errors, warnings and notes mostly
correspond to errors, warnings and style-warnings, but many things
that the cleanup considers to be style-warnings are printed as
warnings rather than notes.  Also, warnings, style-warnings and most
errors aren't really signaled using the condition system.


\subsection{Errors During Macroexpansion}
\cpsubindex{macroexpansion}{errors during}

The compiler handles errors that happen during macroexpansion, turning
them into compiler errors.  If you want to debug the error (to debug a
macro), you can set \code{*break-on-signals*} to \code{error}.  For
example, this definition:

\begin{lisp}
(defun foo (e l)
  (do ((current l (cdr current))
       ((atom current) nil))
      (when (eq (car current) e) (return current))))
\end{lisp}

gives this error:

\begin{example}
In: DEFUN FOO
  (DO ((CURRENT L #) (# NIL)) (WHEN (EQ # E) (RETURN CURRENT)) )
Error: (during macroexpansion)

Error in function LISP::DO-DO-BODY.
DO step variable is not a symbol: (ATOM CURRENT)
\end{example}


\subsection{Read Errors}
\cpsubindex{read errors}{compiler}

The compiler also handles errors while reading the source.  For example:

\begin{example}
Error: Read error at 2:
 "(,/\back{foo})"
Error in function LISP::COMMA-MACRO.
Comma not inside a backquote.
\end{example}

The ``\code{at 2}'' refers to the character position in the source file at
which the error was signaled, which is generally immediately after the
erroneous text.  The next line, ``\code{(,/\back{foo})}'', is the line in
the source that contains the error file position.  The ``\code{/\back{} }''
indicates the error position within that line (in this example,
immediately after the offending comma.)

When in \hemlock{} (or any other EMACS-like editor), you can go to a
character position with:

\begin{example}
M-< C-u \var{position} C-f
\end{example}

Note that if the source is from a \hemlock{} buffer, then the position
is relative to the start of the compiled region or \code{defun}, not the
file or buffer start.

After printing a read error message, the compiler attempts to recover from the
error by backing up to the start of the enclosing top-level form and reading
again with \code{*read-suppress*} true.  If the compiler can recover from the
error, then it substitutes a call to \code{cerror} for the unreadable form and
proceeds to compile the rest of the file normally.

If there is a read error when the file position is at the end of the file
(i.e., an unexpected EOF error), then the error message looks like this:

\begin{example}
Error: Read error in form starting at 14:
 "(defun test ()"
Error in function LISP::FLUSH-WHITESPACE.
EOF while reading #<Stream for file "/usr/me/test.lisp">
\end{example}

In this case, ``\code{starting at 14}'' indicates the character
position at which the compiler started reading, i.e. the position
before the start of the form that was missing the closing delimiter.
The line \w{"\code{(defun test ()}"} is first line after the starting
position that the compiler thinks might contain the unmatched open
delimiter.


\subsection{Error Message Parameterization}
\cpsubindex{error messages}{verbosity}
\cpsubindex{verbosity}{of error messages}

There is some control over the verbosity of error messages.  See also
\varref{undefined-warning-limit}, \code{*efficiency-note-limit*} and
\varref{efficiency-note-cost-threshold}.

\begin{defvar}{}{enclosing-source-cutoff} 
  
  This variable specifies the number of enclosing actual source forms
  that are printed in full, rather than in the abbreviated processing
  path format.  Increasing the value from its default of \code{1}
  allows you to see more of the guts of the macroexpanded source,
  which is useful when debugging macros.
\end{defvar}

\begin{defvar}{}{error-print-length}
  \defvarx{error-print-level}
  
  These variables are the print level and print length used in
  printing error messages.  The default values are \code{5} and
  \code{3}.  If null, the global values of \code{*print-level*} and
  \code{*print-length*} are used.
\end{defvar}

\begin{defmac}{extensions:}{def-source-context}{%
    \args{\var{name} \var{lambda-list} \mstar{form}}}
  
  This macro defines how to extract an abbreviated source context from
  the \var{name}d form when it appears in the compiler input.
  \var{lambda-list} is a \code{defmacro} style lambda-list used to
  parse the arguments.  The \var{body} should return a list of
  subforms that can be printed on about one line.  There are
  predefined methods for \code{defstruct}, \code{defmethod}, etc.  If
  no method is defined, then the first two subforms are returned.
  Note that this facility implicitly determines the string name
  associated with anonymous functions.
\end{defmac}


\section{Types in Python}
\cpsubindex{types}{in python}

A big difference between \python{} and all other \llisp{} compilers
is the approach to type checking and amount of knowledge about types:
\begin{itemize}
  
\item \python{} treats type declarations much differently that other
  Lisp compilers do.  \python{} doesn't blindly believe type
  declarations; it considers them assertions about the program that
  should be checked.
  
\item \python{} also has a tremendously greater knowledge of the
  \clisp{} type system than other compilers.  Support is incomplete
  only for the \code{not}, \code{and} and \code{satisfies} types.
\end{itemize}
See also sections \ref{advanced-type-stuff} and \ref{type-inference}.


\subsection{Compile Time Type Errors}
\cindex{compile time type errors}
\cpsubindex{type checking}{at compile time}

If the compiler can prove at compile time that some portion of the
program cannot be executed without a type error, then it will give a
warning at compile time.  It is possible that the offending code would
never actually be executed at run-time due to some higher level
consistency constraint unknown to the compiler, so a type warning
doesn't always indicate an incorrect program.  For example, consider
this code fragment:
\begin{lisp}
(defun raz (foo)
  (let ((x (case foo
             (:this 13)
             (:that 9)
             (:the-other 42))))
    (declare (fixnum x))
    (foo x)))
\end{lisp}

Compilation produces this warning:

\begin{example}
In: DEFUN RAZ
  (CASE FOO (:THIS 13) (:THAT 9) (:THE-OTHER 42))
--> LET COND IF COND IF COND IF 
==>
  (COND)
Warning: This is not a FIXNUM:
  NIL
\end{example}

In this case, the warning is telling you that if \code{foo} isn't any
of \kwd{this}, \kwd{that} or \kwd{the-other}, then \code{x} will be
initialized to \false, which the \code{fixnum} declaration makes
illegal.  The warning will go away if \code{ecase} is used instead of
\code{case}, or if \kwd{the-other} is changed to \true.

This sort of spurious type warning happens moderately often in the
expansion of complex macros and in inline functions.  In such cases,
there may be dead code that is impossible to correctly execute.  The
compiler can't always prove this code is dead (could never be
executed), so it compiles the erroneous code (which will always signal
an error if it is executed) and gives a warning.

\begin{defun}{extensions:}{required-argument}{}
  
  This function can be used as the default value for keyword arguments
  that must always be supplied.  Since it is known by the compiler to
  never return, it will avoid any compile-time type warnings that
  would result from a default value inconsistent with the declared
  type.  When this function is called, it signals an error indicating
  that a required keyword argument was not supplied.  This function is
  also useful for \code{defstruct} slot defaults corresponding to
  required arguments.  \xlref{empty-type}.
  
  Although this function is a \cmucl{} extension, it is relatively harmless
  to use it in otherwise portable code, since you can easily define it
  yourself:
  \begin{lisp}
    (defun required-argument ()
      (error "A required keyword argument was not supplied."))
    \end{lisp}
\end{defun}

Type warnings are inhibited when the
\code{extensions:inhibit-warnings} optimization quality is \code{3}
(\pxlref{compiler-policy}.)  This can be used in a local declaration
to inhibit type warnings in a code fragment that has spurious
warnings.


\subsection{Precise Type Checking}
\label{precise-type-checks}
\cindex{precise type checking}
\cpsubindex{type checking}{precise}

With the default compilation policy, all type
assertions\footnote{There are a few circumstances where a type
  declaration is discarded rather than being used as type assertion.
  This doesn't affect safety much, since such discarded declarations
  are also not believed to be true by the compiler.}  are precisely
checked.  Precise checking means that the check is done as though
\code{typep} had been called with the exact type specifier that
appeared in the declaration.  \python{} uses \var{policy} to determine
whether to trust type assertions (\pxlref{compiler-policy}).  Type
assertions from declarations are indistinguishable from the type
assertions on arguments to built-in functions.  In \python, adding
type declarations makes code safer.

If a variable is declared to be \w{\code{(integer 3 17)}}, then its
value must always always be an integer between \code{3} and \code{17}.
If multiple type declarations apply to a single variable, then all the
declarations must be correct; it is as though all the types were
intersected producing a single \code{and} type specifier.

Argument type declarations are automatically enforced.  If you declare
the type of a function argument, a type check will be done when that
function is called.  In a function call, the called function does the
argument type checking, which means that a more restrictive type
assertion in the calling function (e.g., from \code{the}) may be lost.

The types of structure slots are also checked.  The value of a
structure slot must always be of the type indicated in any \kwd{type}
slot option.\footnote{The initial value need not be of this type as
  long as the corresponding argument to the constructor is always
  supplied, but this will cause a compile-time type warning unless
  \code{required-argument} is used.} Because of precise type checking,
the arguments to slot accessors are checked to be the correct type of
structure.

In traditional \llisp{} compilers, not all type assertions are
checked, and type checks are not precise.  Traditional compilers
blindly trust explicit type declarations, but may check the argument
type assertions for built-in functions.  Type checking is not precise,
since the argument type checks will be for the most general type legal
for that argument.  In many systems, type declarations suppress what
little type checking is being done, so adding type declarations makes
code unsafe.  This is a problem since it discourages writing type
declarations during initial coding.  In addition to being more error
prone, adding type declarations during tuning also loses all the
benefits of debugging with checked type assertions.

To gain maximum benefit from \python{}'s type checking, you should
always declare the types of function arguments and structure slots as
precisely as possible.  This often involves the use of \code{or},
\code{member} and other list-style type specifiers.  Paradoxically,
even though adding type declarations introduces type checks, it
usually reduces the overall amount of type checking.  This is
especially true for structure slot type declarations.

\python{} uses the \code{safety} optimization quality (rather than
presence or absence of declarations) to choose one of three levels of
run-time type error checking: \pxlref{optimize-declaration}.
\xlref{advanced-type-stuff} for more information about types in
\python{}.


\subsection{Weakened Type Checking}
\label{weakened-type-checks}
\cindex{weakened type checking}
\cpsubindex{type checking}{weakened}

When the value for the \code{speed} optimization quality is greater
than \code{safety}, and \code{safety} is not \code{0}, then type
checking is weakened to reduce the speed and space penalty.  In
structure-intensive code this can double the speed, yet still catch
most type errors.  Weakened type checks provide a level of safety
similar to that of ``safe'' code in other \llisp{} compilers.

A type check is weakened by changing the check to be for some
convenient supertype of the asserted type.  For example,
\code{\w{(integer 3 17)}} is changed to \code{fixnum},
\code{\w{(simple-vector 17)}} to \code{simple-vector}, and structure
types are changed to \code{structure}.  A complex check like:
\begin{example}
(or node hunk (member :foo :bar :baz))
\end{example}
will be omitted entirely (i.e., the check is weakened to \code{*}.)  If
a precise check can be done for no extra cost, then no weakening is
done.

Although weakened type checking is similar to type checking done by
other compilers, it is sometimes safer and sometimes less safe.
Weakened checks are done in the same places is precise checks, so all
the preceding discussion about where checking is done still applies.
Weakened checking is sometimes somewhat unsafe because although the
check is weakened, the precise type is still input into type
inference.  In some contexts this will result in type inferences not
justified by the weakened check, and hence deletion of some type
checks that would be done by conventional compilers.

For example, if this code was compiled with weakened checks:

\begin{lisp}
(defstruct foo
  (a nil :type simple-string))

(defstruct bar
  (a nil :type single-float))

(defun myfun (x)
  (declare (type bar x))
  (* (bar-a x) 3.0))
\end{lisp}

and \code{myfun} was passed a \code{foo}, then no type error would be
signaled, and we would try to multiply a \code{simple-vector} as
though it were a float (with unpredictable results.)  This is because
the check for \code{bar} was weakened to \code{structure}, yet when
compiling the call to \code{bar-a}, the compiler thinks it knows it
has a \code{bar}.

Note that normally even weakened type checks report the precise type
in error messages.  For example, if \code{myfun}'s \code{bar} check is
weakened to \code{structure}, and the argument is \false{}, then the
error will be:

\begin{example}
Type-error in MYFUN:
  NIL is not of type BAR
\end{example}

However, there is some speed and space cost for signaling a precise
error, so the weakened type is reported if the \code{speed}
optimization quality is \code{3} or \code{debug} quality is less than
\code{1}:

\begin{example}
Type-error in MYFUN:
  NIL is not of type STRUCTURE
\end{example}

\xlref{optimize-declaration} for further discussion of the
\code{optimize} declaration.


\section{Getting Existing Programs to Run}
\cpsubindex{existing programs}{to run}
\cpsubindex{types}{portability}
\cindex{compatibility with other Lisps}

Since \python{} does much more comprehensive type checking than other
Lisp compilers, \python{} will detect type errors in many programs
that have been debugged using other compilers.  These errors are
mostly incorrect declarations, although compile-time type errors can
find actual bugs if parts of the program have never been tested.

Some incorrect declarations can only be detected by run-time type
checking.  It is very important to initially compile programs with
full type checks and then test this version.  After the checking
version has been tested, then you can consider weakening or
eliminating type checks.  {\bf This applies even to previously debugged
  programs.}  \python{} does much more type inference than other
\llisp{} compilers, so believing an incorrect declaration does much
more damage.

The most common problem is with variables whose initial value doesn't
match the type declaration.  Incorrect initial values will always be
flagged by a compile-time type error, and they are simple to fix once
located.  Consider this code fragment:

\begin{example}
(prog (foo)
  (declare (fixnum foo))
  (setq foo ...)
  ...)
\end{example}

Here the variable \code{foo} is given an initial value of \false, but
is declared to be a \code{fixnum}.  Even if it is never read, the
initial value of a variable must match the declared type.  There are
two ways to fix this problem.  Change the declaration:

\begin{example}
(prog (foo)
  (declare (type (or fixnum null) foo))
  (setq foo ...)
  ...)
\end{example}

or change the initial value:

\begin{example}
(prog ((foo 0))
  (declare (fixnum foo))
  (setq foo ...)
  ...)
\end{example}

It is generally preferable to change to a legal initial value rather
than to weaken the declaration, but sometimes it is simpler to weaken
the declaration than to try to make an initial value of the
appropriate type.

Another declaration problem occasionally encountered is incorrect
declarations on \code{defmacro} arguments.  This probably usually
happens when a function is converted into a macro.  Consider this
macro:

\begin{lisp}
(defmacro my-1+ (x)
  (declare (fixnum x))
  `(the fixnum (1+ ,x)))
\end{lisp}

Although legal and well-defined \clisp, this meaning of this
definition is almost certainly not what the writer intended.  For
example, this call is illegal:

\begin{lisp}
(my-1+ (+ 4 5))
\end{lisp}

The call is illegal because the argument to the macro is \w{\code{(+ 4
    5)}}, which is a \code{list}, not a \code{fixnum}.  Because of
macro semantics, it is hardly ever useful to declare the types of
macro arguments.  If you really want to assert something about the
type of the result of evaluating a macro argument, then put a
\code{the} in the expansion:

\begin{lisp}
(defmacro my-1+ (x)
  `(the fixnum (1+ (the fixnum ,x))))
\end{lisp}

In this case, it would be stylistically preferable to change this
macro back to a function and declare it inline.  Macros have no
efficiency advantage over inline functions when using \python{}.
\xlref{inline-expansion}.


Some more subtle problems are caused by incorrect declarations that
can't be detected at compile time.  Consider this code:

\begin{example}
(do ((pos 0 (position #\back{a} string :start (1+ pos))))
    ((null pos))
  (declare (fixnum pos))
  ...)
\end{example}

Although \code{pos} is almost always a \code{fixnum}, it is \false{}
at the end of the loop.  If this example is compiled with full type
checks (the default), then running it will signal a type error at the
end of the loop.  If compiled without type checks, the program will go
into an infinite loop (or perhaps \code{position} will complain
because \w{\code{(1+ nil)}} isn't a sensible start.)  Why?  Because if
you compile without type checks, the compiler just quietly believes
the type declaration.  Since \code{pos} is always a \code{fixnum}, it
is never \nil, so \w{\code{(null pos)}} is never true, and the loop
exit test is optimized away.  Such errors are sometimes flagged by
unreachable code notes (\pxlref{dead-code-notes}), but it is still
important to initially compile any system with full type checks, even
if the system works fine when compiled using other compilers.

In this case, the fix is to weaken the type declaration to
\w{\code{(or fixnum null)}}.\footnote{Actually, this declaration is
  totally unnecessary in \python{}, since it already knows
  \code{position} returns a non-negative \code{fixnum} or \false.}
Note that there is usually little performance penalty for weakening a
declaration in this way.  Any numeric operations in the body can still
assume the variable is a \code{fixnum}, since \false{} is not a legal
numeric argument.  Another possible fix would be to say:

\begin{example}
(do ((pos 0 (position #\back{a} string :start (1+ pos))))
    ((null pos))
  (let ((pos pos))
    (declare (fixnum pos))
    ...))
\end{example}

This would be preferable in some circumstances, since it would allow a
non-standard representation to be used for the local \code{pos}
variable in the loop body (see section \ref{ND-variables}.)

In summary, remember that {\em all} values that a variable {\em ever}
has must be of the declared type, and that you should test using safe
code initially.


\section{Compiler Policy}
\label{compiler-policy}
\cpsubindex{policy}{compiler}
\cindex{compiler policy}

The policy is what tells the compiler \var{how} to compile a program.
This is logically (and often textually) distinct from the program
itself.  Broad control of policy is provided by the \code{optimize}
declaration; other declarations and variables control more specific
aspects of compilation.


\subsection{The Optimize Declaration}
\label{optimize-declaration}
\cindex{optimize declaration}
\cpsubindex{declarations}{\code{optimize}}

The \code{optimize} declaration recognizes six different
\var{qualities}.  The qualities are conceptually independent aspects
of program performance.  In reality, increasing one quality tends to
have adverse effects on other qualities.  The compiler compares the
relative values of qualities when it needs to make a trade-off; i.e.,
if \code{speed} is greater than \code{safety}, then improve speed at
the cost of safety.

The default for all qualities (except \code{debug}) is \code{1}.
Whenever qualities are equal, ties are broken according to a broad
idea of what a good default environment is supposed to be.  Generally
this downplays \code{speed}, \code{compile-speed} and \code{space} in
favor of \code{safety} and \code{debug}.  Novice and casual users
should stick to the default policy.  Advanced users often want to
improve speed and memory usage at the cost of safety and
debuggability.

If the value for a quality is \code{0} or \code{3}, then it may have a
special interpretation.  A value of \code{0} means ``totally
unimportant'', and a \code{3} means ``ultimately important.''  These
extreme optimization values enable ``heroic'' compilation strategies
that are not always desirable and sometimes self-defeating.
Specifying more than one quality as \code{3} is not desirable, since
it doesn't tell the compiler which quality is most important.


These are the optimization qualities:
\begin{Lentry}
  
\item[\code{speed}] \cindex{speed optimization quality}How fast the
  program should is run.  \code{speed 3} enables some optimizations
  that hurt debuggability.
  
\item[\code{compilation-speed}] \cindex{compilation-speed optimization
    quality}How fast the compiler should run.  Note that increasing
  this above \code{safety} weakens type checking.
  
\item[\code{space}] \cindex{space optimization quality}How much space
  the compiled code should take up.  Inline expansion is mostly
  inhibited when \code{space} is greater than \code{speed}.  A value
  of \code{0} enables promiscuous inline expansion.  Wide use of a
  \code{0} value is not recommended, as it may waste so much space
  that run time is slowed.  \xlref{inline-expansion} for a discussion
  of inline expansion.
  
\item[\code{debug}] \cindex{debug optimization quality}How debuggable
  the program should be.  The quality is treated differently from the
  other qualities: each value indicates a particular level of debugger
  information; it is not compared with the other qualities.
  \xlref{debugger-policy} for more details.
  
\item[\code{safety}] \cindex{safety optimization quality}How much
  error checking should be done.  If \code{speed}, \code{space} or
  \code{compilation-speed} is more important than \code{safety}, then
  type checking is weakened (\pxlref{weakened-type-checks}).  If
  \code{safety} if \code{0}, then no run time error checking is done.
  In addition to suppressing type checks, \code{0} also suppresses
  argument count checking, unbound-symbol checking and array bounds
  checks.
  
\item[\code{extensions:inhibit-warnings}] \cindex{inhibit-warnings
    optimization quality}This is a \cmucl{} extension that determines how
  little (or how much) diagnostic output should be printed during
  compilation.  This quality is compared to other qualities to
  determine whether to print style notes and warnings concerning those
  qualities.  If \code{speed} is greater than \code{inhibit-warnings},
  then notes about how to improve speed will be printed, etc.  The
  default value is \code{1}, so raising the value for any standard
  quality above its default enables notes for that quality.  If
  \code{inhibit-warnings} is \code{3}, then all notes and most
  non-serious warnings are inhibited.  This is useful with
  \code{declare} to suppress warnings about unavoidable problems.
\end{Lentry}


\subsection{The Optimize-Interface Declaration}
\label{optimize-interface-declaration}
\cindex{optimize-interface declaration}
\cpsubindex{declarations}{\code{optimize-interface}}

The \code{extensions:optimize-interface} declaration is identical in
syntax to the \code{optimize} declaration, but it specifies the policy
used during compilation of code the compiler automatically generates
to check the number and type of arguments supplied to a function.  It
is useful to specify this policy separately, since even thoroughly
debugged functions are vulnerable to being passed the wrong arguments.
The \code{optimize-interface} declaration can specify that arguments
should be checked even when the general \code{optimize} policy is
unsafe.

Note that this argument checking is the checking of user-supplied
arguments to any functions defined within the scope of the
declaration, \code{not} the checking of arguments to \llisp{}
primitives that appear in those definitions.

The idea behind this declaration is that it allows the definition of
functions that appear fully safe to other callers, but that do no
internal error checking.  Of course, it is possible that arguments may
be invalid in ways other than having incorrect type.  Functions
compiled unsafely must still protect themselves against things like
user-supplied array indices that are out of bounds and improper lists.
See also the \kwd{context-declarations} option to
\macref{with-compilation-unit}.


\section{Open Coding and Inline Expansion}
\label{open-coding}
\cindex{open-coding}
\cindex{inline expansion}
\cindex{static functions}

Since \clisp{} forbids the redefinition of standard functions\footnote{See the
proposed X3J13 ``lisp-symbol-redefinition'' cleanup.}, the compiler can have
special knowledge of these standard functions embedded in it.  This special
knowledge is used in various ways (open coding, inline expansion, source
transformation), but the implications to the user are basically the same:
\begin{itemize}
  
\item Attempts to redefine standard functions may be frustrated, since
  the function may never be called.  Although it is technically
  illegal to redefine standard functions, users sometimes want to
  implicitly redefine these functions when they are debugging using
  the \code{trace} macro.  Special-casing of standard functions can be
  inhibited using the \code{notinline} declaration.
  
\item The compiler can have multiple alternate implementations of
  standard functions that implement different trade-offs of speed,
  space and safety.  This selection is based on the compiler policy,
  \pxlref{compiler-policy}.
\end{itemize}


When a function call is {\em open coded}, inline code whose effect is
equivalent to the function call is substituted for that function call.
When a function call is {\em closed coded}, it is usually left as is,
although it might be turned into a call to a different function with
different arguments.  As an example, if \code{nthcdr} were to be open
coded, then

\begin{lisp}
(nthcdr 4 foobar)
\end{lisp}

might turn into

\begin{lisp}
(cdr (cdr (cdr (cdr foobar))))
\end{lisp}

or even 

\begin{lisp}
(do ((i 0 (1+ i))
     (list foobar (cdr foobar)))
    ((= i 4) list))
\end{lisp}

If \code{nth} is closed coded, then

\begin{lisp}
(nth x l)
\end{lisp}

might stay the same, or turn into something like:

\begin{lisp}
(car (nthcdr x l))
\end{lisp}

In general, open coding sacrifices space for speed, but some functions (such as
\code{car}) are so simple that they are always open-coded.  Even when not
open-coded, a call to a standard function may be transformed into a
different function call (as in the last example) or compiled as {\em
static call}. Static function call uses a more efficient calling
convention that forbids redefinition.

\part{Compiler Retargeting}

[\#\#\#

In general, it is a danger sign if a generator references a TN that isn't an
operand or temporary, since lifetime analysis hasn't been done for that use.
We are doing weird stuff for the old-cont and return-pc passing locations,
hoping that the conflicts at the called function have the desired effect.
Other stuff?  When a function returns unknown values, we don't reference the
values locations when a single-value return is done.  But nothing is live at a
return point anyway.



Have a way for template conversion to special-case constant arguments?  
How about:
    If an arg restriction is (:satisfies [$<$predicate function$>$]), and the
    corresponding argument is constant, with the constant value satisfying the
    predicate, then (if any other restrictions are satisfied), the template
    will be emitted with the literal value passed as an info argument.  If the
    predicate is omitted, then any constant will do.

    We could sugar this up a bit by allowing (:member $<$object$>$*) for
    (:satisfies (lambda (x) (member x '($<$object$>$*))))

We could allow this to be translated into a Lisp type by adding a new Constant
type specifier.  This could only appear as an argument to a function type.
To satisfy (Constant $<$type$>$), the argument must be a compile-time constant of
the specified type.  Just Constant means any constant (i.e. (Constant *)).
This would be useful for the type constraints on ICR transforms.


Constant TNs: we count on being able to indirect to the leaf, and don't try to
wedge the information into the offset.  We set the FSC to an appropriate
immediate SC.

    Allow "more operands" to VOPs in define-vop.  You can't do much with the
    more operands: define-vop just fills in the cost information according to
    the loading costs for a SC you specify.  You can't restrict more operands,
    and you can't make local preferences.  In the generator, the named variable
    is bound to the TN-ref for the first extra operand.  This should be good
    enough to handle all the variable arg VOPs (primarily function call and
    return).  Usually more operands are used just to get TN lifetimes to work
    out; the generator actually ignores them.

    Variable-arg VOPs can't be used with the VOP macro.  You must use VOP*.
    VOP* doesn't do anything with these extra operand except stick them on the
    ends of the operand lists passed into the template.  VOP* is often useful
    within the convert functions for non-VOP templates, since it can emit a VOP
    using an already prepared TN-Ref list.
    

    It is pretty basic to the whole primitive-type idea that there is only one
    primitive-type for a given lisp type.  This is really the same as saying
    primitive types are disjoint.  A primitive type serves two somewhat
    unrelated purposes:
     -- It is an abstraction of a Lisp type used to select type specific
        operations.  Originally kind of an efficiency hack, but it lets a
        template's type signature be used both for selection and operand
        representation determination.
     -- It represents a set of possible representations for a value (SCs).  The
        primitive type is used to determine the legal SCs for a TN, and is also
        used to determine which type-coercion/move VOP to use.

]

There are basically three levels of target dependence:

 -- Code in the "front end" (before VMR conversion) deals only with Lisp
    semantics, and is totally target independent.

 -- Code after VMR conversion and before code generation depends on the VM,
    but should work with little modification across a wide range of
    "conventional" architectures.

 -- Code generation depends on the machine's instruction set and other
    implementation details, so it will have to be redone for each
    implementation.  Most of the work here is in defining the translation into
    assembly code of all the supported VOPs.



\chapter{Storage bases and classes}
New interface: instead of CURRENT-FRAME-SIZE, have CURRENT-SB-SIZE \verb+<name>+ which
returns the current element size of the named SB.

How can we have primitive types that overlap, i.e. (UNSIGNED-BYTE 32),
(SIGNED-BYTE 32), FIXNUM?
Primitive types are used for two things:
    Representation selection: which SCs can be used to represent this value?
	For this purpose, it isn't necessary that primitive types be disjoint,
	since any primitive type can choose an arbitrary set of
	representations.  For moves between the overlapping representations,
	the move/load operations can just be noops when the locations are the
	same (vanilla MOVE), since any bad moves should be caught out by type
	checking.
    VOP selection:
	Is this operand legal for this VOP?  When ptypes overlap in interesting
	ways, there is a problem with allowing just a simple ptype restriction,
	since we might want to allow multiple ptypes.  This could be handled
	by allowing "union primitive types", or by allowing multiple primitive
	types to be specified (only in the operand restriction.)  The latter
	would be along the lines of other more flexible VOP operand restriction
	mechanisms, (constant, etc.)



Ensure that load/save-operand never need to do representation conversion.

The PRIMITIVE-TYPE more/coerce info would be moved into the SC.  This could
perhaps go along with flushing the TN-COSTS.  We would annotate the TN with
best SC, which implies the representation (boxed or unboxed).  We would still
need to represent the legal SCs for restricted TNs somehow, and also would have to
come up with some other way for pack to keep track of which SCs we have already
tried.

An SC would have a list of "alternate" SCs and a boolean SAVE-P value that
indicates it needs to be saved across calls in some non-SAVE-P SC.  A TN is
initially given its "best" SC.  The SC is annotated with VOPs that are used for
moving between the SC and its alternate SCs (load/save operand, save/restore
register).  It is also annotated with the "move" VOPs used for moving between
this SC and all other SCs it is possible to move between.  We flush the idea
that there is only c-to-t and c-from-t.

But how does this mesh with the idea of putting operand load/save back into the
generator?  Maybe we should instead specify a load/save function?  The
load/save functions would also differ from the move VOPs in that they would
only be called when the TN is in fact in that particular alternate SC, whereas
the move VOPs will be associated with the primary SC, and will be emitted
before it is known whether the TN will be packed in the primary SC or an
alternate.

I guess a packed SC could also have immediate SCs as alternate SCs, and
constant loading functions could be associated with SCs using this mechanism.

So given a TN packed in SC X and an SC restriction for Y and Z, how do we know
which load function to call?  There would be ambiguity if X was an alternate
for both Y and Z and they specified different load functions.  This seems
unlikely to arise in practice, though, so we could just detect the ambiguity
and give an error at define-vop time.  If they are doing something totally
weird, they can always inhibit loading and roll their own.

Note that loading costs can be specified at the same time (same syntax) as
association of loading functions with SCs.  It seems that maybe we will be
rolling DEFINE-SAVE-SCS and DEFINE-MOVE-COSTS into DEFINE-STORAGE-CLASS.

Fortunately, these changes will affect most VOP definitions very little.


A Storage Base represents a physical storage resource such as a register set or
stack frame.  Storage bases for non-global resources such as the stack are
relativized by the environment that the TN is allocated in.  Packing conflict
information is kept in the storage base, but non-packed storage resources such
as closure environments also have storage bases.
Some storage bases:
\begin{verbatim}
    General purpose registers
    Floating point registers
    Boxed (control) stack environment
    Unboxed (number) stack environment
    Closure environment
\end{verbatim}

A storage class is a potentially arbitrary set of the elements in a storage
base.  Although conceptually there may be a hierarchy of storage classes such
as "all registers", "boxed registers", "boxed scratch registers", this doesn't
exist at the implementation level.  Such things can be done by specifying
storage classes whose locations overlap.  A TN shouldn't have lots of
overlapping SC's as legal SC's, since time would be wasted repeatedly
attempting to pack in the same locations.

There will be some SC's whose locations overlap a great deal, since we get Pack
to do our representation analysis by having lots of SC's.  An SC is basically a
way of looking at a storage resource.  Although we could keep a fixnum and an
unboxed representation of the same number in the same register, they correspond
to different SC's since they are different representation choices.

TNs are annotated with the primitive type of the object that they hold:
    T: random boxed object with only one representation.
    Fixnum, Integer, XXX-Float: Object is always of the specified numeric type.
    String-Char: Object is always a string-char.

When a TN is packed, it is annotated with the SC it was packed into.  The code
generator for a VOP must be able to uniquely determine the representation of
its operands from the SC. (debugger also...)

Some SCs:
    Reg: any register (immediate objects)
    Save-Reg: a boxed register near r15 (registers easily saved in a call)
    Boxed-Reg: any boxed register (any boxed object)
    Unboxed-Reg: any unboxed register (any unboxed object)
    Float-Reg, Double-Float-Reg: float in FP register.
    Stack: boxed object on the stack (on cstack)
    Word: any 32bit unboxed object on nstack.
    Double: any 64bit unboxed object on nstack.

We have a number of non-packed storage classes which serve to represent access
costs associated with values that are not allocated using conflicts
information.  Non-packed TNs appear to already be packed in the appropriate
storage base so that Pack doesn't get confused.  Costs for relevant non-packed
SC's appear in the TN-Ref cost information, but need not ever be summed into
the TN cost vectors, since TNs cannot be packed into them.

There are SCs for non-immediate constants and for each significant kind of
immediate operand in the architecture.  On the RT, 4, 8 and 20 bit integer SCs
are probably worth having.

\begin{verbatim}
Non-packed SCs:
    Constant
    Immediate constant SCs:
        Signed-Byte-<N>, Unsigned-Byte-<N>, for various architecture dependent
	    values of <N>
	String-Char
	XXX-Float
	Magic values: T, NIL, 0.
\end{verbatim}

\chapter{Type system parameterization}

The main aspect of the VM that is likely to vary for good reason is the type
system:

 -- Different systems will have different ways of representing dynamic type
    information.  The primary effect this has on the compiler is causing VMR
    conversion of type tests and checks to be implementation dependent.
    Rewriting this code for each implementation shouldn't be a big problem,
    since the portable semantics of types has already been dealt with.

 -- Different systems will have different specialized number and array types,
    and different VOPs specialized for these types.  It is easy to add this kind
    of knowledge without affecting the rest of the compiler.  All you have to
    do is define the VOPs and translations.

 -- Different systems will offer different specialized storage resources
    such as floating-point registers, and will have additional kinds of
    primitive-types.  The storage class mechanism handles a large part of this,
    but there may be some problem in getting VMR conversion to realize the
    possibly large hidden costs in implicit moves to and from these specialized
    storage resources.  Probably the answer is to have some sort of general
    mechanism for determining the primitive-type for a TN given the Lisp type,
    and then to have some sort of mechanism for automatically using specialized
    Move VOPs when the source or destination has some particular primitive-type.

\#|
How to deal with list/null(symbol)/cons in primitive-type structure?  Since
cons and symbol aren't used for type-specific template selection, it isn't
really all that critical.  Probably Primitive-Type should return the List
primitive type for all of Cons, List and Null (indicating when it is exact).
This would allow type-dispatch for simple sequence functions (such as length)
to be done using the standard template-selection mechanism.  [Not a wired
assumption] 
|\#



\chapter{VOP Definition}

Before the operand TN-refs are passed to the emit function, the following
stuff is done:
 -- The refs in the operand and result lists are linked together in order using
    the Across slot.  This list is properly NIL terminated.
 -- The TN slot in each ref is set, and the ref is linked into that TN's refs
    using the Next slot.
 -- The Write-P slot is set depending on whether the ref is an argument or
    result.
 -- The other slots have the default values.

The template emit function fills in the Vop, Costs, Cost-Function,
SC-Restriction and Preference slots, and links together the Next-Ref chain as
appropriate.


\section{Lifetime model}

\#|
Note in doc that the same TN may not be used as both a more operand and as any
other operand to the same VOP, to simplify more operand LTN number coalescing.
|\#

It seems we need a fairly elaborate model for intra-VOP conflicts in order to
allocate temporaries without introducing spurious conflicts.  Consider the
important case of a VOP such as a miscop that must have operands in certain
registers.  We allocate a wired temporary, create a local preference for the
corresponding operand, and move to (or from) the temporary.  If all temporaries
conflict with all arguments, the result will be correct, but arguments could
never be packed in the actual passing register.  If temporaries didn't conflict
with any arguments, then the temporary for an earlier argument might get packed
in the same location as the operand for a later argument; loading would then
destroy an argument before it was read.

A temporary's intra-VOP lifetime is represented by the times at which its life
starts and ends.  There are various instants during the evaluation that start
and end VOP lifetimes.  Two TNs conflict if the live intervals overlap.
Lifetimes are open intervals: if one TN's lifetime begins at a point where
another's ends, then the TNs don't conflict.

The times within a VOP are the following:

:Load
    This is the beginning of the argument's lives, as far as intra-vop
    conflicts are concerned.  If load-TNs are allocated, then this is the
    beginning of their lives.

(:Argument $<$n$>$)
    The point at which the N'th argument is read for the last time (by this
    VOP).  If the argument is dead after this VOP, then the argument becomes
    dead at this time, and may be reused as a temporary or result load-TN.

(:Eval $<$n$>$)
    The N'th evaluation step.  There may be any number of evaluation steps, but
    it is unlikely that more than two are needed.

(:Result $<$n$>$) 
    The point at which the N'th result is first written into.  This is the
    point at which that result becomes live.

:Save
    Similar to :Load, but marks the end of time.  This is the point at which result
    load-TNs are stored back to the actual location.

In any of the list-style time specifications, the keyword by itself stands for
the first such time, i.e.
\begin{verbatim}
    :argument  <==>  (:argument 0)
\end{verbatim}

Note that argument/result read/write times don't actually have to be in the
order specified, but they must *appear* to happen in that order as far as
conflict analysis is concerned.  For example, the arguments can be read in any
order as long as no TN is written that has a life beginning at or after
(:Argument $<$n$>$), where N is the number of an argument whose reading was
postponed.

[\#\#\# (???)

We probably also want some syntactic sugar in Define-VOP for automatically
moving operands to/from explicitly allocated temporaries so that this kind of
thing is somewhat easy.  There isn't really any reason to consider the
temporary to be a load-TN, but we want to compute costs as though it was and
want to use the same operand loading routines.

We also might consider allowing the lifetime of an argument/result to be
extended forward/backward.  This would in many cases eliminate the need for
temporaries when operands are read/written out of order.
]


\section{VOP Cost model}

Note that in this model, if an operand has no restrictions, it has no cost.
This makes sense, since the purpose of the cost is to indicate the
relative value of packing in different SCs.  If the operand isn't required to
be in a good SC (i.e. a register), then we might as well leave it in memory.
The SC restriction mechanism can be used even when doing a move into the SC is
too complex to be generated automatically (perhaps requiring temporary
registers), since Define-VOP allows operand loading to be done explicitly.


\section{Efficiency notes}

  In addition to
being used to tell whether a particular unsafe template might get emitted, we
can also use it to give better efficiency notes:
 -- We can say what is wrong with the call types, rather than just saying we
    failed to open-code.
 -- We can tell whether any of the "better" templates could possibly apply,
    i.e. is the inapplicability of a template because of inadequate type
    information or because the type is just plain wrong.  We don't want to
    flame people when a template that couldn't possibly match doesn't match,
    e.g. complaining that we can't use fixnum+ when the arguments are known to
    be floats.


This is how we give better efficiency notes:

The Template-Note is a short noun-like string without capitalization or
punctuation that describes what the template ``does", i.e. we say
"Unable to do ~A, doing ~A instead."

The Cost is moved from the Vop-Info to the Template structure, and is used to
determine the ``goodness" of possibly applicable templates.  [Could flush
Template/Vop-Info distinction]  The cost is used to choose the best applicable
template to emit, and also to determine what better templates we might have
been able to use.

A template is possibly applicable if there is an intersection between all of
the arg/result types and the corresponding arg/result restrictions, i.e. the
template is not clearly impossible: more declarations might allow it to be
emitted.


\chapter{Assembler Retargeting}


\chapter{Writing Assembly Code}

VOP writers expect:
\begin{Lentry}
\item[MOVE]
      You write when you port the assembler.)
\item[EMIT-LABEL]
      Assembler interface like INST.  Takes a label you made and says "stick it
      here."
   \item[GEN-LABEL]
      Returns a new label suitable for use with EMIT-LABEL exactly once and
      for referencing as often as necessary.
   \item[INST]
      Recognizes and dispatches to instructions you defined for assembler.
   \item[ALIGN]
      This takes the number of zero bits you want in the low end of the address
      of the next instruction.
   \item[ASSEMBLE]
   \item[ASSEMBLE-ELSEWHERE]
      Get ready for assembling stuff.  Takes a VOP and arbitrary PROGN-style
      body.  Wrap these around instruction emission code announcing the first
      pass of our assembler.
   \item[CURRENT-NFP-TN]
      This returns a TN for the NFP if the caller uses the number stack, or
      nil.
   \item[SB-ALLOCATED-SIZE]
      This returns the size of some storage base used by the currently
      compiling component.
   \item[...]
\end{Lentry}
;;;
;;; VOP idioms
;;;

\begin{Lentry}
\item[STORE-STACK-TN]
\item[LOAD-STACK-TN]
   These move a value from a register to the control stack, or from the
   control stack to a register.  They take care of checking the TN types,
   modifying offsets according to the address units per word, etc.
\end{Lentry}

\chapter{Required VOPS}


Note: the move VOP cannot have any wired temps.  (Move-Argument also?)  This is
so we can move stuff into wired TNs without stepping on our toes.


We create set closure variables using the Value-Cell VOP, which takes a value
and returns a value cell containing the value.  We can basically use this
instead of a Move VOP when initializing the variable.  Value-Cell-Set and
Value-Cell-Ref are used to access the value cell.  We can have a special effect
for value cells so that value cells references can be discovered to be common
subexpressions or loop invariants.




Represent unknown-values continuations as (start, count).  Unknown values
continuations are always outside of the current frame (on stack top).  Within a
function, we always set up and receive values in the standard passing
locations.  If we receive stack values, then we must BLT them down to the start
of our frame, filling in any unsupplied values.  If we generate unknown values
(i.e. PUSH-VALUES), then we set the values up in the standard locations, then
BLT them to stack top.  When doing a tail-return of MVs, we just set them up in
the standard locations and decrement SP: no BLT is necessary.

Unknown argument call (MV-CALL) takes its arguments on stack top (is given a
base pointer).  If not a tail call, then we just set the arg pointer to the
base pointer and call.  If a tail call, we must BLT the arguments down to the
beginning of the current frame.


Implement more args by BLT'ing the more args *on top* of the current frame.
This solves two problems:
\begin{itemize}
\item Any register more arguments can be made uniformly accessibly by copying
    them into memory.  [We can't store the registers in place, since the
    beginning of the frame gets double use for storing the old-cont, return-pc
    and env.]
\item It solves the deallocation problem: the arguments will be deallocated when
    the frame is returned from or a tail full call is done out of it.  So
    keyword args will be properly tail-recursive without any special mechanism
    for squeezing out the more arg once the parsing is done.  Note that a tail
    local call won't blast the more arg, since in local call the callee just
    takes the frame it is given (in this case containing the more arg).
\end{itemize}

More args in local call???  Perhaps we should not attempt local call conversion
in this case.  We already special-case keyword args in local call.  It seems
that the main importance of more args is primarily related to full call: it is
used for defining various kinds of frobs that need to take arbitrary arguments:
\begin{itemize}
\item Keyword arguments
\item Interpreter stubs
\item "Pass through" applications such as dispatch functions
\end{itemize}
Given the marginal importance of more args in local call, it seems unworth
going to any implementation difficulty.  In fact, it seems that it would cause
complications both at the VMR level and also in the VM definition.  This being
the case, we should flush it.


\section{Function Call}



\subsection{Registers and frame format}

These registers are used in function call and return:

A0..A{\it n}
    In full call, the first three arguments.  In unknown values return, the
    first three return values.

CFP
    The current frame pointer.  In full call, this initially points to a
    partial frame large enough to hold the passed stack arguments (zero-length
    if none).

CSP
    The current control stack top pointer. 

OCFP
    In full call, the passing location for the frame to return to.

    In unknown-values return of other than one value, the pointer to returned
    stack values.  In such a return, OCFP is always initialized to point to
    the frame returned from, even when no stack values are returned.  This
    allows OCFP to be used to restore CSP.

LRA
    In full call, the passing location for the return PC.

NARGS
    In full call, the number of arguments passed.  In unknown-values return of
    other than one value, the number of values returned.


\subsection{Full call}

What is our usage of CFP, OCFP and CSP?  

It is an invariant that CSP always points after any useful information so that
at any time an interrupt can come and allocate stuff in the stack.

TR call is also a constraint: we can't deallocate the caller's frame before the
call, since it holds the stack arguments for the call.  

What we do is have the caller set up CFP, and have the callee set CSP to CFP
plus the frame size.  The caller leaves CSP alone: the callee is the one who
does any necessary stack deallocation.

In a TR call, we don't do anything: CFP is left as CFP, and CSP points to the
end of the frame, keeping the stack arguments from being trashed.

In a normal call, CFP is set to CSP, causing the callee's frame to be allocated
after the current frame.


\subsection{Unknown values return}

The unknown values return convention is always used in full call, and is used
in local call when the compiler either can't prove that a fixed number of
values are returned, or decides not to use the fixed values convention to allow
tail-recursive XEP calls.

The unknown-values return convention has variants: single value and variable
values.  We make this distinction to optimize the important case of a returner
who knows exactly one value is being returned.  Note that it is possible to
return a single value using the variable-values convention, but it is less
efficient.

We indicate single-value return by returning at the return-pc+4; variable value
return is indicated by returning at the return PC.

Single-value return makes only the following guarantees:
    A0 holds the value returned.
    CSP has been reset: there is no garbage on the stack.

In variable value return, more information is passed back:
    A0..A2 hold the first three return values.  If fewer than three values are
    returned, then the unused registers are initialized to NIL.

    OCFP points to the frame returned from.  Note that because of our
    tail-recursive implementation of call, the frame receiving the values is
    always immediately under the frame returning the values.  This means that
    we can use OCFP to index the values when we access them, and to restore
    CSP when we want to discard them.

    NARGS holds the number of values returned.

    CSP is always (+ OCFP (* NARGS 4)), i.e. there is room on the stack
    allocated for all returned values, even if they are all actually passed in
    registers.


\subsection{External Entry Points}

Things that need to be done on XEP entry:
 1] Allocate frame
 2] Move more arg above the frame, saving context
 3] Set up env, saving closure pointer if closure
 4] Move arguments from closure to local home
    Move old-cont and return-pc to the save locations
 5] Argument count checking and dispatching

XEP VOPs:
\begin{verbatim}
Allocate-Frame
Copy-More-Arg <nargs-tn> 'fixed {in a3} => <context>, <count>
Setup-Environment
Setup-Closure-Environment => <closure>
Verify-Argument-Count <nargs-tn> 'count {for fixed-arg lambdas}
Argument-Count-Error <nargs-tn> {Drop-thru on hairy arg dispatching}
Use fast-if-=/fixnum and fast-if-</fixnum for dispatching.
\end{verbatim}

Closure vops:
\begin{verbatim}
make-closure <fun entry> <slot count> => <closure>
closure-init <closure> <values> 'slot
\end{verbatim}

Things that need to be done on all function entry:
\begin{itemize}
\item Move arguments to the variable home (consing value cells as necessary)
\item Move environment values to the local home
\item Move old-cont and return-pc to the save locations
\end{itemize}

\section{Calls}

Calling VOP's are a cross product of the following sets (with some members
missing):
   Return values
      multiple (all values)
      fixed (calling with unknown values conventions, wanting a certain
             number.)
      known (only in local call where caller/callee agree on number of
      	     values.)
      tail (doesn't return but does tail call)
   What function
      local
      named (going through symbol, like full but stash fun name for error sys)
      full (have a function)
   Args
      fixed (number of args are known at compile-time)
      variable (MULTIPLE-VALUE-CALL and APPLY)

Note on all jumps for calls and returns that we want to put some instruction
in the jump's delay slot(s).

Register usage at the time of the call:

LEXENV
   This holds the lexical environment to use during the call if it's a closure,
   and it is undefined otherwise.

CNAME
   This holds the symbol for a named call and garbage otherwise.

OCFP
   This holds the frame pointer, which the system restores upon return.  The
   callee saves this if necessary; this is passed as a pseudo-argument.

A0 ... An
   These holds the first n+1 arguments.

NARGS
   This holds the number of arguments, as a fixnum.

LRA
   This holds the lisp-return-address object which indicates where to return.
   For a tail call, this retains its current value.  The callee saves this
   if necessary; this is passed as a pseudo-argument.

CODE
   This holds the function object being called.

CSP
   The caller ignores this.  The callee sets it as necessary based on CFP.

CFP
   This holds the callee's frame pointer.  Caller sets this to the new frame
   pointer, which it remembered when it started computing arguments; this is
   CSP if there were no stack arguments.  For a tail call CFP retains its
   current value.

NSP
   The system uses this within a single function.  A function using NSP must
   allocate and deallocate before returning or making a tail call.

Register usage at the time of the return for single value return, which
goes with the unknown-values convention the caller used.

A0
   This holds the value.

CODE
   This holds the lisp-return-address at which the system continues executing.

CSP
   This holds the CFP.  That is, the stack is guaranteed to be clean, and there
   is no code at the return site to adjust the CSP.

CFP
   This holds the OCFP.

Additional register usage for multiple value return:

NARGS
   This holds the number of values returned.

A0 ... An
   These holds the first n+1 values, or NIL if there are less than n+1 values.

CSP
   Returner stores CSP to hold its CFP + NARGS * \verb+<address units per word>+

OCFP
   Returner stores this as its CFP, so the returnee has a handle on either
   the start of the returned values on the stack.


ALLOCATE FULL CALL FRAME.

If the number of call arguments (passed to the VOP as an info argument)
indicates that there are stack arguments, then it makes some callee frame for
arguments:
\begin{verbatim}
   VOP-result <- CSP
   CSP <- CSP + value of VOP info arg times address units per word.
\end{verbatim}

In a call sequence, move some arguments to the right places.

There's a variety of MOVE-ARGUMENT VOP's.

FULL CALL VOP'S
(variations determined by whether it's named, it's a tail call, there
is a variable arg count, etc.)
\begin{verbatim}
  if variable number of arguments
    NARGS <- (CSP - value of VOP argument) shift right by address units per word.
    A0...An <- values off of VOP argument (just fill them all)
  else
    NARGS <- value of VOP info argument (always a constant)

  if tail call
    OCFP <- value from VOP argument
    LRA <- value from VOP argument
    CFP stays the same since we reuse the frame
    NSP <- NFP
  else
    OCFP <- CFP
    LRA <- compute LRA by adding an assemble-time determined constant to
    	   CODE.
    CFP <- new frame pointer (remembered when starting to compute args)
           This is CSP if no stack args.
    when (current-nfp-tn VOP-self-pointer)
      stack-temp <- NFP

  if named
    CNAME <- function symbol name
    the-fun <- function object out of symbol

  LEXENV <- the-fun (from previous line or VOP argument)
  CODE <- function-entry (the first word after the-fun)
  LIP <- calc first instruction addr (CODE + constant-offset)
  jump and run off temp

  <emit Lisp return address data-block>
  <default and move return values OR receive return values>
  when (current-nfp-tn VOP-self-pointer)
    NFP <- stack-temp
\end{verbatim}
Callee:

\begin{verbatim}
XEP-ALLOCATE-FRAME
  emit function header (maybe initializes offset back to component start,
  			but other pointers are set up at load-time.  Pads
			to dual-word boundary.)
  CSP <- CFP + compile-time determined constant (frame size)
  if the function uses the number stack
    NFP <- NSP
    NSP <- NSP + compile-time determined constant (number stack frame size)
\end{verbatim}

\begin{verbatim}
SETUP-ENVIRONMENT
(either use this or the next one)

CODE <- CODE - assembler-time determined offset from function-entry back to
	       the code data-block address.
\end{verbatim}

\begin{verbatim}
SETUP-CLOSURE-ENVIRONMENT
(either use this or the previous one)
After this the CLOSURE-REF VOP can reference closure variables.

VOP-result <- LEXENV
CODE <- CODE - assembler-time determined offset from function-entry back to
	       the code data-block address.
\end{verbatim}

Return VOP's
RETURN and RETURN-MULTIPLE are for the unknown-values return convention.
For some previous caller this is either it wants n values (and it doesn't
know how many are coming), or it wants all the values returned (and it 
doesn't know how many are coming).


RETURN
(known fixed number of values, used with the unknown-values convention
 in the caller.)
When compiler invokes VOP, all values are already where they should be;
just get back to caller.

\begin{verbatim}
when (current-nfp-tn VOP-self-pointer)
  ;; The number stack grows down in memory.
  NSP <- NFP + number stack frame size for calls within the currently
                  compiling component
	       times address units per word
CODE <- value of VOP argument with LRA
if VOP info arg is 1 (number of values we know we're returning)
  CSP <- CFP
  LIP <- calc target addr
          (CODE + skip over LRA header word + skip over address units per branch)
	  (The branch is in the caller to skip down to the MV code.)
else
  NARGS <- value of VOP info arg
  nil out unused arg regs
  OCFP <- CFP  (This indicates the start of return values on the stack,
  		but you leave space for those in registers for convenience.)
  CSP <- CFP + NARGS * address-units-per-word
  LIP <- calc target addr (CODE + skip over LRA header word)
CFP <- value of VOP argument with OCFP
jump and run off LIP
\end{verbatim}

RETURN-MULTIPLE
(unknown number of values, used with the unknown-values convention in
 the caller.)
When compiler invokes VOP, it gets TN's representing a pointer to the
values on the stack and how many values were computed.

\begin{verbatim}
when (current-nfp-tn VOP-self-pointer)
  ;; The number stack grows down in memory.
  NSP <- NFP + number stack frame size for calls within the currently
                  compiling component
	       times address units per word
NARGS <- value of VOP argument
copy the args to the beginning of the current (returner's) frame.
   Actually some go into the argument registers.  When putting the rest at
   the beginning of the frame, leave room for those in the argument registers.
CSP <- CFP + NARGS * address-units-per-word
nil out unused arg regs
OCFP <- CFP  (This indicates the start of return values on the stack,
	      but you leave space for those in registers for convenience.)
CFP <- value of VOP argument with OCFP
CODE <- value of VOP argument with LRA
LIP <- calc target addr (CODE + skip over LRA header word)
jump and run off LIP
\end{verbatim}

Returnee
The call VOP's call DEFAULT-UNKNOWN-VALUES or RECEIVE-UNKNOWN-VALUES after
spitting out transfer control to get stuff from the returner.

DEFAULT-UNKNOWN-VALUES
(We know what we want and we got something.)
If returnee wants one value, it never does anything to deal with a shortage
of return values.  However, if start at PC, then it has to adjust the stack
pointer to dump extra values (move OCFP into CSP).  If it starts at PC+N,
then it just goes along with the "want one value, got it" case.
If the returnee wants multiple values, and there's a shortage of return
values, there are two cases to handle.  One, if the returnee wants fewer
values than there are return registers, and we start at PC+N, then it fills
in return registers \verb|A1..A<desired values necessary>|; if we start at PC,
then the returnee is fine since the returning conventions have filled in
the unused return registers with nil, but the returnee must adjust the
stack pointer to dump possible stack return values (move OCFP to CSP).
Two, if the returnee wants more values than the number of return registers,
and it starts at PC+N (got one value), then it sets up returnee state as if
an unknown number of values came back:
\begin{verbatim}
   A0 has the one value
   A1..An get nil
   NARGS gets 1
   OCFP gets CSP, so general code described below can move OCFP into CSP
If we start at PC, then branch down to the general "got k values, wanted n"
code which takes care of the following issues:
   If k < n, fill in stack return values of nil for shortage of return
      values and move OCFP into CSP
   If k >= n, move OCFP into CSP
This also restores CODE from LRA by subtracting an assemble-time constant.
\end{verbatim}

RECEIVE-UKNOWN-VALUES
(I want whatever I get.)
We want these at the end of our frame.  When the returnee starts at
PC, it moves the return value registers to OCFP..OCFP[An] ignoring where
the end of the stack is and whether all the return value registers had
values.  The returner left room on the stack before the stack return values
for the register return values.  When the returnee starts at PC+N, bump CSP
by 1 and copy A0 there.
This also restores CODE from LRA by subtracting an assemble-time constant.


Local call

There are three flavors:
   1] KNOWN-CALL-LOCAL
      Uses known call convention where caller and callee agree where all
      the values are, and there's a fixed number of return values.
   2] CALL-LOCAL
      Uses the unknown-values convention, but we expect a particular
      number of values in return.
   3] MULTIPLE-CALL-LOCAL
      Uses the unknown-values convention, but we want all values returned.

ALLOCATE-FRAME

If the number of call arguments (passed to the VOP as an info argument)
indicates that there are stack arguments, then it makes some callee frame for
arguments:
\begin{verbatim}
   VOP-result1 <- CSP
   CSP <- CSP + control stack frame size for calls within the currently
   		   compiling component
   		times address units per word.
   when (callee-nfp-tn <VOP info arg holding callee>)
     ;; The number stack grows down.
     ;; May have to round to dual-word boundary if machines C calling
     ;;    conventions demand this.
     NSP <- NSP - number stack frame size for calls within the currently
     		     compiling component
		  times address units per word
     VOP-result2 <- NSP
\end{verbatim}
KNOWN-CALL-LOCAL, CALL-LOCAL, MULTIPLE-CALL-LOCAL
KNOWN-CALL-LOCAL has no need to affect CODE since CODE is the same for the
caller/returnee and the returner.  This uses KNOWN-RETURN.
With CALL-LOCAL and MULTIPLE-CALL-LOCAL, the caller/returnee must fixup
CODE since the callee may do a tail full call.  This happens in the code
emitted by DEFAULT-UNKNOWN-VALUES and RECEIVE-UNKNOWN-VALUES.  We use these
return conventions since we don't know what kind of values the returner
will give us.  This could happen due to a tail full call to an unknown
function, or because the callee had different return points that returned
various numbers of values.

\begin{verbatim}
when (current-nfp-tn VOP-self-pointer)   ;Get VOP self-pointer with
					 ;DEFINE-VOP switch :vop-var.
  stack-temp <- NFP
CFP <- value of VOP arg
when (callee-nfp-tn <VOP info arg holding callee>)
  <where-callee-wants-NFP-tn>  <-  value of VOP arg
<where-callee-wants-LRA-tn>  <-  compute LRA by adding an assemble-time
				 determined constant to CODE.
jump and run off VOP info arg holding start instruction for callee

<emit Lisp return address data-block>
<case call convention
  known: do nothing
  call: default and move return values
  multiple: receive return values
>
when (current-nfp-tn VOP-self-pointer)   
  NFP <- stack-temp
\end{verbatim}

KNOWN-RETURN
\begin{verbatim}
CSP <- CFP
when (current-nfp-tn VOP-self-pointer)
  ;; number stack grows down in memory.
  NSP <- NFP + number stack frame size for calls within the currently
                  compiling component
	       times address units per word
LIP <- calc target addr (value of VOP arg + skip over LRA header word)
CFP <- value of VOP arg
jump and run off LIP

\end{verbatim}


\chapter{Standard Primitives}


\chapter{Customizing VMR Conversion}

Another way in which different implementations differ is in the relative cost
of operations.  On machines without an integer multiply instruction, it may be
desirable to convert multiplication by a constant into shifts and adds, while
this is surely a bad idea on machines with hardware support for multiplication.
Part of the tuning process for an implementation will be adding implementation
dependent transforms and disabling undesirable standard transforms.

When practical, ICR transforms should be used instead of VMR generators, since
transforms are more portable and less error-prone.  Note that the Lisp code
need not be implementation independent: it may contain all sorts of
sub-primitives and similar stuff.  Generally a function should be implemented
using a transform instead of a VMR translator unless it cannot be implemented
as a transform due to being totally evil or it is just as easy to implement as
a translator because it is so simple.


\section{Constant Operands}

If the code emitted for a VOP when an argument is constant is very different
than the non-constant case, then it may be desirable to special-case the
operation in VMR conversion by emitting different VOPs.  An example would be if
SVREF is only open-coded when the index is a constant, and turns into a miscop
call otherwise.  We wouldn't want constant references to spuriously allocate
all the miscop linkage registers on the off chance that the offset might not be
constant.  See the :constant feature of VOP primitive type restrictions.


\section{Supporting Multiple Hardware Configurations}


A winning way to change emitted code depending on the hardware configuration,
i.e. what FPA is present is to do this using primitive types.  Note that the
Primitive-Type function is VM supplied, and can look at any appropriate
hardware configuration switches.  Short-Float can become 6881-Short-Float,
AFPA-Short-Float, etc.  There would be separate SBs and SCs for the registers
of each kind of FP hardware, with each hardware-specific primitive type
using the appropriate float register SC.  Then the hardware specific templates
would provide AFPA-Short-Float as the argument type restriction.

Primitive type changes:

The primitive-type structure is given a new \%Type slot, which is the CType
structure that is equivalent to this type.  There is also a Guard slot, which,
if true is a function that control whether this primitive type is allowed (due
to hardware configuration, etc.)  

We add new :Type and :Guard keywords to Def-Primitive-Type.  Type is the type
specifier that is equivalent (default to the primitive-type name), and Guard is
an expression evaluated in the null environment that controls whether this type
applies (default to none, i.e. constant T).

The Primitive-Type-Type function returns the Lisp CType corresponding to a
primitive type.  This is the \%Type unless there is a guard that returns false,
in which case it is the empty type (i.e. NIL).

[But this doesn't do what we want it to do, since we will compute the
function type for a template at load-time, so they will correspond to whatever
configuration was in effect then.  Maybe we don't want to dick with guards here
(if at all).  I guess we can defer this issue until we actually support
different FP configurations.  But it would seem pretty losing to separately
flame about all the different FP configurations that could be used to open-code
+ whenever we are forced to closed-code +.

If we separately report each better possibly applicable template that we
couldn't use, then it would be reasonable to report any conditional template
allowed by the configuration.  

But it would probably also be good to give some sort of hint that perhaps it
would be a good time to make sure you understand how to tell the compiler to
compile for a particular configuration.  Perhaps if there is a template that
applies *but for the guard*, then we could give a note.  This way, if someone
thinks they are being efficient by throwing in lots of declarations, we can let
them know that they may have to do more.

I guess the guard should be associated with the template rather than the
primitive type.  This would allow LTN and friends to easily tell whether a
template applies in this configuration.  It is also probably more natural for
some sorts of things: with some hardware variants, it may be that the SBs and
representations (SCs) are really the same, but there are some different allowed
operations.  In this case, we could easily conditionalize VOPs without the
increased complexity due to bogus SCs.  If there are different storage
resources, then we would conditionalize Primitive-Type as well.



\section{Special-case VMR convert methods}

    (defun continuation-tn (cont \&optional (check-p t))
      ...)
Return the TN which holds Continuation's first result value.  In general
this may emit code to load the value into a TN.  If Check-P is true, then
when policy indicates, code should be emitted to check that the value satisfies
the continuation asserted type.

    (defun result-tn (cont)
      ...)
Return the TN that Continuation's first value is delivered in.  In general,
may emit code to default any additional values to NIL.

    (defun result-tns (cont n)
      ...)
Similar to Result-TN, except that it returns a list of N result TNs, one
for each of the first N values.


Nearly all open-coded functions should be handled using standard template
selection.  Some (all?) exceptions:
\begin{itemize}
\item List, List* and Vector take arbitrary numbers of arguments.  Could
    implement Vector as a source transform.  Could even do List in a transform
    if we explicitly represent the stack args using \%More-Args or something.
\item \%Typep varies a lot depending on the type specifier.  We don't want to
    transform it, since we want \%Typep as a canonical form so that we can do
    type optimizations.
\item Apply is weird.
\item Funny functions emitted by the compiler: \%Listify-Rest-Args, Arg,
    \%More-Args, \%Special-Bind, \%Catch, \%Unknown-Values (?), \%Unwind-Protect,
    \%Unwind, \%\%Primitive.
\end{itemize}

\part{Run-Time system}
\chapter{The Type System}

\chapter{The Info Database}

%  					-*- Dictionary: design; Package: C -*-

May be worth having a byte-code representation for interpreted code.  This way,
an entire system could be compiled into byte-code for debugging (the
"check-out" compiler?).

Given our current inclination for using a stack machine to interpret IR1, it
would be straightforward to layer a byte-code interpreter on top of this.


Interpreter:

Instead of having no interpreter, or a more-or-less conventional interpreter,
or byte-code interpreter, how about directly executing IR1?

We run through the IR1 passes, possibly skipping optional ones, until we get
through environment analysis.  Then we run a post-pass that annotates IR1 with
information about where values are kept, i.e. the stack slot.

We can lazily convert functions by having FUNCTION make an interpreted function
object that holds the code (really a closure over the interpreter).  The first
time that we try to call the function, we do the conversion and processing.
Also, we can easily keep track of which interpreted functions we have expanded
macros in, so that macro redefinition automatically invalidates the old
expansion, causing lazy reconversion.

Probably the interpreter will want to represent MVs by a recognizable structure
that is always heap-allocated.  This way, we can punt the stack issues involved
in trying to spread MVs.  So a continuation value can always be kept in a
single cell.

The compiler can have some special frobs for making the interpreter efficient,
such as a call operation that extracts arguments from the stack
slots designated by a continuation list.  Perhaps 
    (values-mapcar fun . lists)
<==>
    (values-list (mapcar fun . lists))
This would be used with MV-CALL.


This scheme seems to provide nearly all of the advantages of both the compiler
and conventional interpretation.  The only significant disadvantage with
respect to a conventional interpreter is that there is the one-time overhead of
conversion, but doing this lazily should make this quite acceptable.

With respect to a conventional interpreter, we have major advantages:
 + Full syntax checking: safety comparable to compiled code.
 + Semantics similar to compiled code due to code sharing.  Similar diagnostic
   messages, etc.  Reduction of error-prone code duplication.
 + Potential for full type checking according to declarations (would require
   running IR1 optimize?)
 + Simplifies debugger interface, since interpreted code can look more like
   compiled code: source paths, edit definition, etc.

For all non-run-time symbol annotations (anything other than SYMBOL-FUNCTION
and SYMBOL-VALUE), we use the compiler's global database.  MACRO-FUNCTION will
use INFO, rather than vice-versa.

When doing the IR1 phases for the interpreter, we probably want to suppress
optimizations that change user-visible function calls:
 -- Don't do local call conversion of any named functions (even lexical ones).
    This is so that a call will appear on the stack that looks like the call in
    the original source.  The keyword and optional argument transformations
    done by local call mangle things quite a bit.  Also, note local-call
    converting prevents unreferenced arguments from being deleted, which is
    another non-obvious transformation.
 -- Don't run source-transforms, IR1 transforms and IR1 optimizers.  This way,
    TRACE and BACKTRACE will show calls with the original arguments, rather
    than the "optimized" form, etc.  Also, for the interpreter it will
    actually be faster to call the original function (which is compiled) than
    to "inline expand" it.  Also, this allows implementation-dependent
    transforms to expand into %PRIMITIVE uses.

There are some problems with stepping, due to our non-syntactic IR1
representation.  The source path information is the key that makes this
conceivable.  We can skip over the stepping of a subform by quietly evaluating
nodes whose source path lies within the form being skipped.

One problem with determining what value has been returned by a form.  With a
function call, it is theoretically possible to precisely determine this, since
if we complete evaluation of the arguments, then we arrive at the Combination
node whose value is synonymous with the value of the form.  We can even detect
this case, since the Node-Source will be EQ to the form.  And we can also
detect when we unwind out of the evaluation, since we will leave the form
without having ever reached this node.

But with macros and special-forms, there is no node whose value is the value of
the form, and no node whose source is the macro call or special form.  We can
still detect when we leave the form, but we can't be sure whether this was a
normal evaluation result or an explicit RETURN-FROM.  

But does this really matter?  It seems that we can print the value returned (if
any), then just print the next form to step.  In the rare case where we did
unwind, the user should be able to figure it out.  

[We can look at this as a side-effect of CPS: there isn't any difference
between a "normal" return and a non-local one.]

[Note that in any control transfer (normal or otherwise), the stepper may need
to unwind out of an arbitrary number of levels of stepping.  This is because a
form in a TR position may yield its to a node arbitrarily far our.]

Another problem is with deciding what form is being stepped.  When we start
evaluating a node, we dive into code that is nested somewhere down inside that
form.  So we actually have to do a loop of asking questions before we do any
evaluation.  But what do we ask about?

If we ask about the outermost enclosing form that is a subform of the the last
form that the user said to execute, then we might offer a form that isn't
really evaluated, such as a LET binding list.  

But once again, is this really a problem?  It is certainly different from a
conventional stepper, but a pretty good argument could be made that it is
superior.  Haven't you ever wanted to skip the evaluation of all the
LET bindings, but not the body?  Wouldn't it be useful to be able to skip the
DO step forms?

All of this assumes that nobody ever wants to step through the guts of a
macroexpansion.  This seems reasonable, since steppers are for weenies, and
weenies don't define macros (hence don't debug them).  But there are probably
some weenies who don't know that they shouldn't be writing macros.

We could handle this by finding the "source paths" in the expansion of each
macro by sticking some special frob in the source path marking the place where
the expansion happened.  When we hit code again that is in the source, then we
revert to the normal source path.  Something along these lines might be a good
idea anyway (for compiler error messages, for example).  

The source path hack isn't guaranteed to work quite so well in generated code,
though, since macros return stuff that isn't freshly consed.  But we could
probably arrange to win as long as any given expansion doesn't return two EQ
forms.

It might be nice to have a command that skipped stepping of the form, but
printed the results of each outermost enclosed evaluated subform, i.e. if you
used this on the DO step-list, it would print the result of each new-value
form.  I think this is implementable.  I guess what you would do is print each
value delivered to a DEST whose source form is the current or an enclosing
form.  Along with the value, you would print the source form for the node that
is computing the value.

The stepper can also have a "back" command that "unskips" or "unsteps".  This
would allow the evaluation of forms that are pure (modulo lexical variable
setting) to be undone.  This is useful, since in stepping it is common that you
skip a form that you shouldn't have, or get confused and want to restart at
some earlier point.

What we would do is remember the current node and the values of all local
variables.  heap before doing each step or skip action.  We can then back up
the state of all lexical variables and the "program counter".  To make this
work right with set closure variables, we would copy the cell's value, rather
than the value cell itself.

[To be fair, note that this could easily be done with our current interpreter:
the stepper could copy the environment alists.]

We can't back up the "program counter" when a control transfer leaves the
current function, since this state is implicitly represented in the
interpreter's state, and is discarded when we exit.  We probably want to ask
for confirmation before leaving the function to give users a chance to "unskip"
the forms in a TR position.

Another question is whether the conventional stepper is really a good thing to
imitate...  How about an editor-based mouse-driven interface?  Instead of
"skipping" and "stepping", you would just designate the next form that you
wanted to stop at.  Instead of displaying return values, you replace the source
text with the printed representation of the value.

It would show the "program counter" by highlighting the *innermost* form that
we are about to evaluate, i.e. the source form for the node that we are stopped
at.  It would probably also be useful to display the start of the form that was
used to designate the next stopping point, although I guess this could be
implied by the mouse position.


Such an interface would be a little harder to implement than a dumb stepper,
but it would be much easier to use.  [It would be impossible for an evalhook
stepper to do this.]


%PRIMITIVE usage:

Note: %PRIMITIVE can only be used in compiled code.  It is a trapdoor into the
compiler, not a general syntax for accessing "sub-primitives".  It's main use
is in implementation-dependent compiler transforms.  It saves us the effort of
defining a "phony function" (that is not really defined), and also allows
direct communication with the code generator through codegen-info arguments.

Some primitives may be exported from the VM so that %PRIMITIVE can be used to
make it explicit that an escape routine or interpreter stub is assuming an
operation is implemented by the compiler.

\chapter{The Debugger}
\cindex{debugger}
\label{debugger}

\credits{by Robert MacLachlan}


\section{Debugger Introduction}

The \cmucl{} debugger is unique in its level of support for source-level
debugging of compiled code.  Although some other debuggers allow access of
variables by name, this seems to be the first \llisp{} debugger that:
\begin{itemize}

\item
Tells you when a variable doesn't have a value because it hasn't been
initialized yet or has already been deallocated, or

\item
Can display the precise source location corresponding to a code
location in the debugged program.
\end{itemize}
These features allow the debugging of compiled code to be made almost
indistinguishable from interpreted code debugging.

The debugger is an interactive command loop that allows a user to examine
the function call stack.  The debugger is invoked when:
\begin{itemize}

\item
A \tindexed{serious-condition} is signaled, and it is not handled, or

\item
\findexed{error} is called, and the condition it signals is not handled, or

\item
The debugger is explicitly invoked with the \clisp{} \findexed{break}
or \findexed{debug} functions.
\end{itemize}

{\it Note: there are two debugger interfaces in \cmucl{}: the TTY
debugger (described below) and the Motif debugger. Since the
difference is only in the user interface, much of this chapter also
applies to the Motif version. \xlref{motif-interface} for a very brief
discussion of the graphical interface.}

When you enter the TTY debugger, it looks something like this:

\begin{example}
Error in function CAR.
Wrong type argument, 3, should have been of type LIST.

Restarts:
  0: Return to Top-Level.

Debug  (type H for help)

(CAR 3)
0]
\end{example}

The first group of lines describe what the error was that put us in the
debugger.  In this case \code{car} was called on \code{3}.  After \code{Restarts:}
is a list of all the ways that we can restart execution after this error.  In
this case, the only option is to return to top-level.  After printing its
banner, the debugger prints the current frame and the debugger prompt.


\section{The Command Loop}

The debugger is an interactive read-eval-print loop much like the normal
top-level, but some symbols are interpreted as debugger commands instead
of being evaluated.  A debugger command starts with the symbol name of
the command, possibly followed by some arguments on the same line.  Some
commands prompt for additional input.  Debugger commands can be
abbreviated by any unambiguous prefix: \code{help} can be typed as
\code{h}, \code{he}, etc.  For convenience, some commands have
ambiguous one-letter abbreviations: \code{f} for \code{frame}.

The package is not significant in debugger commands; any symbol with the
name of a debugger command will work.  If you want to show the value of
a variable that happens also to be the name of a debugger command, you
can use the \code{list-locals} command or the \code{debug:var}
function, or you can wrap the variable in a \code{progn} to hide it from
the command loop.

The debugger prompt is ``\var{frame}\code{]}'', where \var{frame} is the number
of the current frame.  Frames are numbered starting from zero at the top (most
recent call), increasing down to the bottom.  The current frame is the frame
that commands refer to.  The current frame also provides the lexical
environment for evaluation of non-command forms.

\cpsubindex{evaluation}{debugger} The debugger evaluates forms in the lexical
environment of the functions being debugged.  The debugger can only
access variables.  You can't \code{go} or \code{return-from} into a
function, and you can't call local functions.  Special variable
references are evaluated with their current value (the innermost binding
around the debugger invocation)\dash{}you don't get the value that the
special had in the current frame.  \xlref{debug-vars} for more
information on debugger variable access.


\section{Stack Frames}
\cindex{stack frames} \cpsubindex{frames}{stack}

A stack frame is the run-time representation of a call to a function;
the frame stores the state that a function needs to remember what it is
doing.  Frames have:
\begin{itemize}

\item
Variables (\pxlref{debug-vars}), which are the values being operated
on, and

\item
Arguments to the call (which are really just particularly interesting
variables), and

\item
A current location (\pxlref{source-locations}), which is the place in
the program where the function was running when it stopped to call another
function, or because of an interrupt or error.
\end{itemize}


\subsection{Stack Motion}

These commands move to a new stack frame and print the name of the function
and the values of its arguments in the style of a Lisp function call:
\begin{Lentry}

\item[\code{up}]
Move up to the next higher frame.  More recent function calls are considered
to be higher on the stack.

\item[\code{down}]
Move down to the next lower frame.

\item[\code{top}]
Move to the highest frame.

\item[\code{bottom}]
Move to the lowest frame.

\item[\code{frame} [\textit{n}]]
Move to the frame with the specified number.  Prompts for the number if not
supplied.

% \key{S} [\var{function-name} [\var{n}]]
% 
% \item
% Search down the stack for function.  Prompts for the function name if not
% supplied.  Searches an optional number of times, but doesn't prompt for
% this number; enter it following the function.
% 
% \item[\key{R} [\var{function-name} [\var{n}]]]
% Search up the stack for function.  Prompts for the function name if not
% supplied.  Searches an optional number of times, but doesn't prompt for
% this number; enter it following the function.
\end{Lentry}


\subsection{How Arguments are Printed}

A frame is printed to look like a function call, but with the actual argument
values in the argument positions.  So the frame for this call in the source:

\begin{lisp}
(myfun (+ 3 4) 'a)
\end{lisp}

would look like this:

\begin{example}
(MYFUN 7 A)
\end{example}

All keyword and optional arguments are displayed with their actual
values; if the corresponding argument was not supplied, the value will
be the default.  So this call:

\begin{lisp}
(subseq "foo" 1)
\end{lisp}

would look like this:

\begin{example}
(SUBSEQ "foo" 1 3)
\end{example}

And this call:

\begin{lisp}
(string-upcase "test case")
\end{lisp}

would look like this:

\begin{example}
(STRING-UPCASE "test case" :START 0 :END NIL)
\end{example}

The arguments to a function call are displayed by accessing the argument
variables.  Although those variables are initialized to the actual argument
values, they can be set inside the function; in this case the new value will be
displayed.

\code{\amprest} arguments are handled somewhat differently.  The value of
the rest argument variable is displayed as the spread-out arguments to
the call, so:

\begin{lisp}
(format t "~A is a ~A." "This" 'test)
\end{lisp}

would look like this:

\begin{example}
(FORMAT T "~A is a ~A." "This" 'TEST)
\end{example}

Rest arguments cause an exception to the normal display of keyword
arguments in functions that have both \code{\amprest} and \code{\&key}
arguments.  In this case, the keyword argument variables are not
displayed at all; the rest arg is displayed instead.  So for these
functions, only the keywords actually supplied will be shown, and the
values displayed will be the argument values, not values of the
(possibly modified) variables.

If the variable for an argument is never referenced by the function, it will be
deleted.  The variable value is then unavailable, so the debugger prints
\code{\#\textless unused-arg\textgreater} instead of the value.  Similarly, if for any of a number of
reasons (described in more detail in section \ref{debug-vars}) the value of the
variable is unavailable or not known to be available, then
\code{\#\textless unavailable-arg\textgreater} will be printed instead of the argument value.

Printing of argument values is controlled by \code{*debug-print-level*} and
\varref{debug-print-length}.

\subsection{Function Names}
\cpsubindex{function}{names}
\cpsubindex{names}{function}

If a function is defined by \code{defun}, \code{labels}, or \code{flet}, then the
debugger will print the actual function name after the open parenthesis, like:

\begin{example}
(STRING-UPCASE "test case" :START 0 :END NIL)
((SETF AREF) \#\back{a} "for" 1)
\end{example}

Otherwise, the function name is a string, and will be printed in quotes:

\begin{example}
("DEFUN MYFUN" BAR)
("DEFMACRO DO" (DO ((I 0 (1+ I))) ((= I 13))) NIL)
("SETQ *GC-NOTIFY-BEFORE*")
\end{example}

This string name is derived from the \w{\code{def}\var{mumble}} form
that encloses or expanded into the lambda, or the outermost enclosing
form if there is no \w{\code{def}\var{mumble}}.

\subsection{Funny Frames}
\cindex{external entry points}
\cpsubindex{entry points}{external}
\cpsubindex{block compilation}{debugger implications}
\cpsubindex{external}{stack frame kind}
\cpsubindex{optional}{stack frame kind}
\cpsubindex{cleanup}{stack frame kind}

Sometimes the evaluator introduces new functions that are used to implement a
user function, but are not directly specified in the source.  The main place
this is done is for checking argument type and syntax.  Usually these functions
do their thing and then go away, and thus are not seen on the stack in the
debugger.  But when you get some sort of error during lambda-list processing,
you end up in the debugger on one of these funny frames.

These funny frames are flagged by printing ``\code{[}\var{keyword}\code{]}'' after the
parentheses.  For example, this call:

\begin{lisp}
(car 'a 'b)
\end{lisp}

will look like this:

\begin{example}
(CAR 2 A) [:EXTERNAL]
\end{example}

And this call:

\begin{lisp}
(string-upcase "test case" :end)
\end{lisp}

would look like this:

\begin{example}
("DEFUN STRING-UPCASE" "test case" 335544424 1) [:OPTIONAL]
\end{example}

As you can see, these frames have only a vague resemblance to the original
call.  Fortunately, the error message displayed when you enter the debugger
will usually tell you what problem is (in these cases, too many arguments
and odd keyword arguments.)  Also, if you go down the stack to the frame for
the calling function, you can display the original source (\pxlref{source-locations}.)

With recursive or block compiled functions
(\pxlref{block-compilation}), an \kwd{EXTERNAL} frame may appear
before the frame representing the first call to the recursive function
or entry to the compiled block. This is a consequence of the way the
compiler does block compilation: there is nothing odd with your
program. You will also see \kwd{CLEANUP} frames during the execution
of \code{unwind-protect} cleanup code. Note that inline expansion and
open-coding affect what frames are present in the debugger, see
sections \ref{debugger-policy} and \ref{open-coding}.


\subsection{Debug Tail Recursion}
\label{debug-tail-recursion}
\cindex{tail recursion}
\cpsubindex{recursion}{tail}

Both the compiler and the interpreter are ``properly tail recursive.''  If a
function call is in a tail-recursive position, the stack frame will be
deallocated {\em at the time of the call}, rather than after the call returns.
Consider this backtrace:
\begin{example}
(BAR ...) 
(FOO ...)
\end{example}
Because of tail recursion, it is not necessarily the case that
\code{FOO} directly called \code{BAR}.  It may be that \code{FOO} called
some other function \code{FOO2} which then called \code{BAR}
tail-recursively, as in this example:
\begin{example}
(defun foo ()
  ...
  (foo2 ...)
  ...)

(defun foo2 (...)
  ...
  (bar ...))

(defun bar (...)
  ...)
\end{example}

Usually the elimination of tail-recursive frames makes debugging more
pleasant, since theses frames are mostly uninformative.  If there is any
doubt about how one function called another, it can usually be
eliminated by finding the source location in the calling frame (section
\ref{source-locations}.)

The elimination of tail-recursive frames can be prevented by disabling
tail-recursion optimization, which happens when the \code{debug}
optimization quality is greater than \code{2}
(\pxlref{debugger-policy}.)

For a more thorough discussion of tail recursion, \pxlref{tail-recursion}.


\subsection{Unknown Locations and Interrupts}
\label{unknown-locations}
\cindex{unknown code locations}
\cpsubindex{locations}{unknown}
\cindex{interrupts}
\cpsubindex{errors}{run-time}

The debugger operates using special debugging information attached to
the compiled code.  This debug information tells the debugger what it
needs to know about the locations in the code where the debugger can be
invoked.  If the debugger somehow encounters a location not described in
the debug information, then it is said to be \var{unknown}.  If the code
location for a frame is unknown, then some variables may be
inaccessible, and the source location cannot be precisely displayed.

There are three reasons why a code location could be unknown:
\begin{itemize}

\item
There is inadequate debug information due to the value of the \code{debug}
optimization quality.  \xlref{debugger-policy}.

\item
The debugger was entered because of an interrupt such as \code{$\hat{ }C$}.

\item
A hardware error such as ``\code{bus error}'' occurred in code that was
compiled unsafely due to the value of the \code{safety} optimization
quality.  \xlref{optimize-declaration}.
\end{itemize}

In the last two cases, the values of argument variables are accessible,
but may be incorrect.  \xlref{debug-var-validity} for more details on
when variable values are accessible.

It is possible for an interrupt to happen when a function call or return is in
progress.  The debugger may then flame out with some obscure error or insist
that the bottom of the stack has been reached, when the real problem is that
the current stack frame can't be located.  If this happens, return from the
interrupt and try again.

When running interpreted code, all locations should be known.  However,
an interrupt might catch some subfunction of the interpreter at an
unknown location.  In this case, you should be able to go up the stack a
frame or two and reach an interpreted frame which can be debugged.


\section{Variable Access}
\label{debug-vars}
\cpsubindex{variables}{debugger access}
\cindex{debug variables}

There are three ways to access the current frame's local variables in the
debugger.  The simplest is to type the variable's name into the debugger's
read-eval-print loop.  The debugger will evaluate the variable reference as
though it had appeared inside that frame.

The debugger doesn't really understand lexical scoping; it has just one
namespace for all the variables in a function.  If a symbol is the name of
multiple variables in the same function, then the reference appears ambiguous,
even though lexical scoping specifies which value is visible at any given
source location.  If the scopes of the two variables are not nested, then the
debugger can resolve the ambiguity by observing that only one variable is
accessible.

When there are ambiguous variables, the evaluator assigns each one a
small integer identifier.  The \code{debug:var} function and the
\code{list-locals} command use this identifier to distinguish between
ambiguous variables:
\begin{Lentry}

\item[\code{list-locals} \mopt{\var{prefix}}]%%\hfill\\
This command prints the name and value of all variables in the current
frame whose name has the specified \var{prefix}.  \var{prefix} may be a
string or a symbol.  If no \var{prefix} is given, then all available
variables are printed.  If a variable has a potentially ambiguous name,
then the name is printed with a ``\code{\#}\var{identifier}'' suffix, where
\var{identifier} is the small integer used to make the name unique.
\end{Lentry}

\begin{defun}{debug:}{var}{\args{\var{name} \ampoptional{} \var{identifier}}}
  
  This function returns the value of the variable in the current frame
  with the specified \var{name}.  If supplied, \var{identifier}
  determines which value to return when there are ambiguous variables.
  
  When \var{name} is a symbol, it is interpreted as the symbol name of
  the variable, i.e. the package is significant.  If \var{name} is an
  uninterned symbol (gensym), then return the value of the uninterned
  variable with the same name.  If \var{name} is a string,
  \code{debug:var} interprets it as the prefix of a variable name, and
  must unambiguously complete to the name of a valid variable.
  
  This function is useful mainly for accessing the value of uninterned
  or ambiguous variables, since most variables can be evaluated
  directly.
\end{defun}


\subsection{Variable Value Availability}
\label{debug-var-validity}
\cindex{availability of debug variables}
\cindex{validity of debug variables}
\cindex{debug optimization quality}

The value of a variable may be unavailable to the debugger in portions of the
program where \clisp{} says that the variable is defined.  If a variable value is
not available, the debugger will not let you read or write that variable.  With
one exception, the debugger will never display an incorrect value for a
variable.  Rather than displaying incorrect values, the debugger tells you the
value is unavailable.

The one exception is this: if you interrupt (e.g., with \code{$\hat{ }C$}) or if there is
an unexpected hardware error such as ``\code{bus error}'' (which should only happen
in unsafe code), then the values displayed for arguments to the interrupted
frame might be incorrect.\footnote{Since the location of an interrupt or hardware
error will always be an unknown location (\pxlref{unknown-locations}),
non-argument variable values will never be available in the interrupted frame.}
This exception applies only to the interrupted frame: any frame farther down
the stack will be fine.

The value of a variable may be unavailable for these reasons:
\begin{itemize}

\item
The value of the \code{debug} optimization quality may have omitted debug
information needed to determine whether the variable is available.
Unless a variable is an argument, its value will only be available when
\code{debug} is at least \code{2}.

\item
The compiler did lifetime analysis and determined that the value was no longer
needed, even though its scope had not been exited.  Lifetime analysis is
inhibited when the \code{debug} optimization quality is \code{3}.

\item
The variable's name is an uninterned symbol (gensym).  To save space, the
compiler only dumps debug information about uninterned variables when the
\code{debug} optimization quality is \code{3}.

\item
The frame's location is unknown (\pxlref{unknown-locations}) because
the debugger was entered due to an interrupt or unexpected hardware error.
Under these conditions the values of arguments will be available, but might be
incorrect.  This is the exception above.

\item
The variable was optimized out of existence.  Variables with no reads are
always optimized away, even in the interpreter.  The degree to which the
compiler deletes variables will depend on the value of the \code{compile-speed}
optimization quality, but most source-level optimizations are done under all
compilation policies.
\end{itemize}


Since it is especially useful to be able to get the arguments to a function,
argument variables are treated specially when the \code{speed} optimization
quality is less than \code{3} and the \code{debug} quality is at least \code{1}.
With this compilation policy, the values of argument variables are almost
always available everywhere in the function, even at unknown locations.  For
non-argument variables, \code{debug} must be at least \code{2} for values to be
available, and even then, values are only available at known locations.


\subsection{Note On Lexical Variable Access}
\cpsubindex{evaluation}{debugger}
 
When the debugger command loop establishes variable bindings for available
variables, these variable bindings have lexical scope and dynamic
extent.\footnote{The variable bindings are actually created using the \clisp{}
\code{symbol-macrolet} special form.}  You can close over them, but such closures
can't be used as upward funargs.

You can also set local variables using \code{setq}, but if the variable was closed
over in the original source and never set, then setting the variable in the
debugger may not change the value in all the functions the variable is defined
in.  Another risk of setting variables is that you may assign a value of a type
that the compiler proved the variable could never take on.  This may result in
bad things happening.


\section{Source Location Printing}
\label{source-locations}
\cpsubindex{source location printing}{debugger}

One of \cmucl{}'s unique capabilities is source level debugging of compiled
code.  These commands display the source location for the current frame:
\begin{Lentry}

\item[\code{source} \mopt{\var{context}}]%%\hfill\\
This command displays the file that the current frame's function was defined
from (if it was defined from a file), and then the source form responsible for
generating the code that the current frame was executing.  If \var{context} is
specified, then it is an integer specifying the number of enclosing levels of
list structure to print.

\item[\code{vsource} \mopt{\var{context}}]%%\hfill\\
This command is identical to \code{source}, except that it uses the
global values of \code{*print-level*} and \code{*print-length*} instead
of the debugger printing control variables \code{*debug-print-level*}
and \code{*debug-print-length*}.
\end{Lentry}

The source form for a location in the code is the innermost list present
in the original source that encloses the form responsible for generating
that code.  If the actual source form is not a list, then some enclosing
list will be printed.  For example, if the source form was a reference
to the variable \code{*some-random-special*}, then the innermost
enclosing evaluated form will be printed.  Here are some possible
enclosing forms:
\begin{example}
(let ((a *some-random-special*))
  ...)

(+ *some-random-special* ...)
\end{example}

If the code at a location was generated from the expansion of a macro or a
source-level compiler optimization, then the form in the original source that
expanded into that code will be printed.  Suppose the file
\file{/usr/me/mystuff.lisp} looked like this:
\begin{example}
(defmacro mymac ()
  '(myfun))

(defun foo ()
  (mymac)
  ...)
\end{example}
If \code{foo} has called \code{myfun}, and is waiting for it to return, then the
\code{source} command would print:
\begin{example}
; File: /usr/me/mystuff.lisp

(MYMAC)
\end{example}
Note that the macro use was printed, not the actual function call form,
\code{(myfun)}.

If enclosing source is printed by giving an argument to \code{source} or
\code{vsource}, then the actual source form is marked by wrapping it in a list
whose first element is \code{\#:***HERE***}.  In the previous example, 
\w{\code{source 1}} would print:
\begin{example}
; File: /usr/me/mystuff.lisp

(DEFUN FOO ()
  (#:***HERE***
   (MYMAC))
  ...)
\end{example}


\subsection{How the Source is Found}

If the code was defined from \llisp{} by \code{compile} or
\code{eval}, then the source can always be reliably located.  If the
code was defined from a \code{fasl} file created by
\findexed{compile-file}, then the debugger gets the source forms it
prints by reading them from the original source file.  This is a
potential problem, since the source file might have moved or changed
since the time it was compiled.

The source file is opened using the \code{truename} of the source file
pathname originally given to the compiler.  This is an absolute pathname
with all logical names and symbolic links expanded.  If the file can't
be located using this name, then the debugger gives up and signals an
error.

If the source file can be found, but has been modified since the time it was
compiled, the debugger prints this warning:
\begin{example}
; File has been modified since compilation:
;   \var{filename}
; Using form offset instead of character position.
\end{example}
where \var{filename} is the name of the source file.  It then proceeds using a
robust but not foolproof heuristic for locating the source.  This heuristic
works if:
\begin{itemize}

\item
No top-level forms before the top-level form containing the source have been
added or deleted, and

\item
The top-level form containing the source has not been modified much.  (More
precisely, none of the list forms beginning before the source form have been
added or deleted.)
\end{itemize}

If the heuristic doesn't work, the displayed source will be wrong, but will
probably be near the actual source.  If the ``shape'' of the top-level form in
the source file is too different from the original form, then an error will be
signaled.  When the heuristic is used, the the source location commands are
noticeably slowed.

Source location printing can also be confused if (after the source was
compiled) a read-macro you used in the code was redefined to expand into
something different, or if a read-macro ever returns the same \code{eq}
list twice.  If you don't define read macros and don't use \code{\#\#} in
perverted ways, you don't need to worry about this.


\subsection{Source Location Availability}

\cindex{debug optimization quality}
Source location information is only available when the \code{debug}
optimization quality is at least \code{2}.  If source location information is
unavailable, the source commands will give an error message.

If source location information is available, but the source location is
unknown because of an interrupt or unexpected hardware error
(\pxlref{unknown-locations}), then the command will print:

\begin{example}
Unknown location: using block start.
\end{example}

and then proceed to print the source location for the start of the
{\em basic block} enclosing the code location.
\cpsubindex{block}{basic} \cpsubindex{block}{start location} 
It's a bit complicated to explain exactly what a basic block is, but
here are some properties of the block start location:

\begin{itemize}
  
\item The block start location may be the same as the true location.
  
\item The block start location will never be later in the the
  program's flow of control than the true location.
  
\item No conditional control structures (such as \code{if},
  \code{cond}, \code{or}) will intervene between the block start and
  the true location (but note that some conditionals present in the
  original source could be optimized away.)  Function calls {\em do not}
  end basic blocks.
  
\item The head of a loop will be the start of a block.
  
\item The programming language concept of ``block structure'' and the
  \clisp{} \code{block} special form are totally unrelated to the
  compiler's basic block.
\end{itemize}

In other words, the true location lies between the printed location and the
next conditional (but watch out because the compiler may have changed the
program on you.)


\section{Compiler Policy Control}
\label{debugger-policy}
\cpsubindex{policy}{debugger}
\cindex{debug optimization quality}
\cindex{optimize declaration}

The compilation policy specified by \code{optimize} declarations affects the
behavior seen in the debugger.  The \code{debug} quality directly affects the
debugger by controlling the amount of debugger information dumped.  Other
optimization qualities have indirect but observable effects due to changes in
the way compilation is done.

Unlike the other optimization qualities (which are compared in relative value
to evaluate tradeoffs), the \code{debug} optimization quality is directly
translated to a level of debug information.  This absolute interpretation
allows the user to count on a particular amount of debug information being
available even when the values of the other qualities are changed during
compilation.  These are the levels of debug information that correspond to the
values of the \code{debug} quality:
\begin{Lentry}

\item[\code{0}]
Only the function name and enough information to allow the stack to
be parsed.

\item[\code{\w{$>$ 0}}]
Any level greater than \code{0} gives level \code{0} plus all
argument variables.  Values will only be accessible if the argument
variable is never set and
\code{speed} is not \code{3}.  \cmucl{} allows any real value for optimization
qualities.  It may be useful to specify \code{0.5} to get backtrace argument
display without argument documentation.

\item[\code{1}] Level \code{1} provides argument documentation
(printed arglists) and derived argument/result type information.
This makes \findexed{describe} more informative, and allows the
compiler to do compile-time argument count and type checking for any
calls compiled at run-time.

\item[\code{2}]
Level \code{1} plus all interned local variables, source location
information, and lifetime information that tells the debugger when arguments
are available (even when \code{speed} is \code{3} or the argument is set.)  This is
the default.

\item[\code{\w{$>$ 2}}]
Any level greater than \code{2} gives level \code{2} and in addition
disables tail-call optimization, so that the backtrace will contain
frames for all invoked functions, even those in tail positions.

\item[\code{3}]
Level \code{2} plus all uninterned variables.  In addition, lifetime
analysis is disabled (even when \code{speed} is \code{3}), ensuring
that all variable values are available at any known location within
the scope of the binding.  This has a speed penalty in addition to the
obvious space penalty. 
\end{Lentry}

As you can see, if the \code{speed} quality is \code{3}, debugger performance is
degraded.  This effect comes from the elimination of argument variable
special-casing (\pxlref{debug-var-validity}.)  Some degree of
speed/debuggability tradeoff is unavoidable, but the effect is not too drastic
when \code{debug} is at least \code{2}.

\cindex{inline expansion}
\cindex{semi-inline expansion}
In addition to \code{inline} and \code{notinline} declarations, the relative values
of the \code{speed} and \code{space} qualities also change whether functions are
inline expanded (\pxlref{inline-expansion}.)  If a function is inline
expanded, then there will be no frame to represent the call, and the arguments
will be treated like any other local variable.  Functions may also be
``semi-inline'', in which case there is a frame to represent the call, but the
call is to an optimized local version of the function, not to the original
function.


\section{Exiting Commands}

These commands get you out of the debugger.

\begin{Lentry}

\item[\code{quit}]
Throw to top level.

\item[\code{restart} \mopt{\var{n}}]%%\hfill\\
Invokes the \var{n}th restart case as displayed by the \code{error}
command.  If \var{n} is not specified, the available restart cases are
reported.

\item[\code{go}]
Calls \code{continue} on the condition given to \code{debug}.  If there is no
restart case named \var{continue}, then an error is signaled.

\item[\code{abort}]
Calls \code{abort} on the condition given to \code{debug}.  This is
useful for popping debug command loop levels or aborting to top level,
as the case may be.

% (\code{debug:debug-return} \var{expression} \mopt{\var{frame}})
% 
% \item
% From the current or specified frame, return the result of evaluating
% expression.  If multiple values are expected, then this function should be
% called for multiple values.
\end{Lentry}


\section{Information Commands}

Most of these commands print information about the current frame or
function, but a few show general information.

\begin{Lentry}

\item[\code{help}, \code{?}]
Displays a synopsis of debugger commands.

\item[\code{describe}]
Calls \code{describe} on the current function, displays number of local
variables, and indicates whether the function is compiled or interpreted.

\item[\code{print}]
Displays the current function call as it would be displayed by moving to
this frame.

\item[\code{vprint} (or \code{pp}) \mopt{\var{verbosity}}]%%\hfill\\
Displays the current function call using \code{*print-level*} and
\code{*print-length*} instead of \code{*debug-print-level*} and
\code{*debug-print-length*}.  \var{verbosity} is a small integer
(default 2) that controls other dimensions of verbosity.

\item[\code{error}]
Prints the condition given to \code{invoke-debugger} and the active
proceed cases.

\item[\code{backtrace} \mopt{\var{n}}]\hfill\\
Displays all the frames from the current to the bottom.  Only shows
\var{n} frames if specified.  The printing is controlled by
\code{*debug-print-level*} and \code{*debug-print-length*}.

% (\code{debug:debug-function} \mopt{\var{n}})
% 
% \item
% Returns the function from the current or specified frame.
% 
% \item[(\code{debug:function-name} \mopt{\var{n}])]
% Returns the function name from the current or specified frame.
% 
% \item[(\code{debug:pc} \mopt{\var{frame}})]
% Returns the index of the instruction for the function in the current or
% specified frame.  This is useful in conjunction with \code{disassemble}.
% The pc returned points to the instruction after the one that was fatal.
\end{Lentry}


\section{Breakpoint Commands}\cindex{breakpoints}

\cmucl{} supports setting of breakpoints inside compiled functions and
stepping of compiled code.  Breakpoints can only be set at at known
locations (\pxlref{unknown-locations}), so these commands are largely
useless unless the \code{debug} optimize quality is at least \code{2}
(\pxlref{debugger-policy}).  These commands manipulate breakpoints:
\begin{Lentry}
\item[\code{breakpoint} \var{location} \mstar{\var{option} \var{value}}]
%%\hfill\\
Set a breakpoint in some function.  \var{location} may be an integer
code location number (as displayed by \code{list-locations}) or a
keyword.  The keyword can be used to indicate setting a breakpoint at
the function start (\kwd{start}, \kwd{s}) or function end
(\kwd{end}, \kwd{e}).  The \code{breakpoint} command has
\kwd{condition}, \kwd{break}, \kwd{print} and \kwd{function}
options which work similarly to the \code{trace} options.

\item[\code{list-locations} (or \code{ll}) \mopt{\var{function}}]%%\hfill\\
List all the code locations in the current frame's function, or in
\var{function} if it is supplied.  The display format is the code
location number, a colon and then the source form for that location:
\begin{example}
3: (1- N)
\end{example}
If consecutive locations have the same source, then a numeric range like
\code{3-5:} will be printed.  For example, a default function call has a
known location both immediately before and after the call, which would
result in two code locations with the same source.  The listed function
becomes the new default function for breakpoint setting (via the
\code{breakpoint}) command.

\item[\code{list-breakpoints} (or \code{lb})]%%\hfill\\
List all currently active breakpoints with their breakpoint number.

\item[\code{delete-breakpoint} (or \code{db}) \mopt{\var{number}}]%%\hfill\\
Delete a breakpoint specified by its breakpoint number.  If no number is
specified, delete all breakpoints.

\item[\code{step}]%%\hfill\\
Step to the next possible breakpoint location in the current function.
This always steps over function calls, instead of stepping into them
\end{Lentry}


\subsection{Breakpoint Example}

Consider this definition of the factorial function:

\begin{lisp}
(defun ! (n)
  (if (zerop n)
      1
      (* n (! (1- n)))))
\end{lisp}

This debugger session demonstrates the use of breakpoints:

\begin{example}
common-lisp-user> (break) ; Invoke debugger

Break

Restarts:
  0: [CONTINUE] Return from BREAK.
  1: [ABORT   ] Return to Top-Level.

Debug  (type H for help)

(INTERACTIVE-EVAL (BREAK))
0] ll #'!
0: #'(LAMBDA (N) (BLOCK ! (IF # 1 #)))
1: (ZEROP N)
2: (* N (! (1- N)))
3: (1- N)
4: (! (1- N))
5: (* N (! (1- N)))
6: #'(LAMBDA (N) (BLOCK ! (IF # 1 #)))
0] br 2
(* N (! (1- N)))
1: 2 in !
Added.
0] q

common-lisp-user> (! 10) ; Call the function

*Breakpoint hit*

Restarts:
  0: [CONTINUE] Return from BREAK.
  1: [ABORT   ] Return to Top-Level.

Debug  (type H for help)

(! 10) ; We are now in first call (arg 10) before the multiply
Source: (* N (! (1- N)))
3] st

*Step*

(! 10) ; We have finished evaluation of (1- n)
Source: (1- N)
3] st

*Breakpoint hit*

Restarts:
  0: [CONTINUE] Return from BREAK.
  1: [ABORT   ] Return to Top-Level.

Debug  (type H for help)

(! 9) ; We hit the breakpoint in the recursive call
Source: (* N (! (1- N)))
3] 
\end{example}


\section{Function Tracing}
\cindex{tracing}
\cpsubindex{function}{tracing}

The tracer causes selected functions to print their arguments and
their results whenever they are called.  Options allow conditional
printing of the trace information and conditional breakpoints on
function entry or exit.

\begin{defmac}{}{trace}{%
    \args{\mstar{option global-value} \mstar{name \mstar{option
          value}}}}
  
  \code{trace} is a debugging tool that prints information when
  specified functions are called.  In its simplest form:
  \begin{example}
    (trace \var{name-1} \var{name-2} ...)
  \end{example}
  \code{trace} causes a printout on \vindexed{trace-output} each time
  that one of the named functions is entered or returns (the
  \var{names} are not evaluated.)  Trace output is indented according
  to the number of pending traced calls, and this trace depth is
  printed at the beginning of each line of output.  Printing verbosity
  of arguments and return values is controlled by
  \vindexed{debug-print-level} and \vindexed{debug-print-length}.

  Local functions defined by \code{flet} and \code{labels} can be
  traced using the syntax \code{(flet f f1 f2 ...)} or \code{(labels f
    f1 f2 ...)} where \code{f} is the \code{flet} or \code{labels}
  function we want to trace and \code{f1}, \code{f2}, are the
  functions containing the local function \code{f}.
  Invidiual methods can also be traced using the syntax \code{(method
    <name> <qualifiers> <specializers>)}.
  See~\ref{sec:method-tracing} for more information.

  If no \var{names} or \var{options} are are given, \code{trace}
  returns the list of all currently traced functions,
  \code{*traced-function-list*}.
  
  Trace options can cause the normal printout to be suppressed, or
  cause extra information to be printed.  Each option is a pair of an
  option keyword and a value form.  Options may be interspersed with
  function names.  Options only affect tracing of the function whose
  name they appear immediately after.  Global options are specified
  before the first name, and affect all functions traced by a given
  use of \code{trace}.  If an already traced function is traced again,
  any new options replace the old options.  The following options are
  defined:
  \begin{Lentry}
  \item[\kwd{condition} \var{form}, \kwd{condition-after} \var{form},
    \kwd{condition-all} \var{form}] If \kwd{condition} is specified,
    then \code{trace} does nothing unless \var{form} evaluates to true
    at the time of the call.  \kwd{condition-after} is similar, but
    suppresses the initial printout, and is tested when the function
    returns.  \kwd{condition-all} tries both before and after.
    
  \item[\kwd{wherein} \var{names}] If specified, \var{names} is a
    function name or list of names.  \code{trace} does nothing unless
    a call to one of those functions encloses the call to this
    function (i.e. it would appear in a backtrace.)  Anonymous
    functions have string names like \code{"DEFUN FOO"}.  Individual
    methods can also be traced.  See section~\ref{sec:method-tracing}.

  \item[\kwd{wherein-only} \var{names}] If specified, this is just
    like \kwd{wherein}, but trace produces output only if the
    immediate caller of the traced function is one of the functions
    listed in \var{names}.
    
  \item[\kwd{break} \var{form}, \kwd{break-after} \var{form},
    \kwd{break-all} \var{form}] If specified, and \var{form} evaluates
    to true, then the debugger is invoked at the start of the
    function, at the end of the function, or both, according to the
    respective option.
    
  \item[\kwd{print} \var{form}, \kwd{print-after} \var{form},
    \kwd{print-all} \var{form}] In addition to the usual printout, the
    result of evaluating \var{form} is printed at the start of the
    function, at the end of the function, or both, according to the
    respective option.  Multiple print options cause multiple values
    to be printed.
    
  \item[\kwd{function} \var{function-form}] This is a not really an
    option, but rather another way of specifying what function to
    trace.  The \var{function-form} is evaluated immediately, and the
    resulting function is traced.
    
  \item[\kwd{encapsulate \mgroup{:default | t | nil}}] In \cmucl,
    tracing can be done either by temporarily redefining the function
    name (encapsulation), or using breakpoints.  When breakpoints are
    used, the function object itself is destructively modified to
    cause the tracing action.  The advantage of using breakpoints is
    that tracing works even when the function is anonymously called
    via \code{funcall}.
  
    When \kwd{encapsulate} is true, tracing is done via encapsulation.
    \kwd{default} is the default, and means to use encapsulation for
    interpreted functions and funcallable instances, breakpoints
    otherwise.  When encapsulation is used, forms are {\it not}
    evaluated in the function's lexical environment, but
    \code{debug:arg} can still be used.

    Note that if you trace using \kwd{encapsulate}, you will
    only get a trace or breakpoint at the outermost call to the traced
    function, not on recursive calls.

  \end{Lentry}

  In the case of functions where the known return convention is used
  to optimize, encapsulation may be necessary in order to make
  tracing work at all.  The symptom of this occurring is an error
  stating
  \begin{example}
    Error in function \var{foo}: :FUNCTION-END breakpoints are
    currently unsupported for the known return convention.
  \end{example}
  in such cases we recommend using \code{(trace \var{foo} :encapsulate
    t)}
  
  \cpsubindex{tracing}{errors}
  \cpsubindex{breakpoints}{errors}
  \cpsubindex{errors}{breakpoints}
  \cindex{function-end breakpoints}
  \cpsubindex{breakpoints}{function-end}
    

  
  \kwd{condition}, \kwd{break} and \kwd{print} forms are evaluated in
  the lexical environment of the called function; \code{debug:var} and
  \code{debug:arg} can be used.  The \code{-after} and \code{-all}
  forms are evaluated in the null environment.
\end{defmac}

\begin{defmac}{}{untrace}{ \args{\amprest{} \var{function-names}}}
  
  This macro turns off tracing for the specified functions, and
  removes their names from \code{*traced-function-list*}.  If no
  \var{function-names} are given, then all currently traced functions
  are untraced.
\end{defmac}

\begin{defvar}{extensions:}{traced-function-list}
  
  A list of function names maintained and used by \code{trace},
  \code{untrace}, and \code{untrace-all}.  This list should contain
  the names of all functions currently being traced.
\end{defvar}

\begin{defvar}{extensions:}{max-trace-indentation}
  
  The maximum number of spaces which should be used to indent trace
  printout.  This variable is initially set to 40.
\end{defvar}

\begin{defvar}{debug:}{trace-encapsulate-package-names}
  
  A list of package names.  Functions from these packages are traced
  using encapsulation instead of function-end breakpoints.  This list
  should at least include those packages containing functions used
  directly or indirectly in the implementation of \code{trace}.
\end{defvar}


\subsection{Encapsulation Functions}
\cindex{encapsulation}
\cindex{advising}

The encapsulation functions provide a mechanism for intercepting the
arguments and results of a function.  \code{encapsulate} changes the
function definition of a symbol, and saves it so that it can be
restored later.  The new definition normally calls the original
definition.  The \clisp{} \findexed{fdefinition} function always returns
the original definition, stripping off any encapsulation.

The original definition of the symbol can be restored at any time by
the \code{unencapsulate} function.  \code{encapsulate} and \code{unencapsulate}
allow a symbol to be multiply encapsulated in such a way that different
encapsulations can be completely transparent to each other.

Each encapsulation has a type which may be an arbitrary lisp object.
If a symbol has several encapsulations of different types, then any
one of them can be removed without affecting more recent ones.
A symbol may have more than one encapsulation of the same type, but
only the most recent one can be undone.

\begin{defun}{extensions:}{encapsulate}{%
    \args{\var{symbol} \var{type} \var{body}}}
  
  Saves the current definition of \var{symbol}, and replaces it with a
  function which returns the result of evaluating the form,
  \var{body}.  \var{Type} is an arbitrary lisp object which is the
  type of encapsulation.
  
  When the new function is called, the following variables are bound
  for the evaluation of \var{body}:
  \begin{Lentry}
    
  \item[\code{extensions:argument-list}] A list of the arguments to
    the function.
    
  \item[\code{extensions:basic-definition}] The unencapsulated
    definition of the function.
  \end{Lentry}
  The unencapsulated definition may be called with the original
  arguments by including the form
  \begin{lisp}
    (apply extensions:basic-definition extensions:argument-list)
  \end{lisp}

  \code{encapsulate} always returns \var{symbol}.
\end{defun}

\begin{defun}{extensions:}{unencapsulate}{\args{\var{symbol} \var{type}}}
  
  Undoes \var{symbol}'s most recent encapsulation of type \var{type}.
  \var{Type} is compared with \code{eq}.  Encapsulations of other
  types are left in place.
\end{defun}

\begin{defun}{extensions:}{encapsulated-p}{%
    \args{\var{symbol} \var{type}}}
  
  Returns \true{} if \var{symbol} has an encapsulation of type
  \var{type}.  Returns \nil{} otherwise.  \var{type} is compared with
  \code{eq}.
\end{defun}

% section{The Single Stepper}
% 
% \begin{defmac}{}{step}{ \args{\var{form}}}
%   
%   Evaluates form with single stepping enabled or if \var{form} is
%   \code{T}, enables stepping until explicitly disabled.  Stepping can
%   be disabled by quitting to the lisp top level, or by evaluating the
%   form \w{\code{(step ())}}.
%   
%   While stepping is enabled, every call to eval will prompt the user
%   for a single character command.  The prompt is the form which is
%   about to be \code{eval}ed.  It is printed with \code{*print-level*}
%   and \code{*print-length*} bound to \code{*step-print-level*} and
%   \code{*step-print-length*}.  All interaction is done through the
%   stream \code{*query-io*}.  Because of this, the stepper can not be
%   used in Hemlock eval mode.  When connected to a slave Lisp, the
%   stepper can be used from Hemlock.
%   
%   The commands are:
%   \begin{Lentry}
%   
%   \item[\key{n} (next)] Evaluate the expression with stepping still
%     enabled.
%   
%   \item[\key{s} (skip)] Evaluate the expression with stepping
%     disabled.
%   
%   \item[\key{q} (quit)] Evaluate the expression, but disable all
%     further stepping inside the current call to \code{step}.
%   
%   \item[\key{p} (print)] Print current form.  (does not use
%     \code{*step-print-level*} or \code{*step-print-length*}.)
%   
%   \item[\key{b} (break)] Enter break loop, and then prompt for the
%     command again when the break loop returns.
%   
%   \item[\key{e} (eval)] Prompt for and evaluate an arbitrary
%     expression.  The expression is evaluated with stepping disabled.
%   
%   \item[\key{?} (help)] Prints a brief list of the commands.
%   
%   \item[\key{r} (return)] Prompt for an arbitrary value to return as
%     result of the current call to eval.
%   
%   \item[\key{g}] Throw to top level.
%   \end{Lentry}
% \end{defmac}
% 
% \begin{defvar}{extensions:}{step-print-level}
%   \defvarx[extensions:]{step-print-length}
%   
%   \code{*print-level*} and \code{*print-length*} are bound to these
%   values while printing the current form.  \code{*step-print-level*}
%   and \code{*step-print-length*} are initially bound to 4 and 5,
%   respectively.
% \end{defvar}
% 
% \begin{defvar}{extensions:}{max-step-indentation}
%   
%   Step indents the prompts to highlight the nesting of the evaluation.
%   This variable contains the maximum number of spaces to use for
%   indenting.  Initially set to 40.
% \end{defvar}


\section{Specials}
These are the special variables that control the debugger action.

\begin{defvar}{debug:}{debug-print-level}
  \defvarx[debug:]{debug-print-length}
  
  \code{*print-level*} and \code{*print-length*} are bound to these
  values during the execution of some debug commands.  When evaluating
  arbitrary expressions in the debugger, the normal values of
  \code{*print-level*} and \code{*print-length*} are in effect.  These
  variables are initially set to 3 and 5, respectively.
\end{defvar}

\chapter{Object Format}



\label{sec:tagging}

\section{Tagging}

The following is a key of the three bit low-tagging scheme:
\begin{description}
   \item[000] even fixnum
   \item[001] function pointer
   \item[010] even other-immediate (header-words, characters, symbol-value trap value, etc.)
   \item[011] list pointer
   \item[100] odd fixnum
   \item[101] structure pointer
   \item[110] odd other immediate
  \item[111] other-pointer to data-blocks (other than conses, structures,
                                     and functions)
\end{description}

This tagging scheme forces a dual-word alignment of data-blocks on the heap,
but this can be pretty negligible: 
\begin{itemize}
\item   RATIOS and COMPLEX must have a header-word anyway since they are not a
      major type.  This wastes one word for these infrequent data-blocks since
      they require two words for the data.

\item BIGNUMS must have a header-word and probably contain only one other word
      anyway, so we probably don't waste any words here.  Most bignums just
      barely overflow fixnums, that is by a bit or two.

\item   Single and double FLOATS?
      no waste, or
      one word wasted

\item   SYMBOLS have a pad slot (current called the setf function, but unused.)
\end{itemize}
Everything else is vector-like including code, so these probably take up
so many words that one extra one doesn't matter.



\section{GC Comments}

Data-Blocks comprise only descriptors, or they contain immediate data and raw
bits interpreted by the system.  GC must skip the latter when scanning the
heap, so it does not look at a word of raw bits and interpret it as a pointer
descriptor.  These data-blocks require headers for GC as well as for operations
that need to know how to interpret the raw bits.  When GC is scanning, and it
sees a header-word, then it can determine how to skip that data-block if
necessary.  Header-Words are tagged as other-immediates.  See 
``Other-Immediates'', section~\ref{sec:other-immediates} and
``Data-Blocks and Header-Words'', section~\ref{sec:data-blocks-and-header} for comments on
distinguishing header-words from other-immediate data.  This distinction is
necessary since we scan through data-blocks containing only descriptors just as
we scan through the heap looking for header-words introducing data-blocks.

Data-Blocks containing only descriptors do not require header-words for GC
since the entire data-block can be scanned by GC a word at a time, taking
whatever action is necessary or appropriate for the data in that slot.  For
example, a cons is referenced by a descriptor with a specific tag, and the
system always knows the size of this data-block.  When GC encounters a pointer
to a cons, it can transport it into the new space, and when scanning, it can
simply scan the two words manifesting the cons interpreting each word as a
descriptor.  Actually there is no cons tag, but a list tag, so we make sure the
cons is not nil when appropriate.  A header may still be desired if the pointer
to the data-block does not contain enough information to adequately maintain
the data-block.  An example of this is a simple-vector containing only
descriptor slots, and we attach a header-word because the descriptor pointing
to the vector lacks necessary information -- the type of the vector's elements,
its length, etc.

There is no need for a major tag for GC forwarding pointers.  Since the tag
bits are in the low end of the word, a range check on the start and end of old
space tells you if you need to move the thing.  This is all GC overhead.



\section{Structures}

A structure descriptor has the structure lowtag type code, making 
{\tt structurep} a fast operation.  A structure
data-block has the following format:
\begin{verbatim}
    -------------------------------------------------------
    |   length (24 bits) | Structure header type (8 bits) |
    -------------------------------------------------------
    |   structure type name (a symbol)                    |
    -------------------------------------------------------
    |   structure slot 0                                  |
    -------------------------------------------------------
    |   ... structure slot length - 2                     |
    -------------------------------------------------------
\end{verbatim}

The header word contains the structure length, which is the number of words
(other than the header word.)  The length is always at least one, since the
first word of the structure data is the structure type name.


\section{Fixnums}

A fixnum has one of the following formats in 32 bits:
\begin{verbatim}
    -------------------------------------------------------
    |        30 bit 2's complement even integer   | 0 0 0 |
    -------------------------------------------------------
\end{verbatim}
or
\begin{verbatim}
    -------------------------------------------------------
    |        30 bit 2's complement odd integer    | 1 0 0 |
    -------------------------------------------------------
\end{verbatim}

Effectively, there is one tag for immediate integers, two zeros.  This buys one
more bit for fixnums, and now when these numbers index into simple-vectors or
offset into memory, they point to word boundaries on 32-bit, byte-addressable
machines.  That is, no shifting need occur to use the number directly as an
offset.

This format has another advantage on byte-addressable machines when fixnums are
offsets into vector-like data-blocks, including structures.  Even though we
previously mentioned data-blocks are dual-word aligned, most indexing and slot
accessing is word aligned, and so are fixnums with effectively two tag bits.

Two tags also allow better usage of special instructions on some machines that
can deal with two low-tag bits but not three.

Since the two bits are zeros, we avoid having to mask them off before using the
words for arithmetic, but division and multiplication require special shifting.



\section{Other-immediates}
\label{sec:other-immediates}



As for fixnums, there are two different three-bit lowtag codes for
other-immediate, allowing 64 other-immediate types:
\begin{verbatim}
----------------------------------------------------------------
|   Data (24 bits)        | Type (8 bits with low-tag)   | 1 0 |
----------------------------------------------------------------
\end{verbatim}

The type-code for an other-immediate type is considered to include the two
lowtag bits.  This supports the concept of a single ``type code'' namespace for
all descriptors, since the normal lowtag codes are disjoint from the
other-immediate codes.

For other-pointer objects, the full eight bits of the header type code are used
as the type code for that kind of object.  This is why we use two lowtag codes
for other-immediate types: each other-pointer object needs a distinct
other-immediate type to mark its header.

The system uses the other-immediate format for characters, 
the {\tt symbol-value} unbound trap value, and header-words for data-blocks on
the heap.  The type codes are laid out to facilitate range checks for common
subtypes; for example, all numbers will have contiguous type codes which are
distinct from the contiguous array type codes.  See
section~\ref{sec:data-blocks-and-o-i}
for details.


\section{Data-Blocks and Header-Word Format}
\label{sec:data-blocks-and-header}

Pointers to data-blocks have the following format:
\begin{verbatim}
----------------------------------------------------------------
|      Dual-word address of data-block (29 bits)       | 1 1 1 |
----------------------------------------------------------------
\end{verbatim}

The word pointed to by the above descriptor is a header-word, and it has the
same format as an other-immediate:
\begin{verbatim}
----------------------------------------------------------------
|   Data (24 bits)        | Type (8 bits with low-tag) | 0 1 0 |
----------------------------------------------------------------
\end{verbatim}
This is convenient for scanning the heap when GC'ing, but it does mean that
whenever GC encounters an other-immediate word, it has to do a range check on
the low byte to see if it is a header-word or just a character (for example).
This is easily acceptable performance hit for scanning.

The system interprets the data portion of the header-word for non-vector
data-blocks as the word length excluding the header-word.  For example, the
data field of the header for ratio and complex numbers is two, one word each
for the numerator and denominator or for the real and imaginary parts.

For vectors and data-blocks representing Lisp objects stored like vectors, the
system ignores the data portion of the header-word:
\begin{verbatim}
----------------------------------------------------------------
| Unused Data (24 bits)   | Type (8 bits with low-tag) | 0 1 0 |
----------------------------------------------------------------
|           Element Length of Vector (30 bits)           | 0 0 | 
----------------------------------------------------------------
\end{verbatim}

Using a separate word allows for much larger vectors, and it allows {\tt
length} to simply access a single word without masking or shifting.  Similarly,
the header for complex arrays and vectors has a second word, following the
header-word, the system uses for the fill pointer, so computing the length of
any array is the same code sequence.



\section{Data-Blocks and Other-immediates Typing}

\label{sec:data-blocks-and-o-i}
These are the other-immediate types.  We specify them including all low eight
bits, including the other-immediate tag, so we can think of the type bits as
one type -- not an other-immediate major type and a subtype.  Also, fetching a
byte and comparing it against a constant is more efficient than wasting even a
small amount of time shifting out the other-immediate tag to compare against a
five bit constant.
\begin{verbatim}
Number   (< 36)
  bignum                                           10
    ratio                                          14
    single-float                                   18
    double-float                                   22
    complex                                        26
    (complex single-float)                         30
    (complex double-float)                         34

Array   (>= 38 code 118)
   Simple-Array   (>= 38 code 102)
         simple-array                              38
      Vector  (>= 42 code 114)
         simple-string                             42
         simple-bit-vector                         46
         simple-vector                             50
         (simple-array (unsigned-byte 2) (*))      54
         (simple-array (unsigned-byte 4) (*))      58
         (simple-array (unsigned-byte 8) (*))      62
         (simple-array (unsigned-byte 16) (*))     66
         (simple-array (unsigned-byte 32) (*))     70
         (simple-array (signed-byte 8) (*))        74
         (simple-array (signed-byte 16) (*))       78
         (simple-array (signed-byte 30) (*))       82
         (simple-array (signed-byte 32) (*))       86
         (simple-array single-float (*))           90
         (simple-array double-float (*))           94
         (simple-array (complex single-float) (*)  98
         (simple-array (complex double-float) (*)  102
      complex-string                               106
      complex-bit-vector                           110
      (array * (*))   -- general complex vector.   114
   complex-array                                   118

code-header-type                                   122
function-header-type                               126
closure-header-type                                130
funcallable-instance-header-type                   134
unused-function-header-1-type                      138
unused-function-header-2-type                      142
unused-function-header-3-type                      146
closure-function-header-type                       150
return-pc-header-type (a.k.a LRA)                  154
value-cell-header-type                             158
symbol-header-type                                 162
base-character-type                                166
system-area-pointer-type (header type)             170
unbound-marker                                     174
weak-pointer-type                                  178
structure-header-type                              182
fdefn-type                                         186
\end{verbatim}

\section{Strings}

All strings in the system are C-null terminated.  This saves copying the bytes
when calling out to C.  The only time this wastes memory is when the string
contains a multiple of eight characters, and then the system allocates two more
words (since Lisp objects are dual-word aligned) to hold the C-null byte.
Since the system will make heavy use of C routines for systems calls and
libraries that save reimplementation of higher level operating system
functionality (such as pathname resolution or current directory computation),
saving on copying strings for C should make C call out more efficient.

The length word in a string header, see ``Data-Blocks and Header-Word
Format'', section~\ref{sec:data-blocks-and-header}, counts only the characters truly in the Common Lisp string.
Allocation and GC will have to know to handle the extra C-null byte, and GC
already has to deal with rounding up various objects to dual-word alignment.



\section{Symbols and NIL}

Symbol data-block has the following format:
\begin{verbatim}
-------------------------------------------------------
|     5 (data-block words)     | Symbol Type (8 bits) |
-------------------------------------------------------
|                       Value Descriptor              |
-------------------------------------------------------
|  Hash Value (x86/amd64/sparc) Unused (other arch.)  |
-------------------------------------------------------
|                        Property List                |
-------------------------------------------------------
|                          Print Name                 |
-------------------------------------------------------
|                           Package                   |
-------------------------------------------------------
\end{verbatim}

All of these slots are self-explanatory given what symbols must do in Common
Lisp.

The issues with nil are that we want it to act like a symbol, and we need list
operations such as CAR and CDR to be fast on it.  CMU Common Lisp solves this
by putting nil as the first object in static space, where other global values
reside, so it has a known address in the system:
\begin{verbatim}
-------------------------------------------------------  <-- space
|     6 (data-block words)     |         0            |      start
-------------------------------------------------------
|     0 (data-block words)     | Symbol Type (8 bits) |
-------------------------------------------------------  <-- nil
|                           Value/CAR                 |
-------------------------------------------------------
|                         Hash Value/CDR              |
-------------------------------------------------------
|                         Property List               |
-------------------------------------------------------
|                           Print Name                |
-------------------------------------------------------
|                            Package                  |
-------------------------------------------------------
|                              ...                    |
-------------------------------------------------------
\end{verbatim}
In addition, we make the list typed pointer to nil actually point past the
header word of the nil symbol data-block.  This has usefulness explained below.
The value and hash-value of nil are nil.  Therefore, any reference to nil used
as a list has quick list type checking, and CAR and CDR can go right through
the first and second words as if nil were a cons object.

When there is a reference to nil used as a symbol, the system adds offsets to
the address the same as it does for any symbol.  This works due to a
combination of nil pointing past the symbol header-word and the chosen list and
other-pointer type tags.  The list type tag is four less than the other-pointer
type tag, but nil points four additional bytes into its symbol data-block.



\section{Array Headers}

The array-header data-block has the following format:
\begin{verbatim}
----------------------------------------------------------------
| Header Len (24 bits) = Array Rank +5   | Array Type (8 bits) |
----------------------------------------------------------------
|               Fill Pointer (30 bits)                   | 0 0 | 
----------------------------------------------------------------
|               Fill Pointer p (29 bits) -- t or nil   | 1 1 1 |
----------------------------------------------------------------
|               Available Elements (30 bits)             | 0 0 | 
----------------------------------------------------------------
|               Data Vector (29 bits)                  | 1 1 1 | 
----------------------------------------------------------------
|               Displacement (30 bits)                   | 0 0 | 
----------------------------------------------------------------
|               Displacedp (29 bits) -- t or nil       | 1 1 1 | 
----------------------------------------------------------------
|               Range of First Index (30 bits)           | 0 0 | 
----------------------------------------------------------------
                              .
                              .
                              .

\end{verbatim}
The array type in the header-word is one of the eight-bit patterns from 
``Data-Blocks and Other-immediates Typing'', section~\ref{sec:data-blocks-and-header}, indicating that this is a complex
string, complex vector, complex bit-vector, or a multi-dimensional array.  The
data portion of the other-immediate word is the length of the array header
data-block.  Due to its format, its length is always five greater than the
array's number of dimensions.  The following words have the following
interpretations and types:
\begin{description}
   \item[Fill Pointer:]
      This is a fixnum indicating the number of elements in the data vector
      actually in use.  This is the logical length of the array, and it is
      typically the same value as the next slot.  This is the second word, so
      LENGTH of any array, with or without an array header, is just four bytes
      off the pointer to it.
   \item[Fill Pointer P:]
      This is either T or NIL and indicates whether the array uses the
      fill-pointer or not.
   \item[Available Elements:]
      This is a fixnum indicating the number of elements for which there is
      space in the data vector.  This is greater than or equal to the logical
      length of the array when it is a vector having a fill pointer.
   \item[Data Vector:]
      This is a pointer descriptor referencing the actual data of the array.
      This a data-block whose first word is a header-word with an array type as
      described in ``Data-Blocks and Header-Word Format'', section~\ref{sec:data-blocks-and-header} and
      ``Data-Blocks and Other-immediates Typing'', section~\ref{sec:data-blocks-and-o-i}
   \item[Displacement:]
      This is a fixnum added to the computed row-major index for any array.
      This is typically zero.
   \item[Displacedp:]
      This is either t or nil.  This is separate from the displacement slot, so
      most array accesses can simply add in the displacement slot.  The rare
      need to know if an array is displaced costs one extra word in array
      headers which probably aren't very frequent anyway.
   \item[Range of First Index:]
      This is a fixnum indicating the number of elements in the first dimension
      of the array.  Legal index values are zero to one less than this number
      inclusively.  IF the array is zero-dimensional, this slot is
      non-existent.
   \item[... (remaining slots):]
      There is an additional slot in the header for each dimension of the
      array.  These are the same as the Range of First Index slot.
\end{description}


\section{Bignums}

Bignum data-blocks have the following format:
\begin{verbatim}
-------------------------------------------------------
|      Length (24 bits)        | Bignum Type (8 bits) |
-------------------------------------------------------
|             least significant bits                  |
-------------------------------------------------------
                            .
                            .
                            .
\end{verbatim}
The elements contain the two's complement representation of the integer with
the least significant bits in the first element or closer to the header.  The
sign information is in the high end of the last element.




\section{Code Data-Blocks}

A code data-block is the run-time representation of a ``component''.  A component
is a connected portion of a program's flow graph that is compiled as a single
unit, and it contains code for many functions.  Some of these functions are
callable from outside of the component, and these are termed ``entry points''.

Each entry point has an associated user-visible function data-block (of type
{\tt function}).  The full call convention provides for calling an entry point
specified by a function object.

Although all of the function data-blocks for a component's entry points appear
to the user as distinct objects, the system keeps all of the code in a single
code data-block.  The user-visible function object is actually a pointer into
the middle of a code data-block.  This allows any control transfer within a
component to be done using a relative branch.

Besides a function object, there are other kinds of references into the middle
of a code data-block.  Control transfer into a function also occurs at the
return-PC for a call.  The system represents a return-PC somewhat similarly to
a function, so GC can also recognize a return-PC as a reference to a code
data-block.  This representation is known as a Lisp Return Address (LRA).

It is incorrect to think of a code data-block as a concatenation of ``function
data-blocks''.  Code for a function is not emitted in any particular order with
respect to that function's function-header (if any).  The code following a
function-header may only be a branch to some other location where the
function's ``real'' definition is.


The following are the three kinds of pointers to code data-blocks:
\begin{description}
   \item[Code pointer (labeled A below):]
      A code pointer is a descriptor, with other-pointer low-tag bits, pointing
      to the beginning of the code data-block.  The code pointer for the
      currently running function is always kept in a register (CODE).  In
      addition to allowing loading of non-immediate constants, this also serves
      to represent the currently running function to the debugger.
   \item[LRA (labeled B below):]
      The LRA is a descriptor, with other-pointer low-tag bits, pointing
      to a location for a function call.  Note that this location contains no
      descriptors other than the one word of immediate data, so GC can treat
      LRA locations the same as instructions.
   \item[Function (labeled C below):]
      A function is a descriptor, with function low-tag bits, that is user
      callable.  When a function header is referenced from a closure or from
      the function header's self-pointer, the pointer has other-pointer low-tag
      bits, instead of function low-tag bits.  This ensures that the internal
      function data-block associated with a closure appears to be uncallable
      (although users should never see such an object anyway).

      Information about functions that is only useful for entry points is kept
      in some descriptors following the function's self-pointer descriptor.
      All of these together with the function's header-word are known as the
      ``function header''.  GC must be able to locate the function header.  We
      provide for this by chaining together the function headers in a NIL
      terminated list kept in a known slot in the code data-block.
\end{description}

A code data-block has the following format:
\begin{verbatim}
A -->
****************************************************************
|  Header-Word count (24 bits)    |   Code-Type (8 bits)       |
----------------------------------------------------------------
|  Number of code words (fixnum tag)                           |
----------------------------------------------------------------
|  Pointer to first function header (other-pointer tag)        |
----------------------------------------------------------------
|  Debug information (structure tag)                           |
----------------------------------------------------------------
|  First constant (a descriptor)                               |
----------------------------------------------------------------
|  ...                                                         |
----------------------------------------------------------------
|  Last constant (and last word of code header)                |
----------------------------------------------------------------
|  Some instructions (non-descriptor)                          |
----------------------------------------------------------------
|     (pad to dual-word boundary if necessary)                 |

B -->
****************************************************************
|  Word offset from code header (24)   |   Return-PC-Type (8)  |
----------------------------------------------------------------
|  First instruction after return                              |
----------------------------------------------------------------
|  ... more code and LRA header-words                          |
----------------------------------------------------------------
|     (pad to dual-word boundary if necessary)                 |

C -->
****************************************************************
|  Offset from code header (24)  |   Function-Header-Type (8)  |
----------------------------------------------------------------
|  Self-pointer back to previous word (with other-pointer tag) |
----------------------------------------------------------------
|  Pointer to next function (other-pointer low-tag) or NIL     |
----------------------------------------------------------------
|  Function name (a string or a symbol)                        |
----------------------------------------------------------------
|  Function debug arglist (a string)                           |
----------------------------------------------------------------
|  Function type (a list-style function type specifier)        |
----------------------------------------------------------------
|  Start of instructions for function (non-descriptor)         |
----------------------------------------------------------------
|  More function headers and instructions and return PCs,      |
|  until we reach the total size of header-words + code        |
|  words.                                                      |
----------------------------------------------------------------
\end{verbatim}

The following are detailed slot descriptions:
\begin{description}
   \item[Code data-block header-word:]
      The immediate data in the code data-block's header-word is the number of
      leading descriptors in the code data-block, the fixed overhead words plus
      the number of constants.  The first non-descriptor word, some code,
      appears at this word offset from the header.
   \item[Number of code words:]
      The total number of non-header-words in the code data-block.  The total
      word size of the code data-block is the sum of this slot and the
      immediate header-word data of the previous slot.
      header-word.
   \item[Pointer to first function header:]
      A NIL-terminated list of the function headers for all entry points to
      this component.
   \item[Debug information:]
      The DEBUG-INFO structure describing this component.  All information that
      the debugger wants to get from a running function is kept in this
      structure.  Since there are many functions, the current PC is used to
      locate the appropriate debug information.  The system keeps the debug
      information separate from the function data-block, since the currently
      running function may not be an entry point.  There is no way to recover
      the function object for the currently running function, since this
      data-block may not exist.
   \item[First constant ... last constant:]
      These are the constants referenced by the component, if there are any.
\vspace{1ex}
   \item[LRA header word:]
      The immediate header-word data is the word offset from the enclosing code
      data-block's header-word to this word.  This allows GC and the debugger
      to easily recover the code data-block from an LRA.  The code at the
      return point restores the current code pointer using a subtract immediate
      of the offset, which is known at compile time.
\vspace{1ex}
   \item[Function entry point header-word:]
      The immediate header-word data is the word offset from the enclosing code
      data-block's header-word to this word.  This is the same as for the
      return-PC header-word.
   \item[Self-pointer back to header-word:]
      In a non-closure function, this self-pointer to the previous header-word
      allows the call sequence to always indirect through the second word in a
      user callable function.  See section ``Closure Format''.  With a closure,
      indirecting through the second word gets you a function header-word.  The
      system ignores this slot in the function header for a closure, since it
      has already indirected once, and this slot could be some random thing
      that causes an error if you jump to it.  This pointer has an
      other-pointer tag instead of a function pointer tag, indicating it is not
      a user callable Lisp object.
   \item[Pointer to next function:]
      This is the next link in the thread of entry point functions found in
      this component.  This value is NIL when the current header is the last
      entry point in the component.
   \item[Function name:]
      This function's name (for printing).  If the user defined this function
      with DEFUN, then this is the defined symbol, otherwise it is a
      descriptive string.
   \item[Function debug arglist:]
      A printed string representing the function's argument list, for human
      readability.  If it is a macroexpansion function, then this is the
      original DEFMACRO arglist, not the actual expander function arglist.
   \item[Function type:]
      A list-style function type specifier representing the argument signature
      and return types for this function.  For example,
      \begin{verbatim}
(function (fixnum fixnum fixnum) fixnum)
      \end{verbatim}
      or
      \begin{verbatim}
(function (string &key (:start unsigned-byte)) string)
      \end{verbatim}
      This information is intended for machine readablilty, such as by the
      compiler.
\end{description}


\section{Closure Format}

A closure data-block has the following format:
\begin{verbatim}
----------------------------------------------------------------
|  Word size (24 bits)           |  Closure-Type (8 bits)      |
----------------------------------------------------------------
|  Pointer to function header (other-pointer low-tag)          |
----------------------------------------------------------------
|                                 .                            |
|                      Environment information                 |
|                                 .                            |
----------------------------------------------------------------
\end{verbatim}

A closure descriptor has function low-tag bits.  This means that a descriptor
with function low-tag bits may point to either a function header or to a
closure.  The idea is that any callable Lisp object has function low-tag bits.
Insofar as call is concerned, we make the format of closures and non-closure
functions compatible.  This is the reason for the self-pointer in a function
header.  Whenever you have a callable object, you just jump through the second
word, offset some bytes, and go.



\section{Function call}

Due to alignment requirements and low-tag codes, it is not possible to use a
hardware call instruction to compute the LRA.  Instead the LRA
for a call is computed by doing an add-immediate to the start of the code
data-block.

An advantage of using a single data-block to represent both the descriptor and
non-descriptor parts of a function is that both can be represented by a
single pointer.  This reduces the number of memory accesses that have to be
done in a full call.  For example, since the constant pool is implicit in an
LRA, a call need only save the LRA, rather than saving both the
return PC and the constant pool.



\section{Memory Layout}

\cmucl{} has four spaces, read-only, static, dynamic-0, and dynamic-1.
Read-only contains objects that the system never modifies, moves, or reclaims.
Static space contains some global objects necessary for the system's runtime or
performance (since they are located at a known offset at a known address), and
the system never moves or reclaims these.  However, GC does need to scan static
space for references to moved objects.  Dynamic-0 and dynamic-1 are the two
heap areas for stop-and-copy GC algorithms.

What global objects are at the head of static space???
\begin{verbatim}
   NIL
   eval::*top-of-stack*
   lisp::*current-catch-block*
   lisp::*current-unwind-protect*
   FLAGS (RT only)
   BSP (RT only)
   HEAP (RT only)
\end{verbatim}

In addition to the above spaces, the system has a control stack, binding stack,
and a number stack.  The binding stack contains pairs of descriptors, a symbol
and its previous value.  The number stack is the same as the C stack, and the
system uses it for non-Lisp objects such as raw system pointers, saving
non-Lisp registers, parts of bignum computations, etc.



\section{System Pointers}

The system pointers reference raw allocated memory, data returned by foreign
function calls, etc.  The system uses these when you need a pointer to a
non-Lisp block of memory, using an other-pointer.  This provides the greatest
flexibility by relieving contraints placed by having more direct references
that require descriptor type tags.

A system area pointer data-block has the following format:
\begin{verbatim}
-------------------------------------------------------
|     1 (data-block words)        | SAP Type (8 bits) |
-------------------------------------------------------
|             system area pointer                     |
-------------------------------------------------------
\end{verbatim}

``SAP'' means ``system area pointer'', and much of our code contains this naming
scheme.  We don't currently restrict system pointers to one area of memory, but
if they do point onto the heap, it is up to the user to prevent being screwed
by GC or whatever.

\section{Weak Pointers}
\label{sec:weak-pointers}

A weak-pointer data-block has the following format:
\begin{verbatim}
-------------------------------------------------------
|  4 (data-block words) |  Weak pointer Type (8 bits) |
-------------------------------------------------------
|                 weak-pointer-value                  |
-------------------------------------------------------
|                 weak-pointer-broken                 |
-------------------------------------------------------
|                 mark-bit (T or NIL)                 |
-------------------------------------------------------
|                   next                              |
-------------------------------------------------------
\end{verbatim}

The mark-bit is used when gencgc is available.  It's used to note if
this weak pointer has been visited before so that scavenging
weak-pointers isn't an $O(n^2)$ process.

The last slot is an internal slot used by the C runtime to chain all
the weak pointers together for GC.


\chapter{Memory Management}

\section{Stacks and Globals}

\section{Heap Layout}

\section{Garbage Collection}

\chapter{Interface to C and Assembler}


\section{Linkage Table}

The linkage table feature is based on how dynamic libraries dispatch.
A table of functions is used which is filled in with the appropriate
code to jump to the correct address.

For \cmucl{}, this table is stored at
\code{target-foreign-linkage-space-start}. Each entry is
\code{target-foreign-linkage-entry-size} bytes long.

At startup, the table is initialized with default values in
\code{os\_foreign\_linkage\_init}. On x86 platforms, the first entry is
code to call the routine \code{resolve\_linkage\_tramp}. All other
entries jump to the first entry. The function
\code{resolve\_linkage\_tramp} looks at where it was called from to
figure out which entry in the table was used. It calls
\code{lazy\_resolve\_linkage} with the address of the linkage entry.
This routine then fills in the appropriate linkage entry with code to
jump to where the real routine is located, and returns the address of
the entry. On return, \code{resolve\_linkage\_tramp} then just jumps to
the returned address to call the desired function. On all subsequent
calls, the entry no longer points to \code{resolve\_linkage\_tramp} but
to the real function.

This describes how function calls are made. For foreign data,
\code{lazy\_resolve\_linkage} stuffs the address of the actual foreign
data into the linkage table. The lisp code then just loads the address
from there to get the actual address of the foreign data.

For sparc, the linkage table is slightly different. The first entry is
the entry for \code{call\_into\_c} so we never have to look this up. All
other entries are for \code{resolve\_linkage\_tramp}. This has the
advantage that \code{resolve\_linkage\_tramp} can be much simpler since
all calls to foreign code go through \code{call\_into\_c} anyway, and
that means all live Lisp registers have already been saved. Also, to
make life simpler, we lie about \code{closure\_tramp} and
\code{undefined\_tramp} in the Lisp code. These are really functions,
but we treat them as foreign data since these two routines are only
used as addresses in the Lisp code to stuff into a lisp function
header.

On the Lisp side, there are two supporting data structures for the
linkage table: \code{*linkage-table-data*} and
\code{*foreign-linkage-symbols*}. The latter is a hash table whose key
is the foreign symbol (a string) and whose value is an index into
\code{*linkage-table-data*}.

\code{*linkage-table-data*} is a vector with an unlispy layout. Each
entry has 3 parts:

\begin{itemize}
\item symbol name
\item type, a fixnum, 1 = code, 2 = data
\item library list - the library list at the time the symbol is registered.
\end{itemize}

Whenever a new foreign symbol is defined, a new
\code{*linkage-table-data*} entry is created.
\code{*foreign-linkage-symbols*} is updated with the symbol and the
entry number into \code{*linkage-table-data*}.

The \code{*linkage-table-data*} is accessed from C (hence the unlispy
layout), to figure out the symbol name and the type so that the
address of the symbol can be determined.  The type tells the C code
how to fill in the entry in the linkage-table itself.

% (Should say something about genesis too, but I don't know how that
% works other than the initial table is setup with the appropriate first
% entry.)


\chapter{Low-level debugging}

\chapter{Core File Format}

\chapter{Fasload File Format}% -*- Dictionary: design -*-
\section{General}

The purpose of Fasload files is to allow concise storage and rapid
loading of Lisp data, particularly function definitions.  The intent
is that loading a Fasload file has the same effect as loading the
source file from which the Fasload file was compiled, but accomplishes
the tasks more efficiently.  One noticeable difference, of course, is
that function definitions may be in compiled form rather than
S-expression form.  Another is that Fasload files may specify in what
parts of memory the Lisp data should be allocated.  For example,
constant lists used by compiled code may be regarded as read-only.

In some Lisp implementations, Fasload file formats are designed to
allow sharing of code parts of the file, possibly by direct mapping
of pages of the file into the address space of a process.  This
technique produces great performance improvements in a paged
time-sharing system.  Since the Mach project is to produce a
distributed personal-computer network system rather than a
time-sharing system, efficiencies of this type are explicitly {\it not}
a goal for the CMU Common Lisp Fasload file format.

On the other hand, CMU Common Lisp is intended to be portable, as it will
eventually run on a variety of machines.  Therefore an explicit goal
is that Fasload files shall be transportable among various
implementations, to permit efficient distribution of programs in
compiled form.  The representations of data objects in Fasload files
shall be relatively independent of such considerations as word
length, number of type bits, and so on.  If two implementations
interpret the same macrocode (compiled code format), then Fasload
files should be completely compatible.  If they do not, then files
not containing compiled code (so-called ``Fasdump'' data files) should
still be compatible.  While this may lead to a format which is not
maximally efficient for a particular implementation, the sacrifice of
a small amount of performance is deemed a worthwhile price to pay to
achieve portability.

The primary assumption about data format compatibility is that all
implementations can support I/O on finite streams of eight-bit bytes.
By ``finite'' we mean that a definite end-of-file point can be detected
irrespective of the content of the data stream.  A Fasload file will
be regarded as such a byte stream.

\section{Strategy}

A Fasload file may be regarded as a human-readable prefix followed by
code in a funny little language.  When interpreted, this code will
cause the construction of the encoded data structures.  The virtual
machine which interprets this code has a {\it stack} and a {\it table},
both initially empty.  The table may be thought of as an expandable
register file; it is used to remember quantities which are needed
more than once.  The elements of both the stack and the table are
Lisp data objects.  Operators of the funny language may take as
operands following bytes of the data stream, or items popped from the
stack.  Results may be pushed back onto the stack or pushed onto the
table.  The table is an indexable stack that is never popped; it is
indexed relative to the base, not the top, so that an item once
pushed always has the same index.

More precisely, a Fasload file has the following macroscopic
organization.  It is a sequence of zero or more groups concatenated
together.  End-of-file must occur at the end of the last group.  Each
group begins with a series of seven-bit ASCII characters terminated
by one or more bytes of all ones \verb|#xFF|; this is called the
{\it header}.  Following the bytes which terminate the header is the
{\it body}, a stream of bytes in the funny binary language.  The body
of necessity begins with a byte other than \verb|#xFF|.  The body is
terminated by the operation {\tt FOP-END-GROUP}.

The first nine characters of the header must be \verb|FASL FILE| in
upper-case letters.  The rest may be any ASCII text, but by
convention it is formatted in a certain way.  The header is divided
into lines, which are grouped into paragraphs.  A paragraph begins
with a line which does {\it not} begin with a space or tab character,
and contains all lines up to, but not including, the next such line.
The first word of a paragraph, defined to be all characters up to but
not including the first space, tab, or end-of-line character, is the
{\it name} of the paragraph.  A Fasload file header might look something like
this:
\begin{verbatim}
FASL FILE >SteelesPerq>User>Guy>IoHacks>Pretty-Print.Slisp
Package Pretty-Print
Compiled 31-Mar-1988 09:01:32 by some random luser
Compiler Version 1.6, Lisp Version 3.0.
Functions: INITIALIZE DRIVER HACK HACK1 MUNGE MUNGE1 GAZORCH
	   MINGLE MUDDLE PERTURB OVERDRIVE GOBBLE-KEYBOARD
	   FRY-USER DROP-DEAD HELP CLEAR-MICROCODE
	    %AOS-TRIANGLE %HARASS-READTABLE-MAYBE
Macros:    PUSH POP FROB TWIDDLE
\end{verbatim}
{\it one or more bytes of \verb|#xFF|}

The particular paragraph names and contents shown here are only intended as
suggestions.

\section{Fasload Language}

Each operation in the binary Fasload language is an eight-bit
(one-byte) opcode.  Each has a name beginning with ``{\tt FOP-}''.  In	
the following descriptions, the name is followed by operand
descriptors.  Each descriptor denotes operands that follow the opcode
in the input stream.  A quantity in parentheses indicates the number
of bytes of data from the stream making up the operand.  Operands
which implicitly come from the stack are noted in the text.  The
notation ``$\Rightarrow$ stack'' means that the result is pushed onto the
stack; ``$\Rightarrow$ table'' similarly means that the result is added to the
table.  A construction like ``{\it n}(1) {\it value}({\it n})'' means that
first a single byte {\it n} is read from the input stream, and this
byte specifies how many bytes to read as the operand named {\it value}.
All numeric values are unsigned binary integers unless otherwise
specified.  Values described as ``signed'' are in two's-complement form
unless otherwise specified.  When an integer read from the stream
occupies more than one byte, the first byte read is the least
significant byte, and the last byte read is the most significant (and
contains the sign bit as its high-order bit if the entire integer is
signed).

Some of the operations are not necessary, but are rather special
cases of or combinations of others.  These are included to reduce the
size of the file or to speed up important cases.  As an example,
nearly all strings are less than 256 bytes long, and so a special
form of string operation might take a one-byte length rather than a
four-byte length.  As another example, some implementations may
choose to store bits in an array in a left-to-right format within
each word, rather than right-to-left.  The Fasload file format may
support both formats, with one being significantly more efficient
than the other for a given implementation.  The compiler for any
implementation may generate the more efficient form for that
implementation, and yet compatibility can be maintained by requiring
all implementations to support both formats in Fasload files.

Measurements are to be made to determine which operation codes are
worthwhile; little-used operations may be discarded and new ones
added.  After a point the definition will be ``frozen'', meaning that
existing operations may not be deleted (though new ones may be added;
some operations codes will be reserved for that purpose).

\begin{description}
\item[0:] \hspace{2em} {\tt FOP-NOP} \\
No operation.  (This is included because it is recognized
that some implementations may benefit from alignment of operands to some
operations, for example to 32-bit boundaries.  This operation can be used
to pad the instruction stream to a desired boundary.)

\item[1:] \hspace{2em} {\tt FOP-POP} \hspace{2em} $\Rightarrow$ \hspace{2em} table \\
One item is popped from the stack and added to the table.

\item[2:] \hspace{2em} {\tt FOP-PUSH} \hspace{2em} {\it index}(4) \hspace{2em} $\Rightarrow$ \hspace{2em} stack \\
Item number {\it index} of the table is pushed onto the stack.
The first element of the table is item number zero.

\item[3:] \hspace{2em} {\tt FOP-BYTE-PUSH} \hspace{2em} {\it index}(1) \hspace{2em} $\Rightarrow$ \hspace{2em} stack \\
Item number {\it index} of the table is pushed onto the stack.
The first element of the table is item number zero.

\item[4:] \hspace{2em} {\tt FOP-EMPTY-LIST} \hspace{2em} $\Rightarrow$ \hspace{2em} stack \\
The empty list ({\tt ()}) is pushed onto the stack.

\item[5:] \hspace{2em} {\tt FOP-TRUTH} \hspace{2em} $\Rightarrow$ \hspace{2em} stack \\
The standard truth value ({\tt T}) is pushed onto the stack.

\item[6:] \hspace{2em} {\tt FOP-SYMBOL-SAVE} \hspace{2em} {\it n}(4) \hspace{2em} {\it name}({\it n})
\hspace{2em} $\Rightarrow$ \hspace{2em} stack \& table\\
The four-byte operand {\it n} specifies the length of the print name
of a symbol.  The name follows, one character per byte,
with the first byte of the print name being the first read.
The name is interned in the default package,
and the resulting symbol is both pushed onto the stack and added to the table.

\item[7:] \hspace{2em} {\tt FOP-SMALL-SYMBOL-SAVE} \hspace{2em} {\it n}(1) \hspace{2em} {\it name}({\it n}) \hspace{2em} $\Rightarrow$ \hspace{2em} stack \& table\\
The one-byte operand {\it n} specifies the length of the print name
of a symbol.  The name follows, one character per byte,
with the first byte of the print name being the first read.
The name is interned in the default package,
and the resulting symbol is both pushed onto the stack and added to the table.

\item[8:] \hspace{2em} {\tt FOP-SYMBOL-IN-PACKAGE-SAVE} \hspace{2em} {\it index}(4)
\hspace{2em} {\it n}(4) \hspace{2em} {\it name}({\it n})
\hspace{2em} $\Rightarrow$ \hspace{2em} stack \& table\\
The four-byte {\it index} specifies a package stored in the table.
The four-byte operand {\it n} specifies the length of the print name
of a symbol.  The name follows, one character per byte,
with the first byte of the print name being the first read.
The name is interned in the specified package,
and the resulting symbol is both pushed onto the stack and added to the table.

\item[9:] \hspace{2em} {\tt FOP-SMALL-SYMBOL-IN-PACKAGE-SAVE}  \hspace{2em} {\it index}(4)
\hspace{2em} {\it n}(1) \hspace{2em} {\it name}({\it n}) \hspace{2em}
$\Rightarrow$ \hspace{2em} stack \& table\\
The four-byte {\it index} specifies a package stored in the table.
The one-byte operand {\it n} specifies the length of the print name
of a symbol.  The name follows, one character per byte,
with the first byte of the print name being the first read.
The name is interned in the specified package,
and the resulting symbol is both pushed onto the stack and added to the table.

\item[10:] \hspace{2em} {\tt FOP-SYMBOL-IN-BYTE-PACKAGE-SAVE} \hspace{2em} {\it index}(1)
\hspace{2em} {\it n}(4) \hspace{2em} {\it name}({\it n})
\hspace{2em} $\Rightarrow$ \hspace{2em} stack \& table\\
The one-byte {\it index} specifies a package stored in the table.
The four-byte operand {\it n} specifies the length of the print name
of a symbol.  The name follows, one character per byte,
with the first byte of the print name being the first read.
The name is interned in the specified package,
and the resulting symbol is both pushed onto the stack and added to the table.

\item[11:]\hspace{2em} {\tt FOP-SMALL-SYMBOL-IN-BYTE-PACKAGE-SAVE} \hspace{2em} {\it index}(1)
\hspace{2em} {\it n}(1) \hspace{2em} {\it name}({\it n}) \hspace{2em}
$\Rightarrow$ \hspace{2em} stack \& table\\
The one-byte {\it index} specifies a package stored in the table.
The one-byte operand {\it n} specifies the length of the print name
of a symbol.  The name follows, one character per byte,
with the first byte of the print name being the first read.
The name is interned in the specified package,
and the resulting symbol is both pushed onto the stack and added to the table.

\item[12:] \hspace{2em} {\tt FOP-UNINTERNED-SYMBOL-SAVE} \hspace{2em} {\it n}(4) \hspace{2em} {\it name}({\it n})
\hspace{2em} $\Rightarrow$ \hspace{2em} stack \& table\\
Like {\tt FOP-SYMBOL-SAVE}, except that it creates an uninterned symbol.

\item[13:] \hspace{2em} {\tt FOP-UNINTERNED-SMALL-SYMBOL-SAVE} \hspace{2em} {\it n}(1)
\hspace{2em} {\it name}({\it n}) \hspace{2em} $\Rightarrow$ \hspace{2em} stack
\& table\\
Like {\tt FOP-SMALL-SYMBOL-SAVE}, except that it creates an uninterned symbol.

\item[14:] \hspace{2em} {\tt FOP-PACKAGE} \hspace{2em} $\Rightarrow$ \hspace{2em} table \\
An item is popped from the stack; it must be a symbol.	The package of
that name is located and pushed onto the table.

\item[15:] \hspace{2em} {\tt FOP-LIST} \hspace{2em} {\it length}(1) \hspace{2em} $\Rightarrow$ \hspace{2em} stack \\
The unsigned operand {\it length} specifies a number of
operands to be popped from the stack.  These are made into a list
of that length, and the list is pushed onto the stack.
The first item popped from the stack becomes the last element of
the list, and so on.  Hence an iterative loop can start with
the empty list and perform ``pop an item and cons it onto the list''
{\it length} times.
(Lists of length greater than 255 can be made by using {\tt FOP-LIST*}
repeatedly.)

\item[16:] \hspace{2em} {\tt FOP-LIST*} \hspace{2em} {\it length}(1) \hspace{2em} $\Rightarrow$ \hspace{2em} stack \\
This is like {\tt FOP-LIST} except that the constructed list is terminated
not by {\tt ()} (the empty list), but by an item popped from the stack
before any others are.	Therefore {\it length}+1 items are popped in all.
Hence an iterative loop can start with
a popped item and perform ``pop an item and cons it onto the list''
{\it length}+1 times.

\item[17-24:] \hspace{2em} {\tt FOP-LIST-1}, {\tt FOP-LIST-2}, ..., {\tt FOP-LIST-8} \\
{\tt FOP-LIST-{\it k}} is like {\tt FOP-LIST} with a byte containing {\it k}
following it.  These exist purely to reduce the size of Fasload files.
Measurements need to be made to determine the useful values of {\it k}.

\item[25-32:] \hspace{2em} {\tt FOP-LIST*-1}, {\tt FOP-LIST*-2}, ..., {\tt FOP-LIST*-8} \\
{\tt FOP-LIST*-{\it k}} is like {\tt FOP-LIST*} with a byte containing {\it k}
following it.  These exist purely to reduce the size of Fasload files.
Measurements need to be made to determine the useful values of {\it k}.

\item[33:] \hspace{2em} {\tt FOP-INTEGER} \hspace{2em} {\it n}(4) \hspace{2em} {\it value}({\it n}) \hspace{2em}
$\Rightarrow$ \hspace{2em} stack \\
A four-byte unsigned operand specifies the number of following
bytes.	These bytes define the value of a signed integer in two's-complement
form.  The first byte of the value is the least significant byte.

\item[34:] \hspace{2em} {\tt FOP-SMALL-INTEGER} \hspace{2em} {\it n}(1) \hspace{2em} {\it value}({\it n})
\hspace{2em} $\Rightarrow$ \hspace{2em} stack \\
A one-byte unsigned operand specifies the number of following
bytes.	These bytes define the value of a signed integer in two's-complement
form.  The first byte of the value is the least significant byte.

\item[35:] \hspace{2em} {\tt FOP-WORD-INTEGER} \hspace{2em} {\it value}(4) \hspace{2em} $\Rightarrow$ \hspace{2em} stack \\
A four-byte signed integer (in the range $-2^{31}$ to $2^{31}-1$) follows the
operation code.  A LISP integer (fixnum or bignum) with that value
is constructed and pushed onto the stack.

\item[36:] \hspace{2em} {\tt FOP-BYTE-INTEGER} \hspace{2em} {\it value}(1) \hspace{2em} $\Rightarrow$ \hspace{2em} stack \\
A one-byte signed integer (in the range -128 to 127) follows the
operation code.  A LISP integer (fixnum or bignum) with that value
is constructed and pushed onto the stack.

\item[37:] \hspace{2em} {\tt FOP-STRING} \hspace{2em} {\it n}(4) \hspace{2em} {\it name}({\it n})
\hspace{2em} $\Rightarrow$ \hspace{2em} stack \\
The four-byte operand {\it n} specifies the length of a string to
construct.  The characters of the string follow, one per byte.
The constructed string is pushed onto the stack.

\item[38:] \hspace{2em} {\tt FOP-SMALL-STRING} \hspace{2em} {\it n}(1) \hspace{2em} {\it name}({\it n}) \hspace{2em} $\Rightarrow$ \hspace{2em} stack \\
The one-byte operand {\it n} specifies the length of a string to
construct.  The characters of the string follow, one per byte.
The constructed string is pushed onto the stack.

\item[39:] \hspace{2em} {\tt FOP-VECTOR} \hspace{2em} {\it n}(4) \hspace{2em} $\Rightarrow$ \hspace{2em} stack \\
The four-byte operand {\it n} specifies the length of a vector of LISP objects
to construct.  The elements of the vector are popped off the stack;
the first one popped becomes the last element of the vector.
The constructed vector is pushed onto the stack.

\item[40:] \hspace{2em} {\tt FOP-SMALL-VECTOR} \hspace{2em} {\it n}(1) \hspace{2em} $\Rightarrow$ \hspace{2em} stack \\
The one-byte operand {\it n} specifies the length of a vector of LISP objects
to construct.  The elements of the vector are popped off the stack;
the first one popped becomes the last element of the vector.
The constructed vector is pushed onto the stack.

\item[41:] \hspace{2em} {\tt FOP-UNIFORM-VECTOR} \hspace{2em} {\it n}(4) \hspace{2em} $\Rightarrow$ \hspace{2em} stack \\
The four-byte operand {\it n} specifies the length of a vector of LISP objects
to construct.  A single item is popped from the stack and used to initialize
all elements of the vector.  The constructed vector is pushed onto the stack.

\item[42:] \hspace{2em} {\tt FOP-SMALL-UNIFORM-VECTOR} \hspace{2em} {\it n}(1) \hspace{2em} $\Rightarrow$ \hspace{2em} stack \\
The one-byte operand {\it n} specifies the length of a vector of LISP objects
to construct.  A single item is popped from the stack and used to initialize
all elements of the vector.  The constructed vector is pushed onto the stack.

\item[43:] \hspace{2em} {\tt FOP-INT-VECTOR} \hspace{2em} {\it len}(4) \hspace{2em}
{\it size}(1) \hspace{2em} {\it data}($\left\lceil len*count/8\right\rceil$)
\hspace{2em} $\Rightarrow$ \hspace{2em} stack \\
The four-byte operand {\it n} specifies the length of a vector of
unsigned integers to be constructed.   Each integer is {\it size}
bits long, and is packed according to the machine's native byte ordering.
{\it size} must be a directly supported i-vector element size.  Currently
supported values are 1,2,4,8,16 and 32.

\item[44:] \hspace{2em} {\tt FOP-UNIFORM-INT-VECTOR} \hspace{2em} {\it n}(4) \hspace{2em} {\it size}(1) \hspace{2em}
{\it value}(@ceiling$<${\it size}/8$>$) \hspace{2em} $\Rightarrow$ \hspace{2em} stack \\
The four-byte operand {\it n} specifies the length of a vector of unsigned
integers to construct.
Each integer is {\it size} bits big, and is initialized to the value
of the operand {\it value}.
The constructed vector is pushed onto the stack.

\item[45:] \hspace{2em} {\tt FOP-LAYOUT} \hspace{2em} \\
Pops the stack four times to get the name, length, inheritance and depth for a layout object. 

\item[46:] \hspace{2em} {\tt FOP-SINGLE-FLOAT} \hspace{2em} {\it data}(4) \hspace{2em}
$\Rightarrow$ \hspace{2em} stack \\
The {\it data} bytes are read as an integer, then turned into an IEEE single
float (as though by {\tt make-single-float}).

\item[47:] \hspace{2em} {\tt FOP-DOUBLE-FLOAT} \hspace{2em} {\it data}(8) \hspace{2em}
$\Rightarrow$ \hspace{2em} stack \\
The {\it data} bytes are read as an integer, then turned into an IEEE double
float (as though by {\tt make-double-float}).

\item[48:] \hspace{2em} {\tt FOP-STRUCT} \hspace{2em} {\it n}(4) \hspace{2em} $\Rightarrow$ \hspace{2em} stack \\
The four-byte operand {\it n} specifies the length structure to construct.  The
elements of the vector are popped off the stack; the first one popped becomes
the last element of the structure.  The constructed vector is pushed onto the
stack.

\item[49:] \hspace{2em} {\tt FOP-SMALL-STRUCT} \hspace{2em} {\it n}(1) \hspace{2em} $\Rightarrow$ \hspace{2em} stack \\
The one-byte operand {\it n} specifies the length structure to construct.  The
elements of the vector are popped off the stack; the first one popped becomes
the last element of the structure.  The constructed vector is pushed onto the
stack.

\item[50-52:] Unused

\item[53:] \hspace{2em} {\tt FOP-EVAL} \hspace{2em} $\Rightarrow$ \hspace{2em} stack \\
Pop an item from the stack and evaluate it (give it to {\tt EVAL}).
Push the result back onto the stack.

\item[54:] \hspace{2em} {\tt FOP-EVAL-FOR-EFFECT} \\
Pop an item from the stack and evaluate it (give it to {\tt EVAL}).
The result is ignored.

\item[55:] \hspace{2em} {\tt FOP-FUNCALL} \hspace{2em} {\it nargs}(1) \hspace{2em} $\Rightarrow$ \hspace{2em} stack \\
Pop {\it nargs}+1 items from the stack and apply the last one popped
as a function to
all the rest as arguments (the first one popped being the last argument).
Push the result back onto the stack.

\item[56:] \hspace{2em} {\tt FOP-FUNCALL-FOR-EFFECT} \hspace{2em} {\it nargs}(1) \\
Pop {\it nargs}+1 items from the stack and apply the last one popped
as a function to
all the rest as arguments (the first one popped being the last argument).
The result is ignored.

\item[57:] \hspace{2em} {\tt FOP-CODE-FORMAT} \hspace{2em} {\it implementation}(1)
\hspace{2em} {\it version}(1) \\
This FOP specifiers the code format for following code objects.  The operations
{\tt FOP-CODE} and its relatives may not occur in a group until after {\tt
FOP-CODE-FORMAT} has appeared; there is no default format.  The {\it
implementation} is an integer indicating the target hardware and environment.
See {\tt compiler/generic/vm-macs.lisp} for the currently defined
implementations.  {\it version} for an implementation is increased whenever
there is a change that renders old fasl files unusable.

\item[58:] \hspace{2em} {\tt FOP-CODE} \hspace{2em} {\it nitems}(4) \hspace{2em} {\it size}(4) \hspace{2em}
{\it code}({\it size}) \hspace{2em} $\Rightarrow$ \hspace{2em} stack \\
A compiled function is constructed and pushed onto the stack.
This object is in the format specified by the most recent
occurrence of {\tt FOP-CODE-FORMAT}.
The operand {\it nitems} specifies a number of items to pop off
the stack to use in the ``boxed storage'' section.  The operand {\it code}
is a string of bytes constituting the compiled executable code.

\item[59:] \hspace{2em} {\tt FOP-SMALL-CODE} \hspace{2em} {\it nitems}(1) \hspace{2em} {\it size}(2) \hspace{2em}
{\it code}({\it size}) \hspace{2em} $\Rightarrow$ \hspace{2em} stack \\
A compiled function is constructed and pushed onto the stack.
This object is in the format specified by the most recent
occurrence of {\tt FOP-CODE-FORMAT}.
The operand {\it nitems} specifies a number of items to pop off
the stack to use in the ``boxed storage'' section.  The operand {\it code}
is a string of bytes constituting the compiled executable code.

\item[60] \hspace{2em} {\tt FOP-FDEFINITION} \hspace{2em} \\
Pops the stack to get an fdefinition.

\item[61] \hspace{2em} {\tt FOP-SANCTIFY-FOR-EXECUTION} \hspace{2em} \\
A code component is popped from the stack, and the necessary magic is applied 
to the code so that it can be executed.

\item[62:] \hspace{2em} {\tt FOP-VERIFY-TABLE-SIZE} \hspace{2em} {\it size}(4) \\
If the current size of the table is not equal to {\it size},
then an inconsistency has been detected.  This operation
is inserted into a Fasload file purely for error-checking purposes.
It is good practice for a compiler to output this at least at the
end of every group, if not more often.

\item[63:] \hspace{2em} {\tt FOP-VERIFY-EMPTY-STACK} \\
If the stack is not currently empty,
then an inconsistency has been detected.  This operation
is inserted into a Fasload file purely for error-checking purposes.
It is good practice for a compiler to output this at least at the
end of every group, if not more often.

\item[64:] \hspace{2em} {\tt FOP-END-GROUP} \\
This is the last operation of a group.	If this is not the
last byte of the file, then a new group follows; the next
nine bytes must be ``{\tt FASL FILE}''.

\item[65:] \hspace{2em} {\tt FOP-POP-FOR-EFFECT} \hspace{2em} stack \hspace{2em} $\Rightarrow$ \hspace{2em} \\
One item is popped from the stack.

\item[66:] \hspace{2em} {\tt FOP-MISC-TRAP} \hspace{2em} $\Rightarrow$ \hspace{2em} stack \\
A trap object is pushed onto the stack.

\item[67:] \hspace{2em} {\tt FOP-DOUBLE-DOUBLE-FLOAT} \hspace{2em} {\it double-double-float}(8) \hspace{2em} $\Rightarrow$ \hspace{2em} stack \\
The next 8 bytes are read, and a double-double-float number is constructed.

\item[68:] \hspace{2em} {\tt FOP-CHARACTER} \hspace{2em} {\it character}(3) \hspace{2em} $\Rightarrow$ \hspace{2em} stack \\
The three bytes are read as an integer then converted to a character.  This FOP
is currently rather useless, as extended characters are not supported.

\item[69:] \hspace{2em} {\tt FOP-SHORT-CHARACTER} \hspace{2em} {\it character}(1) \hspace{2em}
$\Rightarrow$ \hspace{2em} stack \\
The one byte specifies the code of a Common Lisp character object.  A character
is constructed and pushed onto the stack.

\item[70:] \hspace{2em} {\tt FOP-RATIO} \hspace{2em} $\Rightarrow$ \hspace{2em} stack \\
Creates a ratio from two integers popped from the stack.
The denominator is popped first, the numerator second.

\item[71:] \hspace{2em} {\tt FOP-COMPLEX} \hspace{2em} $\Rightarrow$ \hspace{2em} stack \\
Creates a complex number from two numbers popped from the stack.
The imaginary part is popped first, the real part second.

\item[72] \hspace{2em} {\tt FOP-COMPLEX-SINGLE-FLOAT} {\it real(4)} {\it imag(4)}\hspace{2em} $\Rightarrow$ \hspace{2em} stack \\
Creates a complex single-float number from the following 8 bytes.

\item[73] \hspace{2em} {\tt FOP-COMPLEX-DOUBLE-FLOAT} {\it real(8)} {\it imag(8)}\hspace{2em} $\Rightarrow$ \hspace{2em} stack \\
Creates a complex double-float number from the following 16 bytes.


\item[74:] \hspace{2em} {\tt FOP-FSET} \hspace{2em} \\
Except in the cold loader (Genesis), this is a no-op with two stack arguments.
In the initial core this is used to make DEFUN functions defined at cold-load
time so that global functions can be called before top-level forms are run
(which normally installs definitions.)  Genesis pops the top two things off of
the stack and effectively does (SETF SYMBOL-FUNCTION).

\item[75:] \hspace{2em} {\tt FOP-LISP-SYMBOL-SAVE} \hspace{2em} {\it n}(4) \hspace{2em} {\it name}({\it n})
\hspace{2em} $\Rightarrow$ \hspace{2em} stack \& table\\
Like {\tt FOP-SYMBOL-SAVE}, except that it creates a symbol in the LISP
package.

\item[76:] \hspace{2em} {\tt FOP-LISP-SMALL-SYMBOL-SAVE} \hspace{2em} {\it n}(1)
\hspace{2em} {\it name}({\it n}) \hspace{2em} $\Rightarrow$ \hspace{2em} stack
\& table\\
Like {\tt FOP-SMALL-SYMBOL-SAVE}, except that it creates a symbol in the LISP
package.

\item[77:] \hspace{2em} {\tt FOP-KEYWORD-SYMBOL-SAVE} \hspace{2em} {\it n}(4) \hspace{2em} {\it name}({\it n})
\hspace{2em} $\Rightarrow$ \hspace{2em} stack \& table\\
Like {\tt FOP-SYMBOL-SAVE}, except that it creates a symbol in the
KEYWORD package.

\item[78:] \hspace{2em} {\tt FOP-KEYWORD-SMALL-SYMBOL-SAVE} \hspace{2em} {\it n}(1)
\hspace{2em} {\it name}({\it n}) \hspace{2em} $\Rightarrow$ \hspace{2em} stack
\& table\\
Like {\tt FOP-SMALL-SYMBOL-SAVE}, except that it creates a symbol in the
KEYWORD package.

\item[79-80:] Unused

\item[81:] \hspace{2em} {\tt FOP-NORMAL-LOAD}\\
This FOP is used in conjunction with the cold loader (Genesis) to read
top-level package manipulation forms.  These forms are to be read as though by
the normal loaded, so that they can be evaluated at cold load time, instead of
being dumped into the initial core image.  A no-op in normal loading.

\item[82:] \hspace{2em} {\tt FOP-MAYBE-COLD-LOAD}\\
Undoes the effect of {\tt FOP-NORMAL-LOAD}. 

\item[83:] \hspace{2em} {\tt FOP-ARRAY} \hspace{2em} {\it rank}(4)
\hspace{2em} $\Rightarrow$ \hspace{2em} stack\\
This operation creates a simple array header (used for simple-arrays with rank
/= 1).  The data vector is popped off of the stack, and then {\it rank}
dimensions are popped off of the stack (the highest dimensions is on top.)

\item[84:] \hspace{2em} {\tt FOP-SINGLE-FLOAT-VECTOR} \hspace{2em} {\it length}(4) {\it data}(n)
 \hspace{2em} $\Rightarrow$ \hspace{2em} stack\\
Creates a {\it (simple-array single-float (*))} object.  The number of single-floats is {\it length}.

\item[85:] \hspace{2em} {\tt FOP-DOUBLE-FLOAT-VECTOR} \hspace{2em} {\it length}(4) {\it data}(n)
 \hspace{2em} $\Rightarrow$ \hspace{2em} stack\\
Creates a {\it (simple-array double-float (*))} object.  The number of double-floats is {\it length}.

\item[86:] \hspace{2em} {\tt FOP-COMPLEX-SINGLE-FLOAT-VECTOR} \hspace{2em} {\it length}(4) {\it data}(n)
 \hspace{2em} $\Rightarrow$ \hspace{2em} stack\\
Creates a {\it (simple-array (complex single-float) (*))} object.  The number of complex single-floats is {\it length}.

\item[87:] \hspace{2em} {\tt FOP-COMPLEX-DOUBLE-FLOAT-VECTOR} \hspace{2em} {\it length}(4) {\it data}(n)
 \hspace{2em} $\Rightarrow$ \hspace{2em} stack\\
Creates a {\it (simple-array (complex double-float) (*))} object.  The number of complex double-floats is {\it length}.

\item[88:] \hspace{2em} {\tt FOP-DOUBLE-DOUBLE-FLOAT-VECTOR} \hspace{2em} {\it length}(4) {\it data}(n)
 \hspace{2em} $\Rightarrow$ \hspace{2em} stack\\
Creates a {\it (simple-array double-double-float (*))} object.  The number of double-double-floats is {\it length}.

\item[89:] \hspace{2em} {\tt FOP-COMPLEX-DOUBLE-DOUBLE-FLOAT} \hspace{2em} {\it data}(32)
 \hspace{2em} $\Rightarrow$ \hspace{2em} stack\\
Creates a {\it (complex double-double-float)} object from the following 32 bytes of data.

\item[90:] \hspace{2em} {\tt FOP-COMPLEX-DOUBLE-DOUBLE-FLOAT-VECTOR} \hspace{2em} {\it length}(4) {\it data}(n)
 \hspace{2em} $\Rightarrow$ \hspace{2em} stack\\
Creates a {\it (simple-arra (complex double-double-float) (*))} object.  The number of complex double-double-floats is {\it length}.

\item[91-139:] Unused

\item[140:] \hspace{2em} {\tt FOP-ALTER-CODE} \hspace{2em} {\it index}(4)\\
This operation modifies the constants part of a code object (necessary for
creating certain circular function references.)  It pops the new value and code
object are off of the stack, storing the new value at the specified index.

\item[141:] \hspace{2em} {\tt FOP-BYTE-ALTER-CODE} \hspace{2em} {\it index}(1)\\
Like {\tt FOP-ALTER-CODE}, but has only a one byte offset.

\item[142:] \hspace{2em} {\tt FOP-FUNCTION-ENTRY} \hspace{2em} {\it index}(4)
\hspace{2em} $\Rightarrow$ \hspace{2em} stack\\
Initializes a function-entry header inside of a pre-existing code object, and
returns the corresponding function descriptor.  {\it index} is the byte offset
inside of the code object where the header should be plunked down.  The stack
arguments to this operation are the code object, function name, function debug
arglist and function type.

\item[143:] \hspace{2em} {\tt FOP-MAKE-BYTE-COMPILED-FUNCTION} \hspace{2em} {\it size}(1) \hspace{2em} $\Rightarrow$ \hspace{2em} stack\\
Create a byte-compiled function.  {\it FIXME:} describe what's on the stack.

\item[144:] \hspace{2em} {\tt FOP-ASSEMBLER-CODE} \hspace{2em} {\it length}(4)
\hspace{2em} $\Rightarrow$ \hspace{2em} stack\\
This operation creates a code object holding assembly routines.  {\it length}
bytes of code are read and placed in the code object, and the code object
descriptor is pushed on the stack.  This FOP is only recognized by the cold
loader (Genesis.)

\item[145:] \hspace{2em} {\tt FOP-ASSEMBLER-ROUTINE} \hspace{2em} {\it offset}(4)
\hspace{2em} $\Rightarrow$ \hspace{2em} stack\\
This operation records an entry point into an assembler code object (for use
with {\tt FOP-ASSEMBLER-FIXUP}).  The routine name (a symbol) is on stack top.
The code object is underneath.  The entry point is defined at {\it offset}
bytes inside the code area of the code object, and the code object is left on
stack top (allowing multiple uses of this FOP to be chained.)  This FOP is only
recognized by the cold loader (Genesis.)

\item[146:] Unused

\item[147:] \hspace{2em} {\tt FOP-FOREIGN-FIXUP} \hspace{2em} {\it len}(1)
\hspace{2em} {\it name}({\it len})
\hspace{2em} {\it offset}(4) \hspace{2em} $\Rightarrow$ \hspace{2em} stack\\
This operation resolves a reference to a foreign (C) symbol.  {\it len} bytes
are read and interpreted as the symbol {\it name}.  First the {\it kind} and the
code-object to patch are popped from the stack.  The kind is a target-dependent
symbol indicating the instruction format of the patch target (at {\it offset}
bytes from the start of the code area.)  The code object is left on
stack top (allowing multiple uses of this FOP to be chained.)

\item[148:] \hspace{2em} {\tt FOP-ASSEMBLER-FIXUP} \hspace{2em} {\it offset}(4)
\hspace{2em} $\Rightarrow$ \hspace{2em} stack\\
This operation resolves a reference to an assembler routine.  The stack args
are ({\it routine-name}, {\it kind} and {\it code-object}).  The kind is a
target-dependent symbol indicating the instruction format of the patch target
(at {\it offset} bytes from the start of the code area.)  The code object is
left on stack top (allowing multiple uses of this FOP to be chained.)

\item[149:] \hspace{2em} {\tt FOP-CODE-OBJECT-FIXUP} 
\hspace{2em} $\Rightarrow$ \hspace{2em} stack\\
{\it FIXME:} Describe what this does!

\item[150:] \hspace{2em} {\tt FOP-FOREIGN-DATA-FIXUP} 
\hspace{2em} $\Rightarrow$ \hspace{2em} stack\\
{\it FIXME:} Describe what this does!

\item[151-156:] Unused

\item[157:] \hspace{2em} {\tt FOP-LONG-CODE-FORMAT} \hspace{2em} {\it implementation}(1)
\hspace{2em} {\it version}(4) \\
Like FOP-CODE-FORMAT, except that the version is 32 bits long.

\item[158-199:] Unused

\item[200:] \hspace{2em} {\tt FOP-RPLACA} \hspace{2em} {\it table-idx}(4)
\hspace{2em} {\it cdr-offset}(4)\\

\item[201:] \hspace{2em} {\tt FOP-RPLACD} \hspace{2em} {\it table-idx}(4)
\hspace{2em} {\it cdr-offset}(4)\\
These operations destructively modify a list entered in the table.  {\it
table-idx} is the table entry holding the list, and {\it cdr-offset} designates
the cons in the list to modify (like the argument to {\tt nthcdr}.)  The new
value is popped off of the stack, and stored in the {\tt car} or {\tt cdr},
respectively.

\item[202:] \hspace{2em} {\tt FOP-SVSET} \hspace{2em} {\it table-idx}(4)
\hspace{2em} {\it vector-idx}(4)\\
Destructively modifies a {\tt simple-vector} entered in the table.  Pops the
new value off of the stack, and stores it in the {\it vector-idx} element of
the contents of the table entry {\it table-idx.}

\item[203:] \hspace{2em} {\tt FOP-NTHCDR} \hspace{2em} {\it cdr-offset}(4)
\hspace{2em} $\Rightarrow$ \hspace{2em} stack\\
Does {\tt nthcdr} on the top-of stack, leaving the result there.

\item[204:] \hspace{2em} {\tt FOP-STRUCTSET} \hspace{2em} {\it table-idx}(4)
\hspace{2em} {\it vector-idx}(4)\\
Like {\tt FOP-SVSET}, except it alters structure slots.

\item[205-254:] Unused
\item[255:] \hspace{2em} {\tt FOP-END-HEADER} \\ Indicates the end of a group header,
as described above.
\end{description}


\appendix
\chapter{Glossary}% -*- Dictionary: int:design -*-

% Note: in an entry, any word that is also defined should be \it
% should entries have page references as well?

\begin{description}
\item[assert (a type)]
In Python, all type checking is done via a general type assertion
mechanism.  Explicit declarations and implicit assertions (e.g. the arg to
+ is a number) are recorded in the front-end (implicit continuation)
representation.  Type assertions (and thus type-checking) are "unbundled"
from the operations that are affected by the assertion.  This has two major
advantages:
\begin{itemize}
\item Code that implements operations need not concern itself with checking
operand types.

\item Run-time type checks can be eliminated when the compiler can prove that
the assertion will always be satisfied.
\end{itemize}
See also {\it restrict}.

\item[back end] The back end is the part of the compiler that operates on the
{\it virtual machine} intermediate representation.  Also included are the
compiler phases involved in the conversion from the {\it front end}
representation (or {\it ICR}).

\item[bind node] This is a node type the that marks the start of a {\it lambda}
body in {\it ICR}.  This serves as a placeholder for environment manipulation
code.

\item[IR1] The first intermediate representation, also known as {\it ICR}, or
the Implicit Continuation Represenation.

\item[IR2] The second intermediate representation, also known as {\it VMR}, or
the Virtual Machine Representation.

\item[basic block] A basic block (or simply "block") has the pretty much the
usual meaning of representing a straight-line sequence of code.  However, the
code sequence ultimately generated for a block might contain internal branches
that were hidden inside the implementation of a particular operation.  The type
of a block is actually {\tt cblock}.  The {\tt block-info} slot holds an 
{\tt VMR-block} containing backend information.

\item[block compilation] Block compilation is a term commonly used to describe
the compile-time resolution of function names.  This enables many
optimizations.

\item[call graph]
Each node in the call graph is a function (represented by a {\it flow graph}.)
The arcs in the call graph represent a possible call from one function to
another.  See also {\it tail set}.

\item[cleanup]
A cleanup is the part of the implicit continuation representation that
retains information scoping relationships.  For indefinite extent bindings
(variables and functions), we can abandon scoping information after ICR
conversion, recovering the lifetime information using flow analysis.  But
dynamic bindings (special values, catch, unwind protect, etc.) must be
removed at a precise time (whenever the scope is exited.)  Cleanup
structures form a hierarchy that represents the static nesting of dynamic
binding structures.  When the compiler does a control transfer, it can use
the cleanup information to determine what cleanup code needs to be emitted.

\item[closure variable]
A closure variable is any lexical variable that has references outside of
its {\it home environment}.  See also {\it indirect value cell}.

\item[closed continuation] A closed continuation represents a {\tt tagbody} tag
or {\tt block} name that is closed over.  These two cases are mostly
indistinguishable in {\it ICR}.

\item[home] Home is a term used to describe various back-pointers.  A lambda
variable's "home" is the lambda that the variable belongs to.  A lambda's "home
environment" is the environment in which that lambda's variables are allocated.

\item[indirect value cell]
Any closure variable that has assignments ({\tt setq}s) will be allocated in an
indirect value cell.  This is necessary to ensure that all references to
the variable will see assigned values, since the compiler normally freely
copies values when creating a closure.

\item[set variable] Any variable that is assigned to is called a "set
variable".  Several optimizations must special-case set variables, and set
closure variables must have an {\it indirect value cell}.

\item[code generator] The code generator for a {\it VOP} is a potentially
arbitrary list code fragment which is responsible for emitting assembly code to
implement that VOP.

\item[constant pool] The part of a compiled code object that holds pointers to
non-immediate constants.

\item[constant TN]
A constant TN is the {\it VMR} of a compile-time constant value.  A
constant may be immediate, or may be allocated in the {\it constant pool}.

\item[constant leaf]
A constant {\it leaf} is the {\it ICR} of a compile-time constant value.

\item[combination]
A combination {\it node} is the {\it ICR} of any fixed-argument function
call (not {\tt apply} or {\tt multiple-value-call}.)  

\item[top-level component]
A top-level component is any component whose only entry points are top-level
lambdas.

\item[top-level lambda]
A top-level lambda represents the execution of the outermost form on which
the compiler was invoked.  In the case of {\tt compile-file}, this is often a
truly top-level form in the source file, but the compiler can recursively
descend into some forms ({\tt eval-when}, etc.) breaking them into separate
compilations.

\item[component] A component is basically a sequence of blocks.  Each component
is compiled into a separate code object.  With {\it block compilation} or {\it
local functions}, a component will contain the code for more than one function.
This is called a component because it represents a connected portion of the
call graph.  Normally the blocks are in depth-first order ({\it DFO}).

\item[component, initial] During ICR conversion, blocks are temporarily
assigned to initial components.  The "flow graph canonicalization" phase
determines the true component structure.

\item[component, head and tail]
The head and tail of a component are dummy blocks that mark the start and
end of the {\it DFO} sequence.  The component head and tail double as the root
and finish node of the component's flow graph.

\item[local function (call)]
A local function call is a call to a function known at compile time to be
in the same {\it component}.  Local call allows compile time resolution of the
target address and calling conventions.  See {\it block compilation}.

\item[conflict (of TNs, set)]
Register allocation terminology.  Two TNs conflict if they could ever be
live simultaneously.  The conflict set of a TN is all TNs that it conflicts
with.

\item[continuation]
The ICR data structure which represents both:
\begin{itemize}
\item The receiving of a value (or multiple values), and

\item A control location in the flow graph.
\end{itemize}
In the Implicit Continuation Representation, the environment is implicit in the
continuation's BLOCK (hence the name.)  The ICR continuation is very similar to
a CPS continuation in its use, but its representation doesn't much resemble (is
not interchangeable with) a lambda.

\item[cont] A slot in the {\it node} holding the {\it continuation} which
receives the node's value(s).  Unless the node ends a {\it block}, this also
implicitly indicates which node should be evaluated next.

\item[cost] Approximations of the run-time costs of operations are widely used
in the back end.  By convention, the unit is generally machine cycles, but the
values are only used for comparison between alternatives.  For example, the
VOP cost is used to determine the preferred order in which to try possible
implementations.
    
\item[CSP, CFP] See {\it control stack pointer} and {\it control frame
pointer}.

\item[Control stack] The main call stack, which holds function stack frames.
All words on the control stack are tagged {\it descriptors}.  In all ports done
so far, the control stack grows from low memory to high memory.  The most
recent call frames are considered to be ``on top'' of earlier call frames.

\item[Control stack pointer] The allocation pointer for the {\it control
stack}.  Generally this points to the first free word at the top of the stack.

\item[Control frame pointer] The pointer to the base of the {\it control stack}
frame for a particular function invocation.  The CFP for the running function
must be in a register.

\item[Number stack] The auxiliary stack used to hold any {\it non-descriptor}
(untagged) objects.  This is generally the same as the C call stack, and thus
typically grows down.

\item[Number stack pointer] The allocation pointer for the {\it number stack}.
This is typically the C stack pointer, and is thus kept in a register.

\item[NSP, NFP] See {\it number stack pointer}, {\it number frame pointer}.

\item[Number frame pointer] The pointer to the base of the {\it number stack}
frame for a particular function invocation.  Functions that don't use the
number stack won't have an NFP, but if an NFP is allocated, it is always
allocated in a particular register.  If there is no variable-size data on the
number stack, then the NFP will generally be identical to the NSP.

\item[Lisp return address] The name of the {\it descriptor} encoding the
"return pc" for a function call.

\item[LRA] See {\it lisp return address}.  Also, the name of the register where
the LRA is passed.


\item[Code pointer] A pointer to the header of a code object.  The code pointer
for the currently running function is stored in the {\tt code} register.

\item[Interior pointer] A pointer into the inside of some heap-allocated
object.  Interior pointers confuse the garbage collector, so their use is
highly constrained.  Typically there is a single register dedicated to holding
interior pointers.

\item[dest]
A slot in the {\it continuation} which points the the node that receives this
value.  Null if this value is not received by anyone.

\item[DFN, DFO] See {\it Depth First Number}, {\it Depth First Order}.

\item[Depth first number] Blocks are numbered according to their appearance in
the depth-first ordering (the {\tt block-number} slot.)  The numbering actually
increases from the component tail, so earlier blocks have larger numbers.

\item[Depth first order] This is a linearization of the flow graph, obtained by
a depth-first walk.  Iterative flow analysis algorithms work better when blocks
are processed in DFO (or reverse DFO.)


\item[Object] In low-level design discussions, an object is one of the
following:
\begin{itemize}
\item a single word containing immediate data (characters, fixnums, etc)
\item a single word pointing to an object (structures, conses, etc.)
\end{itemize}
These are tagged with three low-tag bits as described in the section
\ref{tagging} This is synonymous with {\it descriptor}.
In other parts of the documentation, may be used more loosely to refer to a
{\it lisp object}.

\item[Lisp object]
A Lisp object is a high-level object discussed as a data type in the Common
Lisp definition.

\item[Data-block]
A data-block is a dual-word aligned block of memory that either manifests a
Lisp object (vectors, code, symbols, etc.) or helps manage a Lisp object on
the heap (array header, function header, etc.).

\item[Descriptor]
A descriptor is a tagged, single-word object.  It either contains immediate
data or a pointer to data.  This is synonymous with {\it object}.  Storage
locations that must contain descriptors are referred to as descriptor
locations.

\item[Pointer descriptor]
A descriptor that points to a {\it data block} in memory (i.e. not an immediate
object.)

\item[Immediate descriptor]
A descriptor that encodes the object value in the descriptor itself; used for
characters, fixnums, etc.

\item[Word]
A word is a 32-bit quantity.

\item[Non-descriptor]
Any chunk of bits that isn't a valid tagged descriptor.  For example, a
double-float on the number stack.  Storage locations that are not scanned by
the garbage collector (and thus cannot contain {\it pointer descriptors}) are
called non-descriptor locations.  {\it Immediate descriptors} can be stored in
non-descriptor locations.


\item[Entry point] An entry point is a function that may be subject to
``unpredictable'' control transfers.  All entry points are linked to the root
of the flow graph (the component head.)  The only functions that aren't entry
points are {\it let} functions.  When complex lambda-list syntax is used,
multiple entry points may be created for a single lisp-level function.
See {\it external entry point}.

\item[External entry point] A function that serves as a ``trampoline'' to
intercept function calls coming in from outside of the component.  The XEP does
argument syntax and type checking, and may also translate the arguments and
return values for a locally specialized calling calling convention.

\item[XEP] An {\it external entry point}.

\item[lexical environment] A lexical environment is a structure that is used
during VMR conversion to represent all lexically scoped bindings (variables,
functions, declarations, etc.)  Each {\tt node} is annotated with its lexical
environment, primarily for use by the debugger and other user interfaces.  This
structure is also the environment object passed to {\tt macroexpand}.

\item[environment] The environment is part of the ICR, created during
environment analysis.  Environment analysis apportions code to disjoint
environments, with all code in the same environment sharing the same stack
frame.  Each environment has a ``{\it real}'' function that allocates it, and
some collection {\tt let} functions.   Although environment analysis is the
last ICR phase, in earlier phases, code is sometimes said to be ``in the
same/different environment(s)''.  This means that the code will definitely be
in the same environment (because it is in the same real function), or that is
might not be in the same environment, because it is not in the same function.

\item[fixup]  Some sort of back-patching annotation.  The main sort encountered
are load-time {\it assembler fixups}, which are a linkage annotation mechanism.

\item[flow graph] A flow graph is a directed graph of basic blocks, where each
arc represents a possible control transfer.  The flow graph is the basic data
structure used to represent code, and provides direct support for data flow
analysis.  See component and ICR.

\item[foldable] An attribute of {\it known functions}.  A function is foldable
if calls may be constant folded whenever the arguments are compile-time
constant.  Generally this means that it is a pure function with no side
effects.


FSC
full call
function attribute
function
	"real" (allocates environment)
	meaning function-entry
	more vague (any lambda?)
funny function
GEN (kill and...)
global TN, conflicts, preference
GTN (number)
IR ICR VMR  ICR conversion, VMR conversion (translation)
inline expansion, call
kill (to make dead)
known function
LAMBDA
leaf
let call
lifetime analysis, live (tn, variable)
load tn
LOCS (passing, return locations)
local call
local TN, conflicts, (or just used in one block)
location (selection)
LTN (number)
main entry
mess-up (for cleanup)
more arg (entry)
MV
non-local exit
non-packed SC, TN
non-set variable
operand (to vop)
optimizer (in icr optimize)
optional-dispatch
pack, packing, packed
pass (in a transform)
passing 
	locations (value)
	conventions (known, unknown)
policy (safe, fast, small, ...)
predecessor block
primitive-type
reaching definition
REF
representation
	selection
	for value
result continuation (for function)
result type assertion (for template) (or is it restriction)
restrict
	a TN to finite SBs
	a template operand to a primitive type (boxed...)
	a tn-ref to particular SCs

return (node, vops)
safe, safety
saving (of registers, costs)
SB
SC (restriction)
semi-inline
side-effect
	in ICR
	in VMR
sparse set
splitting (of VMR blocks)
SSET
SUBPRIMITIVE
successor block
tail recursion
	tail recursive
	tail recursive loop
	user tail recursion

template
TN
TNBIND
TN-REF
transform (source, ICR)
type
	assertion
	inference
		top-down, bottom-up
	assertion propagation
        derived, asserted
	descriptor, specifier, intersection, union, member type
        check
type-check (in continuation)
UNBOXED (boxed) descriptor
unknown values continuation
unset variable
unwind-block, unwinding
used value (dest)
value passing
VAR
VM
VOP
XEP

\end{description}

\end{document}
